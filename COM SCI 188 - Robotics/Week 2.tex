% document formatting
\documentclass[10pt]{article}
\usepackage[utf8]{inputenc}
\usepackage[left=1in,right=1in,top=1in,bottom=1in]{geometry}
\usepackage[T1]{fontenc}
\usepackage{xcolor}

% math symbols, etc.
\usepackage{amsmath, amsfonts, amssymb, amsthm}
\usepackage{xcolor}

% lists
\usepackage{enumerate}

% images
\usepackage{graphicx} % for images

% code blocks
\usepackage{minted, listings} 

% verbatim greek
\usepackage{alphabeta}

\newcommand{\dd}{\text{d}}

\graphicspath{{./assets/images/Week 2}}

\title{CS 188 Robotics Week 2} 

\author{Aidan Jan}

\date{\today}

\begin{document}
\maketitle 

\section*{Rigid Body Motions}

\subsection*{Representing Position}
[FILL 11]

\subsection*{2D Transformation: Translation}
Translate the point $p$ to $p'$ with $T = (\dd x, \dd y)$:
\[p' = T + p\]
\[\begin{bmatrix} x' \\ y' \end{bmatrix} = \begin{bmatrix} d_x \\ d_y \end{bmatrix} + \begin{bmatrix} x \\ y \end{bmatrix}\]
[FILL 12, graph only]

\subsection*{2D Transformation: Rotation}
\[p' = R \cdot p\]
Here we are doing a counter-clockwise rotation
[FILL 13]
The triangle here helps us visualize the rotation.  However, we are still considering one 2D point $p$.
\begin{align*}
    \begin{bmatrix} x' \\ y' \end{bmatrix} &= \begin{bmatrix} \cos \theta & -\sin \theta \\ \sin \theta & \cos \theta \end{bmatrix} \cdot \begin{bmatrix} x \\ y \end{bmatrix}\\
    x' &= x \cos \theta - y \sin \theta\\
    y' &= x \sin \theta + y \cos \theta
\end{align*}

\subsection*{Combining Rotation and Transformation}
\[p' = R \cdot p + T\]
In general, a matrix multiplication lets us linearly combine components of a vector.
\begin{itemize}
	\item It is sufficient for representing rotation, but we can't add a constant :(
\end{itemize}
\[\begin{bmatrix} a & b \\ c & d \end{bmatrix} \times \begin{bmatrix} x \\ y \end{bmatrix} = \begin{bmatrix} ax + by \\ cx + dy \end{bmatrix}\]

\subsection*{Homogeneous Coordinates}
\begin{itemize}
	\item The solution?  Stick a "1" at the end of every vector.
	\item Now, we can do rotation AND translation
	\item This is called "homogeneous coordinates"
	\[\begin{bmatrix} a & b & c \\ d & e & f \\ 0 & 0 & 1 \end{bmatrix} \times \begin{bmatrix} x \\ y \\ 1 \end{bmatrix} = \begin{bmatrix} ax + by + c \\ dx + ey + f \\ 1 \end{bmatrix}\]
    \item Our old way of representing point is called "Cartesian coordinate system"
\end{itemize}

\subsection*{Cartesian and Homogeneous Coordinate}
\begin{itemize}
	\item A point in cartesian coordinate $\langle x, y \rangle$ can be represented by $\langle sx, sy, s \rangle$ in homogeneous coordinate, where $s$ is any scalar number.
	\begin{itemize}
        \item For example, $\langle 2, 3 \rangle$ in cartesian coordinate can be represented as $\langle 2, 3, 1 \rangle$ or $\langle 4, 6, 2 \rangle$, or $\langle 1, 1.5, 0.5\rangle$, etc. in homogeneous coordinates
        \item A point in homogeneous coordinate $\langle x, y, z \rangle$ can be converted to cartesian coordinates by dividing the last element $\langle x/z, y/z \rangle$
        \item Similarly for higher dimensions
    \end{itemize}
\end{itemize}

\subsection*{Transformation Matrices}
Representing rotation and translation homogeneous coordinates
\begin{itemize}
	\item 2D Translation
	\[\begin{bmatrix} x' \\ y' \\ 1 \end{bmatrix} = \begin{bmatrix} 1 & 0 & t_x \\ 0 & 1 & t_y \\ 0 & 0 & 1 \end{bmatrix} \begin{bmatrix} x \\ y \\ 1 \end{bmatrix}\]
	\item 2D Rotation
	\[\begin{bmatrix} x' \\ y' \\ 1 \end{bmatrix} = \begin{bmatrix} \cos \theta & -\sin \theta & 0 \\ \sin \theta & \cos \theta & 0 \\ 0 & 0 & 1 \end{bmatrix} \begin{bmatrix} x \\ y \\ 1 \end{bmatrix}\]
\end{itemize}
Now we can represent both the rotation and translation operation with one \underline{transformation matrix}.
\[\begin{bmatrix} x' \\ y' \\ 1 \end{bmatrix} = \begin{bmatrix} \cos \theta & -\sin \theta & t_x \\ \sin \theta & \cos \theta & t_y \\ 0 & 0 & 1 \end{bmatrix} \begin{bmatrix} x \\ y \\ 1 \end{bmatrix}\]
Note: Following the matrix multiplication rule, a transformation matrix always apply rotation first, then translation.
\begin{itemize}
	\item Matrix multiplication is \textit{not} commutative.
\end{itemize}

\section*{3D Transformation}
Our examples so far were all in 2D, but we often want a 3D representation
[FILL 21]

\subsection*{Right Hand Rule}
[FILL 24]
Most of robotics system's coordinate system follows the right hand rule
\begin{itemize}
	\item Not always true (e.g., in some graphics and physics engine directX Unity)
	\item Therefore, be careful!
\end{itemize}

\subsection*{3D Transformation: Translation}
A 3D point $(x, y, z)$, translation by $t_x, t_y, t_z$:
\[\begin{bmatrix} x' \\ y' \\ z' \end{bmatrix} = \begin{bmatrix} t_x \\ t_y \\ t_z \end{bmatrix} + \begin{bmatrix} x \\ y \\ z \end{bmatrix}\]
[FILL 26]

\subsection*{3D Transformation: Rotation}
\begin{itemize}
	\item A rotation in 2D is around a point
	\item A rotation in 3D is around an \underline{axis} (a line with direction)
	\begin{itemize}
        \item rotation direction also follows right hand rule (thumb points to the axis direction, other fingers points towards the \textbf{positive} rotation direction)
        \item It is a 3D space, not just 1D
        \item most common choices for rotation axes are the $x$, $y$, $z$-axes (Euler angle representation)
    \end{itemize}
    [FILL 27]
\end{itemize}

\subsection*{3D Rotation Matrices}
\begin{align*}
R_x (\theta) &= \begin{bmatrix} 1 & 0 & 0 \\ 0 & \cos \theta & -\sin \theta \\ 0 & \sin \theta & \cos \theta \end{bmatrix}\\
R_y (\theta) &= \begin{bmatrix} \cos \theta & 0 & \sin \theta \\ 0 & 1 & 0 \\ -\sin \theta & 0 & \cos \theta \end{bmatrix}\\
R_z (\theta) &= \begin{bmatrix} \cos \theta & -\sin \theta & 0 \\ \sin \theta & \cos \theta & 0 \\ 0 & 0 & 1 \end{bmatrix}
\end{align*}

\subsection*{Reference Frames (Coordinate System)}
\begin{itemize}
	\item Up to now we have look at transformation in a single reference frame.  However, in a complex robotic system we often need to define many reference frames.
	\item The same 3D point might have different coordinate if we use different reference frames, next we will learn how to transform between different reference frames.
\end{itemize}
Example: green dot's coordinate is $(2, 1)$ in blue reference frame, but its coordinate is $(4, 4)$ in red reference frame.
[FILL 30]
Changing coordinate frame is like translating between two different languages that describes the same thing.

\subsubsection*{Examples:}
[FILL 31, full]

\subsection*{Changing Reference Frames}
\begin{itemize}
	\item We define two coordinate frames A and B
	\item A Point $P$:
	\begin{itemize}
        \item $P$'s coordinate in Frame A is $^A P = (4, 4)$
        \item $P$'s coordinate in Frame B is $^B P = (2, 1)$
    \end{itemize}
    \item Transformations between reference frames we will use the notation $^{\textcolor{red}{A}} T_{\textcolor{blue}{B}}$ (\textcolor{blue}{FROM} frame is in the bottom right and the \textcolor{red}{TO} frame is in the top left.)
    \item To transform $^B P$'s reference frame from B to A, we just need to apply $^A T_B$ to $^B P$.
    \[^A P = \:^A T_B \cdot \:^B P\]
\end{itemize}
How do we compute $^A T_B$?
\begin{itemize}
	\item Suppose the point $P$ is \underline{rigidly} attached to reference Frame B.
	\item No matter where the reference B, point $P$ is its coordinates with respect to Frame B is always given by $^B P = (2, 1)$.
\end{itemize}
[FILL 37, edited, 2x2 grid]
First, let's make Frame B identical to Frame A.  Now, $^A P = \:^B P = (2, 1)$.  Now, simply \underline{translate} Frame B together with $d = (2, 3)$, we will get the $^A P = \:^B P + d$.\\
Therefore in this case,
\[^A T_B = \begin{bmatrix} 1 & 0 & 2 \\ 0 & 1 & 3 \\ 0 & 0 & 1 \end{bmatrix}\]
(There is no rotation in this case, only translation)
\begin{itemize}
	\item If there is a rotation, first rotate the frame so it is aligned with the target, then do a translation.
\end{itemize}
[FILL 47, edited]
If we combine this rotation and translation into one transformation matrix, we get:
\[T = \begin{bmatrix} R_\theta & d \\ 0_n & 1 \end{bmatrix}\]
This is the transformation $^A T_B$ that change the coordinate frame from B to A.
\begin{itemize}
	\item However, geometrically it describes the motion from Frame A to B.
	\item $^A T_B$ also describes Frame B's "pose" in Frame A, where the rotation component R describes the B's orientation in Frame A, and the translation represents B's position in Frame A.
\end{itemize}
[FILL 52]

\subsection*{Change of Basis Summary}
What is $^A T_B$?
\begin{itemize}
	\item $^A T_B$ is a rigid transformation matrix (3x3 matrix in 2D, 4x4 in 3D)
	\item $^A T_B$ represents the transform that \textbf{change the coordinate frame from B to A:} $^A P = \:^A T_B \:^B P$
	\item $^A T_B$ geometrically describes the motion from Frame A to B.
	\item $^A T_B$ is also the \underline{pose} of coordinate frame (B) in the coordinate frame (A); that describes the \underline{position} and \underline{orientation} of Frame B in Frame A.
\end{itemize}

\subsection*{Composing Transformation}
[FILL 54, fill]

\subsection*{Chained 3D Rotation}
We can chain a sequence of Euler angle rotations (multiple sequence of rotation matrix) to get a general 3D rotation.
\[R = R_z(\alpha) R_y(\beta) R_x(\gamma) = \begin{bmatrix} \cos \alpha & -\sin \alpha & 0 \\ \sin \alpha \cos \alpha & 0 \\ 0 & 0 & 1 \end{bmatrix} \begin{bmatrix} \cos \beta & 0 & \sin \beta \\ 0 & 1 & 0 \\ -\sin \beta & 0 & \cos \beta \end{bmatrix} \begin{bmatrix} 1 & 0 & 0 \\ 0 & \cos \gamma & -\sin \gamma \\ 0 & \sin \gamma & \cos \gamma \end{bmatrix}\]
There are a few things to note when writing down the sequence of rotation:
\begin{enumerate}
    \item Rotation matrix is non-commutative - order matters!
    \item Be aware of which sequence convention you are using when describing the 2nd and 3rd rotations: \textbf{extrinsic} rotation (fixed global frame), or \textbf{intrinsic} rotation? (last rotated coordinate system) - they are different.
\end{enumerate}

\subsection*{Extrinsic vs. Intrinsic Rotation}
[FILL 57, 58, incl. text, edited]
The final rotation $R$ is the same.  However, the order of describing rotation sequence is opposite in each convention.
\begin{itemize}
	\item (Use premultiply!)
\end{itemize}

\subsection*{Rotation Matrix}
Rotation matrix has a number of highly useful properties:
\begin{itemize}
	\item R is an orthonormal matrix: Its columns are orthogonal unit vectors. ($R^{-1} = R^T$)
	\begin{itemize}
        \item This does not apply to general transformation matrices.
    \end{itemize}
	\item determinant of the matrix $|R| = 1$
	\item The length of the vector is unchanged after transformation
\end{itemize}

\subsection*{Other 3D Rotation Representations}
There are many ways to specify rotation
\begin{itemize}
	\item Rotation matrix
	\item Euler angles: 3 angles about 3 axes
	\item Axis-angle representation
	\item Quaternions
\end{itemize}

\subsection*{Axis Angle Representation}
Parameterize a 3D rotation by two quantities: a unit vector $e$ indicating the direction of an axis of rotation, and an angle $\theta$ describing the magnitude of the rotation about the axis.
\begin{itemize}
	\item Euler's rotation theorem: any rotation or sequence of rotations of a rigid body in a three-dimensional space is equivalent to a single rotation about a single fixed axis.
\end{itemize}
[FILL 63]

\subsection*{Quaternions}
Uses a unit four-dimensional vector $(x, y, z, w)$ to represent rotation.
\begin{itemize}
	\item If the rotation is $(v_1, v_2, v_3, \theta)$ in angle-axis representation, it can be written in quaternion as:
\end{itemize}
\begin{align*}
    x &= v_1 \sin \frac{\theta}{2}\\
    y &= v_2 \sin \frac{\theta}{2}\\
    z &= v_3 \sin \frac{\theta}{2}\\
    w &= \cos \frac{\theta}{2}\\
\end{align*}
\[x^2 + y^2 + z^2 + w^2 = 1\]
\begin{itemize}
	\item the above is a 4-dimensional vector on a 4D sphere.
\end{itemize}
Quaternions are a very popular parameterization due to the following properties:
\begin{itemize}
	\item More \underline{compact} than the matrix representation (4 numbers instead of 9 numbers)
	\item The quaternion elements vary \underline{continuously} over the unit sphere in $\mathbb{R}^4$ as the orientation changes, avoiding \underline{discontinuous} jumps (it is important for many optimization or learning algorithms).
\end{itemize}
To inverse a quaternion:
\begin{itemize}
	\item keep the rotation axis, rotate backward
	\item Inverse of $(x, y, z, w)$ is $(x, y, z, -w)$
	\item $(x, y, z, w)$ is equivalent to $(-x, -y, -z, -w)$
\end{itemize}

\end{document}