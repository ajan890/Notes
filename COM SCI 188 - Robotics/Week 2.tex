% document formatting
\documentclass[10pt]{article}
\usepackage[utf8]{inputenc}
\usepackage[left=1in,right=1in,top=1in,bottom=1in]{geometry}
\usepackage[T1]{fontenc}
\usepackage{xcolor}

% math symbols, etc.
\usepackage{amsmath, amsfonts, amssymb, amsthm}
\usepackage{xcolor}

% lists
\usepackage{enumerate}

% images
\usepackage{graphicx} % for images

% code blocks
\usepackage{minted, listings} 

% verbatim greek
\usepackage{alphabeta}

\newcommand{\dd}{\text{d}}

\graphicspath{{./assets/images/Week 2}}

\title{CS 188 Robotics Week 2} 

\author{Aidan Jan}

\date{\today}

\begin{document}
\maketitle 

\section*{Rigid Body Motions}

\subsection*{Representing Position}
\begin{center} 
	\includegraphics*[width=0.8\textwidth]{L1_1.png} 
\end{center}

\subsection*{2D Transformation: Translation}
Translate the point $p$ to $p'$ with $T = (\dd x, \dd y)$:
\[p' = T + p\]
\[\begin{bmatrix} x' \\ y' \end{bmatrix} = \begin{bmatrix} d_x \\ d_y \end{bmatrix} + \begin{bmatrix} x \\ y \end{bmatrix}\]
\begin{center} 
	\includegraphics*[width=0.3\textwidth]{L1_2.png} 
\end{center}


\subsection*{2D Transformation: Rotation}
\[p' = R \cdot p\]
Here we are doing a counter-clockwise rotation
\begin{center} 
	\includegraphics*[width=0.4\textwidth]{L1_3.png} 
\end{center}

The triangle here helps us visualize the rotation.  However, we are still considering one 2D point $p$.
\begin{align*}
    \begin{bmatrix} x' \\ y' \end{bmatrix} &= \begin{bmatrix} \cos \theta & -\sin \theta \\ \sin \theta & \cos \theta \end{bmatrix} \cdot \begin{bmatrix} x \\ y \end{bmatrix}\\
    x' &= x \cos \theta - y \sin \theta\\
    y' &= x \sin \theta + y \cos \theta
\end{align*}

\subsection*{Combining Rotation and Transformation}
\[p' = R \cdot p + T\]
In general, a matrix multiplication lets us linearly combine components of a vector.
\begin{itemize}
	\item It is sufficient for representing rotation, but we can't add a constant :(
\end{itemize}
\[\begin{bmatrix} a & b \\ c & d \end{bmatrix} \times \begin{bmatrix} x \\ y \end{bmatrix} = \begin{bmatrix} ax + by \\ cx + dy \end{bmatrix}\]

\subsection*{Homogeneous Coordinates}
\begin{itemize}
	\item The solution?  Stick a "1" at the end of every vector.
	\item Now, we can do rotation AND translation
	\item This is called "homogeneous coordinates"
	\[\begin{bmatrix} a & b & c \\ d & e & f \\ 0 & 0 & 1 \end{bmatrix} \times \begin{bmatrix} x \\ y \\ 1 \end{bmatrix} = \begin{bmatrix} ax + by + c \\ dx + ey + f \\ 1 \end{bmatrix}\]
    \item Our old way of representing point is called "Cartesian coordinate system"
\end{itemize}

\subsection*{Cartesian and Homogeneous Coordinate}
\begin{itemize}
	\item A point in cartesian coordinate $\langle x, y \rangle$ can be represented by $\langle sx, sy, s \rangle$ in homogeneous coordinate, where $s$ is any scalar number.
	\begin{itemize}
        \item For example, $\langle 2, 3 \rangle$ in cartesian coordinate can be represented as $\langle 2, 3, 1 \rangle$ or $\langle 4, 6, 2 \rangle$, or $\langle 1, 1.5, 0.5\rangle$, etc. in homogeneous coordinates
        \item A point in homogeneous coordinate $\langle x, y, z \rangle$ can be converted to cartesian coordinates by dividing the last element $\langle x/z, y/z \rangle$
        \item Similarly for higher dimensions
    \end{itemize}
\end{itemize}

\subsection*{Transformation Matrices}
Representing rotation and translation homogeneous coordinates
\begin{itemize}
	\item 2D Translation
	\[\begin{bmatrix} x' \\ y' \\ 1 \end{bmatrix} = \begin{bmatrix} 1 & 0 & t_x \\ 0 & 1 & t_y \\ 0 & 0 & 1 \end{bmatrix} \begin{bmatrix} x \\ y \\ 1 \end{bmatrix}\]
	\item 2D Rotation
	\[\begin{bmatrix} x' \\ y' \\ 1 \end{bmatrix} = \begin{bmatrix} \cos \theta & -\sin \theta & 0 \\ \sin \theta & \cos \theta & 0 \\ 0 & 0 & 1 \end{bmatrix} \begin{bmatrix} x \\ y \\ 1 \end{bmatrix}\]
\end{itemize}
Now we can represent both the rotation and translation operation with one \underline{transformation matrix}.
\[\begin{bmatrix} x' \\ y' \\ 1 \end{bmatrix} = \begin{bmatrix} \cos \theta & -\sin \theta & t_x \\ \sin \theta & \cos \theta & t_y \\ 0 & 0 & 1 \end{bmatrix} \begin{bmatrix} x \\ y \\ 1 \end{bmatrix}\]
Note: Following the matrix multiplication rule, a transformation matrix always apply rotation first, then translation.
\begin{itemize}
	\item Matrix multiplication is \textit{not} commutative.
\end{itemize}

\section*{3D Transformation}
Our examples so far were all in 2D, but we often want a 3D representation
\begin{center} 
	\includegraphics*{L1_4.png} 
\end{center}


\subsection*{Right Hand Rule}
\begin{center} 
	\includegraphics*[width=\textwidth]{L1_5.png} 
\end{center}

Most of robotics system's coordinate system follows the right hand rule
\begin{itemize}
	\item Not always true (e.g., in some graphics and physics engine directX Unity)
	\item Therefore, be careful!
\end{itemize}

\subsection*{3D Transformation: Translation}
A 3D point $(x, y, z)$, translation by $t_x, t_y, t_z$:
\[\begin{bmatrix} x' \\ y' \\ z' \end{bmatrix} = \begin{bmatrix} t_x \\ t_y \\ t_z \end{bmatrix} + \begin{bmatrix} x \\ y \\ z \end{bmatrix}\]
\begin{center} 
	\includegraphics*[width=0.5\textwidth]{L1_6.png} 
\end{center}


\subsection*{3D Transformation: Rotation}
\begin{itemize}
	\item A rotation in 2D is around a point
	\item A rotation in 3D is around an \underline{axis} (a line with direction)
	\begin{itemize}
        \item rotation direction also follows right hand rule (thumb points to the axis direction, other fingers points towards the \textbf{positive} rotation direction)
        \item It is a 3D space, not just 1D
        \item most common choices for rotation axes are the $x$, $y$, $z$-axes (Euler angle representation)
    \end{itemize}
    \begin{center} 
        \includegraphics*[width=0.6\textwidth]{L1_7.png} 
    \end{center}
    
\end{itemize}

\subsection*{3D Rotation Matrices}
\begin{align*}
R_x (\theta) &= \begin{bmatrix} 1 & 0 & 0 \\ 0 & \cos \theta & -\sin \theta \\ 0 & \sin \theta & \cos \theta \end{bmatrix}\\
R_y (\theta) &= \begin{bmatrix} \cos \theta & 0 & \sin \theta \\ 0 & 1 & 0 \\ -\sin \theta & 0 & \cos \theta \end{bmatrix}\\
R_z (\theta) &= \begin{bmatrix} \cos \theta & -\sin \theta & 0 \\ \sin \theta & \cos \theta & 0 \\ 0 & 0 & 1 \end{bmatrix}
\end{align*}

\subsection*{Reference Frames (Coordinate System)}
\begin{itemize}
	\item Up to now we have look at transformation in a single reference frame.  However, in a complex robotic system we often need to define many reference frames.
	\item The same 3D point might have different coordinate if we use different reference frames, next we will learn how to transform between different reference frames.
\end{itemize}
Example: green dot's coordinate is $(2, 1)$ in blue reference frame, but its coordinate is $(4, 4)$ in red reference frame.
\begin{center} 
	\includegraphics*[width=0.3\textwidth]{L1_8.png} 
\end{center}
Changing coordinate frame is like translating between two different languages that describes the same thing.

\subsubsection*{Examples:}
\begin{center} 
	\includegraphics*[width=\textwidth]{L1_9.png} 
\end{center}


\subsection*{Changing Reference Frames}
\begin{itemize}
	\item We define two coordinate frames A and B
	\item A Point $P$:
	\begin{itemize}
        \item $P$'s coordinate in Frame A is $^A P = (4, 4)$
        \item $P$'s coordinate in Frame B is $^B P = (2, 1)$
    \end{itemize}
    \item Transformations between reference frames we will use the notation $^{\textcolor{red}{A}} T_{\textcolor{blue}{B}}$ (\textcolor{blue}{FROM} frame is in the bottom right and the \textcolor{red}{TO} frame is in the top left.)
    \item To transform $^B P$'s reference frame from B to A, we just need to apply $^A T_B$ to $^B P$.
    \[^A P = \:^A T_B \cdot \:^B P\]
\end{itemize}
How do we compute $^A T_B$?
\begin{itemize}
	\item Suppose the point $P$ is \underline{rigidly} attached to reference Frame B.
	\item No matter where the reference B, point $P$ is its coordinates with respect to Frame B is always given by $^B P = (2, 1)$.
\end{itemize}
\begin{center} 
	\includegraphics*[width=0.8\textwidth]{L1_10.png} 
\end{center}
First, let's make Frame B identical to Frame A.  Now, $^A P = \:^B P = (2, 1)$.  Now, simply \underline{translate} Frame B together with $d = (2, 3)$, we will get the $^A P = \:^B P + d$.\\
Therefore in this case,
\[^A T_B = \begin{bmatrix} 1 & 0 & 2 \\ 0 & 1 & 3 \\ 0 & 0 & 1 \end{bmatrix}\]
(There is no rotation in this case, only translation)
\begin{itemize}
	\item If there is a rotation, first rotate the frame so it is aligned with the target, then do a translation.
\end{itemize}
\begin{center} 
	\includegraphics*[width=\textwidth]{L1_11.png} 
\end{center}
If we combine this rotation and translation into one transformation matrix, we get:
\[T = \begin{bmatrix} R_\theta & d \\ 0_n & 1 \end{bmatrix}\]
This is the transformation $^A T_B$ that change the coordinate frame from B to A.
\begin{itemize}
	\item However, geometrically it describes the motion from Frame A to B.
	\item $^A T_B$ also describes Frame B's "pose" in Frame A, where the rotation component R describes the B's orientation in Frame A, and the translation represents B's position in Frame A.
\end{itemize}
\begin{center} 
	\includegraphics*[width=\textwidth]{L1_12.png} 
\end{center}


\subsection*{Change of Basis Summary}
What is $^A T_B$?
\begin{itemize}
	\item $^A T_B$ is a rigid transformation matrix (3x3 matrix in 2D, 4x4 in 3D)
	\item $^A T_B$ represents the transform that \textbf{change the coordinate frame from B to A:} $^A P = \:^A T_B \:^B P$
	\item $^A T_B$ geometrically describes the motion from Frame A to B.
	\item $^A T_B$ is also the \underline{pose} of coordinate frame (B) in the coordinate frame (A); that describes the \underline{position} and \underline{orientation} of Frame B in Frame A.
\end{itemize}

\subsection*{Composing Transformation}
\begin{center} 
	\includegraphics*[width=\textwidth]{L1_13.png} 
\end{center}


\subsection*{Chained 3D Rotation}
We can chain a sequence of Euler angle rotations (multiple sequence of rotation matrix) to get a general 3D rotation.
\[R = R_z(\alpha) R_y(\beta) R_x(\gamma) = \begin{bmatrix} \cos \alpha & -\sin \alpha & 0 \\ \sin \alpha \cos \alpha & 0 \\ 0 & 0 & 1 \end{bmatrix} \begin{bmatrix} \cos \beta & 0 & \sin \beta \\ 0 & 1 & 0 \\ -\sin \beta & 0 & \cos \beta \end{bmatrix} \begin{bmatrix} 1 & 0 & 0 \\ 0 & \cos \gamma & -\sin \gamma \\ 0 & \sin \gamma & \cos \gamma \end{bmatrix}\]
There are a few things to note when writing down the sequence of rotation:
\begin{enumerate}
    \item Rotation matrix is non-commutative - order matters!
    \item Be aware of which sequence convention you are using when describing the 2nd and 3rd rotations: \textbf{extrinsic} rotation (fixed global frame), or \textbf{intrinsic} rotation? (last rotated coordinate system) - they are different.
\end{enumerate}

\subsection*{Extrinsic vs. Intrinsic Rotation}
\begin{center} 
	\includegraphics*[width=\textwidth]{L1_14.png} 
\end{center}

The final rotation $R$ is the same.  However, the order of describing rotation sequence is opposite in each convention.
\begin{itemize}
	\item (Use premultiply!)
\end{itemize}

\subsection*{Rotation Matrix}
Rotation matrix has a number of highly useful properties:
\begin{itemize}
	\item R is an orthonormal matrix: Its columns are orthogonal unit vectors. ($R^{-1} = R^T$)
	\begin{itemize}
        \item This does not apply to general transformation matrices.
    \end{itemize}
	\item determinant of the matrix $|R| = 1$
	\item The length of the vector is unchanged after transformation
\end{itemize}

\subsection*{Other 3D Rotation Representations}
There are many ways to specify rotation
\begin{itemize}
	\item Rotation matrix
	\item Euler angles: 3 angles about 3 axes
	\item Axis-angle representation
	\item Quaternions
\end{itemize}

\subsection*{Axis Angle Representation}
Parameterize a 3D rotation by two quantities: a unit vector $e$ indicating the direction of an axis of rotation, and an angle $\theta$ describing the magnitude of the rotation about the axis.
\begin{itemize}
	\item Euler's rotation theorem: any rotation or sequence of rotations of a rigid body in a three-dimensional space is equivalent to a single rotation about a single fixed axis.
\end{itemize}
\begin{center} 
	\includegraphics*[width=0.3\textwidth]{L1_15.png} 
\end{center}


\subsection*{Quaternions}
Uses a unit four-dimensional vector $(x, y, z, w)$ to represent rotation.
\begin{itemize}
	\item If the rotation is $(v_1, v_2, v_3, \theta)$ in angle-axis representation, it can be written in quaternion as:
\end{itemize}
\begin{align*}
    x &= v_1 \sin \frac{\theta}{2}\\
    y &= v_2 \sin \frac{\theta}{2}\\
    z &= v_3 \sin \frac{\theta}{2}\\
    w &= \cos \frac{\theta}{2}\\
\end{align*}
\[x^2 + y^2 + z^2 + w^2 = 1\]
\begin{itemize}
	\item the above is a 4-dimensional vector on a 4D sphere.
\end{itemize}
Quaternions are a very popular parameterization due to the following properties:
\begin{itemize}
	\item More \underline{compact} than the matrix representation (4 numbers instead of 9 numbers)
	\item The quaternion elements vary \underline{continuously} over the unit sphere in $\mathbb{R}^4$ as the orientation changes, avoiding \underline{discontinuous} jumps (it is important for many optimization or learning algorithms).
\end{itemize}
For example:
\begin{itemize}
	\item $(0, 0, 0, 1)$ is the identity quaternion.
	\item $(1, 0, 0, 0)$ rotates along $x$-axis by $\pi$.  (Since $w = 0$, therefore $\theta = \pi$).
\end{itemize}
To inverse a quaternion:
\begin{itemize}
	\item keep the rotation axis, rotate backward
	\item Inverse of $(x, y, z, w)$ is $(x, y, z, -w)$
	\item $(x, y, z, w)$ is equivalent to $(-x, -y, -z, -w)$
\end{itemize}

\section*{Forward Kinematics}
\subsection*{Forward Kinematics of 2-link Manipulator}
Given joint angles, calculate position of end-effector
\begin{center} 
	\includegraphics*[width=\textwidth]{L1_16.png} 
\end{center}

\begin{align*}
    x &= l_1 \cos \theta_1 + l_2 \cos(\theta_1 + \theta_2)\\
    y &= l_1 \sin \theta_1 + l_2 \sin(\theta_1 + \theta_2)
\end{align*}

\subsection*{Forward Kinematics of RRR open-chain}
\begin{center} 
	\includegraphics*[width=\textwidth]{L1_17.png} 
\end{center}

\subsection*{Denavit-Hartenberg (DH) parameters}
\[T_{0n} (\theta_1, \dots, \theta_n) = T_{01} (\theta_1) T_{12} (\theta_2) \cdots T_{n - 1, n} (\theta_n) \hfill T_i = \begin{bmatrix} \cos \theta_i & -\sin \theta_i \cos \alpha_i & \sin \theta_i \sin \alpha_i & a_i \cos \theta_i \\ \sin \theta_i & \cos \theta_i \cos \alpha_i & -\cos \theta_i \sin \alpha_i & a_i \sin \theta_i \\ 0 & \sin \alpha_i & \cos \alpha_i & d_i \\ 0 & 0 & 0 & 1 \end{bmatrix}\]
\begin{itemize}
	\item The length of the mutually perpendicular line, denoted by the scalar $a_{i - 1}$, is called the \textbf{link length} of link $i - 1$.  Despite its name, this link length does not necessarily correspond to the actual length of the physical link.
	\item The \textbf{link twist} $\alpha_{i - 1}$ is the angle from $\hat{z}_{i - 1}$ to $\hat{z}_{i - 1}$, measured about $\hat{x}_{i - 1}$.
	\item The \textbf{link offset} $d_i$ is the distance from the intersection of $\hat{x}_{i - 1}$ and $\hat{z}_i$ to the origin of the link-0$i$ frame (the positive direction is defined to be along the $\hat{z}_i$-axis).
	\item The \textbf{joint angle} $\phi_i$ is the angle from $\hat{x}_{i - 1}$ to $\hat{x}_i$, measured about the $\hat{z}_i$-axis.
\end{itemize}
\begin{center} 
	\includegraphics*[width=\textwidth]{L1_18.png} 
\end{center}


\section*{Inverse Kinematics}
Given the end-effector position, calculate joint angles
\begin{center} 
	\includegraphics*[width=\textwidth]{L1_19.png} 
\end{center}

\[\theta_2 = \pi \pm \alpha \hspace{1cm} \alpha = \cos^{-1} \left(\frac{l_i^2 + l_2^2 - r^2}{2l_1 l_2}\right)\]
If $\alpha \neq 0$, there are two distinct values of $\theta_2$ which give the appropriate radius - the \textit{flip solution} is shown dashed above.
\[\theta_1 = \arctan2(y, x) \pm \beta \hspace{1cm} \beta = \cos^{-1} \left(\frac{r^2 + l_1^2 - l_2^2}{2l_1 r}\right)\]
Solve for $\phi$ and use this to get $\theta_1$ for \textbf{both} possible $\theta_2$ values
\begin{itemize}
	\item Inverse kinematics for joints > 2 is generally not solvable (no closed-form solution)
	\item More than one solution (redundancy)
	\item A hard (and well-studied problem)
\end{itemize}

\subsection*{"Solving" Inverse Kinematics}
Inverse kinematics for joints > 2 is generally not solvable.  In this case, what do we do?
\begin{enumerate}
    \item Numerical IK
    \begin{itemize}
        \item Iterative solvers (like Newton-Raphson, Jacobian pseudo-inverse).
        \item Solves for joint angles by minimizing position / orientation error.
        \item Used in most general-purpose robot controllers.
    \end{itemize}
    \item Optimization-Based IK
    \begin{itemize}
        \item Define an objective function (like minimizing joint torque or staying within limits).
        \item Add constraints (collision, joint bounds, etc.)
        \item Solvers: gradient descent, SQP, or even MPC
    \end{itemize}
    \item Learning-Based IK
    \begin{itemize}
        \item Neural networks or reinforcement learning.
        \item Especially helpful in high-DoF or redundant robots (like humanoids or octopuses).
    \end{itemize}
\end{enumerate}

\subsection*{Dynamics}
Given joint velocities, find the end-effector velocity
\begin{align*}
    x &= L_1 \cos(\theta_1) + L_2 \cos(\theta_1 + \theta_2)\\
    y &= L_2 \sin(\theta_1) + L_2 \sin(\theta_1 + \theta_2)
\end{align*}
First, differentiate with respect to joint angles
\begin{align*}
    \frac{\partial x}{\partial \theta_1} &= -L_1 \sin \theta_1 - L_2 \sin(\theta_1 + \theta_2)\\
    \frac{\partial x}{\partial \theta_2} &= -L_2 \sin(\theta_1 + \theta_2)\\
    \frac{\partial y}{\partial \theta_1} &= L_1 \cos \theta_1 + L_2 \cos(\theta_1 + \theta_2)\\
    \frac{\partial y}{\partial \theta_2} &= L_2 \cos(\theta_1 + \theta_2)
\end{align*}

\subsection*{Aside: Jacobian Matrix}
For \textit{Jacobian} matrix: the matrix of all first-order partial derivatives of a vector-valued function
\[J = \begin{bmatrix} \frac{\partial x}{\partial \theta_1} & \frac{\partial x}{\partial \theta_2} \\ \frac{\partial y}{\partial \theta_1} & \frac{\partial y}{\partial \theta_2} \end{bmatrix}\]
Velocity of end-effector:
\[\begin{pmatrix} \dot{x} \\ \dot{y} \end{pmatrix} = \begin{bmatrix} \frac{\partial x}{\partial \theta_1} & \frac{\partial x}{\partial \theta_2} \\ \frac{\partial y}{\partial \theta_1} & \frac{\partial y}{\partial \theta_2} \end{bmatrix} \begin{bmatrix} \dot{\theta_1} \\ \dot{\theta_2} \end{bmatrix}\]

\subsubsection*{Inverse of Jacobian}
Jacobian is used for inverse dynamics
\[\begin{bmatrix} \dot{\theta}_1 \\ \dot{\theta}_2 \end{bmatrix} = \begin{bmatrix} \frac{\partial x}{\partial \theta_1} & \frac{\partial x}{\partial \theta_2} \\ \frac{\partial y}{\partial \theta_1} & \frac{\partial y}{\partial \theta_2} \end{bmatrix}^{-1} \begin{bmatrix} \dot{x} \\ \dot{y} \end{bmatrix}\]
\begin{itemize}
	\item Can be used for closed-loop control
	\item Manipulator has singularity when determinant of Jacobian is zero
	\item Difficult to control around singularity
\end{itemize}
\begin{center} 
	\includegraphics*[width=\textwidth]{L2_1.png} 
\end{center}

\section*{Control}
\begin{itemize}
	\item Many tasks in robotics are defined by \textit{achievement goals}
	\begin{itemize}
        \item Go to the end of the maze
        \item Push that box over here
        \item Typically AI (search, ML) algorithms
    \end{itemize}
    \item Other tasks in robotics are defined by \textit{maintenance goals}
    \begin{itemize}
        \item Drive at 0.5 m/s
        \item Balance on one leg
    \end{itemize} 
    \item Control theory is generally used for low-level maintenance goals
    \item General notions:
    \begin{itemize}
        \item output = Controller(input)
        \item output is control signal to actuator (e.g., motor voltage/current)
        \item input is either goal state or goal state error (e.g., desired motor velocity)
    \end{itemize}
\end{itemize}

\subsection*{Control Systems}
\begin{itemize}
	\item Provides an output or response for a given input or stimulus
	\begin{itemize}
        \item Input: desired response
        \item Output: actual response
        \item E.g., pressing 4th floor button on an elevator
    \end{itemize}
\end{itemize}
\begin{center} 
	\includegraphics*[width=\textwidth]{L2_2.png} 
\end{center}
Output $\neq$ Input
\begin{itemize}
	\item Transient Response:
	\begin{itemize}
        \item Instantaneous change of input but gradual change of output
    \end{itemize}
	\item State-state Error:
	\begin{itemize}
        \item Accuracy of leveling
        \item Steady state error is inherent!
    \end{itemize}
\end{itemize}

\subsection*{Open Loop (feedforward) Control}
\begin{itemize}
	\item Open loop controller:
	\begin{itemize}
        \item output = FF(goal)
    \end{itemize}
	\item E.g.: motor speed controller (linear):
	\begin{itemize}
        \item Is applied voltage on motor: $V$
        \item Is goal speed: $s$
        \item Is gain term (from calibration): $k$
        \item $V = ks$
    \end{itemize}
	\item How do we know we've reached the goal?
	\begin{itemize}
        \item Weakness:
        \begin{itemize}
            \item Varying load on motor: \textit{motor may not maintain goal speed}
        \end{itemize}
        \item More likely scenario in robotics:
        \begin{center} 
            \includegraphics*[width=0.9\textwidth]{L2_3.png} 
        \end{center}
    \end{itemize}    
\end{itemize}

\subsection*{Uses of Open Loop Control}
\begin{itemize}
	\item Industrial machines may use open loop control
	\begin{itemize}
        \item Biological systems use it, in various movements
        \item These are called ballistic movements
        \item Ballistic movements cannot be corrected while they are executed
        \item E.g., pouncing, reflex reaching and withdrawal, etc.
    \end{itemize}
\end{itemize}
\begin{center} 
	\includegraphics*[width=\textwidth]{L2_4.png} 
\end{center}

\subsection*{Definition of Feedback Control}
\begin{itemize}
	\item Feedback control is a means of getting a system to achieve and maintain a desired state by \underline{continuously feeding back} the current state and comparing it to the desired state, then adjusting the current state to minimize the difference.
	\item Feedback control systems are drawn in a traditional diagrammatic way
\end{itemize}
\begin{center} 
	\includegraphics*[width=0.6\textwidth]{L2_5.png}
\end{center}

\subsection*{Comparing Methods}
\begin{itemize}
	\item Feedforward
	\begin{itemize}
        \item Anticipative
        \item Previous plan
        \item Doesn't wait
    \end{itemize}
	\item Feedback
	\begin{itemize}
        \item Reactive / Responsive
        \item Seeks to correct errors
        \item Always too late!
    \end{itemize}
\end{itemize}

\subsection*{Goal / Desired State}
\begin{itemize}
	\item The desired state is also called the \textbf{set point} or the \textbf{goal state} of the system.
	\item Goal state can be:
	\begin{itemize}
        \item External (e.g., a thermostat monitors and controls the temperature of the house)
        \item Internal (e.g., a robot can monitor its battery and control its energy usage)
    \end{itemize}
    \item If the desired and current states are the same, then what?
    \begin{itemize}
        \item Then there is nothing to do!
    \end{itemize}
    \item What if they are not?
\end{itemize}

\subsection*{Measuring Error}
\begin{itemize}
	\item The controller first computes the difference between the current and desired states
	\item The difference is called \textbf{error}
	\item What is the controller's job?
	\begin{itemize}
        \item To minimize the error at all times
    \end{itemize}
    \item Depending on the type of sensors, the error may be measured with different amounts of information.
\end{itemize}
\begin{center} 
	\includegraphics*[width=\textwidth]{L2_6.png} 
\end{center}
\subsection*{Zero / non-zero Error}
\begin{itemize}
	\item The least information we can have about the error is whether it exists.
	\begin{itemize}
        \item i.e., whether the current state is the desired state
    \end{itemize}
    \item This is called \textbf{zero / non-zero error}
    \item Zero / non-zero error provides very little information, yet useful control systems can be constructed with it.  (E.g., some forms of reinforcement learning work this way)
    \item What other information would be helpful?
\end{itemize}

\subsection*{Error Magnitude and Direction}
\begin{itemize}
	\item Additional information about the error is its \underline{magnitude}, i.e., the absolute difference (distance) between the current state and the desired state.
	\item The last part of the error information is its \underline{direction}, i.e., whether the difference is positive or negative.
	\item Control is easiest if frequent \textbf{feedback} provides both magnitude and direction.
\end{itemize}

\subsection*{Closed loop (feedback) control}
\begin{itemize}
	\item Feedback controller:
	\begin{itemize}
        \item output = FB(error)
        \item error = goal state - measured state
        \item controller attempts to minimize error
    \end{itemize}
	\item Feedback control requires sensors:
	\begin{itemize}
        \item Binary (at goal / not at goal)
        \item Direction (less than / greater than)
        \item Magnitude (very bad, bad, good)
    \end{itemize}
    \item Control is easiest when direction and magnitude are available.
\end{itemize}

\subsubsection*{Example}
\begin{itemize}
	\item Using feedback control to implement a wall-following behavior
	\begin{itemize}
        \item What type of goal is this?  Achievement or maintenance?
        \item What would the error be?
    \end{itemize}
	\item What sensors could you use?
	\begin{itemize}
        \item Would the sensor provide magnitude and direction of the error?
    \end{itemize}
	\item How can we word the controller?
	\item What will this robot's behavior look like?
\end{itemize}

\subsection*{Simplest Control: ON/off or bang/bang}
\begin{itemize}
	\item If error > 0, turn on actuators
	\item If error $\leq$ 0, turn off actuators
\end{itemize}
\begin{center} 
	\includegraphics*[width=\textwidth]{L2_7.png} 
\end{center}

\subsection*{Oscillation and the Set Point}
\begin{itemize}
	\item The behavior of a feedback system oscillates around the desired state
	\item E.g., the robot's movement will oscillate around the desired distance from the wall
	\item How can we decrease oscillation?
\end{itemize}
\begin{center} 
	\includegraphics*[width=0.6\textwidth]{L2_8.png} 
\end{center}

\subsection*{Control Theory}
\begin{itemize}
	\item \textit{Control theory} is the science that studies the behavior of control systems
	\item Three main types of simple linear controllers:
	\begin{itemize}
        \item P: proportional control
        \item PD: proportional derivative control
        \item PID: proportional integral derivative control
    \end{itemize}
    \item All use direction and magnitude of error
\end{itemize}

\subsection*{Proportional Control}
\begin{itemize}
	\item Act in proportion to the error
	\item A proportional controller has an output $o$ proportional to its input $i$:
	\[o = K_p e\]
    \item $K_p$ is a proportionality constant (gain)
\end{itemize}
\begin{center} 
	\includegraphics*[width=\textwidth]{L2_9.png}
\end{center}

\subsection*{What is a Gain?}
\begin{itemize}
	\item How do we decide how much to turn, or how fast to go?  (i.e., the magnitude of the system's response)
	\item Those parameters ($K_p$) are called \textbf{gains}, and are very important in control
	\item Determining the right gains is difficult
	\item It can be done
	\begin{itemize}
        \item analytically (mathematics)
        \item empirically (trial and error)
    \end{itemize}
\end{itemize}

\subsection*{Effect of Gains}
\begin{center} 
	\includegraphics*[width=0.7\textwidth]{L2_10.png} 
\end{center}

\subsection*{Physics and Gains}
\begin{itemize}
	\item The physical properties of the system directly affect the gain values.
	\begin{itemize}
        \item E.g., the velocity profile of a motor (how fast it can accelerate and decelerate)
        \item the backlash and friction in the gears,
        \item the friction on the surface, in the air, etc.
    \end{itemize}
    \item All of these influence what the system actually does in response to a command
\end{itemize}

\subsection*{Setting Gains}
\begin{itemize}
	\item Analytical approaches
	\begin{itemize}
        \item require that the system be well understood and characterized mathematically
    \end{itemize}
    \item Trial and error (ad hoc, system-specific) approaches
    \begin{itemize}
        \item require that the system be tested extensively
    \end{itemize}
    \item Gains can also be tuned by the system itself, automatically, by trying different values at run-time
\end{itemize}

\subsection*{Gains and Oscillations}
\begin{itemize}
	\item Real systems have \textit{momentum}
	\begin{itemize}
        \item System takes some time to respond to commands
    \end{itemize}
    \item Large gains can lead to \textit{oscillations}
    \begin{itemize}
        \item System overshoots, recovers, overshoots again
    \end{itemize}
    \item Very large gains can lead to \textit{divergent} oscillations
    \begin{itemize}
        \item Overshoot gets larger and larger over time.
    \end{itemize}
\end{itemize}
\begin{center} 
	\includegraphics*[width=0.6\textwidth]{L2_11.png} 
\end{center}

\subsection*{Control Near the Set Point}
\begin{itemize}
	\item Setting gains is difficult, and simply increasing the proportional gain does not remove oscillations
	\item While at low values this may work, as the gain increases, the oscillations increase as well.
	\item The problem has to do with the distance from the set point\dots
	\item How can we solve this?
\end{itemize}

\subsection*{Damping}
\begin{itemize}
	\item \textbf{Damping} is the process of systematically decreasin oscillations
	\item What do you think it means to be properly damped?
	\item A system is \underline{properly damped} if it does not oscillate with increasing magnitude, i.e., if its oscillations are either avoided, or decrease to the desired set point within a reasonable time period
\end{itemize}
\begin{center} 
	\includegraphics*[width=0.6\textwidth]{L2_12.png} 
\end{center}

\subsection*{Derivative Control}
\begin{itemize}
	\item Act in proportion to the \textit{rate of change} of the error
	\item A \underline{derivative controller} has an output $o$ proportional to the \underline{derivative} of its input:
	\[o = K_d \frac{\dd e}{\dd t}\]
    \item $K_d$ is a proportionality constant
\end{itemize}
\begin{center} 
	\includegraphics*[width=\textwidth]{L2_13.png} 
\end{center}

\subsection*{PD Control}
\begin{itemize}
	\item PD control combined P and D control: $o = K_p e + K_d \frac{\dd e}{\dd t}$
	\item P component minimizes error
	\item D component provides damping
	\item Gains $K_p$ and $K_d$ must be tuned together
\end{itemize}
\begin{center} 
	\includegraphics*[width=\textwidth]{L2_14.png} 
\end{center}

\subsection*{Integral Control}
\begin{itemize}
	\item Act in proportion to the \textit{accumulated} error
    \item Output proportioanl to the integral of its input:
    \[o = K_i \int e(t) \dd t\]
    \item $K_i$ is a proportionality constant
    \item Integral control is useful for eliminating steady-state errors
\end{itemize}
\begin{center} 
	\includegraphics*[width=\textwidth]{L2_15.png} \\
    \includegraphics*[width=\textwidth]{L2_16.png} \\
\end{center}

\subsection*{PID Control}
\begin{itemize}
	\item PID control combined P, I, and D control:
	\[o = K_p e + K_i \int e(t) \dd t + K_d \frac{\dd e}{\dd t}\]
    \begin{itemize}
        \item P component minimizes instantaneous error
        \item I component minimizes cumulative error
        \item D component provides damping
    \end{itemize}
    \item Gains must be tuned together
    \begin{center} 
        \includegraphics*[width=0.9\textwidth]{L2_17.png} 
    \end{center}
\end{itemize}

\subsection*{System Response Terminology}
\begin{itemize}
	\item Rise Time
	\begin{itemize}
        \item Time for the waveform to go from 0.1 to 0.9 of its final value
        \item Measures how quickly the controller responds
    \end{itemize}
	\item Percent Overshoot
	\begin{itemize}
        \item Amount the waveform overshoots the final value, at the peak time
        \item Measures how big the oscillations are
    \end{itemize}
	\item Settling time
	\begin{itemize}
        \item Time for the response to reach, and stay within, 2\% of its final value
        \item Measures how fast the controller reaches steady-state
    \end{itemize}
\end{itemize}
\begin{center} 
	\includegraphics*[width=\textwidth]{L2_18.png} 
\end{center}

\subsection*{PID Control Summary}
\begin{center} 
	\includegraphics*[width=\textwidth]{L2_19.png} 
\end{center}

\subsection*{Control is one component\dots}
\begin{itemize}
	\item Getting robots to do what you want requires many components!
	\item Feedback Control plays a part at the lower level: turning wheels, moving actuators
	\item Higher-level goals (coordination, interaction, collaboration) require other approaches
	\begin{itemize}
        \item These approaches can better represent and handle those challenges
    \end{itemize}
    \item Control Architectures are the AI portion that determines how the robot makes decisions
\end{itemize}

\end{document}