\documentclass{article}
\usepackage{setspace}
\onehalfspacing
\usepackage[utf8]{inputenc}
\usepackage[bb=boondox]{mathalfa}
\usepackage{graphicx}

%%Let's you change margins
\usepackage[left=1in,right=1in,top=1in,bottom=1in]{geometry}

%%Math symbols, proof environments
\usepackage{amsmath,amsthm,amssymb, graphicx, tikz}

%%Use this package for matrices
\usepackage{array}

\newcommand{\example}{\textbf{Example: }}

\title{CS 188 Robotics Week 1} 

\author{Aidan Jan}

\date{\today}

\begin{document}
\maketitle

\section*{Robots}
\subsection*{What is a robot}
\begin{itemize}
    \item "A robot is defined as \underline{intelligence} embodied in an engineered \underline{construct}, with the ability to process information, \underline{sense}, \underline{plan}, and \underline{move} within or substantially \underline{alter} its working environment.
    \item [FILL]
    \item Therefore, the following count as robots:
    \begin{itemize}
        \item Roombas
        \item Automatic sliding doors (which may use facial recognition, or just a simple proximity sensor)
    \end{itemize}
\end{itemize}

\subsection*{Robotics}
\begin{itemize}
    \item Robots must be able to move (physically), or interact with its environment in some way.  There are three sectors of robotics:
    \begin{itemize}
        \item \textbf{Kinematics}: the study of motion \textit{without} considering forces or torques
        \item \textbf{Dynamics}: the study of motion considering the forces and torques that caused it.
        \item \textbf{Control}: how to execute the desired motion
        \item \textbf{Perception}: how to understand the world using sensors
        \item \textbf{Planning}: how to reach a goal
    \end{itemize}
\end{itemize}

\subsection*{Course Objectives} [FILL format]
\begin{itemize}
    \item Develop a foundational understanding of kinematics, dynamics, and control for modeling and managing robotic motion
    \item Become familiar with sensors and pereption algorithms to interpret environmental data for robotic decision-making
    \item Understand principles of state estimation, as well as task and motion planning, to enable reliable and efficient robot behaviors.
    \item Explore basic ideas of AI in robotics, including imitation learning and human-robot interactions, for advanced autonomous capabilities.
    \item Gain hands-on experience in simulation tools to design, test, and refine robitic systems in a virtual environment
    \item Reflect on the ethical implications of robotics, fostering responsible development and deployment [FILL]
\end{itemize}

\subsection*{Designing a Robot}
\textbf{Considerations:}
\begin{enumerate}
    \item Tasks and Operating Environments
    \begin{itemize}
        \item Define specific tasks the robot will perform.
        \item Analyze working environments: indoor/outdoor, structured/unstructured, temperature, terrain, obstacles, etc.
    \end{itemize}
    \item Hardware Design
    \begin{itemize}
        \item Mechanical Structure: Chassis, joints, degrees of freedom
        \item Actuators: Motors, servos, pneumatic or hydraulic systems
        \item Power System: Battery type, power efficiency, backup options
    \end{itemize}
    \item Firmware and Embedded Systems
    \begin{itemize}
        \item Computing Units: Microcontrollers, onboard processors
        \item Sensor integration: Cameras, IMUs, LiDAR, GPS, force sensors
        \item Communication Interfaces: Wired/wireless protocols (e.g., I2C, SPI, UART, CAN, Wi-Fi, Bluetooth)
    \end{itemize}
    \item Software Architecture
    \begin{itemize}
        \item Control Algorithms: Motion planning, PID control, pathfinding
        \item Autonomy and Intelligence: SLAM, AI/ML models, obstacle avoidance
        \item User Interface: Remote control, dashboards, or autonomous modes.
    \end{itemize}
\end{enumerate}




\end{document}