% document formatting
\documentclass[10pt]{article}
\usepackage[utf8]{inputenc}
\usepackage[left=1in,right=1in,top=1in,bottom=1in]{geometry}
\usepackage[T1]{fontenc}
\usepackage{xcolor}

% math symbols, etc.
\usepackage{amsmath, amsfonts, amssymb, amsthm}

% lists
\usepackage{enumerate}

% images
\usepackage{graphicx} % for images

% code blocks
\usepackage{minted, listings} 

% verbatim greek
\usepackage{alphabeta}

\graphicspath{{./assets/images/Week 1}}

\title{CS 188 Robotics Week 1} 

\author{Aidan Jan}

\date{\today}

\begin{document}
\maketitle 

\section*{Robots}
\subsection*{What is a robot}
\begin{itemize}
    \item "A robot is defined as \underline{intelligence} embodied in an engineered \underline{construct}, with the ability to process information, \underline{sense}, \underline{plan}, and \underline{move} within or substantially \underline{alter} its working environment.
    \item Here intelligence includes a \underline{broad class of methods} that enable a robot to solve problems or to make contextually appropriate decisions and act upon them.
    \item Therefore, the following count as robots:
    \begin{itemize}
        \item Roombas
        \item Automatic sliding doors (which may use facial recognition, or just a simple proximity sensor)
    \end{itemize}
\end{itemize}

\subsection*{Robotics}
\begin{itemize}
    \item Robots must be able to move (physically), or interact with its environment in some way.  There are three sectors of robotics:
    \begin{itemize}
        \item \textbf{Kinematics}: the study of motion \textit{without} considering forces or torques
        \item \textbf{Dynamics}: the study of motion considering the forces and torques that caused it.
        \item \textbf{Control}: how to execute the desired motion
        \item \textbf{Perception}: how to understand the world using sensors
        \item \textbf{Planning}: how to reach a goal
    \end{itemize}
\end{itemize}

\subsection*{Course Objectives}
\begin{itemize}
    \item Develop a foundational understanding of \textit{kinematics, dynamics, and control} for modeling and managing robotic motion
    \item Become familiar with \textit{sensors and pereption algorithms} to interpret environmental data for robotic decision-making
    \item Understand principles of \textit{state estimation}, as well as \textit{task and motion planning}, to enable reliable and efficient robot behaviors.
    \item Explore basic ideas of \textit{AI in robotics}, including imitation learning and human-robot interactions, for advanced autonomous capabilities.
    \item Gain \textit{hands-on experience} in simulation tools to design, test, and refine robitic systems in a virtual environment
    \item Reflect on the \textit{ethical implications} of robotics, fostering responsible development and deployment of robotic technologies
\end{itemize}

\subsection*{Designing a Robot}
\textbf{Considerations:}
\begin{enumerate}
    \item Tasks and Operating Environments
    \begin{itemize}
        \item Define specific tasks the robot will perform.
        \item Analyze working environments: indoor/outdoor, structured/unstructured, temperature, terrain, obstacles, etc.
    \end{itemize}
    \item Hardware Design
    \begin{itemize}
        \item Mechanical Structure: Chassis, joints, degrees of freedom
        \item Actuators: Motors, servos, pneumatic or hydraulic systems
        \item Power System: Battery type, power efficiency, backup options
    \end{itemize}
    \item Firmware and Embedded Systems
    \begin{itemize}
        \item Computing Units: Microcontrollers, onboard processors
        \item Sensor integration: Cameras, IMUs, LiDAR, GPS, force sensors
        \item Communication Interfaces: Wired/wireless protocols (e.g., I2C, SPI, UART, CAN, Wi-Fi, Bluetooth)
    \end{itemize}
    \item Software Architecture
    \begin{itemize}
        \item Control Algorithms: Motion planning, PID control, pathfinding
        \item Autonomy and Intelligence: SLAM, AI/ML models, obstacle avoidance
        \item User Interface: Remote control, dashboards, or autonomous modes.
    \end{itemize}
\end{enumerate}

\subsection*{Where are we?}
\subsubsection*{Logistics and Warehouse Robots}

\subsubsection*{Space Robots}
\begin{center} 
	\includegraphics*[width=\textwidth]{L1_1.png} 
\end{center}
\subsubsection*{Deepsea Robots}
\begin{center} 
	\includegraphics*[width=\textwidth]{L1_2.png} 
\end{center}
\subsubsection*{Healthcare and Medical Robots (?)}
\begin{center} 
	\includegraphics*[width=\textwidth]{L1_3.png} 
\end{center}
\subsubsection*{Agricultural Robots}
\begin{center} 
	\includegraphics*[width=\textwidth]{L1_4.png} 
\end{center}
\subsubsection*{Disaster Response Robots}
\begin{center} 
	\includegraphics*[width=\textwidth]{L1_5.png} 
\end{center}
\subsubsection*{Service and Hospitality Robots}
\begin{center} 
	\includegraphics*[width=\textwidth]{L1_6.png} 
\end{center}
\subsubsection*{Education, Entertainment, and Companion Robots}
\begin{center} 
	\includegraphics*[width=\textwidth]{L1_7.png} 
\end{center}
\subsubsection*{Humanoids | Tesla Optimus, Unitree H1, 1x, Figure}
\begin{center} 
	\includegraphics*[width=\textwidth]{L1_8.png} 
\end{center}

\subsection*{[FILL]}
[FILL]

\subsection*{Rigid Body in 3D space:}
[FILL image]
How many DoF does it have?
\begin{itemize}
    \item 6 = 3 for position + 3 for orientation
\end{itemize}

\subsection*{Joints and DoF}
[FILL image, full]

\subsection*{How many DoF do you have?}
[FILL image]
Whole body: ~30 DoF (major joints)

\subsection*{Gr$\ddot{\text{u}}$bler's Formula}
\begin{align*}
    \text{def} &= m(N - 1) - \sum_{i = 1}^J c_i \\
    &= m(N - 1) - \sum_{i = 1}^J(m - f_i)\\
    &= m(N - 1 - J) + \sum_{i = 1}^J f_i
\end{align*}
\begin{itemize}
    \item The term $m(N - 1)$ represents the number of rigid body freedoms
    \item The term $\sum_{i = 1}^J c_i$ represents the number of joint constraints.
    \item We would use $m = 3$ for two-dimensional rigid bodies (planar mechanisms), and $m = 6$ for three-dimensional rigid bodies (spatial mechanisms).
\end{itemize}

\subsection*{Acrobot (Double Pendulum)}
[FILL image]
\begin{itemize}
    \item You can only move the first joint; the second is completely free moving.
    \item The tip is called the \textbf{end effector}.
    \item The entire circle that the robot can reach is called the \textbf{work space.}
    \item [FILL]
\end{itemize}

\section*{Motors and Gears}
\subsection*{Action and Actuation}
\begin{itemize}
	\item [FILL]
\end{itemize}
\subsection*{Definition of Effector}
\begin{itemize}
	\item An effector is any device that has an effect on the environment.
	\item A robot's effectors are used to purposefully \underline{create an effect} on the environment.
	\item E.g., legs, wheels, arms, fingers, \dots
	\item \textit{The role of the controller is to get the effectors to produce the desired effect on the environment, based on the robot's task}
\end{itemize}

\subsection*{Definition of Actuator}
\begin{itemize}
	\item An actuator is the mechanism that enables the effector to execute an action.
	\item E.g., electric motors, hydraulic or pneumatic cylinders, pumps, \dots
	\item Actuators and effectors are \textbf{not} the same thing.
\end{itemize}

\subsection*{Electric Motors}
[FILL image]
\begin{itemize}
	\item \textbf{AC Motor}
	\begin{itemize}
        \item Hard to control speed directly
        \item Cheaper and more durable $\rightarrow$ common in household appliances, HVAC, pumps, and fans
    \end{itemize}
    \item \textbf{DC Motor}
    \begin{itemize}
        \item [FILL]
        \item \textbf{The most common actuator in mobile robotics is the direct current (DC) motor}
        \item Advantages: [FILL]
        \item Disadvantages: [FILL]
    \end{itemize}
\end{itemize}

\subsubsection*{How do DC motors work?}
\begin{itemize}
	\item DC motors consist of permanent magnets with loops of wire inside
	\item When current is applied, the wire loops generate a \textbf{magnetic field}, which reacts against the outside field of the static magnets
	\item The interaction of the fields produces the movement of the shaft or armature
	\item A \textbf{commutator} switches the direction of the current flow, yielding continuous motion
\end{itemize}

\subsubsection*{Types of DC Motors}
\begin{itemize}
	\item Brushed motors (mechanical commutation)
	\begin{itemize}
        \item Low-voltage, low-torque, cheap
    \end{itemize}
    \item Brushless motors (electric commutation)
    \begin{itemize}
        \item High voltage, high-torque, expensive
        \item No friction or wear of brushes
    \end{itemize}
\end{itemize}
[FILL images]

\subsection*{Motor Efficiency}
\begin{itemize}
	\item As any physical system, DC motors are not perfectly efficient
	\item [FILL]
	\item Good DC motors can be made to have efficiencies in the 90th percentile
	\item Cheap DC motors can be as low as 50\%
	\item [FILL]
\end{itemize}

\subsection*{Speed and Torque}
\begin{itemize}
	\item Motor \underline{speed} $w$ is proportional to induced voltage $V$.
	\[w = k_v V\]
    \item \underline{Torque} is a force that acts in a rotational manner
    \[t = r \times F\]
    \item Motor \underline{torque} $t$ is proportional to applied current $I$:
    \[t = k_I I\]
    \item Motors have a maximum speed (no-load speed) and a maximum torque (stall torque)
\end{itemize}

\subsection*{Speed/Torque Relationship}
[FILL image (graph)]

\subsection*{Motor Power}
\begin{itemize}
	\item Output power is the product of speed and torque:
	\[P = w \times t\]
    \item At stall torque and no-load speed, the power is zero!
\end{itemize}
\begin{center} 
	[FILL]
\end{center}

\subsection*{Power as a function of $\tau$, $\omega$}
\begin{align*}
    P_{motor}(\omega) &= -\frac{\tau_s}{\omega_n} [FILL]
    P_{motor}(\tau) &= [FILL]
\end{align*}
\begin{center} 
	[FILL]
\end{center}

\subsection*{Operating Voltage and Speed}
\begin{itemize}
	\item Motors have maximum voltage
	\item [FILL]
\end{itemize}

\subsection*{DC motors and Robots}
\begin{itemize}
	\item DC motors have high-speed, low torque
	\item Typical speed range:
	\begin{itemize}
        \item 9000 to 12000 RPM
        \item 150 to 200 Hz
    \end{itemize}
    \item Robots require low-speed, high torque.  
    \begin{itemize}
        \item What do we do about this? (We use gears!)
    \end{itemize}
\end{itemize}

\subsection*{Gearing}
\begin{itemize}
	\item Gears are used to [FILL]
\end{itemize}

\subsection*{Gear Fundamentals}
\begin{itemize}
	\item The force $F$ at the edge of a gear of radius $r$ is given by:
	\[F = \tau / r\]
    \item The linear speed $v$ at the edge of a gear of radius $r$ is given by:
    \[v = \omega r\]
\end{itemize}

\subsection*{Combining Gears}
\begin{itemize}
	\item Meshing gears have equal linear speeds.
	\[v_1 = v_2\]
    \item Thus the output speed is:
    \[v = \omega r, \:\:\therefore \omega_2 = \frac{r_1}{r_2} \omega_1\]
    \item And the output torque is:
    \[F = \tau / r \:\:\therefore \tau_2 = \frac{r_2}{r_1}\tau_1\]
    \item [FILL]
\end{itemize}

\subsubsection*{Examples:}
\begin{itemize}
	\item Gearing down:
	\[r_1 = 1, r_2 = 2\]
    \begin{itemize}
        \item 2:1 gear ratio doubles the torque and halves speed
    \end{itemize}    
    \item Gearing up:
    \[r_1 = 2, r_2 = 1\]
    \begin{itemize}
        \item 1:2 gear ratio halves torque and doubles speed
    \end{itemize}
\end{itemize}

\subsection*{Gear Stages}
\begin{itemize}
	\item Usually, it is not possible to achieve a sufficient gear ratio with a single pair of gears
	\item Gears can be arranged \textit{in stages}
	\item The tortal gear ratio is the product of gear ratios for each stage
	\begin{itemize}
        \item E.g., $3:1 \times 3:1 = 9:1$
    \end{itemize}
\end{itemize}

\subsection*{Types of Gears}
[FILL image]

\subsection*{Backlash}
\begin{itemize}
	\item Simple gears suffer from \textit{backlash} (teeth not meshing completely)
	\item Although sometimes this is needed, it reduces the control you have
\end{itemize}
[FILL image]

\subsection*{Control of Motors}
\subsubsection*{Controlling Speed: Pulse Width Modulation (PWM)}
[FILL image]
\subsubsection*{Controlling Direction: H-Bridge}
[FILL image]

\subsection*{[FILL]}
[FILL]

\subsection*{Servo Motors}
\begin{itemize}
	\item Servo motors are adapted DC motors:
	\begin{itemize}
        \item Gear reduction
        \item [FILL]
    \end{itemize}
    \item [FILL]
\end{itemize}

\subsection*{PWM Position Control}
[FILL image]
\begin{itemize}
	\item Not defined by PWM duty cycle but only \textbf{duration} of the pulse!
	\item Pulse width must be very accurate
	\begin{itemize}
        \item Noise in width $\Rightarrow$ noise in position
    \end{itemize}
    \item Pulse rate may be variable
    \begin{itemize}
        \item Noise in rate $\Rightarrow$ no change
    \end{itemize}
\end{itemize}




\end{document}