% document formatting
\documentclass[10pt]{article}
\usepackage[utf8]{inputenc}
\usepackage[left=1in,right=1in,top=1in,bottom=1in]{geometry}
\usepackage[T1]{fontenc}
\usepackage{xcolor}

% math symbols, etc.
\usepackage{amsmath, amsfonts, amssymb, amsthm}
\usepackage{xcolor}
\usepackage[dvipsnames]{xcolor}
\usepackage[table,xcdraw]{xcolor}

% lists
\usepackage{enumerate}
\usepackage{enumitem}
\usepackage{tabularx}

% images
\usepackage{graphicx} % for images

% code blocks
\usepackage{minted, listings} 

% verbatim greek
\usepackage{alphabeta}

% colors
\definecolor{paint_red}{RGB}{237, 28, 36}
\definecolor{paint_green}{RGB}{34, 177, 76}
\definecolor{paint_blue}{RGB}{63, 72, 204}
\definecolor{paint_purple}{RGB}{163, 73, 164}

\newcommand{\dd}{\text{d}}

\graphicspath{{./assets/images/Week 7}}

\title{CS 188 Robotics Week 7} 

\author{Aidan Jan}

\date{\today}

\begin{document}
\maketitle 

\section*{Human Robot Interaction}
\begin{itemize}
	\item Proposed definitions:
	\begin{itemize}
        \item A field of study dedicated to \underline{understanding}, \underline{designing}, and \underline{evaluating} robotic systems for use by or with humans.  Interaction requires \underline{communication} between robots and humans
        \item Define models of \underline{humans' expectations} regarding robot interaction to guide robot design and algorithmic development that would allow more \underline{natural} and \underline{effective} interaction between humans and robots
    \end{itemize}
	\item The HRI problem is to \textit{understand and shape the interactions between one or more humans and one or more robots}
\end{itemize}

\subsection*{A Little Bit of History}
\begin{itemize}
	\item HRI as a field started to emerge in the mid-1990s
	\item Interdisciplinary in nature, requiring contributions from cognitive science, linguistics, and psychology; from engineering, mathematics, and computer science; and from human factors engineering and design.
\end{itemize}

\subsection*{(Some) Topics in HRI}
\begin{center}
    \begin{tabularx}{\textwidth}{XX}
        \textbullet \hspace{3mm} Expression and Gase & \textbullet \hspace{3mm} Robot Futures and Singularity \\
        \textbullet \hspace{3mm} Proxemics & \textbullet \hspace{3mm} Shared Autonomy \\
        \textbullet \hspace{3mm} Speech & \textbullet \hspace{3mm} Exoskeletons \\
        \textbullet \hspace{3mm} Perception & \textbullet \hspace{3mm} Assistive Robotics \\
        \textbullet \hspace{3mm} Interaction Design & \textbullet \hspace{3mm} Educational Robotics \\
        \textbullet \hspace{3mm} Manipulation & \textbullet \hspace{3mm} Philosophy \\
        \textbullet \hspace{3mm} Decision-Making & \textbullet \hspace{3mm} Ethics \\
        \textbullet \hspace{3mm} Mental Models & \textbullet \hspace{3mm} Law \\
        \textbullet \hspace{3mm} Perspective-Taking & \textbullet \hspace{3mm} \dots \\
        \textbullet \hspace{3mm} Systems Engineering & 
        \end{tabularx} 
\end{center}

\subsection*{Types of Interactions with Robots}
\begin{itemize}
	\item Intentional
	\begin{itemize}
        \item E.g., human-robot search and rescue team
    \end{itemize}
	\item Incidental
	\begin{itemize}
        \item E.g., valuum cleaning robot bumping into your foot
    \end{itemize}
	\item Explicit
	\begin{itemize}
        \item E.g., assistive robot lifting a patient out of bed
    \end{itemize}
	\item Implicit
	\begin{itemize}
        \item E.g., UAV flying over your home
    \end{itemize}
\end{itemize}

\subsection*{Example: Autonomous Cars}
\begin{itemize}
	\item Intentional physical contact
	\begin{itemize}
        \item Car is physically transporting a person
    \end{itemize}
	\item Unintentional physical contact
	\begin{itemize}
        \item In case of an accident
    \end{itemize}
	\item Explicit social interactions
	\begin{itemize}
        \item Alerting people in the car about changing the route home
    \end{itemize}
	\item Implicit social interactions
	\begin{itemize}
        \item Opinions human drivers in other cars may have about the autonomous car's driving behavior
    \end{itemize}
	\item Plus ethical considerations!
	\begin{itemize}
        \item Who has access to the recorded data from the cameras in the car?
        \item Should the car protect the driver or pedestrians in the case of an inevitable accident?
    \end{itemize} 
\end{itemize}

\subsection*{Human Robot Interaction Design}
\begin{itemize}
	\item A designer can control five attributes that affect the interactions between humans and robots:
	\begin{itemize}
        \item Level and behavior of autonomy
        \item Nature of information exchange
        \item Structure of the team
        \item Adaptation, learning, and training of people and the robot
        \item Shape of the task
    \end{itemize}
\end{itemize}

\subsection*{Attribute 1: Autonomy}
\begin{itemize}
	\item Designing autonomy consists of mapping inputs from the environment into actuator movements, representational schemas, or speech acts
    \item Autonomy is only useful insofar as it supports beneficial interaction between a human and a robot
\end{itemize}

\subsection*{Levels of Autonomy in HRI}
\begin{itemize}
	\item Teleoperation
	\begin{itemize}
        \item A human remotely controls a mobile robot or robotic arm
        \item Cons: Higher cognitive load of the operator
    \end{itemize}
	\item Mediated Teleoperation
	\begin{itemize}
        \item Human remotely controls the robot; the robot autonomously intervenes as necessary
    \end{itemize}
	\item Supervisory Control
	\begin{itemize}
        \item Human supervises the behavior of an autonomous system and intervenes as necessary
    \end{itemize}
	\item Peer-to-peer Collaboration
	\begin{itemize}
        \item High-level supervision and direction of the robot; human provides goals and robot maintains knowledge about the world, task, and constraints
        \item Cons:
        \begin{itemize}
            \item Difficult to create robots with the appropriate cognitive skills to interact naturally or efficient with a human
            \item Must be able to flexibly exhibit "full autonomy" at appropriate times
        \end{itemize}
    \end{itemize}
\end{itemize}

\subsubsection*{Teleoperation Example}
\begin{itemize}
	\item Da Vinci Surgical Robot (Intuitive Surgical)
\end{itemize}
\begin{center} 
	\includegraphics*[width=0.5\textwidth]{L1_1.png} 
\end{center}

\subsubsection*{Mediated Teleoperation Example}
\begin{center} 
	\includegraphics*[width=\textwidth]{L1_2.png} 
\end{center}

\subsection*{Attribute 2: Information Exchange}
\begin{itemize}
	\item Efficient interactions characterized by productive exchanges between the human and robot
	\item Measures of the efficiency of an interaction:
	\begin{itemize}
        \item Interaction time required for intent and/or instructions to be communicated to the robot
        \item Cognitive or mental workload of an interaction
        \item Amount of situation awareness produced by the interaction (or reduced because of interruptions from the robot),
        \item Amount of shared understanding or common ground between humans and robots
    \end{itemize}
\end{itemize}

\subsubsection*{Visual Displays}
\begin{itemize}
	\item Typically presented as graphical user interfaces or augmented reality interfaces
\end{itemize}
\begin{center} 
	\includegraphics*[width=\textwidth]{L1_3.png} 
\end{center}

\subsubsection*{Gestures}
\begin{itemize}
	\item Gestures, including hand and facial movements and by movement-based signaling of intent
\end{itemize}
\begin{center} 
	\includegraphics*[width=0.7\textwidth]{L1_4.png} \\
    \includegraphics*[width=0.7\textwidth]{L1_5.png} \\
    \includegraphics*[width=0.7\textwidth]{L1_6.png} \\
\end{center}

\subsubsection*{Speech and natural language}
\begin{itemize}
	\item Auditory speech
	\item Text-based responses
	\item Robot must understand responses from human and must be able to perform appropriate task
\end{itemize}
\begin{center} 
	\includegraphics*[width=\textwidth]{L1_7.png} 
\end{center}

\subsubsection*{Non-speech audio}
\begin{itemize}
	\item Frequently used in alerting
\end{itemize}
\begin{center} 
	\includegraphics*[width=0.7\textwidth]{L1_8.png} 
\end{center}

\subsubsection*{Physical Interaction and Haptics}
\begin{itemize}
	\item Used remotely in augmented reality or in teleoperation to invoke a sense of presence especially in telemanipulation tasks
    \item Used proximately to promote emotional, social, and assistive exchanges
\end{itemize}
\begin{center} 
	\includegraphics*[width=\textwidth]{L1_9.png} 
\end{center}

\subsubsection*{Safety in HRI}
\begin{itemize}
	\item Industrial Robots
	\begin{center} 
        \includegraphics*[width=0.7\textwidth]{L1_10.png} 
    \end{center}
	\item Human-robot interaction
	\begin{center} 
        \includegraphics*[width=0.9\textwidth]{L1_11.png} 
    \end{center}
\end{itemize}

\subsubsection*{The Uncanny Valley}
\begin{center} 
    \includegraphics*[width=0.7\textwidth]{L1_12.png} 
\end{center}

\subsection*{Attribute 3: Team Structure}
\begin{itemize}
	\item HRI is not restricted to a single human and a single robot
	\item For example:
	\begin{itemize}
        \item Robots used in search and rescue are typically managed by two or more people, each with special roles in the team
        \item Unmanned/Uninhabited Air Vehicles (UAVs) are typically managed by at least two people: a “pilot”, who is responsible for navigation and control and a sensor/payload operator, who is responsible for managing cameras, sensors, and other payloads
        \item One human may also interact with multiple robots
    \end{itemize}
\end{itemize}

\subsection*{Attribute 4: Adaptation, learning, and training}
\begin{itemize}
	\item Training human operators
	\begin{itemize}
        \item Minimize needed training
        \begin{itemize}
            \item i.e., the Roomba should work even without taking a class in robotics
            \item Robots for children are becoming more popular
        \end{itemize}
        \item Specialized robots or tasks may require additional training
    \end{itemize}
	\item Training robots
	\begin{itemize}
        \item Improving perceptual capabilities through efficient communication
        \item Improving reasoning and planning through interaction
        \item Improving autonomous capabilities
    \end{itemize}
\end{itemize}

\subsection*{Attribute 5: Task-shaping}
\begin{itemize}
	\item Robotic technology is introduced to allow a human to do a task that they could not do before, or to make the task easier or more pleasant for the human
    \item Consider how the task should be done and will be done when new technology is introduced
\end{itemize}

\subsubsection*{Assistive Robots}
\begin{itemize}
	\item Seek to provide physical, mental, or social support to persons who could benefit from it such as the elderly or disabled
	\item Challenges:
	\begin{itemize}
        \item Providing safe physical contact
        \item Moving within very close proximity
        \item Cognitive and emotive computing
        \item Gesture and speech
    \end{itemize}
\end{itemize}

\subsubsection*{Physically Assistive Robots}
\begin{itemize}
	\item Robot physically interacts with the user to provide assistance in some task
\end{itemize}
\begin{center} 
    \includegraphics*[width=\textwidth]{L1_13.png} 
\end{center}

\subsubsection*{Socially Assistive Robots}
\begin{itemize}
	\item Seeks to supplement and augment the support of clinicians and caregivers through individualized, socially mediated interventions with robots
    \item No physical interaction with robots
\end{itemize}
\begin{center} 
    \includegraphics*[width=\textwidth]{L1_14.png} 
\end{center}

\subsection*{Human Subject Studies in HRI}
\begin{itemize}
	\item Must evaluate the effectiveness of the system in context of the interaction with real (naive) users
	\item Can measure:
	\begin{itemize}
        \item Qualitative
        \begin{itemize}
            \item How does the user perceive the (look, feel, sound) of the robot?
            \item How much does the user trust the robot?
        \end{itemize}
        \item Quantitive
        \begin{itemize}
            \item How long does it take the user to complete the task?
            \item How accurately does the robot follow the user's commands?
            \item How much of the user's attention is required by the task?
        \end{itemize}
    \end{itemize}
\end{itemize}

\subsection*{Designing Human Subject Studies}
\begin{itemize}
	\item Who are your subjects and how would you recruit them?
    \item Where would you perform the study?
    \item What do you want to discover? (what is the hypothesis or research question)?
    \item What would the protocol be?
    \item What would the performance metrics be?
    \item How would you acquire and analyze the data?    
\end{itemize}



\end{document}