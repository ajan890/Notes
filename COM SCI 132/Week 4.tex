% document formatting
\documentclass[10pt]{article}
\usepackage[utf8]{inputenc}
\usepackage[left=1in,right=1in,top=1in,bottom=1in]{geometry}
\usepackage[T1]{fontenc}
\usepackage{xcolor}

% math symbols, etc.
\usepackage{amsmath, amsfonts, amssymb, amsthm}

% lists
\usepackage{enumerate}

% images
\usepackage{graphicx} % for images

% code blocks
\usepackage{minted, listings} 
\lstset{
  basicstyle=\ttfamily,
  mathescape
}
% verbatim greek
\usepackage{alphabeta}

\graphicspath{{./assets/images}}

\newcommand{\solution}{\textbf{Solution:}} 

\title{COM SCI 132 Week 4}

\author{Aidan Jan}
\date{\today}

\begin{document}
\maketitle
\section*{Translation of Expressions}
\[\varepsilon, k \rightarrow \text{code}, k'\]
\begin{itemize}
    \item $\varepsilon$ = minijava expression
    \item $k$ = find unused number
    \item code = sparrow code
    \item $k'$ = first unused number after we translated $\varepsilon$
\end{itemize}
To compile minijava into sparrow, we use an indexed list.  We go expression by expression through the minijava program, and load all the variables into the list.\\\\
Consider the syntax:
\[5, k \rightarrow (t_k = 5), k + 1\]
\begin{itemize}
    \item This means to put the \textit{expression} (in this case 5), into index $k$ of the list.
    \item The end result is $t_k = 5$, and $k$ is incremented to point to the next index.
\end{itemize}
More generally,
\[v, k \rightarrow (t_k = v), k + 1\]
where $v$ represents the value of a local variable.

\subsection*{Addition Example}
\[\epsilon_1, k + 1 \rightarrow \text{code}_1, k_1.\hspace{2cm} e_2, k_1 \rightarrow \text{code}_2, k\]

Suppose we have $e_1 + e_2$.  How does that translate?
\begin{itemize}
    \item In the indexed list, index $k$ is taken by the result of $e_1 + e_2$ (1 'slot').
    \item From index $k + 1$ to $k_1$ is the space to store code$_1$. (many 'slots')
    \item From index $k_1$ to $k_2$ is the space to store code$_2$.  (many 'slots')
\end{itemize}
We write:
\begin{align*}
e_1 + e_2, t_k &\rightarrow \text{code}_1 \text{\hspace{1cm}// result is in $t_{k + 1}$}\\
&\rightarrow \text{code}_2 \text{\hspace{1cm}// result is in $t_{k_1}$}
\end{align*}
What is $t_k$ in the first line?  Well, $t_k = t_{k + 1} + t_{k_1}$

\subsection*{Java Example}
To implement the above example in Java,
\begin{minted}{java}
class Result {
    String code;
    int t;

    public Result(String code, int t) {
        this.code = code;
        this.t = t;
    }

}

class Translator {
    Result visit(IntConst n, int k) {
        return new Result(
            "t" + k + "=" n, k + 1
        )
    }
    Result visit(Plus n, int k) {
        Result r1 = n.e1.accept(this, k + 1);
        Result r2 = n.e2.accept(this, r1.c);

        return new Result(
            r1.code + r2.code + "t" + k + "= t" + (k + 1) + "+ t" + r1.c
        );
    }
}
\end{minted}

\section*{Translation of Statements}
\[s, k \rightarrow \text{code}, k'\]
\begin{itemize}
    \item $s$ = minijava statement
    \item $k$ = first free number in indexed list
    \item code = sparrow code
    \item $k'$ = first free number after translation
\end{itemize}
\[\frac{e, k \rightarrow \text{code}, k_e}{(v = e), k \rightarrow \text{code}, v = t_k}\]
\begin{itemize}
    \item $v$ represents a local variable.
    \item This essentially stores the code in $t_k$.
\end{itemize}
\[\frac{s_1, k \rightarrow \text{code}_1, k_1 \hspace{1cm } s_2, k_1 \rightarrow \text{code}_2, k_2}{(s_1 \:;\: s_2), k \rightarrow {\text{code}_1 \choose \text{code}_2}, k_2}\]
\begin{itemize}
    \item $(s_1 \:;\: s_2)$ = minijava statements
    \item $k$ = first free number
\end{itemize}
In this example, in the indexed list, indices $k$ to $k_1$ are taken up by $s_1$, and $k_1$ to $k_2$ are taken up by $s_2$.

\subsection*{If-Else Example}
\[\frac{e, k \rightarrow \text{code}_e, k_e \hspace{1cm} s_1, k_e \rightarrow \text{code}_1, k_1 \hspace{1cm} s_2, k_1 \rightarrow \text{code}_2, k_2}{\text{if }(e) s_1 \text{ else } s_2, k \rightarrow \text{code}_e \text{~~~~// result in $t_k$}}\]
\begin{itemize}
    \item (if $(e) s_1$ else $s_2$) = minijava statement
    \item $k$ = first free number
\end{itemize}
For the if-else statement, we have two branches.  As a result, we must allocate space to store two separate sections of code.  The issue with our indexed list is that we execute it sequentially - we need some way to tell the computer to only execute one branch.  The solution?  GOTO statements!\\\\
Our block would look something like:
\begin{lstlisting}
    if0 $t_k$ goto else$_{k_2}$          // if0 is the same as JZ in assembly - jump if $t_k == 0$
        code$_1$                // code for if if statement evaluates to true (1)
        goto end$_{k_2}$            // jump over the else branch
    else$_{k_2}$:
        code$_2$                // code for if if statement evaluates to false (0)
    end$_{k_2}$: 
        ...

\end{lstlisting}
In this example, the spaces in the indexed list are:
\begin{itemize}
    \item Indices $k$ to $k_e$ are taken by the conditional of the if statement
    \item $k_e$ to $k_1$ is the True branch of the if statement
    \item $k_1$ to $k_2$ is the False branch of the if statement
    \item $k_2$ is the label for else$_{k_2}$
    \item $k_2 + 1$ is the label for end$_{k_2}$
\end{itemize}
As a result, the $k$ value after the if statement would be $k_2 + 2$.  The next statement or expression would begin there.

\subsection*{While Example}
\[\frac{e, k \rightarrow \text{code}_e, k_e \hspace{1cm} s, k_e \rightarrow \text{code}_s, k_s}{(\text{while}(e)\: s), k \rightarrow \text{loop}_{ks} \::\: k_e}\]
In this statement, we have two sections of code, one for the conditional, and one for the statements executed by the while loop.  However, we must be able to loop in the code.  This can also be done with GOTO statements!\\\\
Our block would look something like:
\begin{lstlisting}
    loop$_{k_s}$:
        code$_e$                // result in $t_k$, conditional of while loop
        if0 $t_k$ goto end$_{k_s}$       // if conditional evaluates to false, don't loop.
        code$_s$                // statements
        goto loop$_{k_s}$           // after body of while loop executes, go back to the conditional.
    end$_{k_s}$:
        ...
\end{lstlisting}
In this example, the spaces in the indexed list are:
\begin{itemize}
    \item $k_e$ to $k_s$ is taken by the conditional
    \item $k_s$ to $k$ is taken by the statements
    \item $k$ is the label for loop$_{k_s}$
    \item $k + 1$ is the label for end$_{k_s}$
\end{itemize}
While and If statements are the same construction in Sparrow - the major difference is that the while loop jumps backwards and the if statement jumps forwards.

\subsection*{If-Elif-Else Statements (Nested If)}
We write elif as a nested if statement.
\[\frac{\hspace{1cm}\frac{e_1, k \rightarrow \text{code}_{e_1}, k_{e_1} \hspace{0.5cm} s_1, k_{e_1} \rightarrow \text{code}_{s_1}, k_{s_1} \hspace{0.5cm} e_2, k_{s_1} \rightarrow \text{code}_{s_2}, k_{e_1} \hspace{0.5cm} s_2, k_{e_2} \rightarrow \text{code}_{s_2}, k_{s_2} \hspace{0.5cm}\dots}{\text{if}(e_1) s_2 \text{ else } s_2, k_{e_1} \rightarrow \text{code}, k_{if}}\hspace{1cm}}{\text{if}(e_1) s_1 \text{ else } (\text{if} (e_2) s_2 \text{ else } s_3). k \rightarrow }\]
This example would give the block:
\begin{lstlisting}
code$_{e_1}$
if0
code$_{s_1}$
goto

else$_{k_{if}}$:
    code

end$_{k_{if}}$:

\end{lstlisting}


\subsection*{Arrays}
\subsubsection*{Allocation}
\begin{lstlisting}
    e ::= new int[e] | e$_1$[e$_2$] | e.length
    s ::= v[e$_2$] = e$_3$
\end{lstlisting}

\begin{lstlisting}
    void m([]int a) {
        ... a.length...
    }
\end{lstlisting}
In the sparrow heap, arrays are stored as the following:
\begin{itemize}
    \item index 0 is the length of the array.  (arrays are fixed-length)
    \item indices 1- is the contents of the array.
    \item In minijava, each index is four bytes (because an int is 4 bytes)
    \begin{itemize}
        \item $\text{offset} = 4(\text{index} + 1)$
    \end{itemize}
\end{itemize}

\[\frac{e, k + 1 \rightarrow \text{code}_e, k_e}{(\texttt{new int [e]}), k \rightarrow \text{code}_e \hspace{1cm}\texttt{// result in }v_{k + 1}}\]
\begin{itemize}
    \item \texttt{(new int [e])} is a minijava expression
    \item $k$ is the first free number
    \item The memory slot the element is stored in $v_{k + 1}$ corresponds to the $k + 1$ in the 'numerator'.
    \item $k_e$ is the offset value - that is, $v_{k + 1} + 1$
    \item Since this code is for declaring a \texttt{new} array, we must allocate the space in heap.
    \begin{itemize}
        \item Incrementing $t_{k_e} = v_{k + 1} + 1$ is equivalent to doing $t_{k_e} = 4 \cdot t_{k_e}$
        \item $t_{k_e + 1}$ = \texttt{alloc}($t_{k_e}$) // this statement allocates the entire array
        \item Ideally, after the allocate, we would traverse through and set all the values to 0.  However, allocating the array means you can't read the values there before, so the step \textit{can} be skipped.
        \item To set to 0, move the pointer, then set 0.  Then move the pointer again, then set 0, etc...  Continue until you reach the correct length.
    \end{itemize}  
\end{itemize}   

\subsubsection*{Length Function}
\[\frac{e, k + 1 \rightarrow \texttt{code}, k_e}{\texttt{e.length}, k \rightarrow \texttt{code}_e \hspace{1cm}\texttt{// result in $t_{k + 1}$}}\]
\begin{itemize}
    \item \texttt{e.length} is a minijava expression
    \item $k$ is the first free number
    \begin{itemize}
        \item $t_k = [t_{k + 1} + 0]$
    \end{itemize}
    \item The result of \texttt{code$_e$} is stored in $t_{k + 1}$ because $t_k$ contains the length of the array.
\end{itemize}   

\subsubsection*{Array Access}
\[\frac{e_1, k + 1 \rightarrow \texttt{code}_1, k_1 \hspace{1cm} e_2, k_1 \rightarrow \texttt{code}_2, k_2}{e_1[e_2], k \rightarrow \texttt{code}_1, \texttt{code}_2}\]
\begin{itemize}
    \item \texttt{code}$_2$ is stored in $t_{k + 1}$
    \item \texttt{code}$_3$ is stored in $t_{k_1}$
    \item $t_{k_2} = 4(t_{k_1} + 1)$
    \item $t_k = [t_{k + 1} + t_{k_2}]$
\end{itemize}
Note that:
\begin{itemize}
    \item $t_k$ represents the location in heap where the number from the access is stored.  That is, $e_1$[$e_2$] is stored at that location.
    \item $t_k = [t_{k + 1} + t_{k_2}]$ because $t_{k + 1}$ represents the first index of the content of the array, and $t_{k_2}$ is the byte offset to the index from the top of the array.
\end{itemize}


\[\frac{e_2, k \rightarrow \texttt{code}_2, k_2 \hspace{1cm} e_3, k_2 \rightarrow \texttt{code}_3, k_3}{(v[e_2] = e_3), k \rightarrow \texttt{code}_2, \texttt{code}_3}\]
\begin{itemize}
    \item $v[e_2] = e_3$ is a minijava expression
    \item $k$ is the first free number
    \item \texttt{code}$_2$ is stored in $t_k$
    \item \texttt{code}$_3$ is stored in $t_{k_2}$
    \item $t_{k_3} = 4(t_k + 1)$, and [v + $t_{k_3}$] = $t_{k_2}$
\end{itemize}

\section*{Classes and Objects}
Objects are instantiated like groups of variables in a heap.  The code for the methods and function are stored like how they are allocated in C.\
\begin{itemize}
    \item Methods in form \texttt{m(int a)} translates to \texttt{$c_m$(this a)} - a combination of a class and method name.
\end{itemize}
\[e ::= \texttt{new} C() \:\vert\: x \:\vert\: \texttt{this} \:\vert\: e_1 . m(e_2)\]
\begin{itemize}
    \item $C$ is the class name
    \item $x$ is a field name.  (e.g., C.x)
\end{itemize}
\subsection*{Accessing methods in classes}
\[\frac{e_1, k + 1 \rightarrow \texttt{code}_1, k_1 \hspace{1cm} e_2, k_1 \rightarrow \texttt{code}_2, k_2}{e_1 \:.\: m(e_2), k \rightarrow \texttt{code}_1, \texttt{code}_2}\]
\begin{itemize}
    \item $e_1 \:.\: m(e_2)$ is a minijava expression (the dot is part of minijava syntax)
    \item \texttt{code}$_1$ is result in $t_{k + 1}$
    \item \texttt{code}$_2$ is result in $t_{k_1}$
    \item $t+k = C-m (t_{k + 1}, t_{k_1})$
    \begin{itemize}
        \item C-m is the function name in sparrow.
        \item Like normal functions in sparrow, \texttt{code}$_1$ is the parameters, \texttt{code}$_2$ is the body of the function
    \end{itemize}
\end{itemize}

\subsection*{Instantiating Objects}
\[\texttt{new } C(), k \rightarrow t_k = \texttt{alloc}(4 \cdot \#fields(c)), k + 1\]
\begin{itemize}
    \item \texttt{new }$C()$ is the minijava expression
    \item $k$ is the first free number
    \item $\texttt{alloc}(4 \cdot \#fields(c))$ allocates a space in heap large enough to store all the fields of the object.
\end{itemize}


\section*{Back to Type Checking (with class inheritance)}
\begin{verbatim}
    class C {
        t m (s a) {
            ...
        }
    }

    class D extends C {
        ... 
        t m (s a) {
            ...
        }
        ...
    }
\end{verbatim}
\begin{itemize}
    \item Minijava does not allow \textit{overloading}, the practice of having two methods of the same name but different parameters in the same class
    \item It does, however, allow \textit{overwriting}, or having an inheriting class overwrite a class in its parent.  In the above example, the two methods have type t, the same method name m, and single parameter of type s and name a.
\end{itemize}
In this case, having a type checker simplifies a lot of code, since now the compiler just has to figure out which method to call.  If it is an object of Class C, it calls the method in class C.  If it is an object of Class D, it calls the method in class D, because the one in D overwrites the one in C.

\subsection*{Method Tables}
\begin{verbatim}
class C {
    ...
    void m() {
        ...
    }
    void q() {
        ...
    }
}
\end{verbatim}

\begin{itemize}
    \item Method table would store the locations of the start of C-m and C-q.
    \item A pointer to the method table location in heap is stored in the first index of each object.
    \item Calling a class method makes the compiler
    \begin{enumerate}
        \item Load method table
        \item Load the method you want
        \item Call the method
    \end{enumerate}
\end{itemize}
\subsection*{Calling a Method}
\[\frac{e_1, k + 1 \rightarrow \texttt{code}_1, k_1 \hspace{1cm} e_2, k_1 \rightarrow \texttt{code}_2, k_2}{e_1 \:.\: m(e_2), k \rightarrow \texttt{code}_1, \texttt{code}_2}\]
\begin{itemize}
    \item $e_1 \:.\: m(e_2)$ is the minijava expression (method call)
    \item $k$ is the first free number
    \item \texttt{code}$_1$ is stored in $t_{k + 1}$
    \item \texttt{code}$_2$ is stored in $t_{k_1}$
    \item $t_{k_2} = [t_{k + 1} + 0]$
    \item $t_{k_2} + 1 = [t_{k_2} + 4 \cdot \text{(index of m)}]$
    \item $t_k$ = call $t_{k_2 + 1}$($t_{k + 1}$, $t_{k_2}$)
\end{itemize}
\subsection*{A View of Method Tables}
\begin{verbatim}
class C {
    void m() {...} 
    void p() {...}
}

class D extends C {
    void m() {...} // override!
    void q() {...} // only in D object!

}

/*
C Object Method Table:
[C-m][C-p]
0    4

D Object Method Table:
[D-m][C-p][D-q]
0    4    8
*/
\end{verbatim}



\end{document}