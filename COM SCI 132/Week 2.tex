% document formatting
\documentclass[10pt]{article}
\usepackage[utf8]{inputenc}
\usepackage[left=1in,right=1in,top=1in,bottom=1in]{geometry}
\usepackage[T1]{fontenc}
\usepackage{xcolor}

% math symbols, etc.
\usepackage{amsmath, amsfonts, amssymb, amsthm}

% lists
\usepackage{enumerate}

% images
\usepackage{graphicx} % for images

% code blocks
\usepackage{minted, listings} 

% verbatim greek
\usepackage{alphabeta}

\graphicspath{{./assets/images}}

\newcommand{\solution}{\textbf{Solution:}} 

\title{COM SCI 132 Week 2}

\author{Aidan Jan}
\date{\today}

\begin{document}
\maketitle
\section*{The Java Compiler Compiler (JCC)}
\begin{itemize}
    \item Can be thought of as "Lex and Yacc for Java"
    \item It is based on LL(k) rather than LALR(1).
    \item Grammars are written in EBNF.
    \item The Java Compiler Compiler transforms an EBNF grammar into an LL(k) parser.
    \item The JavaCC grammar can have embedded action code written in Java, just like a Yacc grammar can have embeeded action code written in C.
    \item The lookahead can be changed by writing LOOKAHEAD(\dots).
    \item The whole input is given in just one file (not two).
\end{itemize}
\textbf{Input format:} one file containing a header, token specifications for lexical analysis, and grammar.\\
\textbf{Example of a token specification:}
\begin{verbatim}
TOKEN:
{
    < INTEGER_LITERAL: ( ["1"-"9"] (["0"-"9"])* | "0" ) >
}
\end{verbatim}
\textbf{Example of a production:}
\begin{verbatim}
void StatementListReturn():
{}
{
    ( Statement() )* "return" Expression() ";"
}
\end{verbatim}
\textbf{Generating a parser with JavaCC}
\begin{verbatim}
javacc fortran.jj // generates a parser with a specified name
javac Main.java // Main.java contains a call of the parser
java Main < prog.f // parses the program prog.f    
\end{verbatim}

\section*{The Visitor Pattern}
According to Gamma, Helm, Johnson, Vlissides: \textbf{Design Patterns}, 1995:
\begin{quotation}
For \textbf{object-oriented programming}, the Visitor pattern \textbf{enables} the definition of a \textbf{new operation} on an \textbf{object structure without changing the classes} of the objects.
\end{quotation}

\subsection*{First Approach: Instanceof and Type Casts}
The running Java example: summing an integer list.
\begin{verbatim}
interface List {}

class Nil implements List {}

class Cons immplements List {
    int head;
    List tail;
}

List l;
\end{verbatim}
Suppose we have a \texttt{List l} in the code to parse.  Is \texttt{l} a \texttt{Nil} or a \texttt{Cons}?  \\
One way to check is to do type casting:
\begin{verbatim}
List l;  // The List-object
int sum = 0;
boolean proceed = true;
while(proceed) {
    if (l instanceof Nil) proceed = false;
    else if (l instanceof Cons) {
        sum += ((Cons) l).head;  // Notice these two type casts!
        l = ((Cons) l).tail;
    }
}
\end{verbatim}
\textbf{Advantage:} The code is written without touching the classes Nil and Cons.\\
\textbf{Drawback:} The code constantly uses type casts and \texttt{instanceof} to determine what class of object it is considering.\\

\subsection*{Second Approach: Dedicated Methods}
The first approach is \textbf{not} object-oriented!  To access parts of an object, the classcal approach is to use dedicated methods which both access and act on the subobjects.\\
Instead, of writing a \texttt{sum()} method that operates on list types, just simply add a .sum() method to the \texttt{List} interface.  All classes that implement the interface can then override the method.

\begin{verbatim}
class Nil implements List {
    public int sum() { return 0; }
}

class Cons implements List {
    int head;
    List tail;
    public int sum() {
        return head + tail.sum();
    }
}
\end{verbatim}
\textbf{Advantage:} The type casts and \texttt{instanceof} operations have disappeared, and the code can be written in a systematic way.\\
\textbf{Drawback:} For each new operation on \texttt{List}-objects, write new dedicated methods and recompile all classes.

\subsection*{Third Approach: The Visitor Pattern}
\textbf{The Idea:}
\begin{itemize}
    \item Divide the code into an object structure and a Visitor (akin to Functional Programming!)
    \item Insert an \texttt{accept} method in each class.  Each accept method takes a Visitor as an argument.
    \item A Visitor contains a \texttt{visit} method for each class (overloading!) A method for a class $C$ takes an argument of type $C$.
\end{itemize}
\begin{verbatim}
interface List {
    void accept(Visitor v);
}

interface Visitor {
    void visit(Nil x);
    void visit(Cons x);
}
\end{verbatim}
The purpose of the \texttt{accept} methods is to invoke the \texttt{visit} method in the Visitor which can handle the current object.

\begin{verbatim}
class Nil implements List {
    public void accept(Visitor v) {
        v.visit(this);
    }
}

class Cons implements List {
    int head;
    List tail;
    public void accept(Visitor v) {
        v.visit(this);
    }
}
\end{verbatim}
\textbf{Notice:} the \texttt{accept} method in both classes is the same!  Due to the way the Visitor is implemented, the method always calls \texttt{visit}.  This is beneficial because it allows the compiler to easily parse classes by appending the same code to each class.  In fact, the \texttt{accept} method is always implemented the way it is shown above, for any class!\\\\
The control flow goes back and forth between the \texttt{visit} methods in the Visitor and the \texttt{accept} methods in the object structure.

\begin{verbatim}
class SumVisitor implements Visitor {
    int sum = 0;
    public void visit(Nil x) {}
    public void visit(Cons x) {
        sum += x.head;
        x.tail.accept(this);
    }
}
.....

SumVisitor sv = new SumVisitor();
l.accept(sv);
System.out.println(sv.sum);
\end{verbatim}
\textbf{Also notice:} since the \texttt{accept} method is titled \texttt{accept} in every class, the \texttt{visit} method in the Visitor can just call the \texttt{accept} method, and it automatically calls the method for the correct class!

\subsection*{Visitors: Summary}
\begin{itemize}
    \item Visitor makes adding new operations easy.  Simply write a new visitor.
    \item A visitor gathers related operations.  It also separates unrelated ones.
    \item Adding new classes to the object structure is hard.  Key consideration: are you most likely to change the algorithm applied over an object structure, or are you most like to change the classes of objects that make up the structure.
    \item Visitors can accumulate state.
    \item Visitor can break encapsulation.  Visitor's approach assumes that the interface of the data sturcture classes is powerful enough tot let visitors do their job.  As a result, the patten often forces you to provide public operations that access internal state, which may compromise its encapsulation.
\end{itemize}

\section*{The Java Tree Builder}
The Java Tree Builder (JTB) is a front for The Java Compiler Compiler.  It supports the building of syntax trees which can be traversed using visitors.\\\\
JTB transforms a bare JavaCC grammar into three components:
\begin{enumerate}
    \item a JavaCC grammar with embedded Java code for building a syntax tree;
    \item one class for every form of syntax tree node; and
    \item a default visitor which can do a depth-first traversal of a syntax tree.
\end{enumerate}
The produced JavaCC grammar cam then be processed by the JCC to give a parser which produces syntax trees.  The produced syntax trees can now be traversed by a Java program by writing subclasses of the default visitor.
\begin{center}
    \includegraphics*[scale=0.9]{W2_1.png}
\end{center}
\subsubsection*{Example (simplified)}
JTB produces a syntax-tree-node class for \texttt{Assignment:}
\begin{verbatim}
public class Assignment implements Npde {
    PrimaryExpression f0; AssignmentOperator f1;
    Expression f2;

    public Assignment (PrimaryExpression n0, AssignmentOperator n1, Expression n2) {
        f0 = n0;
        f1 = n1;
        f2 = n2;
    }

    public void accept(visitor.Visitor v) {
        v.visit(this);
    }
}
\end{verbatim}
Notice the \texttt{accept} method; it invokes the method \texttt{visit} for \texttt{Assignment} in the default visitor.\\\\
The default visitor looks like this:
\begin{verbatim}
public class DepthFirstVisitor implements Visitor {
    ...
    //
    // f0 -> PrimaryExpression()
    // f1 -> AssignmentOperator()
    // f2 -> Expression()
    //
    public void visit(Assignment n) {
        n.f0.accept(this);
        n.f1.accept(this);
        n.f2.accept(this);
    }
}
\end{verbatim}
Notice the body of the method which visits each of the three subtrees of the \texttt{Assignment} node.

\section*{Type Checking with Trees}
Consider the following:
\begin{center}
    \texttt{e ::= c | true | false | !e | e + e}, where \texttt{c} represents a numerical constant.
\end{center}
Expressions can be written into trees then type checked.  For example, if the first $e$ is derived to $e + e$, then the new $e$ must be integers.  If $!e$ is derived, then $e$ must be a boolean.  Cases like \texttt{true + 5} or \texttt{!3} are not allowed.

\subsection*{Notation}
\[\frac{\text{hypothesis}_1 \dots \text{hypothesis}_n}{\text{conclusion}}\]
is read as
\begin{verbatim}
    if (hypothesis1 and ... and hypothesisN)
    then: conclusion
\end{verbatim}
In other words, if the numerator can be derived (e.g., proven true), then the conclusion can also be derived.\\
In this case, we care about this notation for type checking.  In this problem, we care about hypotheses in form $e\::\:t$, where
\begin{itemize}
    \item $e$ = expression
    \item $t$ = data type (e.g., t ::= int | boolean)
\end{itemize}
We can then derive expressions like:
\[\frac{e_1 \::\: \text{int}, e_2 \::\: \text{int}}{e_1 + e_2 \::\: \text{int}}\]
We can recursively go up the tree through every expression and check the types of all the expressions in this way.  If some mismatch is detected, then the compiler returns an error.\\
Example: (pseudocode)
\begin{minted}{java}
class TypeChecker implements DepthFirstVisitor {
    Type visit (IntConst n) {
        return "int";
    }

    Type visit(Plus n) {
        Type t0 = n.f0.accept(this)
        Type t2 = n.f2.accept(this)
        if (t0 == "int" && t2 == "int") {
            return "int";
        } else {
            throw new Exception();
        }
    }
}
\end{minted}

\subsection*{Some other derivation examples}

\[\frac{\hspace{2cm}}{\vdash x = e}\]
Assignment does not take any other expressions to derive; it simply just \textit{is}.

\[\frac{\vdash e \::\: \text{boolean}, \vdash s_1, \vdash s_2}{\vdash \text{if}(e): s_1 \text{ else}: s_2}\]
If/else can be derived if given that $e$ is a boolean, and $s_1$ and $s_2$ have some type.


\[\frac{\vdash e \::\: \text{boolean}, \vdash s}{\vdash \text{while} (e): \:s}\]
Similarly, while can be derived if given that $e$ is a boolean, and $s$ has some type.

\subsection*{Overloading}
Consider:
\begin{verbatim}
class a {
    int a;
    void a (int a) {
        boolean a;
    }
}
\end{verbatim}
\texttt{a} has five different appearances: class name, field name, method name, parameter name, and local variable name.  How do you type \texttt{a}?  Also, how many of the \texttt{a} are identical?\\\\
It turns out that none of the \texttt{a} are identical; they are all allowed to coexist.  As a result, we have to track these variables.  Thus, in the symbol table, it must be "chunked" per method - use a stack to keep track.  (This is similar to the environment manager in CS131).

\subsection*{Arrays}
In minijava, arrays are used as such:
\begin{center}
    \texttt{e ::= new int[e] | e1[e2] | e.length}\\
    \texttt{s ::= id[e2] = e3}
\end{center}
Note: now, types are not limited to int and bool, but also extended to int[].\\
Additional derivations:
\begin{align*}
    &\frac{A \vdash e \::\: \text{int}}{A \vdash \text{new int[e]} \::\: \text{int[]}}\\
    &\frac{A \vdash e_1 \::\: \text{int[]}, \:A \vdash e_2 \::\: \text{int}}{A \vdash e_1[e_2]\::\: \text{int}}
\end{align*}
$A$ represents the symbol table.
\texttt{A $\vdash$ e : t} represents "in symboltable $A$, the expression $e$ has type $t$."\\
Similarly, \texttt{A $\vdash$ s} represents "in symboltable $A$, the statement $s$ tyoe checks."\\
We can extend our visit function from earlier to include a symbol table.
\begin{verbatim}
Type visit(Exp n, Symboltable A) {
    ...
    return;
}
\end{verbatim}








\end{document}