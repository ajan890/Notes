% document formatting
\documentclass[10pt]{article}
\usepackage[utf8]{inputenc}
\usepackage[left=1in,right=1in,top=1in,bottom=1in]{geometry}
\usepackage[T1]{fontenc}
\usepackage{xcolor}

% math symbols, etc.
\usepackage{amsmath, amsfonts, amssymb, amsthm}

% lists
\usepackage{enumerate}

% images
\usepackage{graphicx} % for images

% code blocks
\usepackage{minted, listings} 
\lstset{
  basicstyle=\ttfamily,
  mathescape
}
% verbatim greek
\usepackage{alphabeta}

\graphicspath{{./assets/images}}

\newcommand{\solution}{\textbf{Solution:}} 

\title{COM SCI 132 Week 3}

\author{Aidan Jan}
\date{\today}

\begin{document}
\maketitle
\section*{Reducing Variables}
A program can have many variables, but a modern computer only has 16 registers.  We must find some way to reduce the number of variables!\\
\begin{itemize}
    \item Liveness Analysis $\rightarrow$ Interference Graph
\end{itemize}
Consider the example:
\begin{verbatim}
    a = 1
    b = 10
    c = 9 + a
    d = a + c
    e = c + d
    f = b + 8
    c = f + e
    f = e + c
    b = c + 5
    return b + f
\end{verbatim}
Notice that we have six variables.  How do we compress this?
\begin{verbatim}
                   |                  |  a  b  c  d  e  f
                   |  a = 1           |  *
                a  |                  |  | 
                   |  b = 10          |  |  *
          c  b  a  |                  |  |  |
                   |  c = 9 + a       |  |  |  *
          c  b  a  |                  |  |  |  |
                   |  d = a + c       |  O  |  |  *
       d  c  b     |                  |     |  |  |
                   |  e = c + d       |     |  O  O  *
    e        b     |                  |     |        |
                   |  f = b + 8       |     O        |  *
 f  e              |                  |              |  |
                   |  c = f + e       |        *     |  O
    e     c        |                  |        |     |
                   |  f = e + c       |        |     O  *
 f        c        |                  |        |        |
                   |  b = c + 5       |     *  O        |
 f           b     |                  |     |           |
                   |  return b + f    |     O           O
\end{verbatim}
Above is a \textbf{liveness graph}.
\begin{itemize}
    \item The middle section shows the code.
    \item The right section shows the "liveness" of each variable.
    \begin{itemize}
        \item A variable becomes "alive" (denoted with \texttt{*}) when it is assigned a value.
        \item A variable stays alive until the last time it is used before its next assignment or the end of the program.
        \item A variable "dies" or can be deallocated after the last time it is used.  (denoted with \texttt{O}).
    \end{itemize}   
    \item The left section shows which variables are alive between each statement.
    \begin{itemize}
        \item Note that these are written in between each line of code instead of on the same line.
        \item The number of registers required is the maximum number of variables on a line plus 1.  (In this case, 3 variables + 1 = 4 registers)
        \begin{itemize}
            \item The (+1) comes from the need to have an extra register to temporarily store a variable as it is being assigned.
            \item This is similar to why you need an extra variable to swap the values in two variables!
        \end{itemize}
        \item This rule ONLY works when the code is straight executing (e.g., no branches or loops).
    \end{itemize}
\end{itemize}

\section*{Liveness with loops - Context-free Graph}
Consider the following:
\begin{verbatim}
        c = 0
        a = 0
    L1: b = a + 1
        c = c + 2
        a = b * 2
        if (a < 10) goto L1
        return c
\end{verbatim}

\begin{tabular}{| l | c | c || c | c || c | c || c | c || c | c || c | c || c | c |}
    \hline
    \multicolumn{3}{|c||}{ } & \multicolumn{2}{c ||}{init} & \multicolumn{2}{c ||}{iter 1} & \multicolumn{2}{c ||}{iter 2} & \multicolumn{2}{c ||}{iter 3} & \multicolumn{2}{c ||}{iter 4} & \multicolumn{2}{c |}{iter 5}\\
    \hline
    statement & def & use & in & out & in & out & in & out & in & out & in & out & in & out\\
    \hline
%    \texttt{c = 0} & c & - & - & - & - & a & - & ac & c & ac \\
    \texttt{a = 0} & a & - & - & - & - & a & - & ac & c & ac & c & ac & & \\
    \texttt{L1: b = a + 1} & b & a & - & - & a & c & ac & bc & ac & bc & ac & bc & & \\
    \texttt{c = c + 2} & c & c & - & - & c & b & bc & b & bc & bc & bc & bc & & \\
    \texttt{a = b * 2} & a & b & - & - & b & a & b & ac & bc & ac & bc & ac & & \\
    \texttt{if (a < 10) goto L1} & - & a & - & - & a & ac & ac & ac & ac & ac & ac & ac & & \\
    \texttt{return c} & - & c & - & - & c & - & c & - & c & - & c & - & & \\
    \hline
\end{tabular}
\begin{itemize}
    \item \texttt{in} and \texttt{out} refer to which variables are live going into the statement and which variables are live coming out of the statement.
    \item We can stop after iter 4 because iter 4 matches iter 3, signifying that it will no longer change.
\end{itemize}
\textbf{Liveness Equations:}
\begin{align*}
    \text{in}[n] &= \text{use}[n] \cup (\text{out}[n] - \text{def}[n])\\
    \text{out}[n] &= \bigcup_{s \in \text{succ}[n]} \text{in}[s]
\end{align*}
\textbf{Number of Iterations Time Complexity:}
\begin{itemize}
    \item The number of iterations is $O(n^2)$.
    \item In the worst case scenario, every box is filled with every variable at the end of all the iterations.  We have $2n$ boxes to fill in: $n$ "in" boxes and $n$ "out" boxes.
    \item In the worst case scenario, each iteration only adds one variable to one box.  There are $n$ variables.
    \item Therefore, the worst case scenario would be of order $2n \cdot n = n^2$.
\end{itemize}
\textbf{Full Program Time Complexity:}
\begin{itemize}
    \item When there are $n$ variables...
    \item $O(n^2)$ iterations
    \item $O(n)$ set unions per iteration
    \item $O(n)$ time to do a set union
    \item Total = $O(n)^4$ (polynomial time!)
\end{itemize}
Note that this is not the most efficient algorithm - this is a sloppy way to do the algorithm, in other words, the naive way.  With more involved analyzing, this algorithm can be reduced to $O(n^2)$.

\end{document}