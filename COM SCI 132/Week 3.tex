% document formatting
\documentclass[10pt]{article}
\usepackage[utf8]{inputenc}
\usepackage[left=1in,right=1in,top=1in,bottom=1in]{geometry}
\usepackage[T1]{fontenc}
\usepackage{xcolor}

% math symbols, etc.
\usepackage{amsmath, amsfonts, amssymb, amsthm}

% lists
\usepackage{enumerate}

% images
\usepackage{graphicx} % for images

% code blocks
\usepackage{minted, listings} 

% verbatim greek
\usepackage{alphabeta}

\graphicspath{{./assets/images}}

\newcommand{\solution}{\textbf{Solution:}} 

\title{COM SCI 132 Week 3}

\author{Aidan Jan}
\date{\today}

\begin{document}
\maketitle
\section*{Type Checking Continued}
Review: we have expressions \texttt{A $\vdash e \::\: t$} and statements \texttt{A $\vdash$ s}, where
\begin{itemize}
    \item $A$ represents the symbol table (type environment)
    \begin{itemize}
        \item Must be searched in the order: local variables, parameters, then fields
    \end{itemize}
    \item $s$ represents a statement
    \item $e$ represents an expression
    \item $t$ represents a data type (out of \{\texttt{int, bool, int[], C}\})
    \begin{itemize}
        \item \texttt{C} represents some user-defined class
    \end{itemize}
\end{itemize}

\section*{Type Checking Methods}
Methods are written in the format:
\[\texttt{$t_r$ m ($t_a$ a) \{$t_l$ x; s; return e\}}\]
\begin{itemize}
    \item $t_r$ is the return type
    \item \texttt{m} is the method name
    \item $t_a$ is the type of the parameter \texttt{a}
    \item \texttt{a} is the parameter
    \item $t_l$ is the type of the local variable
    \item \texttt{x} is a local variable
    \item \texttt{s} is a statement
    \item \texttt{e} is the return value (which must have type $t_r$)
\end{itemize}
Additionally,
\[\frac{\texttt{A = fields $\cdot$ (a : $t_a$, k : $l_l$), A $\vdash$ s, A $\vdash$ e : $t_r$}}{\texttt{$t_r$ m ($t_a$ a) \{$t_l$ x; s; return e\}}}\]
For a method call,
\[\frac{A \vdash e_0\::\: \texttt{C}, \textbf{c}, A \vdash e\::\:t_a}{A \vdash e_0 \cdot m(e)\::\: t_r}\]
where \textbf{c} refers to
\begin{align*}
&\texttt{class c \{}\hspace{13cm}\\
&\texttt{~~~~// fields}\\
&\texttt{~~~~...}\\
&\texttt{~~~~$t_r$ m ($t_a$ a) \{ s \}}\\
&\texttt{\}}
\end{align*}
and $t_a$ represents the type of the parameter in \textbf{c}.

\section*{Objects}
\begin{itemize}
    \item In Java (and miniJava), objects are created with the \texttt{new} keyword.
    \item This stores the object in the symbol table, along with any object variables (fields) and their types.
\end{itemize}   
\subsection*{Subtyping}
Consider the following representations of a number: byte, short, int, long, double.  In increasing order, byte has 8 bits of storage, a short 16, an int 32, and a long and double 64.  Due to the increasing bit lengths, a 'bigger' data type can contain 'smaller' types.  For example,
\begin{verbatim}
    int a = 0;
    long b = 0;
    b = a;
\end{verbatim} 
The above is possible since a long is big enough to store all the data an int contains.  However,
\begin{verbatim}
    a = b;
\end{verbatim}
is not possible, because an int cannot contain a long.
\subsection*{Subtyping with Classes}
A class can inherit another class with the keyword \texttt{extends}.  When a class is inherited, the class inheriting gains all the functions and private variables (fields) of the inherited class.  For example,
\begin{verbatim}
    class A { ... }
    class B extends A { ... }
    
    A a = new A(...);
    B b = new B(...);

    A = B;
\end{verbatim}
Setting A to B is valid since A can contain the data B has, in that all of A's fields will be filled.  However, setting \texttt{B = A} is invalid since B is a subtype of A, and B has less functionality than A.\\
Example: (ColorPoint $\subseteq$ Point)
\begin{minted}{java}
class Point {
    public Point() { ... }
    public void move() { ... }


}
class ColorPoint extends Point {
    public ColorPoint() { ... }
    public void color() { this.move(); ... }

}

class Main {
    public static void main(String[] args) {
        Point p;
        ColorPoint q;

        p = q; // legal!
        q = p; // illegal!

        q.color();
    }

}
\end{minted}
Remember that if $t_e \subseteq t_x$, then
\[\frac{x \::\: t_x, e \::\: t_e}{\vdash x = e}\]
Everything done on a \texttt{p} can be done on a \texttt{q}, but not the reverse, because \texttt{q} extends \texttt{p}.


\end{document}