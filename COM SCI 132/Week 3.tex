% document formatting
\documentclass[10pt]{article}
\usepackage[utf8]{inputenc}
\usepackage[left=1in,right=1in,top=1in,bottom=1in]{geometry}
\usepackage[T1]{fontenc}
\usepackage{xcolor}

% math symbols, etc.
\usepackage{amsmath, amsfonts, amssymb, amsthm}

% lists
\usepackage{enumerate}

% images
\usepackage{graphicx} % for images

% code blocks
\usepackage{minted, listings} 

% verbatim greek
\usepackage{alphabeta}

\graphicspath{{./assets/images}}

\newcommand{\solution}{\textbf{Solution:}} 

\title{COM SCI 132 Week 3}

\author{Aidan Jan}
\date{\today}

\begin{document}
\maketitle
\section*{Type Checking Continued}
Review: we have expressions \texttt{A $\vdash e \::\: t$} and statements \texttt{A $\vdash$ s}, where
\begin{itemize}
    \item $A$ represents the symbol table (type environment)
    \begin{itemize}
        \item Must be searched in the order: local variables, parameters, then fields
    \end{itemize}
    \item $s$ represents a statement
    \item $e$ represents an expression
    \item $t$ represents a data type (out of \{\texttt{int, bool, int[], C}\})
    \begin{itemize}
        \item \texttt{C} represents some user-defined class
    \end{itemize}
\end{itemize}

\section*{Type Checking Methods}
Methods are written in the format:
\[\texttt{$t_r$ m ($t_a$ a) \{$t_l$ x; s; return e\}}\]
\begin{itemize}
    \item $t_r$ is the return type
    \item \texttt{m} is the method name
    \item $t_a$ is the type of the parameter \texttt{a}
    \item \texttt{a} is the parameter
    \item $t_l$ is the type of the local variable
    \item \texttt{x} is a local variable
    \item \texttt{s} is a statement
    \item \texttt{e} is the return value (which must have type $t_r$)
\end{itemize}
Additionally,
\[\frac{\texttt{A = fields $\cdot$ (a : $t_a$, k : $l_l$), A $\vdash$ s, A $\vdash$ e : $t_r$}}{\texttt{$t_r$ m ($t_a$ a) \{$t_l$ x; s; return e\}}}\]
For a method call,
\[\frac{A \vdash e_0\::\: \texttt{C}, \textbf{c}, A \vdash e\::\:t_a}{A \vdash e_0 \cdot m(e)\::\: t_r}\]
where \textbf{c} refers to
\begin{align*}
&\texttt{class c \{}\hspace{13cm}\\
&\texttt{~~~~// fields}\\
&\texttt{~~~~...}\\
&\texttt{~~~~$t_r$ m ($t_a$ a) \{ s \}}\\
&\texttt{\}}
\end{align*}
and $t_a$ represents the type of the parameter in \textbf{c}.

\section*{Objects}
\begin{itemize}
    \item In Java (and miniJava), objects are created with the \texttt{new} keyword.
    \item This stores the object in the symbol table, along with any object variables (fields) and their types.
\end{itemize}   
\subsection*{Subtyping}
Consider the following representations of a number: byte, short, int, long, double.  In increasing order, byte has 8 bits of storage, a short 16, an int 32, and a long and double 64.  Due to the increasing bit lengths, a 'bigger' data type can contain 'smaller' types.  For example,
\begin{verbatim}
    int a = 0;
    long b = 0;
    b = a;
\end{verbatim} 
The above is possible since a long is big enough to store all the data an int contains.  However,
\begin{verbatim}
    a = b;
\end{verbatim}
is not possible, because an int cannot contain a long.
\subsection*{Subtyping with Classes}
A class can inherit another class with the keyword \texttt{extends}.  When a class is inherited, the class inheriting gains all the functions and private variables (fields) of the inherited class.  For example,
\begin{verbatim}
    class A { ... }
    class B extends A { ... }
    
    A a = new A(...);
    B b = new B(...);

    A = B;
\end{verbatim}
Setting A to B is valid since A can contain the data B has, in that all of A's fields will be filled.  However, setting \texttt{B = A} is invalid since B is a subtype of A, and B has less functionality than A.\\
Example: (ColorPoint $\subseteq$ Point)
\begin{minted}{java}
class Point {
    public Point() { ... }
    public void move() { ... }


}
class ColorPoint extends Point {
    public ColorPoint() { ... }
    public void color() { this.move(); ... }

}

class Main {
    public static void main(String[] args) {
        Point p;
        ColorPoint q;

        p = q; // legal!
        q = p; // illegal!

        q.color();
    }

}
\end{minted}
Remember that if $t_e \subseteq t_x$, then
\[\frac{x \::\: t_x, e \::\: t_e}{\vdash x = e}\]
Everything done on a \texttt{p} can be done on a \texttt{q}, but not the reverse, because \texttt{q} extends \texttt{p}.

\subsection*{Sparrow}
\begin{itemize}
    \item Program       \texttt{p ::= $F_1$ \dots $F_m$}
    \item FunDecl       \texttt{F ::= func f ($id_1$ \dots $id_f$) b}
    \item Block         \texttt{b ::= $i_1$ \dots $i_n$ return id}
    \item Instruction   \texttt{i ::= l: | id=@f | id = id + id | \dots | id = [id + c] | [id + c] = id |\\ id = id | id = alloc(id) | print(id) | goto l | if0 id goto l | id = call id(id\dots id)}
    \begin{itemize}
        \item \dots includes subtract, greater than, less than, etc. operators
        \item in \texttt{[id + c]}, \texttt{id} is the heap address, \texttt{c} is the offset (measured in bytes).
        \item \texttt{l:} is a label
        \item \texttt{if0} is a conditional check that executes the goto if the variable is zero.  This is equivalent to the JZ processor command in x86 assembly.
        \item Using \texttt{call} on an id works because identifiers can also be functions.
    \end{itemize}
\end{itemize}
\subsection*{Sparrow Rules}
\[\frac{\text{hypothesis$_0$ \dots hypothesis$_n$}}{\text{conclusion}}\]
\begin{itemize}
    \item Values:  integers \texttt{c}; heap address with offset (a, c); function names \texttt{f}
    \item The heap: \texttt{H}: map from heap addresses to tuples of values
    \item Environment: \texttt{E}: map from identifiers to values
    \item Program state: $p, H, b^*, E, b$
    \begin{itemize}
        \item $p$ is the program (never modified; program being executed stays the same)
        \item $H$ is the heap (heap changes as program is executed)
        \item $b^*$ is the function being executed (in a way, a bigger block, changes only when change of control occurs)
        \item $E$ is the environment
        \item $b$ is the block of code currently being executed.  (e.g., a loop, conditional block, etc., same as $b^*$ at the start.)
    \end{itemize}
\end{itemize}  

\subsection*{Program States}
\subsubsection*{Assignment to constant}
Suppose we add an extra statement in front of $b$, such that the statement is executed first.  Like:
\[(p, H, b^*, E, (\texttt{id = c}) \cdot b)\]
What happens on the next step?
\[(p, H, b^*, E \cdot [id \rightarrow c], b)\]
\begin{itemize}
    \item An id is now assigned to the number c and stored in the environment
    \item The next statement being executed is the first statement in $b$
\end{itemize}
\subsubsection*{Assignment to expression}
What if we have $(p, H, b^*, E, id = (id_1 - id_2) \cdot b)$?\\
The next step is more complex since we must check that $id_1$ and $id_2$ are both integers.
\[\begin{cases} (p, H, b^*, E \cdot [id \rightarrow (c_1 - c_2)], b) & \text{if $id_1$ and $id_2$ map to constants} \\ \text{Error} & \text{otherwise} \end{cases}\]
\subsubsection*{Assignment to variable on heap}
If we now have $(p, H, b^*, E, (id = [id_1 + c]) \cdot b)$, where $id$ is being assigned to another variable stored in the heap, then,
\[\begin{cases} (p, H, b^*, E \cdot [id \rightarrow ((H(a_1))(c_1 + c))], b) & \text{if E($id_1$) = $(a_1, c_1)$ and $(c_1 + c) \in H$} \\ \text{Error} & \text{otherwise}\end{cases}\]
\begin{itemize}
    \item Both variable checks for $id_1$ being an integer, and a range check of location $[id_1 + c]$ being in the heap must be passed to not result in an error
    \item $H(a_1)$ is a tuple
\end{itemize}
\subsubsection*{Assignment of location on heap to identifier}
If we are assigning a location in the heap to some identifier, so $(p, H, b^*, E, ([id_1 + c]) \cdot b)$, then the next step would be
\[\begin{cases} (p, H \cdot [a_1 \rightarrow t], b^*, E, b) & \text{if $E(id_1) = (a_1, c_1)$ and $(c_1 + c) \in \text{dom}(H(a_1))$ and $t = H(a_1) \cdot [(c_1 + c) \rightarrow E[id]]$} \\ \text{Error} & \text{otherwise}\end{cases}\]
\begin{itemize}
    \item Range check still must be done
    \item $H(a_1)$ is a tuple
    \item Notice that this time, the heap changes instead of the environment
\end{itemize}
\subsubsection*{Function calls}
If we have a function call: $(p, H, b^*, E, (id = call id_0(id_1)) \cdot b)$, many things change.\\
First, we need to perform checks for the following:
\begin{itemize}
    \item $E[id_0] = f$
    \item $p$ contains the function $f (id'_1 \dots id'_f)$
\end{itemize}   
If the checks pass, we get:
\[(p, H', b^*, E \cdot [id \rightarrow E'(id')], b)\]
where $E'(id')$ is the result of executing from the following program state:
\[(p, H', b', E', b' \cdots \texttt{return } id')\]
\begin{itemize}
    \item Since a change in control occurs, the heap can be a completely new heap on the next instruction.
    \item However, on the \texttt{return} statement, the program must return to the original heap.  As a result, an identifier in the environment is assigned to the \textbf{return value} of the function, if there is one.
    \item Note that the $H'$ carries over into the original program state, since the other function can also modify the same heap.
\end{itemize}


\end{document}