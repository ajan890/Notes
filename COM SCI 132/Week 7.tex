% document formatting
\documentclass[10pt]{article}
\usepackage[utf8]{inputenc}
\usepackage[left=1in,right=1in,top=1in,bottom=1in]{geometry}
\usepackage[T1]{fontenc}
\usepackage{xcolor}

% math symbols, etc.
\usepackage{amsmath, amsfonts, amssymb, amsthm}

% lists
\usepackage{enumerate}

% images
\usepackage{graphicx} % for images

% code blocks
\usepackage{minted, listings} 
\lstset{
  basicstyle=\ttfamily,
  mathescape
}
% verbatim greek
\usepackage{alphabeta}

\graphicspath{{./assets/images}}

\newcommand{\solution}{\textbf{Solution:}} 

\title{COM SCI 132 Week 7}

\author{Aidan Jan}
\date{\today}

\begin{document}
\maketitle

\section*{LR Parsing}
\subsection*{Review}
Recall that
\begin{itemize}
    \item For a grammar $G$, with start symbol $S$, any string $\alpha$ such that $S \Rightarrow^* \alpha$ is called a \textit{sentential form}
    \item If $\alpha \in V_t^*$, then $\alpha$ is called a \textit{sentence} in $L(G)$
    \item Otherwise it is just a sentential form (not a sentence in $L(G)$)
    \item A \textit{left-sentential form} is a sentential form that occurs in the leftmost derivation of some sentence.
    \item A \textit{right-sentential form} is a sentential form that occurs in the rightmost derivation of some sentence.
\end{itemize}

\subsection*{Bottom-up parsing}
The goal: Given an input string $w$ and a grammar $G$, construct a parse tree by starting at the leaves and working to the root.\\
The parser repeatedly matches a \textit{right sentential} form from the language against the tree's upper frontier.  At each match, it applies a \textit{reduction} to build on the frontier:
\begin{itemize}
    \item each reduction matches an upper frontier of the partially built tree to the RHS of some production
    \item each reduction adds a node on top of the frontier
\end{itemize}
The final result is a rightmost derivation, in reverse.

\subsection*{Example}
Consider the grammar:
\begin{verbatim}
    1 | S -> aABe
    2 | A -> Abc
    3 |   |  b
    4 | B -> d
\end{verbatim}
and the input string \texttt{abbcde}.
\begin{center}
\begin{tabular}{c|c}
    Prod'n & Sentential Form \\
    \hline
    3 & \texttt{a$\boxed{\texttt{b}}$bcde} \\
    2 & \texttt{a$\boxed{\texttt{Abc}}$de} \\
    4 & \texttt{aA$\boxed{\texttt{d}}$e} \\
    1 & $\boxed{\texttt{aABe}}$ \\
    - & \texttt{S}
\end{tabular}
\end{center}
The trick appears to be scanning the input and finding valid sentential forms.

\section*{Handles}
What are we trying to find?
\begin{itemize}
    \item A substring $\alpha$ of the tree's upper frontier that matches some production $A \rightarrow \alpha$ where reducing $\alpha$ to $A$ is one step in the reverse of a rightmost derivation.
\end{itemize}
We call such a string a \textit{handle}.  Formally:
\begin{itemize}
    \item a \textit{handle} of a right-sentential form $\gamma$ is a production $A \rightarrow \beta$ and a position in $\gamma$ where $\beta$ may be found and replaced by $A$ to produce the previous right-sentential form in a rightmost derivation of $\gamma$.
    \item (i.e., if $S \Rightarrow_{rm}^* \alpha Aw \Rightarrow_{rm} \alpha \beta w$, then $A \rightarrow \beta$ in the position following $\alpha$ is a handle of $\alpha \beta w$)
\end{itemize}
Because $\gamma$ is a right-sentential form, the substring to the right of a handle contains only terminal symbols.
\begin{center}
    \includegraphics*[scale=1]{W7_1.png}
\end{center}

\subsection*{Theorem:}
\begin{itemize}
    \item If $G$ is unambiguous then every right-sentential form has a unique handle.
\end{itemize}
\textit{Proof: (by definition)}
\begin{enumerate}
    \item $G$ is unambiguous $\Rightarrow$ rightmost derivation is unique
    \item $\Rightarrow$ a unique production $A \rightarrow \beta$ applied to take $\gamma_{i - 1}$ to $\gamma_1$
    \item $\Rightarrow$ a unique position $k$ at which $A \rightarrow \beta$ is applied
    \item $\Rightarrow$ a unique handle $A \rightarrow \beta$
\end{enumerate} 

\subsubsection*{Example}
\begin{center}
    \includegraphics*[scale=0.9]{W7_2.png}
\end{center}

\subsection*{Handle-pruning}
The process to construct a bottom-up parse is called \textit{handle-pruning}.\\
To construct a rightmost derivation

\[S = \gamma_0 \Rightarrow \gamma_1 \Rightarrow \gamma_2 \Rightarrow \cdots \Rightarrow \gamma_{n - 1} \Rightarrow \gamma_n = w\]
we set $i$ to $n$ and apply the following simple algorithm:
\begin{flalign*}
    &\texttt{for i = n downto 1}&&\\
    &\texttt{~~~~find the handle $A_i \rightarrow \beta_i$ in $\gamma_i$}&&\\
    &\texttt{~~~~replace $\beta_i$ with $A_i$ to generate $\gamma_{i - 1}$}&& 
\end{flalign*}
This takes $2n$ steps, where $n$ is the length of the derivation

\subsection*{Stack Implementation}
One scheme to implement a handle-pruning, bottom-up parser is called a \textit{shift-reduce} parser.  Shift-reduce parsers use a \textit{stack} and an \textit{input buffer}
\begin{enumerate}
    \item initialize stack with \$
    \item Repeat until the top of the stack is the goal symbol and the input token is \$
    \begin{enumerate}
        \item \textit{find the handle}  
        \begin{itemize}
            \item if we don't have a handle on top of the stack, \textit{shift} an input symbol onto the stack
        \end{itemize}   
        \item \textit{prune the handle}\\
            if we have a handle $A \rightarrow \beta$ on the stack, \textit{reduce}.
            \begin{enumerate}
                \item pop $\vert \beta \vert$ symbols off the stack
                \item push $A$ onto the stack
            \end{enumerate}
    \end{enumerate}
\end{enumerate}

\subsubsection*{Example}
\begin{center}
    \includegraphics*[scale=1]{W7_3.png}
\end{center}
\begin{enumerate}
    \item Shift until top of stack is the right end of a handle
    \item Find the left end of the handle and reduce
\end{enumerate}
For this example: 5 shifts + 9 reduces + 1 accept

\section*{Shift-reduce parsing}
\textit{Shift-reduce parsers are simple to understand}\\
A shift-reduce parser has just four canonical actions:
\begin{enumerate}
    \item \textit{shift} - next input symbol is shifted onto the top of the stack
    \item \textit{reduce} - right end of handle is on top of the stack; locate left end of handle within the stack; pop handle off stack and push appropriate non-terminal LHS
    \item \textit{accept} - terminate parsing and signal success
    \item \textit{error} - call an error recovery routine
\end{enumerate}
The key problem: to recognize handles (not covered in this course).

\subsection*{LR(k) Grammars}
Informally, we say that a grammar $G$ is LR($k$) if, given a rightmost derivation
\[S = \gamma_0 \Rightarrow \gamma_1 \Rightarrow \gamma_2 \Rightarrow \cdots \Rightarrow \gamma_n = w\]
we can, for each right-sequential form in the derivation,
\begin{enumerate}
    \item isolate the handle of each right-sequential form, and
    \item determine the production by which to reduce
\end{enumerate}
by scanning $\gamma_i$ from left to right, going at most $k$ symbols beyond the right end of the handle of $\gamma_i$.\\
Formally, a grammar $G$ is LR($k$) if and only if:
\begin{enumerate}
    \item $S \Rightarrow_{rm}^* \alpha Aw \Rightarrow_{rm} \alpha \beta w$, and
    \item $S \Rightarrow^*_{rm} \gamma Bx \Rightarrow_{rm} \alpha \beta y$, and
    \item \texttt{FIRST}$_k$(w) = \texttt{FIRST}$_k$(y) $\Rightarrow \alpha Ay = \gamma Bx$ 
\end{enumerate}
i.e., Assume sentential forms $\alpha \beta w$ and $\alpha \beta y$ with common prefix $\alpha \beta$ and common k-symbol lookahead \texttt{FIRST}$_k$(y) = \texttt{FIRST}$_k$(w), such that $\alpha \beta w$ reduces to $\alpha Aw$ and $\alpha \beta y$ reduces to $\gamma Bx$.\\
But, the common prefix means $\alpha \beta y$ also reduces to $\alpha A y$, for the same result.\\
Thus $\alpha Ay = \gamma Bx$

\subsection*{Why Study LR Grammars?}
LR(1) grammars are often used to construct parsers.  We call these parsers LR(1) parsers.
\begin{itemize}
    \item everyone's favorite parser
    \item virtually all context-free programming language constructs can be expressed in an LR(1) form
    \item LR grammars are the most general grammars parsable by a deterministic, bottom-up parser
    \item efficient parsers can be implemented for LR(1) grammars
    \item LR parsers detect an error as soon as possible in a left-to-right scan of the input
    \item LR grammars describe a proper superset of the languages recognized by predictive (i.e., LL) parsers
    \begin{itemize}
        \item LL(k): recognize use of a production $A \rightarrow \beta$ seeing first $k$ symbols of $\beta$
        \item LR(k): recognize occurrence of $\beta$ (the handle) having seen all of what is derived from $\beta$ plus $k$ symbols of lookahead.
    \end{itemize}
\end{itemize}

\subsection*{Left Versus Right Recursion}
Right Recursion:
\begin{itemize}
    \item needed for termination in predictive parsers
    \item requires more stack space
    \item right associative operators
\end{itemize}
Left recursion:
\begin{itemize}
    \item works fine in bottom-up parsers
    \item limits required stack space
    \item left associative operatives
\end{itemize}
Rule of thumb:
\begin{itemize}
    \item right recursion for top-down parsers
    \item left recursion for bottom-up parsers
\end{itemize}

\subsection*{Parsing Review}
\begin{itemize}
    \item R. descent: A hand coded recursive descent parser directly encodes a grammar (typically an LL(1) grammar) into a series of mutually recursive procedures.  It has most of the linguistic limitations of LL(1).
    \item LL(k): An LL($k$) parser must be able to recognize the use of a production after seeing only the first $k$ symbols of its right hand side.
    \item LR(k): An LR($k$) parser must be able to recognize the occurrence of the right hand side of a production after having seen all that is derived from that right hand side with $k$ symbols of lookahead.
    \item Dilemmas:
    \begin{itemize}
        \item LL dilemma: pick $A \rightarrow b$ or $A \rightarrow c$?
        \item LR dilemma: pick $A \rightarrow b$ or $B \rightarrow b$?
    \end{itemize}
\end{itemize}

\section*{Type Checking Generices}
A \textit{generic} is a type that has to be parameterized.  For example, the compiler needs to be told which object is stored in a given List.

\begin{minted}{java}
    class List<A> extends Object {
        public <R> R accept(ListVisitor<R, A> f) {
            // ... do stuff
            return this.accept(f);
        }
    }

    class ListNull<A> extends List<A> {
        public<R> R accept(ListVisitor<R, A> f) {
            return f.visit(this)
        }
    }

\end{minted}
The above program above will in fact, type check.
\subsection*{Example}
Grammar:
\begin{flalign*}
    &(Class) \texttt{ L} ::= \texttt{class C<$\bar x$> extends N \{$\bar s \bar f$; K $\bar M$\}}&&\\
    &(Constructor) \texttt{ K} ::= \texttt{C($\bar T \bar f$) \{super(f); this.$\bar f$ = $\bar f$\}}&&\\
    &(Method) \texttt{ M} ::= \texttt{<$\bar Y$> U m($\bar U \bar x$) \{ return e; \}}&&\\
    &(Expression) \texttt{ e} ::= \texttt{x | e.f | e.<$\bar v$> m($\bar e$) | new N($\bar e$)}&&\\
    &(Type) \texttt{ S, T, U, V} ::= \texttt{X | N}&&\\
    &\hspace{1cm}\texttt{ N} ::= \texttt{C<$\bar T$>}&&
\end{flalign*}
Type Rules:
\[\Gamma + e \::\: T\]
where 
\begin{itemize}
    \item $\Gamma$ represents the type environment
    \item $e$ represents an expression
    \item $T$ represents a type
\end{itemize}
\begin{flalign}
&\frac{\Gamma \vdash e \::\: T \hspace{1cm} \texttt{fields}(T) = \bar T \bar f}{\Gamma \vdash e.f_i \::\: T_i}&&\\
&\frac{\texttt{fields}(N) = \bar T \bar f \hspace{1cm} \Gamma + \bar e \::\: \bar S \hspace{1cm} \bar S \leq \bar T}{\Gamma \vdash \texttt{new } N(\bar e) \::\: N}&&\\
&\frac{\Gamma + e_0 \::\: T_0 \hspace{1cm}\texttt{mtype}(m, T_0) = <\bar Y> \bar U \rightarrow U \hspace{1cm} \bar S \leq [\bar V / \bar Y] \bar U}{\Gamma \vdash e_0.<\bar V> m(\bar e) \::\: [\bar V / \bar Y] U}&&
\end{flalign}
\begin{itemize}
    \item \texttt{mtype} refers to method type
    \item $<\bar Y> \bar U$ are the types of formal parameters and formal parameters, respectively.
    \item Type expression (3) is created via substitution with (1) and (2).
    \item $\bar S \leq \bar T$ is to check class extends; S and T are in a class-superclass relationship.
    \item Substitution Notation: $[\bar V / \bar X]$ means the same thing as $[\bar X := \bar V]$
    \item \texttt{mtype} is a placeholder function for method type checking
\end{itemize}
At this point, we have type checked methods and classes.  However, we still have to do the following:
\begin{itemize}
    \item Override methods
    \item Fields
    \item Subtyping
    \item \texttt{mtype}
\end{itemize}
\subsection*{Subtyping}
Recall that our types:
\begin{flalign*}
    &(Type) \texttt{ S, T, U, V} ::= \texttt{X | N}&&\\
    &\hspace{1cm}\texttt{ N} ::= \texttt{C<$\bar T$>}&&
\end{flalign*}
Subtyping rules:
\[T \leq T\]
(one type is a subtype of another type)

\begin{flalign}
&\frac{S \leq T \hspace{1cm} T \leq U}{S \leq U}&&\\
&\frac{\texttt{class } C<\bar x> \texttt{extends } N \{\dots\}}{C<\bar T> \leq [\bar T / \bar X]N}&&
\end{flalign}

\begin{itemize}
    \item (5) states that if $C$ is a use of $N$, as in, $N$ is a parent class to $C$, then a use of $T$ in $C$ (as a parameter of $C$) is a subtype of the use of $T$ in $N$.
    \begin{itemize}
        \item In simpler terms, $C$ parameterized with $T$ would be a subtype of $N$ parameterized with $T$, because $C$ is a subtype of $N$.
    \end{itemize}
\end{itemize}

\subsection*{Fields}
\begin{flalign}
&\frac{\texttt{class } C<\bar x> \texttt{extends } N \{\bar S \bar f; \dots\} \hspace{1cm} \texttt{fields }([\bar T / \bar x]N) = \bar V \bar g}{\texttt{fields}(C<\bar T>) = \bar U \bar g, [\bar T / \bar X]\bar f}&&
\end{flalign}

\begin{itemize}
    \item $\bar S \bar f$ is one parameter, where $S$ is the type, $f$ is the field.
\end{itemize}   

\subsection*{\texttt{mtype} (Method Typing)}

\begin{flalign}
&\frac{\texttt{class } C<\bar X> \texttt{ extends } N \{\dots \bar M\} \hspace{1cm} (<\bar Y> \: U \: m (\bar U \: \bar x) \{\texttt{return }e\}) \in \bar M}{\texttt{mtype} (m, C<\bar T>) = [\bar T / \bar X](<\bar Y> \bar U \rightarrow U)}&&\\
&\frac{\texttt{class } C<\bar x> \texttt{ extends } N \{\dots \bar M\} \hspace{1cm} m \notin \bar M}{\texttt{mtype }(m, C<\bar T>) = \texttt{mtype }(m, [\bar T / \bar X] \bar N)}&&
\end{flalign}

\begin{itemize}
    \item $\bar M$ are the methods of the class $C$.
\end{itemize}

\subsection*{Method Overriding}

\begin{flalign}
&\frac{\texttt{If \{mtype}(m, N) = <\bar z> \bar U \rightarrow U \texttt{\} Then \{} \bar T = [Y / \bar Z] \bar U \hspace{1cm} T = [\bar Y / \bar Z]U\}}{\texttt{override}(m, N, <\bar Y> \bar T \rightarrow T)}&&
\end{flalign}
In the override statement\dots
\begin{itemize}
    \item $m$ = method name
    \item $N$ = superclass of the current class
    \item $<\bar Y> \bar T \rightarrow \bar T$ = method header
\end{itemize}


\end{document}