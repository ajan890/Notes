% document formatting
\documentclass[10pt]{article}
\usepackage[utf8]{inputenc}
\usepackage[left=1in,right=1in,top=1in,bottom=1in]{geometry}
\usepackage[T1]{fontenc}
\usepackage{xcolor}

% math symbols, etc.
\usepackage{amsmath, amsfonts, amssymb, amsthm}

% lists
\usepackage{enumerate}

% images
\usepackage{graphicx} % for images

% code blocks
\usepackage{minted, listings} 
\lstset{
  basicstyle=\ttfamily,
  mathescape
}
% verbatim greek
\usepackage{alphabeta}

\graphicspath{{./assets/images}}

\newcommand{\solution}{\textbf{Solution:}} 

\title{COM SCI 132 Week 6}

\author{Aidan Jan}
\date{\today}

\begin{document}
\section*{Register Allocation}
Register Allocation can be split into two phases:
\begin{itemize}
    \item Liveness analysis
    \item Graph coloring
\end{itemize}
Last week it was proved that liveness analysis can be done in polynomial time.  But what about graph coloring?

\section*{Graph Coloring}
\textbf{The problem statement:}\\
Given a Graph $G = (V, E)$, where $V$ represents the set of vertices and $E$ represents the set of edges, such that $E \subseteq V \times V$, the goal is to give a color to each vertex such that neighbors have different colors.
\begin{itemize}
    \item From an algorithms class we know that this problem is \textbf{NP-complete}.  In other words, unsolvable in polynomial time.
\end{itemize}

\subsection*{Approximate Graph Coloring}
There exists an algorithm called \textit{Linear Scan Register Allocation} that does a good job of graph coloring in linear time in the size of graph.\\\\
Note: It does a \textit{good} job, not an \textit{optimal} job.  As in, this algorithm returns a suboptimal solution in exchange for fast execution.

\subsection*{Example}
Let's start with a basic program with liveness analysis done:
\begin{verbatim}
                    | a  b  c |         
    a = 1           | *       |   (1)
    b = 2           | |  *    |   (2)
    c = a + 3       | O  |  * |   (3)
    print(b + c)    |    O  O |   (4)
\end{verbatim}
Linear Scan Register Allocation is a greedy algorithm!  There is no backtracking.
\begin{itemize}
    \item (1) Assign $r_1$ to $a$.
    \item (2) Assign the next free register, $r_2$ to $b$.
    \item (3) Deallocate $r_1$ since $a$ is no longer used, and assign the next free register, now $r_1$, to $c$.
    \item (4) After the print statement, deallocate $r_1$ and $r_2$ since they are no longer used.
\end{itemize}
The steps would be:
\begin{enumerate}
    \item Order the intervals after start points.
    \item Process the intervals from left to right.
    \item The only situation where we change our mind about an assignment is when we 'spill' a variable to memory.  (e.g., write it to the stack instead of register).
    \begin{itemize}
        \item Pick the live range that extends the furthest into the future (see next subsection.)
    \end{itemize}
\end{enumerate}

\pagebreak
\subsection*{Live Ranges Example}
Suppose we have 6 variables but only 4 registers.
\begin{verbatim}
                    | a  b  c  d  e  f |
    a = 1           | *                |  (1)
    b = 10          | |  *             |  (2)
    c = a + 9       | |  |  *          |  (3)
    d = a + c       | O  |  |  *       |  (4)
    e = c + d       |    |  |  O  *    |  (5)
    f = b + 8       |    |  |     |  * |  (6)
    c = f + e       |    |  |     |  | |  (7)
    f = e + c       |    |  |     O  | |  (8)
    b = c + 5       |    |  O        | |  (9)
    return b + f    |    O           O |  (10)
\end{verbatim}
\begin{itemize}
    \item (1) Assign register $r_1$ to $a$.
    \item (2) Assign register $r_2$ to $b$.
    \item (3) Assign register $r_3$ to $c$.
    \item (4) Assign register $r_1$ to $d$. (deallocate $a$)
    \item (5) Assign register $r_1$ to $e$. (deallocate $d$)
    \item (6) Assign register $r_4$ to $f$
\end{itemize}
Below is the interference graph for this system, where $r$, $s$, $t$, and $u$ correspond to $r_1$, $r_2$, $r_3$, and $r_4$.
\begin{center}
    \includegraphics*[scale=0.7]{W6_1.png}
\end{center}
What if in this case, we only have three registers?\\
Then, instead of allocating $r_2$ to $b$, we would store $b$ in memory.  The execution would look like:
\begin{itemize}
    \item (1) Assign register $r_1$ to $a$.
    \item (2) Spill $b$ to memory ($M$).
    \item (3) Assign $r_3$ to $c$.
    \item (4) Assign $r_1$ to $d$.
    \item (5) Assign $r_1$ to $e$.
    \item (6) Assign $r_2$ to $f$.
\end{itemize}
The interference graph would now look like:
\begin{center}
    \includegraphics*[scale=0.7]{W6_2.png}    
\end{center}
Similarly, if we only had two registers, we would send $f$ to memory too, since it is the variable that lasts the longest into the future (other than $b$, which is already in memory).\\
Note that when we are storing variables in memory, we \textbf{always} have to have two registers free to cycle values in and out of memory.

\section*{Aside - Sparrow}
Consider the expression \texttt{x = a + b}.  In Sparrow-V, we would need three lines:
\begin{verbatim}
    u = [b]
    u = s + u
    [x] = u
\end{verbatim}
Where \texttt{u} is a free register.  \texttt{[b]} and \texttt{[x]} are the variables.\\
Note that you cannot assign \texttt{[x]} to the expression \texttt{s + u}, because of the way machine languages work; you cannot assign to expressions.

\subsection*{Colors and Cliques}
A clique is defined as a group of variables that all share existence at the same time.  The \textit{maximal} clique is the clique with the most members.  For the example above, the maximal clique would be variables $b$, $c$, $e$, and $f$, as those four variables are alive at the same time.
A general rule is:
\[\text{The minimal number of colors} \geq \text{The size of the maximal clique}\]
This makes sense intuitively; we need at least $n$ registers to store $n$ coexisting variables.\\
When does this number go larger than the size of the maximal clique?  Consider the following example:
\begin{center}
    \includegraphics*[scale=0.5]{W6_3.png}
\end{center}
Here, the maximal clique is size 2, since based on the graph, each variable is only bordering two others.  However, we need three registers to color the graph.

\subsection*{Loops}
What if instead of having a straight line program, we have a control flow program?  (e.g., loops)

\begin{verbatim}
                            | a  b  c |       | c  a  b |
    (c from outer scope)    |       | |       | |       |
    a = 0                   | *     | |       | |  *    |
L1: b = a + 1               | |  *  | |       | |  |  * |
    c = c + b               | |  |  | |  -->  | |  |  | |
    a = b * 2               | |  O  | |       | |  |  O |
    if (a < 10) goto L1     | *     | |       | |  *    |
    return c                |       O |       | O       |
\end{verbatim}
On the right liveness analysis, we arrange them on order of appearance; in this case, if we only have two registers, then \texttt{C} can be sent to memory.  \texttt{a} and \texttt{b} can be assigned to registers $r_1$ and $r_2$.\\
Note that in this case, \texttt{a} may stay alive after the goto statement, so its entry in the liveness analysis has a closed circle (\texttt{*}) on both ends.  (e.g., it is inclusive on both sides).

\section*{Activation Records}
\subsection*{Procedure Abstraction}
Separate Compilation:
\begin{itemize}
    \item Allows us to build large programs
    \item Keeps compile times reasonable
    \item Requires independent procedures
\end{itemize}
The linkage convention:
\begin{itemize}
    \item a social contract: division of responsibility
    \item machine dependent
    \item ensures that every procedure inherits a valid run-time environment and that it restores one for its parent.
\end{itemize}
Code for linkage:
\begin{itemize}
    \item Generated at \textit{compile time}.
    \item Executes at \textit{run time}.
\end{itemize}

\subsection*{Caller and Callee}
The essentials:
\begin{itemize}
    \item \textit{on entry:} establish \texttt{p}'s environment
    \item \textit{at a call:} preserve \texttt{p}'s environment
    \item \textit{on exit:} tear down \texttt{p}'s environment
    \item \textit{in between:} addressability and proper lifetimes
\end{itemize}
Each system has a standard linkage.
\begin{center}
    \includegraphics*[scale=1]{W6_4.png}
\end{center}

\subsection*{An activation record = a stack frame}
Assume that each procedure activation has an associated \textit{activation record} or \textit{frame} (at run time).
Assumptions:
\begin{itemize}
    \item \texttt{RISC} architecture
    \item Can expand an allocated block
    \item Locals stored in frame
\end{itemize}
\begin{center}
    \includegraphics*[scale=1]{W6_5.png}
\end{center}

\subsubsection*{Procedure Linkages}
The linkage divides responsibility between \textit{caller} and \textit{callee}.
\begin{center}
    \includegraphics*[scale=0.8]{W6_6.png}
\end{center}

\subsection*{Calls: Saving and restoring registers}
\begin{tabular}{|c|c|c|c|}
    \hline
    & caller's registers & callee's registers & all registers \\
    \hline
    callee saves & 1 & 3 & 5\\
    caller saves & 2 & 4 & 6\\
    \hline
\end{tabular}
\begin{enumerate}
    \item Hard: call includes bitmap of caller's registers.
    \item Easy: the caller saves and restores its own registers.
    \item Easy: the callee saves and restores its own registers.
    \item Hard: the caller uses a bitmap in callee's stack frame to save and restore.
    \item Easy: the callee saves and restores all registers.
    \item Easy: the caller saves and restores all registers.
\end{enumerate}

\section*{RISC-V Registers}
\begin{tabular}{|c|c||c|c|}
\hline
Register & Name & Saver & Description \\
\hline
\texttt{x0} & \texttt{zero} & - & Hard-wired zero \\
\texttt{x1} & \texttt{ra} & Caller & Return address \\
\texttt{x2} & \texttt{sp} & Callee & Stack pointer \\
\texttt{x3} & \texttt{gp} & - & Global pointer \\
\texttt{x4} & \texttt{tp} & - & Thread pointer \\
\texttt{x5-7} & \texttt{t0-2} & Caller & Temporaries \\
\texttt{x8} & \texttt{s0/fp} & Callee & Saved register / frame pointer \\
\texttt{x9} & \texttt{s1} & Callee & Saved register \\
\texttt{x10-11} & \texttt{a0-1} & Caller & Function arguments / return values \\
\texttt{x12-17} & \texttt{a2-7} & Caller & Function arguments \\
\texttt{x18-27} & \texttt{s2-11} & Callee & Saved registers \\
\texttt{x28-31} & \texttt{t3-6} & Caller & Temporaries \\
\hline
\end{tabular}\\\\
\textbf{Philosophy:} Use a general calling sequence only when necessary; omit portions of it where possible.








\end{document}