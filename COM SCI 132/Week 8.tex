% document formatting
\documentclass[10pt]{article}
\usepackage[utf8]{inputenc}
\usepackage[left=1in,right=1in,top=1in,bottom=1in]{geometry}
\usepackage[T1]{fontenc}
\usepackage{xcolor}

% math symbols, etc.
\usepackage{amsmath, amsfonts, amssymb, amsthm}

% lists
\usepackage{enumerate}

% images
\usepackage{graphicx} % for images

% code blocks
\usepackage{minted, listings} 
\lstset{
  basicstyle=\ttfamily,
  mathescape
}
% verbatim greek
\usepackage{alphabeta}

\graphicspath{{./assets/images}}

\newcommand{\solution}{\textbf{Solution:}} 

\title{COM SCI 132 Week 8}

\author{Aidan Jan}
\date{\today}

\begin{document}
\maketitle

\section*{How to Compile Lambda Expressions}
In summary: we want to convert recursion to iteration.
\begin{center}
    Lambda Expressions $\rightarrow$ Tail Form $\rightarrow$ First-Order Form $\rightarrow$ Imperative Form\\
    \begin{tabular}{c|c|c}
        \textbf{Form} & \textbf{Item} & \textbf{Approach}\\
        \hline
        Tail & functions never return & continuations\\
        First-order & functions are all top level & data structures\\
        Imperative & functions take no arguments & register allocation
    \end{tabular}
\end{center}

\subsection*{Recursion vs. Iteration}
\textbf{Recursion:}
\begin{verbatim}
    static Function<Integer, Integer> // in class Test
    fact = n -> n == 0 ? 1 : n * Test.fact.apply(n - 1);
\end{verbatim}
\textbf{Iteration:}
\begin{verbatim}
    static int factIter(int n) {
        int a = 1;
        while (n != 0) { a = n * a; n--; }
        return a;
    }
\end{verbatim}
We want to convert the recursion to iteration.

\subsection{Recursion to Tail Form}
\textbf{Recursion:}
\begin{verbatim}
    static Function<Integer, Integer> // in class Test
    fact = n -> n == 0 ? 1 : n * Test.fact.apply(n - 1);
\end{verbatim}
\textbf{Tail Form:}
Uses continuation passing style (CPS).
\begin{verbatim}
    static BiFunction<Integer, Function<Integer, Integer>, Integer>
    factCPS = (n, k) -> n == 0 ? k.apply(1) : Test.factCPS.apply(n - 1, v -> k.apply(n * v));
\end{verbatim}
To call a function that is written in CPS, use \texttt{.apply()}.
\begin{verbatim}
        factCPS.apply(4, v1 -> v1)
    =   factCPS.apply(3, v2 -> (v1 -> v1)
                               .apply(4 * v2))
    =   factCPS.apply(2, v3 -> (v2 -> (v1 -> v1)
                                      .apply(4 * v2))
                               .apply(3 * v3))
    =   factCPS.apply(1, v4 -> (v3 -> (v2 -> (v1 -> v1)
                                             .apply(4 * v2))
                                      .apply(3 * v3))
                               .apply(2 * v4))
    =   factCPS.apply(0, v5 -> (v4 -> (v3 -> (v2 -> (v1 -> v1)
                                                    .apply(4 * v2))
                                             .apply(3 * v3))
                                      .apply(2 * v4))
                               .apply(1 * v5))                         
\end{verbatim}
If we execute this\dots
\begin{verbatim}
    factCPS.apply(0, v5 -> (v4 -> (v3 -> (v2 -> (v1 -> v1)
                                                .apply(4 * v2))
                                         .apply(3 * v3))
                                  .apply(2 * v4))
                           .apply(1 * v5))  
    = factCPS.apply(1, (v4 -> (v3 -> (v2 -> (v1 -> v1)
                                                  .apply(4 * v2))
                                           .apply(3 * v3))
                                    .apply(2 * v4))
                              .apply(1))  
    = factCPS.apply(2, (v3 -> (v2 -> (v1 -> v1)
                                        .apply(4 * v2))
                                 .apply(3 * v3))
                          .apply(2))
    = factCPS.apply(6 -> (v2 -> (v1 -> v1)
                                .apply(4 * v2))
                         .apply(6))
    = factCPS.apply(24 -> (v1 -> v1)
                          .apply(24))
    = factCPS.apply(24)
    = 24                              
\end{verbatim}
\subsection*{Tail Form Grammar}
The grammar is as follows:
\begin{align*}
TailForm &::= Simple\\
&~~| ~~~Simple\texttt{.apply}(Simple_1, \dots, Simple_n)\\
&~~| ~~~Simple \texttt{ ?} TailForm \::\: TailForm\\
\\
Simple &::= Identifier\\
&~~| ~~~Constant\\
&~~| ~~~Simple \: PrimitiveOperation \: Simple\\
&~~| ~~~Identifier \texttt{ -> } TailForm
\end{align*}
\begin{itemize}
    \item Evaluation of a Tail Form expression ($TailForm$) has \textbf{one} call which is the last operation.
    \item Evaluation of a Simple expression ($Simple$) has \textbf{no} calls.
\end{itemize}

\end{document}