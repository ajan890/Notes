% document formatting
\documentclass[10pt]{article}
\usepackage[utf8]{inputenc}
\usepackage[left=1in,right=1in,top=1in,bottom=1in]{geometry}
\usepackage[T1]{fontenc}
\usepackage{xcolor}

% math symbols, etc.
\usepackage{amsmath, amsfonts, amssymb, amsthm}

% lists
\usepackage{enumerate}

% images
\usepackage{graphicx} % for images

% code blocks
\usepackage{minted, listings} 

\graphicspath{{./assets/images}}

\newcommand{\solution}{\textbf{Solution:}} 

\title{CHEM 20B Week 1}

\author{Aidan Jan}
\date{\today}

\begin{document}
\maketitle
\section*{Gases}
\begin{itemize}
    \item Eleven elements are gases under normal conditions
    \item Lower \textit{molar mass} compounds such as carbon dioxide, hydrogen chloride
    \item A remarkable characteristics of gases is that many of their physical properties are very similar, particularly at low pressures, regardless of the identity of the gas.
\end{itemize}   

\subsection*{Characteristics of Gases}
\begin{enumerate}
    \item Compressibility: Gases are more compressible than solids and liquids; suggests that there is a lot of space between the molecules of gases
    \item Motion: Gas expands rapidly to fill the space available to it.  Because balloons are spherical, we can infer that the motion of the molecules is chaotic, not favoring any single direction.
\end{enumerate}

\subsection*{Units}
\begin{itemize}
    \item There are two types of energy
    \begin{enumerate}
        \item \textbf{Kinetic Energy}: relating to how fast each atom moves.
            \[\text{K.E.} = \frac{1}{2}mv^2\]
        \item \textbf{Potential Energy}: energy that can turn into Kinetic Energy
    \end{enumerate}
\end{itemize}

\subsection*{Force}
\begin{itemize}
    \item $F$ = Force = Energy/distance
    \item Force is measured in $J/m = \text{kg} \cdot \frac{m}{s^2}$, or Newtons (N).
\end{itemize}

\subsection*{Pressure}
\begin{itemize}
    \item Pressure is $\frac{\text{force}}{\text{area}}$, or $P = \frac{F}{A}$.
    \item Colliding gases exert a pressure on the sides of the container walls.
    \item The more vigorous the motion, the stronger the force and hence the higher the pressure.
    \item Pressure is measured in atm, bars, mmHg, and Pascals (Energy/Volume).
    \begin{itemize}
        \item Pascal ($Pa$) = $\frac{J}{m^3} = \frac{N}{m^2}$
        \item 1 atm = $1.01325 \times 10^5$ Pa
        \item 1 bar = $1 \times 10^5$ Pa
        \item 1 atm = 760 torr (any temperature)
        \item 1 atm = 760 mm Hg (at $0^\circ$C)
    \end{itemize}
\end{itemize}

\section*{PV=nRT}
\begin{itemize}
    \item P = pressure
    \item V = volume
    \item T = temperature
    \item n = number of moles
    \item R = "gas constant" = $\frac{8.3 J}{K \times mol}$
\end{itemize}
\subsection*{Pressure and Boyle's Law}
\begin{itemize}
    \item T fixed.
    \item The more mercury he added, the more the trapped air was compressed
    \item PV will be proportional to the amount of material
    \item $P \cdot V$ is unchanged if the amount of material is unchanged, and T is fixed.
\end{itemize}

\subsection*{Moles and Avogadro's Law}
\begin{itemize}
    \item For the same T and P, the ratios of the volumes between 2 different gases = ratio of the numbers of moles
    \item H$_2$O $\rightarrow$ H$_2$ + $1/2$O$_2$
    \item Under fixed P and T, $V_{H_2} = 2 \cdot V_{O_2}$ by experimental observation.
    \begin{itemize}
        \item This implies that 
            \begin{align*}
                \frac{PV_{H_2}}{n_{H_2}} &= \frac{PV_{O_2}}{n_{O_2}}\\
                PV &= nf(T)
            \end{align*}
            where $f$(T) is universal.
    \end{itemize}
\end{itemize}
\subsubsection*{Defining the Temperature in Kelvin}
\begin{itemize}
\item Fahrenheit and Celsius scale completely arbitrary designations
\item We need a temperature that is based on a physical law.
\item The simplest way to do that is to use the gas law.
    \[\frac{PV}{n} = RT\]
\end{itemize}
\subsection*{Temperature and Charles' Law}
Charles observed that all gases expand by the same relative amount between the same initial and final temperature under sufficient low pressure.
\[t = C\left(\frac{V}{V_0} - 1\right)\]
\begin{itemize}
    \item $V_0$ is the volume of the freezing point of water
    \item $C$ is a constant that is the same for all gases.
\end{itemize}
Experimentally: put a container (with a movable plug) with air in boiling water, at P = 1 bar, and find that V is, say, 50.000 L.  When the same container is then put into ice water, we'll find that V = 36.601 L.
\begin{align*}
\frac{PV \text{(at $T_{boil}$)}}{PV \text{(at $T_{freeze}$)}} = \frac{1\text{bar} \cdot 50.000 \text{L}}{1 \text{bar} \cdot 36.601 \text{L}} &= 1.36609\\
\frac{nRT_{\text{boil}}}{nRT_{\text{freeze}}} = \frac{T_{\text{boil}}}{T_{\text{freeze}}} &= 1.36609\\
T_{\text{freeze}} &= \frac{100K}{0.36609}\\
\end{align*}
\[\therefore T_{\text{freeze}} = 273.16\text{K}\sim273\text{K}, T_{\text{boil}} = 373.16\text{K}\]
In the same experiment, suppose $n = 1.61$ mol.

\[R = \frac{PV}{nT} = \frac{1 \text{bar} \cdot 50.00 \text{L}}{1.61 \text{mol} \cdot 373.16 \text{K}} = \frac{0.0831 \text{bar} \cdot \text{L}}{\text{mol} \cdot \text{K}}\]

Simplifying, 
\[R = 8.3 \frac{\text{J}}{\text{K} \cdot \text{mol}}\]

\begin{itemize}
    \item For a fixed amount of gas under constant pressure, volume is directly proportional to temperature.
    \[V = V_0 (1 + \frac{t}{273.15^\circ \text{C}})\]
    \item T (Kelvin) = t ($^\circ$C) + 273.15 (fixed pressure and fixed amount of gas)
\end{itemize}

\section*{The Ideal Gas Law}
Combining all of the following:
\begin{itemize}
    \item Boyle's Law:
    \[V \propto \frac{1}{p} \text{  (at constant temperature, fixed amount of gas)}\]
    \item Charles' Law:
    \[V \propto T \text{  (at constant pressure, fixed amount of gas)}\]
    \item Avogadro's Law:
    \[V \propto n \text{  (at constant temperature and pressure)}\]
\end{itemize}
We get...
\[\boxed{PV = nRT}\]
R = 8.3145 J mol$^{-1}$ K$^{-1}$

\pagebreak
\section*{Kinetic Theory of Gases}
Assumptions: 
\begin{enumerate}
    \item A gas consists of a collection of molecules in ceaseless random motion.
    \item Gas molecules are infinitesimally small points.
    \item The molecules move in straight lines until they collide.
    \item The molecules do not influence one another except during collisions.
    \item The collisions are elastic.  No energy is lost during a collision.
\end{enumerate}
\[\text{Root-mean-square speed }v_{rms} = \sqrt{\bar v^2} = \sqrt{\frac{3RT}{M}}\]
\[T = \frac{Mv^2_{rms}}{3R}\]

\section*{Physical Derivation of the Ideal Gas Law}
Gases always expand to fill all of the space they are given.  As a result, the pressure is the force exerted from gas particles hitting the walls of the container.\\\\
Suppose that all the gas particles in a rectangular chamber (with side lengths $l_x$, $l_y$, and $l_z$) are identical, and have mass $m$.  The pressure exerted on the walls is 
\[P = \frac{F}{A}\]
where $P$ represents pressure, $F$ is the force exerted on the walls by particles, and $A$ is the surface area of the wall.\\\\
For one gas molecule,
\[F = ma = m \cdot \frac{\Delta v}{\Delta t} = \frac{\Delta(mv)}{\Delta t} = \frac{\Delta p}{\Delta t}\]
Recall, $\Delta p$ is the change in momentum of a particle.\\
The change in momentum can be calculated by 
\[\Delta p = p_{\text{final}} - p_{\text{initial}}\]
A particle heading to a wall with momentum $-mv_x$ would bounce off with momentum $mv_x$.  Therefore, the absolute change in momentum, $\Delta p = 2 \cdot mv_x$.\\\\
By definition of velocity,
\[v = \frac{d}{t}\]
where $v$ represents velocity, $d$ represents distance, and $t$ represents time.\\\\
Thus, if the particle travels from one wall of the rectangular box to the other, bounces off, and returns to its starting point, then $d = 2 \cdot l_x$ and $v = v_x$.  Solving for time, $t = 2 \cdot \frac{l_x}{v_x}$.
\[\therefore F = \frac{\Delta p}{\Delta t} = \frac{2 \cdot mv_x}{2 \cdot \frac{l_x}{v_x}} = \frac{mv_x^2}{l_x}\]
This is the force for one particle.  Since all of the particles have the same mass, then the total force would be 
\[F_{\text{total}} = F_1 + F_2 + F_3 + \cdots + F_n\]
where $F_i$ represents the contribution of force from the $i$-th particle.\\\\
There are $N$ gas particles in the container.  Therefore, the average square of x-component of velocity for all the particles will be:
\[v_x^2 = \frac{v_{x_1}^2 + v_{x_2}^2 + \cdots + v_{x_N}^2}{N}\]
The total force is then:
\[F = \frac{m}{l_x} \cdot N \cdot \bar v_x^2\]
Since there are three dimensions to the cubic container, and the particles have random direction, it can be assumed that for an average particle, $\bar v_x = \bar v_y = \bar v_z$.  Therefore, the mean square speed $\bar v^2 = \bar v_x^2 + \bar v_y^2 + \bar v_z^2 = 3\bar v_x^2$
\begin{align*}
P = \frac{F}{A} &= \frac{m \cdot N \cdot \frac{1}{3}\bar v_x^2}{l_x \cdot l_y \cdot l_z}\\
&= \frac{m \cdot N \cdot \bar v^2}{3V}
\end{align*}
Rearranging,
\[PV = \frac{1}{3} \cdot N \cdot m \cdot \bar v^2\]
On the side, find the energy of $N_A$ molecules.
\[\bar E = N_A \cdot \frac{1}{2}m\bar v^2\]
$N_A$ is equal to $\frac{N}{n}$
Average kinetic energy for 1 molecule is then $\frac{\bar E}{N_A} = \bar \epsilon = \frac{3RT}{2N_A}.$
Simplifying,
\[\bar \epsilon = \frac{3}{2} K_B T\]
where $K_B = \frac{R}{N_A}$.\\\\
Additionally, $N_A \cdot m = M$, so $\bar E = \frac{1}{2}M \bar v^2 = \frac{3}{2}RT$.  Solving for velocity, 
\begin{align*}
    \frac{1}{2}Mv^2 &= \frac{3}{2}RT\\
    M \bar v^2 &= 3RT\\
    v^2 &= \frac{3RT}{M}
\end{align*}
Substituting into the $PV$ equation from earlier,
\[PV = \frac{1}{3} \cdot N \cdot m \cdot (\frac{3RT}{M})\]
Simplifying,
\[PV = N \cdot m \cdot \frac{RT}{M}\]
By definition earlier, $N_A \cdot n = N$.  Therefore,
\[PV = N_A \cdot n \cdot m \cdot \frac{RT}{M}\]
Also, $N_A \cdot m = M$, so
\[PV = M \cdot n \cdot \frac{RT}{M}\]
Simplifying,
\[\boxed{PV=nRT}\]

\section*{Real Gases: Intermolecular Forces}
There are many differences between ideal and real gases, since the assumptions made when defining an ideal gas do not apply to real gases.

\subsection*{Compressibility factor}
\[z = \frac{PV}{nRT}\]
Remember that in this case, $PV \neq nRT$!

\subsection*{The van der Waals Equation of State}
\[\left(P + a \cdot \frac{n^2}{V^2}\right)(V - nb) = nRT\]
Rearranging,
\[P = \frac{nRT}{V - nb} - a \cdot \frac{n^2}{V^2}\]
This equation takes into account the intermolecular forces between the molecules.\\\\
The attractive forces are represented by:
\[P + a \cdot \frac{n^2}{V^2}\]
while the repulsive forces are represented by:
\[V_{\text{effective}} = V - nb\]
\begin{itemize}
    \item Attractive forces hold molecules together, which means there are fewer independent molecules, which in turn reduces the rate of collisions with the wall.  Therefore, $P_{real} < P_{ideal}$.
    \item Repulsive forces decreases the available space for a molecule, which increases the rate of collision with the wall.  Therefore, $P_{real} > P_{ideal}$.
\end{itemize}

\subsection*{Molecule-Wall collisions}
Rate of collisions of gas molecules with a section of wall of area A:
\[Z_w \propto \frac{N}{V} \cdot \bar v \cdot A\]
\[Z_w \propto \frac{1}{4} \cdot \frac{N}{V} \cdot \bar v \cdot A = \frac{1}{4} \cdot \frac{N}{V} \cdot \sqrt{\frac{8RT}{\pi M}} \cdot A\]

The rate of effusion is represented by $Z_w$.




\end{document}