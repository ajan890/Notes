% document formatting
\documentclass[10pt]{article}
\usepackage[utf8]{inputenc}
\usepackage[left=1in,right=1in,top=1in,bottom=1in]{geometry}
\usepackage[T1]{fontenc}
\usepackage{xcolor}

% math symbols, etc.
\usepackage{amsmath, amsfonts, amssymb, amsthm}

% lists
\usepackage{enumerate}

% images
\usepackage{graphicx} % for images

% code blocks
\usepackage{minted, listings} 

\graphicspath{{./assets/images/}}

\newcommand{\solution}{\textbf{Solution:}} 

\title{CHEM 20B Week 3}

\author{Aidan Jan}
\date{\today}

\begin{document}
\maketitle
\section*{Intermolecular Forces (continued)}
\subsection*{Repulsive Forces}
\begin{itemize}
    \item When two atoms or molecules are so close together that their respective core electrons repel each other, overwhelming the attractive forces.
    \item Very short range
    \item $\frac{1}{R^{12}}$
\end{itemize}

\subsection*{Hydrogen Bonding}
\begin{itemize}
    \item A special case of dipole-dipole
    \item Molecule with N-H, O-H, or F-H wtih polar molecule with lone pair on N, O, or F.
    \item N, O, and F are very electronegative and can almost steal electrons from hydrogen, leaving unshielded nucleus (p+), which can interact with lone pairs of N, O, or F.
    \item It is weaker than ionic and covalent bonds but stronger than any intermolecular force.
\end{itemize}
\begin{center}
    \includegraphics*[scale=0.5]{W3_1}
\end{center}

\pagebreak
\subsection*{Intermolecular Forces in Liquids}
Trends in the boiling points of hydrides of some main group elements and the noble gases
\begin{center}
    \includegraphics*[scale=0.5]{W3_2}
\end{center}

\subsection*{Review: Types of Intermolecular Forces}
\begin{tabular}{|c|c|}
    \hline
    Ion-Ion Interaction & ion + ion \\
    \hline
    Ion-Dipole Interaction & ion + polar molecule \\
    \hline
    Hydrogen Bonding & Molecule with N-H, O-H, or F-H + polar molecule with lone pair on N, O, or F \\
    \hline
    Dipole-Dipole Interaction & polar + polar molecule \\
    \hline
    Ion-Induced Dipole Interaction & ion + nonpolar molecule \\
    \hline
    Dipole-Induced Dipole Interaction & polar molecule + nonpolar molecule \\
    \hline
    London Dispersion Forces & nonpolar molecule + nonpolar molecule\\
    \hline
\end{tabular}

\subsection*{Steps for comparing properties of molecules}
\begin{enumerate}
    \item Identify the compound: ion vs. polar vs. nonpolar
    \item Identify the types of intermolecular forces
    \item If two molecules have the same types of intermolecular forces, compare \textbf{size (molecular weight)}, then shape
\end{enumerate}

\subsection*{Intermolecular Forces Affect Many Physical Properties}
\begin{itemize}
    \item \textbf{Melting point:} solid $\rightarrow$ liquid
    \item \textbf{Boiling point:} liquid $\rightarrow$ gas
    \item \textbf{Vapor pressure:} pressure caused by molecules that escape from liquid, (to escape from liquid, need to break all intermolecular forces)
    \item Stronger intermolecular forces:
        \begin{itemize}
            \item Higher boiling points (need higher temperature to break intermolecular forces)
            \item Higher melting points (need higher temperature to break intermolecular forces)
            \item Lower vapor pressure (harder for molecules to escape from liquid)
        \end{itemize}
\end{itemize}

\subsection*{Phase Transition}
\begin{itemize}
    \item Boiling point: temperature in which the vapor pressure of a liquid equals the external pressure.
    \item Normal boiling point is the temperature at which the vapor pressure of the liquid equals 1 atm.
\end{itemize}

\subsection*{Phase Diagrams}
\begin{itemize}
    \item m.p. = normal melting point:
        \begin{itemize}
            \item T(solid $\rightarrow$ liquid) at 1atm
        \end{itemize}    
    \item b.p. = normal boiling point:
        \begin{itemize}
            \item T(liquid $\rightarrow$ gas) at 1atm, 373K (100$^\circ$ celsius)
        \end{itemize}
    \item t.p. = triple point:
        \begin{itemize}
            \item The pressure and temperature where the solid, liquid, and gas states coexist
            \item (For H$_2$O: 0.01 Celsius, 0.006 atm)
        \end{itemize}
    \item T$_c$, P$_c$: above which where there are no liquid or gas phase transitions, just gradual transition; called "supercritical region"
        \begin{itemize}
            \item 
        \end{itemize}
\end{itemize}

\begin{center}
    \includegraphics*[scale=0.5]{W3_3}
\end{center}

\section*{Composition of Solutions}
\begin{itemize}
    \item Mole Fraction
    \[X_1 = \frac{n_1}{n_1 + n_2}\]
    \item Concentration: number of moles per unit volume
    \begin{itemize}
        \item SI Unit: mol/m$^3$ (large for chemical work)
    \end{itemize}
    \item Molarity:
    \[molarity = \frac{moles \: solute}{liters \: solution} = mol L^{-1} = M = molar\]
    \item Molality:
    \[molality = \frac{moles \: solute}{kilograms \: solvent} = mol kg^{-1}\]
\end{itemize}

\section*{Solutions}
\begin{itemize}
    \item Solute + Solvent
    \item Aqueous solution: solvent = water
    \item Species that dissolve in water:
        \begin{itemize}
            \item Polar molecules:
            \begin{itemize}
                \item Glucose C$_6$H$_{12}$O$_6$
                \item Sucrose C$_{12}$H$_{22}$O$_{11}$
            \end{itemize}
            \item Ionic solic: NaCl(aq)
            \item NaCl(aq) $\rightarrow$ Na$^+$(aq) + Cl$^-$(aq)
        \end{itemize}
\end{itemize}

\subsection*{Precipitation Reaction}
\begin{itemize}
    \item A \textbf{precipitate} is an insoluble solid formed by a reaction in solution.
    \item Example:
    \begin{itemize}
        \item Molecular Equation:
        \[\text{AgNO}_3(aq) + \text{KCl}(aq) \longrightarrow \text{AgCl}(s) + \text{KNO}_3(aq)\]
        \item Complete Ionic Equation:
        \[\text{Ag}^+(aq) + \text{NO}_3^-(aq) + \text{K}^+(aq) + \text{Cl}^-(aq) \longrightarrow \text{AgCl}(s) + \text{K}^+(aq) + \text{NO}_3^-(aq)\]
        \item Net Ionic Equation:
        \[\text{Ag}^+(aq) + \text{Cl}^-(aq) \longrightarrow \text{AgCl}(s)\]
    \end{itemize}
\end{itemize}


\end{document}