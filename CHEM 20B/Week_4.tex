% document formatting
\documentclass[10pt]{article}
\usepackage[utf8]{inputenc}
\usepackage[left=1in,right=1in,top=1in,bottom=1in]{geometry}
\usepackage[T1]{fontenc}
\usepackage{xcolor}

% math symbols, etc.
\usepackage{amsmath, amsfonts, amssymb, amsthm}

% lists
\usepackage{enumerate}

% images
\usepackage{graphicx} % for images

% code blocks
\usepackage{minted, listings} 

\graphicspath{{./assets/images/Week 4}}

\newcommand{\solution}{\textbf{Solution:}} 

\title{CHEM 20B Week 4}

\author{Aidan Jan}
\date{\today}
\begin{document}
\maketitle
\section*{Thermochemistry}
\textbf{Thermochemistry:} study of heat transfer during chemical reactions.
\begin{itemize}
    \item Chemical reactions are usually studied at constant pressure, heat transfers in reactions are measured at constant pressure
    \[q_p = \Delta H = H_f - H_i = H_{products} - H_{reactants} = \Delta H_{reaction}\]
    \item $\Delta H_{reaction}$ is called the \textbf{reaction enthalpy}
    \begin{itemize}
        \item $\Delta H_{reaction} > 0$: endothermic
        \item $\Delta H_{reaction} < 0$: exothermic
    \end{itemize}
\end{itemize}

\subsection*{Hess's Law}
\begin{itemize}
    \item If a reaction is carried out in a series of steps, $\Delta H$ for the reaction is the sum of $\Delta H$ for each of the steps.
    \item Enthalpy is a state function:
    \begin{itemize}
        \item $\Delta H$ is sensitive to the states of the reactants and products
        \item Depends on the amount of matter
    \end{itemize}
    \item If a reaction is carried out in a series of steps, $\Delta H$ for the overall reaction equals the sum of the enthalpy changes for the individual steps
\end{itemize}
\begin{center}
    \includegraphics*[scale=0.6]{W4_1.png}
\end{center} 

\subsection*{Standard Enthalpy of Formation $\Delta H^0_f$}
\begin{itemize}
    \item The most useful thermochemical data are tables of the standard enthalpy of formation $\Delta H_f^0$ for compounds, defined as the enthalpy of formation of a compound in its standard state from the elements in their standard states at \textbf{1 atm} and \textbf{298.15 K}.
    \item The enthalpy of each element at the standard state, in its most stable form at the standard state, is assigned to be zero.
    \item For example:
    \begin{itemize}
        \item $\Delta H^0_f$ (O$_2$ gas, 298K, 1atm) = 0
        \item $\Delta H^0_f$ (C, graphite, 298K, 1atm) = 0
    \end{itemize}
    \item \underline{Enthalpy of formation of molecules} is then defined as the enthalpy of reaction to make the molecules from their atomic ingredients in their stable form.
    \item For example:
    \[\Delta H^0_f(H_2O(l)) \equiv \Delta H_{reaction}\left(H_2(g) + \frac{1}{2}O_2(g) \rightarrow H_2O(l)\text{, T = 298.15 K}\right) = -285.8 \frac{kJ}{mol}\]
    \item The change in standard state enthalpy for any reaction can be calculated from the standard state enthalpy of formation of its proucts and reactants as
    \[\Delta H^0 = \sum_{i = 1}^{prod} n_i \Delta H_i^0 - \sum_{j = 1}^{react} n_j \Delta H_j^0\]
    \item For the general reaction of the form $aA + bB \rightarrow cC + dD$, the standadrd enthalpy change is:
    \[\Delta H^0 = c\Delta H_f^0(C) + d\Delta H_f^0(D) - a\Delta H_f^0(A) - b\Delta H_f^0(B)\]
\end{itemize}
\subsection*{Reversible Processes in Ideal Gases - Isothermal processes:}
\begin{itemize}
    \item In this case, heat will flow (in either direction) to offset the cost of PV work, whereas T remains constant.
    \begin{align*}
        dw &= -P\text{d}V \\
        w &= -\int_{V_1}^{V_2} P\text{d}V
    \end{align*}
    \item For an ideal gas, P = nRT/V.  Because T is constant it comes outside the integral to give:
    \begin{align*}
        w &= -nRT\in_{V_1}^{V_2}\frac{1}{V} \text{d}V \\
        w &= -nRT\ln\left(\frac{V_2}{V_1}\right)\\
        q &= +nRT\ln\left(\frac{V_2}{V_1}\right)\\
    \end{align*}
    \[\Delta U = 0\]
    \[\Delta H = 0\]
\end{itemize}



\end{document}