% document formatting
\documentclass[10pt]{article}
\usepackage[utf8]{inputenc}
\usepackage[left=1in,right=1in,top=1in,bottom=1in]{geometry}
\usepackage[T1]{fontenc}
\usepackage{xcolor}

% math symbols, etc.
\usepackage{amsmath, amsfonts, amssymb, amsthm}

% lists
\usepackage{enumerate}

% images
\usepackage{graphicx} % for images

% code blocks
\usepackage{minted, listings} 

\graphicspath{{./assets/images/Week 6}}

\newcommand{\solution}{\textbf{Solution:}} 
\newcommand{\der}{\text{d}}

\title{CHEM 20B Week 6}

\author{Aidan Jan}
\date{\today}
\begin{document}
\maketitle
\section*{Chemical Equilibrium}
\begin{itemize}
    \item In principle, every chemical reaction is reversible... capable of moving in the forward of backward reaction
    \item At equilibrium the two opposing reactions occur at the same rate.
    \item Concentrations of chemical species do not change once equilibrium is established.
    \item Example: N$_2(g)$ + 3 H$_2(g) \leftrightharpoons$ 2 NH$_3(g)$
    \begin{center}
        \includegraphics*[scale=0.8]{W6_1.png}
    \end{center}
\end{itemize}

\subsection*{The Empirical Law of Mass Action}
\[aA + bB \leftrightharpoons cC + dD\]
\begin{itemize}
    \item A and B are reactants, C and D are products.
    \item K, the ratio at equilibrium, is always the same.  It is a constant and depends only on the temperature.
    \[K_C = \frac{[C]_{eq}^c [D]_{eq}^d}{[A]_{eq}^a [B]_{eq}^b}\]
    \[K_P = \frac{(P_C)_{eq}^c (P_D)_{eq}^d}{(P_A)_{eq}^a (P_B)_{eq}^b}\]
    \item $K_C$ is for concentration in aqueous reactions, $K_P$ is for pressure in gaseous reactions.
    \item Essentially,
    \[K = \frac{\text{Products}}{\text{Reactants}}\]
\end{itemize}

\subsubsection*{Example}
\[\text{3 H}_2(g) + \text{N}_2(g) \leftrightharpoons \text{2 NH}_3(g)\]
\[K_P = \frac{(P_{\text{NH}_3}^2)_{eq}}{(P_{\text{H}_2}^3)(P_{\text{N}_2})}\]

\subsection*{Law of Mass Action for Reactions involving Pure Substances and Multiple Phases}
For example: Zn$(s)$ + 2 H$_2$O$^+ (aq) \leftrightharpoons$ Zn$^{2+}(aq)$ + H$_2(g)$ + 2 H$_2$O$(l)$
\[K = \frac{([\text{Zn}^{2+}])_{eq}(P_{\text{H}_2})_{eq}}{([\text{H}_3\text{O}^+])^2_{eq}}\]
\begin{itemize}
    \item (s) = pure solid
    \item (aq) = dissolved species (aqueous)
    \item (g) = gas
    \item (l) = pure liquid
\end{itemize}
\textbf{General Rules for writing the mass action law:}
\begin{enumerate}
    \item Gases enter the equilibrium expression as partial pressures, measured in atm.
    \item Dissolved species enter as concentrations, in Molarity.
    \item Pure solids and pure liquids do not appear in equilibrium expressions.
    \item Products appear in the numerator and reactants appear in the denominator; each raised to a power equal its coefficient in the balanced chemical equation.
    \item K is dimensionless.
\end{enumerate}
\section*{Thermodynamics Description of the Equilibrium State}
In a reaction among ideal gases, the gas pressure is changing $P_1 \rightarrow P_2$ at a constant temperature.
\begin{itemize}
    \item $\Delta T = 0$
\end{itemize}
\[\Delta S = nR \ln \left(\frac{V_2}{V_1}\right) \hspace{1cm} \text{isothermal, ideal gas}\]
\[G = H - TS\]
\[\Delta G = \Delta (H - TS) = \Delta H - T \Delta S = -T \Delta S = -nRT\ln\left(\frac{V_2}{V_1}\right) = nRT\ln\left(\frac{P_2}{P_1}\right)\]
Importantly,
\[\Delta G = nRT\ln\left(\frac{P_2}{P_1}\right) = nRT\ln\left(\frac{P}{P_{ref}}\right)\]
Since usually the beginning pressure $P_1 = 1$ atm, $P_{ref} = 1$ atm.  Thus,
\[\Delta G = nRT\ln P\]

\subsection*{The Equilibrium Expression for Reactions in the Gas Phase:}
\[3 \text{NO}(g) \leftrightharpoons \text{N}_2\text{O}(g) + \text{NO}_2(g)\]
\textbf{Derive $\Delta G$}
\begin{center}
    \includegraphics*[scale=0.8]{W6_2.png}
\end{center}
\begin{align*}
    \Delta G_1 &= nRT\ln\left(\frac{P_2}{P_1}\right) = 3RT\ln\left(\frac{P_{ref}}{P_{NO}}\right) = RT\ln\left(\frac{P_{ref}}{P_{NO}}\right)^3\\
    \Delta G_2 &= \Delta G^0\\
    \Delta G_3 &= RT\ln\left(\frac{P_{N_2 O}}{P_{ref}}\right) + RT\ln\left(\frac{P_{NO_2}}{P_{ref}}\right) = RT\ln \left[\left(\frac{P_{N_2 O}}{P_{ref}}\right)\left(\frac{P_{NO_2}}{P_{ref}}\right)\right]\\
\\
\\
    \Delta G &= \Delta G_1 + \Delta G_2 + \Delta G_3\\
    &= RT\ln \left(\frac{P_{ref}}{P_{NO}}\right)^3 + \Delta G^0 + RT\ln\left[ \left(\frac{P_{N_2 O}}{P_{ref}}\right)\left(\frac{P_{NO_2}}{P_{ref}}\right)\right]\\
    &= -RT\ln \left(\frac{P_{NO}}{P_{ref}}\right)^3 + \Delta G^0 + RT\ln\left[ \left(\frac{P_{N_2 O}}{P_{ref}}\right)\left(\frac{P_{NO_2}}{P_{ref}}\right)\right]\\
    &= \Delta G^0 + RT\ln\left[\frac{\left(\frac{P_{N_2 O}}{P_{ref}}\right)\left(\frac{P_{NO_2}}{P_{ref}}\right)}{\left(\frac{P_{NO}}{P_{ref}}\right)^3}\right]\\
    &= \Delta G^0 + RT \ln Q
\end{align*}
When the reaction arrives at equilibrium, $\Delta G = 0$ and $Q = K$
$\Delta G^0 = -RT \ln K$

\subsection*{Reactions in Ideal Solutions}
$\Delta G$ for $n$ moles of solute, as ideal (dilute) solution changes in concentration form $c_!$ to $c_2$ mol/L, is
\[\Delta G = nRT \ln \left(\frac{c_2}{c_1} =\right) = nRT \ln \left( \frac{c}{c_{ref}}\right) = nRT \ln c\]
\[c_{ref} = 1 M\]
\[\Delta G = \Delta G^0 + RT \ln \left[\frac{\left(\frac{[C]}{c_{ref}}\right)^c\left(\frac{[D]}{c_{ref}}\right)^d}{\left(\frac{[A]}{c_{ref}}\right)^a \left(\frac{[B]}{c_{ref}}\right)^b}\right]\]
\[\Delta G = \Delta G^0 + RT \ln Q\]
When the reaction arrives at equilibrium, $\Delta G = 0$ and $Q = K$.
\[\Delta G^0 = -RT \ln K\]
\[K = \left[\frac{[C]^c[D]^d}{[A]^a[B]^b}\right]\]

\subsection*{Properties of K}
Consider the reaction:
\[aA + bB \leftrightharpoons cC\]
\begin{itemize}
    \item The reverse reaction: Invert K.  $K' = 1 / K$
    \item A balanced chemical equation is multiplied by a constant: Raise K to a power equal to that constant. $K' = K^C$, where $C$ is the constant multiplied.
    \item When two reactions are added, the $K$-values multiply.  $K' = K_A \cdot K_B$, where $K_A$ and $K_B$ are the $K$-values of the two reactions.
\end{itemize}

\section*{The Reaction Quotient:}
\[Q = \frac{(P_C)^c (P_D)^d}{(P_A)^a (P_B)^b}\]
Where the partial pressures are the actual values measured at any point during the reactions, not just at equilibrium.
\begin{itemize}
    \item $Q < K$: reaction moves to the right
    \item $Q = K$: system at equilibrium, no reaction
    \item $Q > K$: reaction moves to the left
\end{itemize}

\begin{center}
    \includegraphics*[scale=0.8]{W6_3.png}
\end{center}

\section*{Le Chatelier's Principle}
\begin{itemize}
    \item A system in equilibrium that is subjected to a stress will react in a way that tends to counteract the stress.  \textbf{System rearranges to mitigate change.}
    \item Effects of changing the concentration of a reactant or product:
    \begin{itemize}
        \item Stress: Increase the concentration or pressure of species A
        \item Response: The reaction will move in the appropriate direction to decrease A
    \end{itemize}
    \item Effects of Changing the Volume
    \begin{itemize}
        \item Stress: Increase pressure or decrease Volume
        \item Response: The reaction will move in the direction to produce fewer gaseous molecules to decrease the pressure
    \end{itemize}
    \item Effects of Changing the temperature
    \begin{itemize}
        \item Stress: Increase temperature
        \item Response: The reaction will move in the appropriate direction to absorb heat and decrease the temperature.
    \end{itemize}
\end{itemize}

\subsection*{The Effect of a Temperature Change on the Equilibrium Constant}
The values of K at two different temperatures are related by van't Hoff equation:
\[\ln \left(\frac{K_2}{K_1}\right) = - \frac{\Delta H^0}{R} \left[\frac{1}{T_2} - \frac{1}{T_1}\right]\]
K increases with temperature increase for endothermic reactions and decreases with temperature increase for exothermic reactions.

\end{document}
