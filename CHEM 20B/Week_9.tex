% document formatting
\documentclass[10pt]{article}
\usepackage[utf8]{inputenc}
\usepackage[left=1in,right=1in,top=1in,bottom=1in]{geometry}
\usepackage[T1]{fontenc}
\usepackage{xcolor}

% math symbols, etc.
\usepackage{amsmath, amsfonts, amssymb, amsthm}

% lists
\usepackage{enumerate}

% images
\usepackage{graphicx} % for images

% code blocks
\usepackage{minted, listings} 

\graphicspath{{./assets/images/Week 9}}

\newcommand{\solution}{\textbf{Solution:}} 
\newcommand{\der}{\text{d}}

\title{CHEM 20B Week 9}

\author{Aidan Jan}
\date{\today}
\begin{document}
\maketitle
\section*{Electrochemistry}
\textbf{General Rules to Balance Oxidation-Reduction Equation:}\
\begin{enumerate}
    \item Write the two unbalanced half-reactions.
    \item Balance both half-reactions for all atoms except O and H.
    \item Balance each half-reaction for O by adding H$_2$O, and then balance for H by adding H$^+$.
    \item Balance each half-reaction for charge by adding elecrons to the side with greater positive charge.
    \item Multiply each half-reaction by a factor to make the electron count the same in both half-reactions.
    \item Add the two balanced reactions together and cancel species that appear on both sides of the equation.
    \begin{itemize}
        \item Stop here for acidic solutions
    \end{itemize}
    \item For a basic solution, "neutralize" the excess H$^+$ by adding OH$^-$ and cancel any water (if possible).
\end{enumerate}
\subsection*{Example: Balancing Reaction of MnO$_4^-$ and Br$^-$ in Basic Solution}
\begin{enumerate}
    \item Write the two unbalanced half-reactions
    \begin{eqnarray*}
        Br^- &\rightarrow& BrO_3^- \\
        MnO_4^- &\rightarrow& MnO_2 
    \end{eqnarray*}
    \item Balance both half-reactions for all atoms except $O$ and $H$.
    \begin{eqnarray*}
        Br^- &\rightarrow& BrO_3^- \\
        MnO_4^- &\rightarrow& MnO_2 
    \end{eqnarray*}
    \item Balance each half-reaction for $O$ by adding $H_2 O$, and then balance for $H$ by adding $H^+$.
    \begin{eqnarray*}
        Br^- + 3 H_2O&\rightarrow& BrO_3^- + 6 H^+\\
        4 H^+ + MnO_4^- &\rightarrow& MnO_2 + 2H_2O
    \end{eqnarray*}
    \item Balance each half-reaction for charge by adding electrons to the side with greater positive charge.
    \begin{eqnarray*}
        Br^- + 3 H_2O&\rightarrow& BrO_3^- + 6 H^+ + 6e^-\\
        4 H^+ + MnO_4^- + 3e^- &\rightarrow& MnO_2 + 2H_2O
    \end{eqnarray*}
    \item Multiply each half-reaction by a factor to make the electron count the same in both half-reactions.
    \begin{eqnarray*}
        Br^- + 3 H_2O&\rightarrow& BrO_3^- + 6 H^+ + 6e^-\\
        8 H^+ + 2MnO_4^- + 6e^- &\rightarrow& 2MnO_2 + 4H_2O
    \end{eqnarray*}
    \item Add the two balanced reactions together and cancel species that appear on both sides of the equation.
    \begin{align*}
        Br^- + 3 H_2O + 8 H^+ + 2MnO_4^- + 6e^-&\rightarrow BrO_3^- + 6 H^+ + 6e^- + 2MnO_2 + 4H_2O\\
        2 H^+ + 2MnO_4^- + Br^- &\rightarrow BrO_3^- + 2MnO_2 + H_2O
    \end{align*}
    \item Since the reaction occurs in a basic solution, "neutralize" the excess $H^+$ by adding $OH^-$ and cancel any water (if possible).
    \begin{align*}
        2 H^+ + 2 OH^- + 2MnO_4^- + Br^- &\rightarrow BrO_3^- + 2MnO_2 + H_2O + 2 OH^- \\
        2MnO_4^- + Br^- + H_2O &\rightarrow BrO_3^- + 2MnO_2 + 2 OH^- \\
    \end{align*}
\end{enumerate} 



\pagebreak
\section*{Electrochemical cells}
An \textbf{electrochemical cell} is a device in which an electric current is either produced by a spontaneous chemical reaction or used to bring about a nonspontaneous reaction.
\begin{itemize}
    \item \textbf{Galvanic cell:} electrochemical cell in which a spontaneous chemical reaction is used to generate an electric current.
    \item For example, batteries
    \begin{itemize}
        \item If we simply mix two species that undergo a redox reaction, energy is released as heat, but no electricity is generated.  However, if we separate the reactants and provide a pathway for the electrons to travel, the electrons can do work, generating an electric current.
    \end{itemize}
    \item A galvanic cell consists of \textbf{two electrodes:}
    \begin{itemize}
        \item \textbf{Anode:} the electrode at which oxidation (Loss of electrons) occurs.
        \item \textbf{Cathode:} the electrode at which reduction (gain of electrons) occurs.
        \item Electrons flow from the anode to cathode.
    \end{itemize}
    \begin{center}
        \includegraphics*[scale=0.9]{W9_1.png}
    \end{center}
\end{itemize}

\section*{Voltage}
\begin{itemize}
    \item Electrostatic Potential: $\Phi$ (V)
    \item Electrostatic Potential Energy: $E_p$ (J)
    \item The SI unit for potential in the volt:
    \[1 V = 1 J \cdot C^{-1}\]
    \[\Delta E_p = q \Delta \Phi\]
\end{itemize}

\section*{Faraday's Laws}
\begin{itemize}
    \item The mass of a given substance that is produced or consumed in an electrochemical reaction is proportional to the quantity of electric charge passed.
    \item The magnitude of the charge on a single electron:
    \[\vert -e \vert = e = 1.60 \times 10^{-19} C\]
    \item The magnitude of the charge of one mole of electrons:
    \[Q = (6.022 \times 10^{23})(1.60 \times 10^{-19} C) = 96485.34 C\]
    \[Q = e \cdot N\]
    \item Faraday constant: Denoted as $F$, the magnitude of the charge per mole of electrons:
    \[F = 96,485.34 \:C \cdot mol^{-1}\]
    \item Electric Current: is the amount of charge that flows through a circuit per second, measured in the SI unit, the ampere (A):
    \[1\:C = (1\:A) (1\:s)\]
    \item \textit{The amount of charge that has passed when a current of 1 amperes has flowed for t seconds is}
    \[Q = I \cdot t\]
    \item The number of moles of electrons $n$ transferred:
    \[n = \frac{I \cdot t}{96485.34}\]
\end{itemize}

\end{document}
