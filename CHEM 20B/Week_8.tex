% document formatting
\documentclass[10pt]{article}
\usepackage[utf8]{inputenc}
\usepackage[left=1in,right=1in,top=1in,bottom=1in]{geometry}
\usepackage[T1]{fontenc}
\usepackage{xcolor}

% math symbols, etc.
\usepackage{amsmath, amsfonts, amssymb, amsthm}

% lists
\usepackage{enumerate}

% images
\usepackage{graphicx} % for images

% code blocks
\usepackage{minted, listings} 

\graphicspath{{./assets/images/Week 8}}

\newcommand{\solution}{\textbf{Solution:}} 
\newcommand{\der}{\text{d}}

\title{CHEM 20B Week 8}

\author{Aidan Jan}
\date{\today}
\begin{document}
\maketitle
\section*{Acid-Base Equilibria}
\subsection*{Arrhenius Definition of Acids and Bases}
\begin{itemize}
    \item Acids: when dissolved in water increases the concentration of H$_3$O$^+$
    \item Bases: when dissolved in water increases the concentration of OH$^-$
\end{itemize}

\subsection*{Brønsted-Lowry Definition of Acids and Bases}
\begin{itemize}
    \item Acids: Proton donor
    \item Bases: Proton acceptor
\end{itemize}

\subsection*{Lewis Definition of Acids and Bases}
\begin{itemize}
    \item Electron pair acceptor
    \item Electron pair donor
\end{itemize}

\subsection*{Properties of Acids and Bases in Aqueous Solutions: The Brønsted-Lowry Scheme}
\textbf{Autoionization of Water}
\[H_2 O(l) + H_2 O(l) \leftrightharpoons H_3O^+(aq) + OH^- (aq)\]

$K_w$, the acid-base constant of water is determined by 
\[K_w[H_3O^+][OH^-]\]
This constant varies by temperature.

\begin{center}
\textbf{Temperature Dependence of $K_w$}\\
\begin{tabular}{| c | c | c |}
    \hline
    \textbf{Temperature ($^\circ$ C)} & \textbf{$K_w$} & \textbf{pH of Water} \\
    \hline
    $0$ & $0.114 \times 10^{-14}$ & $7.47$ \\
    $10$ & $0.292 \times 10^{-14}$ & $7.27$ \\
    $20$ & $0.681 \times 10^{-14}$ & $7.08$ \\
    $25$ & $1.01 \times 10^{-14}$ & $7.00$ \\
    $30$ & $1.47 \times 10^{-14}$ & $6.92$ \\
    $40$ & $2.92 \times 10^{-14}$ & $6.77$ \\
    $50$ & $5.47 \times 10^{-14}$ & $6.63$ \\
    $60$ & $9.61 \times 10^{-14}$ & $6.51$ \\
    \hline
\end{tabular}
\end{center}

\subsection*{Strong Acids}
Strong Acids ionize completely in aqueous solution.
\[H_2 O(l) + HA(aq) \rightarrow H_3O^+(aq) + A^-(aq)\]
\textbf{Common Strong Acids}
\begin{itemize}
    \item HBr (aq)
    \item HCl (aq)
    \item HI (aq)
    \item HNO$_3$ (aq)
    \item HClO$_4$ (aq)
    \item HClO$_3$ (aq)
    \item H$_2$SO$_4$ (aq)
\end{itemize}

\subsection*{Strong Bases}
Strong Bases react completely to give $OH^-$ when put in water.
\[H_2O(l) + NH_2^-(aq) \rightarrow OH^-(aq) + H_2(aq)\]
\[H_2O(l) + H^-(aq) \rightarrow OH^-(aq) + H_2(aq)\]

\textbf{Common Strong Bases}
\begin{itemize}
    \item Group 1 hydroxides
    \item Alkaline earth metal hydroxides
    \item Group 1 and Group 2 oxides
\end{itemize}

\section*{The pH Function}
\[pH = -\log_{10}[H_3O^+]\]
At 25$^\circ$C,
\begin{itemize}
    \item pH < 7 = Acidic solution
    \item pH = 7 = Neutral solution
    \item pH > 7 = Basic solution
\end{itemize}

Similar to how $[H_3O^+][OH^-] = K_w$, $pH + pOH = pK_w$.\\
As a consequence,
\begin{itemize}
    \item $[H^+] = 10^{-pH}$
    \item $[OH^-] = 10^{-pOH}$
\end{itemize}

\subsection*{Acid and Base Strength}
Acid strength is based on the extent to which they are ionized in solution.
\[HA (aq) \leftrightharpoons H^+ (aq) + A^- (aq)\]
The Acid Ionization Constant, $K_a$ is a quantitative measure of the strength of the acid.
\[K \equiv K_a = \frac{[H^+][A^-]}{[HA]}\]
If:
\begin{itemize}
    \item $K_a \gg 1 \rightarrow$ HA is a strong acid.
    \item $K_a \ll 1 \rightarrow$ HA is a weak acid.
\end{itemize}
Convenient characterization of strength of acid is $pK_a$.
\[pK_a = -\log_{10}(K_a)\]
\textbf{For Example:}\\
$K_a = 10^7 \rightarrow pK_a = -7$ (strong acid)\\
$K_a = 10^-5 \rightarrow pK_a = 5$ (weak acid) \\

\noindent Similarly to acids, base strength is represented by $K_b$, which is inversely related to the strength of its conjugate acid.
\[H_2O(l) + B(aq) \leftrightharpoons OH^- (aq) + BH^+ (aq)\]
\[K \equiv K_b = \frac{[OH^-][BH^+]}{[B]}\]
Similarly, a convenient characterization of strength of base is $pK_b$.
\[pK_b = -\log_{10}(K_b)\]
If:
\begin{itemize}
    \item $K_b \gg 1 \rightarrow$ strong base, many $OH^-$ produced, little $[B]$ left.
    \item $K_b \ll 1 \rightarrow$ weak base, most $[B]$ remains.
\end{itemize}
Importantly, 
\[K_b K_a = K_w\]
\[pK_b + pK_a = pK_w\]


\end{document}
