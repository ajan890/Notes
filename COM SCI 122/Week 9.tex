% document formatting
\documentclass[10pt]{article}
\usepackage[utf8]{inputenc}
\usepackage[left=1in,right=1in,top=1in,bottom=1in]{geometry}
\usepackage[T1]{fontenc}
\usepackage{xcolor}

% math symbols, etc.
\usepackage{amsmath, amsfonts, amssymb, amsthm}

% lists
\usepackage{enumerate}

% images
\usepackage{graphicx} % for images

% code blocks
\usepackage{minted, listings} 

% verbatim greek
\usepackage{alphabeta}

\graphicspath{{./assets/images}}

\newcommand{\solution}{\textbf{Solution:}} 
\newcommand{\example}{\textbf{Example: }}
\newcommand{\R}{\mathbb{R}}

\title{COM SCI 122 Week 9}

\author{Aidan Jan}
\date{\today}

\begin{document}
\maketitle

\section*{Applications of Hidden Markov Models (HMMs)}
From Wikipedia:\\\\
A \textbf{hidden Markov model} (HMM) is a is a Markov model in which the observations are dependent on a latent (or hidden) Markov process (referred to as $X$). An HMM requires that there be an observable process $Y$ whose outcomes depend on the outcomes of $X$ in a known way. Since $X$ cannot be observed directly, the goal is to learn about state of $X$ by observing $Y$. By definition of being a Markov model, an HMM has an additional requirement that the outcome of $Y$ at time $t = t_0$ must be "influenced" exclusively by the outcome of $X$ at $t = t_0$ and that the outcomes of $X$ and $Y$ at $t < t_0$ must be conditionally independent of $Y$ at $t = t_0$ given $X$ at time $t = t_0$. Estimation of the parameters in an HMM can be performed using maximum likelihood estimation. For linear chain HMMs, the Baum-Welch algorithm can be used to estimate parameters.\\\\
Hidden Markov models are known for their applications to thermodynamics, statistical mechanics, physics, chemistry, economics, finance, signal processing, information theory, pattern recognition—such as speech, handwriting, gesture recognition, part-of-speech tagging, musical score following, partial discharges and bioinformatics.

\subsubsection*{HMMs for Analyzing and Modeling Epigenetic Data}
\begin{center} 
	\includegraphics*[width=\textwidth]{W9_1.png} 
\end{center}

\subsubsection*{HMMs for Genotype Imputation and Phasing}
\begin{center} 
	\includegraphics*[width=\textwidth]{W9_2.png} 
\end{center}

\subsubsection*{HMMs for Identifying Protein Coding Genes}
\begin{center} 
	\includegraphics*[width=\textwidth]{W9_3.png} 
\end{center}

\subsubsection*{Profile HMMs for Protein Sequence Alignment}
\begin{center} 
	\includegraphics*[width=\textwidth]{W9_4.png} 
\end{center}

\subsubsection*{HMMs for Calling Copy Number Variation}
\begin{center} 
	\includegraphics*[width=\textwidth]{W9_5.png} 
\end{center}

\subsection*{Motivating Example}
\begin{itemize}
	\item In genomics problems we often want to "annotate" or "cluster" positions along a sequence
	\item Considering each position along a sequence in isolation is often inadequate.
	\item For example, promoter regions (i.e., regions near transcription start sites) have elevated frequency of G or C nucleotides
	\item Would like to annotate each base of the genome as likely promoter ('P') or background ('B') based on this
\end{itemize}
\begin{center} 
	\includegraphics*[width=\textwidth]{W9_6.png} 
\end{center}

\subsubsection*{Annotation Strategies for Example}
\textbf{Idea 1:}
\begin{itemize}
	\item Assign all G or C nucleotides to promoter ('P') and A or T nucleotides to background ('B')
	\item Problem: This ignores spatial context leading to many incorrect annotations and a segmentation that is overly choppy
\end{itemize}
\begin{center} 
	\includegraphics*[width=\textwidth]{W9_7.png} 
\end{center}
\textbf{Idea 2:}
\begin{itemize}
	\item Bin the genome into fixed length bins.  Classify each bin indepenently based on the frequency of G or C, for example, promoter if majority nucleotides are G or C and background otherwise
	\item Potential drawbacks: Signal could be split across consecutive bins leading to incorrect classifications.  Can be unclear how to pick length or if a single length is adequate.
\end{itemize}
\begin{center} 
	\includegraphics*[width=\textwidth]{W9_8.png} 
\end{center}
\textbf{Idea 3:}
\begin{itemize}
	\item Use a sliding window.  Classify each window based on frequency of G or C, for example promoter if majority nucleotides are G or C and background otherwise.
	\item Potential drawbacks: Different windows overlapping the same base can lead to different annotations.  Can be unclear how to pick length or if a single length is adequate
\end{itemize}
\begin{center} 
	\includegraphics*[width=\textwidth]{W9_9.png} 
\end{center}
Under assumption any base overlapping a window that receives 'P' annotation also receives a 'P' annotation.\\\\
\textbf{Idea 4:}
\begin{itemize}
	\item Use a probabilistic model that can enable principled annotation of individual nucleotides while considering the sequential context
	\item This model we will discuss is the \textbf{Hidden Markov Model (HMMs)}.  It is used for many genomic applications
\end{itemize}

\subsection*{Markov Chain}
Has $K$ - states, denoted $s_1, \dots, s_k$.
\begin{center} 
	\includegraphics*[width=0.5\textwidth]{W9_10.png} 
\end{center}
In this example, $K = 2$ and denote $s_1$ and $s_2$ by 'B' and 'P', respectively.
\begin{itemize}
	\item There are discrete sequential positions denoted $i = 1$, $i = 2$, $\dots$
	\item At the $i$-th position the chain is in exactly one of the $K$ states, denoted by $\pi_i$, and $\pi_i \in \{s_1, s_2, \dots, s_k\}$
	\item We start the chain ($i = 1$) based on an initial probability distribution determined by $\tau_1, \dots, \tau_k$
	\item The current state determines the probability distribution of the next state.  We have parameters
	\[a_{kj} = P(\pi_{i + 1} = s_j | \pi_i = s_k)\]
    corresponding to arcs in the figure below
    \item Probabilities out of a state sum to 1.
\end{itemize}
\begin{center} 
	\includegraphics*[width=\textwidth]{W9_11.png} 
\end{center}
\subsubsection*{Markov Property or Assumption:}
$\pi_{i + 1}$ is conditionally independent of $\{\pi_{i - 1}, \pi_{i - 2}, \dots, \pi_1\}$ given $\pi_i$ that is
\[P(\pi_{i + 1 = s_j | \pi_i = s_k}) = P(\pi_{i + 1} = s_j| \pi_i = s_k, \text{ any earlier history})\]
\subsection*{Example Path}
Consider the path:
\begin{center}
\begin{tabular}{ll}
$\pi_1 = B$ & $i = 1$ (starting)\\
$\pi_2 = B$ & $i = 2$ (prob. 0.9)\\
$\pi_3 = P$ & $i = 3$ (prob. 0.1)
\end{tabular}
\end{center}
\begin{itemize}
	\item HMMs extend Markov chains by allowing states to "emit" observations with different probabilities that depend on the state
	\item Let $\{\sigma_1, \dots, \sigma_M\}$ be a set of symbols that are possible observations (e.g., $\{A, C, G, T\}$)
	\item Let $x = x_1 x_2 \dots x_n$ denote a sequence of observations (e.g., AACGGC)
	\item We have emission parameters corresponding to probability of observing a symbol emitted in a given state:
	\[b_k(\sigma_m) = P(x_i = \sigma_m | \pi_i = s_k)\]
    for $m = 1, \dots, M$ possible symbole $\sigma_m$ and $k = 1, \dots, K$ states
\end{itemize}
We will use the following emission parameters for this example:
\begin{center} 
	\includegraphics*[scale=0.4]{W9_12.png} 
\end{center}
\begin{itemize}
	\item First, we start from $\pi_1$.  $\pi_1 = B, i = 1$.  Therefore, the probability is $1/4$.  Suppose we generate $x_1 = A$; according to our emission parameters, the probability is $1/4$
    \item When we move to the next base, $\pi_2 = B, i = 2$.  The probability is $9/10$, according to our model.  ($B \rightarrow B$)  Suppose we generate $x_2 = G$; according to our emission parameters, the probability is $1/4$.
    \item Finally, to the third base, $\pi_3 = P, i = 3$.  The probability is $1/10$, according to our model.  ($B \rightarrow P$)  Suppose we generate $x_3 = C$; according to our emission parameters, the probability is $1/3$.
\end{itemize}

\subsection*{HMM Definition}
An HMM, $\lambda$, is a 5-tuple consisting of:
\begin{itemize}
	\item $\Sigma$ - alphabet of $M$ emitted of symbols $\{\sigma_1, \dots, \sigma_M\}$
	\item A set $S$ of $K$ hidden states $\{s_1, \dots, s_k\}$
	\item A set $S$ of $K$ initial state probabilities $\{\tau_1, \dots, \tau_K\}$
	\item A $K$ by $K$ set of state transition probabilities
	\[a_{kj} = P(\pi_{i + 1} = s_j | \pi_i = s_k)\]
    \item A $K$ by $M$ set of state emission probabilities
    \[b_k(\sigma_m) = P(x_i = \sigma_m|\pi_i = s_k)\]
\end{itemize}

\subsection*{HMM Problems and Algorithms}
\begin{enumerate}
    \item State Emission (Forward-Backward Algorithm, Dynamic Programming)
    \item Most Probable Path (Viterbi, Dynamic Programming)
    \item Learning a HMM (Baum-Welch, Expectation-Maximization)
\end{enumerate}

\section*{Problem 1: State Emission}
Given a sequence of observations of length $n$, $x = x_1 x_2 \dots x_n$, and a model $\lambda$ we want to compute:
\[P(\pi_i = s_k | x, \lambda)\]
that is the probability at position $i$ the state is $s_k$ given the observation sequence.\\\\
For example, we observe the sequence AGC and want to infer what is the probability at position 3 we are in state 'P'?
\[P(\pi_3 = P|x=AGC, \lambda)\]

\subsection*{Brute Force Approach to State Estimation}
\begin{align*}
    &P(\pi_i = s_k | x, \lambda)\\
    &= \sum_{\pi s.t. \pi_i = s_k} P(\pi|x, \lambda)\\
    &= \sum_{\pi s.t. \pi_i = s_k} P(\pi, x | \lambda) / P(x | \lambda)\\
    &= \frac{\sum_{\pi s.t. \pi_i = s_k} P(\pi|\lambda) P(x|\pi, \lambda)}{\sum_\pi P(\pi, x|\lambda)}\
    &= \frac{\sum_{\pi s.t. \pi_i = s_k} P(\pi|\lambda) P(x|\pi, \lambda)}{\sum_\pi P(\pi|\lambda) P(x|\pi, \lambda)}
\end{align*}
\begin{itemize}
	\item $\pi$ - hidden path variables
	\item $x$ - observations
	\item $i$ - position
	\item $s_k$ - state
	\item $\lambda$ - model parameters
\end{itemize}
For example:
\begin{align*}
    &P(\pi_i = s_k | x, \lambda)\\
    &= P(\pi = BBP|x = AGC, \lambda) + P(\pi = BPP|x = AGC, \lambda) + P(\pi = PBP|x = AGC, \lambda) + P(\pi = PPP|x = AGC, \lambda)\\
    &= [P(\pi = BBP, x = AGC|\lambda) + P(\pi = BPP, x = AGC|\lambda) + P(\pi = PBP, x = AGC|\lambda) + P(\pi = PPP, x = AGC|\lambda)] / P(x = AGC|\lambda)\\
    &= [P(\pi = BBP|\lambda) P(x = AGC|\pi = BBP, \lambda) + P(\pi = BPP|\lambda) P(x = AGC|\pi = BPP, \lambda) + P(\pi = PBP|\lambda) P(x = AGC|\pi = PBP, \lambda) + P(\pi = PPP|\lambda) P(x = AGC|\pi = PPP, \lambda)] / [P(\pi = BBB, x = AGC|\lambda) + P(\pi = BBP, x = AGC|\lambda) + P(\pi = BPB, x = AGC|\lambda) + P(\pi = BPP, x = AGC|\lambda) + P(\pi = PBB, x = AGC|\lambda) + P(\pi = PBP, x = AGC|\lambda) + P(\pi = PPB, x = AGC|\lambda) + P(\pi = PPP, x = AGC|\lambda)]
\end{align*}
The terms in the final equation can be computed directly.\\\\
How do we compute $P(\pi|\lambda)$ for an arbitrary path $\pi$ of length 3?
\begin{align*}
    P(\pi|\lambda) &= P(\pi_1, \pi_2, \pi_3|\lambda)\\
    &= P(\pi_1|\lambda) P(\pi_2, \pi_3|\lambda, \pi_1)\\
    &= P(\pi_1|\lambda) P(\pi_2|\lambda, \pi_1) P(\pi_3|\lambda, \pi_1, \pi_2)\\
    &= P(\pi_1|\lambda) P(\pi_2|\lambda, \pi_1) P(\pi_3|\lambda, \pi_2)
\end{align*}
How do we compute $P(x|\pi, \lambda)$ for an arbitrary path $\pi$ of length 3 and observation sequence $x$?
\begin{align*}
    P(x|\pi, \lambda) &= P(x_1, x_2, x_3 | \pi_1, \pi_2, \pi_3, \lambda)\\
    &= P(x_1|\pi_1, \lambda) P(x_2, \pi_2, \lambda) P(x_3, \pi_3, \lambda)
\end{align*}
Example in the case $\pi = BBP$ and $x = AGC$
\begin{center} 
	\includegraphics*[width=\textwidth]{W9_13.png} 
\end{center}
In general, what is the complexity of the brute force approach for a sequence of length $n$ and $K$ states?
\begin{itemize}
	\item O($nk^n$)
	\item This sucks! We need a more efficient approach.
\end{itemize}

\subsection*{State Estimation with Forward and Backward Variables}
Define:
\begin{itemize}
	\item Forward variables: $\alpha_i(k) = P(x_1 x_2 \dots x_i, \pi_i = s_k | \lambda)$
	\item Backward variables: $\beta_i(k) = P(x_{i + 1} x_{i + 2} \dots x_n | \pi_i = s_k, \lambda)$
	\item Forward variables $\alpha_i(k)$ is the probability under the model $\lambda$ that both:
	\begin{itemize}
        \item We have seen the first $i$ observations
        \item We are in state $s_k$ at the $i$-th position
    \end{itemize}
    \item Backward variables $\beta_i(k)$ is the probability under the model $\lambda$ given we are in state $s_k$ at position $i$ that:
    \begin{itemize}
        \item We have seen the last $n - i$ observations
    \end{itemize}
    \item Claim:
    \[P(\pi_i = s_k | x, \lambda) = \frac{\alpha_i(k) \beta_i(k)}{\sum_{j = 1}^K \alpha_i(j) \beta_i(j)}\]
    \begin{itemize}
        \item We can efficiently compute state estimates variables for all positions $i$ and states $k$ if we can efficiently compute forward and backward variables for them.
        \item We need to show that this claim is true and that we can compute the forward and backward variables efficiently.
    \end{itemize}
\end{itemize}
\subsubsection*{Proof}
Let's break down the claim:
\begin{align*}
    &= \alpha_i(k) \beta_i(k) = P(x_1 x_2 \dots x_i, \pi_i = s_k|\lambda) P(x_{i + 1} x_{i + 2} \dots x_n | \pi_i = s_k, \lambda)\\
    &=P(x_1 x_2 \dots x_i, \pi_i = s_k | \lambda) P(x_{i + 1} x_{i + 2} \dots x_n | x_1 x_2 \dots x_i, \pi_i = s_k, \lambda)\\
    &= P(x_1 x_2 \dots x_i, x_{i + 1} x_{i + 2} \dots x_n, \pi_i = s_k | \lambda)\\
    &= P(x, \pi_i = s_k | \lambda)
\end{align*}
Therefore, we get
\[\sum_{j = 1}^K \alpha_i(j) \beta_i(j) = \sum_{j = 1}^K P(x, \pi_i = s_k | \lambda) = P(x | \lambda)\]
We have shown the claim is true.  Now, we have to find a way to compute the variables.
\subsubsection*{Computing Forward Variables}
\[\alpha_i(k) = P(x_1 x_2 \dots x_i, \pi_i = s_k | \lambda)\]
Forward variable $\alpha_i(k)$ is the probability under the mdoel that both:
\begin{itemize}
	\item We have seen the first $i$ observations
	\item We are in state $s_k$ at the $i$-th position
\end{itemize}
We use dynamic programming to compute this efficiently.
\begin{align*}
    \intertext{Beginning with the base case:}
    \alpha_1(k) &= P(x_1, \pi_1 = s_k | \lambda) \hspace{1cm} \text{for } k = 1, \dots, K\\
    &= P(\pi_1 = s_k | \lambda) P(x_1 | \pi_1 = s_k, \lambda)\\
    &= \tau_k b_k (x_1)
\end{align*}
\begin{itemize}
	\item $\tau_k$ is the initial state probability
	\item $b_k$ is the emission probability for the first observation
\end{itemize}
\begin{align*}
    \intertext{Now, for the induction step:}
    \alpha_{i = 1}(j) &= P(x_1 x_2 \dots x_i x_{i + 1}, \pi_{i + 1} = s_j | \lambda)\\
    &= \sum_{k = 1}^K P(x_1 x_2 \dots x_i, \pi_i = s_k, x_{i + 1}, \pi_{i + 1 = s_j | \lambda})\\
    &= \sum_{k = 1}^K P(x_{i + 1}, \pi_{i + 1} = s_j | x_1 x_2 \dots x_i, \pi_i = s_k, \lambda) P(x_1 x_2 \dots x_i, \pi_i = s_k | \lambda)\\
    &= \sum_{k = 1}^K P(x_{i + 1}, \pi_{i + 1} = s_j | \pi_i = s_k, \lambda) \alpha_i(k)\\
    &= \sum_{k = 1}^K P(\pi_{i + 1} = s_j | \pi_i = s_k, \lambda) P(x_{i + 1} | \pi_{i + 1} = s_j, \lambda) \alpha_i(k)\\
    &= \sum_{k = 1}^K a_{kj} b_j (x_{i + 1})\alpha_i(k)
\end{align*}
\begin{itemize}
	\item where $\alpha_{kj}$ is the transmission probability
	\item $b_j(x_{i + 1})$ is the emission probability for the current observation
	\item $\alpha_i(k)$ is the previous forward variable
	\item The time complexity to compute all forward variables is O($nK^2$) since time for one position and state is $O(K)$ and need to do it for all $nK$ pairs of positions and states.
\end{itemize}

\subsection*{Forward Variable Example}
\begin{center} 
	\includegraphics*[width=\textwidth]{W9_14.png} 
\end{center}
Repeat for $\alpha_2$ and $\alpha_3$ \dots
\begin{center} 
	\includegraphics*[width=\textwidth]{W9_15.png} \\
    \rule{\textwidth}{2pt}
    \includegraphics*[width=\textwidth]{W9_16.png} \\
\end{center}

\subsection*{Computing Backwards Variables}
\[\beta_i(k) = P(x_{i + 1} x_{i + 2} \dots x_n | \pi_i = s_k, \lambda)\]
Backward variable $\beta_i(k)$ is the probability under the model give we are in state $s_k$ at position $i$ that:
\begin{itemize}
	\item We have seen the last $n - i$ observations
\end{itemize}
Base case: $\beta_n (k) = 1$ for $k = 1, \dots, K$.  (By definition, probability of observing nothing after the last position)
\begin{align*}
    \beta_i(k) &= P(x_{i + 1} x_{i + 2} \dots x_n | \pi_i = s_k, \lambda)\\
    &= \sum_{j = 1}^K P(x_{i + 1} x_{i + 2} \dots x_n, \pi_{i + 1} = s_j | \pi_i = s_k, \lambda)\\
    &= \sum_{j = 1}^K P(\pi_{i + 1} = s_j | \pi_i = s_k, \lambda) P(x_{i + 1} x_{i + 2} | \pi_i = s_k, \pi_{i + 1} = s_j, \lambda)\\
    &= \sum_{j = 1}^K P(\pi_{i + 1} = s_j | \pi_i = s_k, \lambda) P(x_{i + 1} x_{i + 2} \dots x_n | \pi_{i + 1} = s_j, \lambda)\\
    &= \sum_{j = 1}^K P(\pi_{i + 1} = s_j | \pi_i = s_k, \lambda) P(x_{i + 1} | \pi_{i + 1} = s_j, \lambda) P(x_{i + 2} \dots x_n | \\pi_{i + 1} = s_j, \lambda)\\
    &= \sum_{j = 1}^K a_{kj} b_j (x_{i + 1}) \beta_{i + 1}(j)
\end{align*}
\begin{itemize}
	\item $a_{kj}$ is the transition probability
	\item $b_j(x_{i + 1})$ is the emission probability for the next observation
	\item $\beta_{i + 1}(j)$ is the next backward variable
	\item The time complexity to compute all backwards variables is O($nK^2$) since time for one position and state is O($K$) and need to do it for all $nK$ pairs of positions and states
\end{itemize}

\subsection*{Backward Variable Example}
\begin{center} 
	\includegraphics*[width=0.9\textwidth]{W9_17.png}\\
    \rule{\textwidth}{2pt}
    \includegraphics*[width=0.9\textwidth]{W9_18.png}\\
\end{center}

\subsection*{State Estimation Example}
\begin{center} 
	\includegraphics*[width=0.9\textwidth]{W9_19.png} 
\end{center}
From this result, we can calculate:
\begin{center} 
	\includegraphics*[width=0.9\textwidth]{W9_20.png} 
\end{center}
\begin{itemize}
	\item One limitation of this is that forward and backward variables can approach limits of floating point representations.  (We're multiplying lots of fractions)
	\item To increase numerical stability, we can first normalize the $\alpha_i$ and $\beta_i$ values by constants.
\end{itemize}

\subsection*{Scaling}
\[P(\pi_{i = s_i | x, \lambda}) = \frac{\alpha_i(k) \beta_i(k)}{\sum_{j = 1}^K \alpha_i(j) \beta_i(j)} = \frac{c_i \times \alpha_i(k) \beta_i(k)}{\sum_{j = 1}^K c_i \times \alpha_i (j) \beta_i(j)} \]
\begin{itemize}
	\item One strategy is to normalize $\alpha_i(j)$ based on the sum of their values after computing them for a position $i$ in the forward step
	\item e.g., replace $\alpha_i(k)$ with $\frac{\alpha_i(k)}{\sum_{j = 1}^K \alpha_i(j)}$
	\item similarly for the $\beta_i(j)$ values in the backwards step, replace $\beta_i(k)$ with $\frac{\beta_i(k)}{\sum_{j = 1}^K \beta_i(j)}$
\end{itemize}

\subsection*{Forward Variable Example with Scaling}
\begin{center} 
	\includegraphics*[width=\textwidth]{W9_21.png}\\
    \rule{\textwidth}{2pt}
    \includegraphics*[width=\textwidth]{W9_22.png} 
\end{center}
The final forward variables are scaled relative to each other, such that they would add to 1.  Repeat this process for every other forward variable.

\subsection*{Computing Data Likelihood}
To compute the data likelihood $P(x | \lambda)$ before scaling, we did:
\begin{itemize}
	\item Forward variables: $\alpha_i(k) = P(x_1 x_2 \dots x_i, \pi_i = s_k | \lambda)$
	\item Backward variables: $\beta_i(k) = P(x_{i + 1} x_{i + 2} \dots x_n | \pi_i = s_k, \lambda)$
	\[P(x | \lambda) = \sum_{j = 1}^K \alpha_i (j) \beta_i(j)\]
    \item For $i = n$, this becomes:
    \[P(x | \lambda) = \sum_{j = 1}^K \alpha_n(j)\]
\end{itemize}
If we have scaling, we now have:
\[P(x | \lambda) = \left(\prod_{i = 1}^n \frac{1}{c_i}\right) \sum_{j = 1}^K \alpha_n(j)\]

\subsection*{Posterior Decoding}
\begin{itemize}
	\item How can we use the state estimates to obtain hard assignments (e.g., 'P' or 'B' assignments?)
	\item We simply take the state with the maximum posterior probability at each position.
\end{itemize}
Recall from earlier:
\begin{center}
\begin{tabular}{ll}
    $P(\pi_1 = P|x, \lambda) = 0$ & $P(\pi_1 = B|x, \lambda) = 1$\\
    $P(\pi_2 = P|x, \lambda) = 0.15$ & $P(\pi_2 = B|x, \lambda) = 0.85$\\
    $P(\pi_3 = P|x, \lambda) = 0.24$ & $P(\pi_3 = B|x, \lambda) = 0.76$
\end{tabular}
\end{center}
Therefore, our posterior decoding is BBB.

\section*{Problem 2: Most Probable Path}
\subsection*{Most Probable Path Given Observations}
\begin{itemize}
	\item What is the most probable path $\pi$ given a sequence of observations $x_1 x_2 \dots x_n$ that is
	\[\arg \max_{\pi} P(\pi | x_1 x_2 \dots x_n, \lambda)\]
    \item Brute force approach:
    \begin{align*}
        &\arg \max_{\pi} P(\pi | x_1 x_2 \dots x_n, \lambda)\\
        &= \arg \max_{\pi} \frac{P(x_1 x_2 \dots x_n, \pi | \lambda)}{P(x_1 x_2 \dots x_n | \lambda)}\\
        &= \arg \max_{\pi} P(x_1 x_2 \dots x_n, \pi | \lambda)\\
        &= \arg \max_{\pi} P(x_1 x_2 \dots x_n | \pi, \lambda) P(\pi | \lambda)
    \end{align*}
\end{itemize}

\subsubsection*{Brute Force Example}
\begin{center} 
	\includegraphics*[width=\textwidth]{W9_23.png} 
\end{center}
We would pick BBB since it has the highest probability.
\begin{itemize}
	\item The time complexity of this approach is O($nK^n$) time.
	\item We need a more efficient approach!
\end{itemize}

\subsection*{Efficient Most Probable Path Computation}
Let's define:
\begin{align*}
    \delta_i(k) &= \max_{\pi_1 \pi_2 \dots \pi_{i - 1}} P(\pi_1 \pi_2 \dots \pi_{i - 1}, \pi_i = s_k, x_1 x_2 \dots x_i | \lambda)\\
    mpp_i(k) &= \arg \max_{\pi_1 \pi_2 \dots \pi_{i - 1}} P(\pi_1 \pi_2 \dots \pi_{i - 1}, \pi_i = s_k, x_1 x_2 \dots x_i | \lambda)
\end{align*}
$mpp_i(k)$ is the path of length $i - 1$ with maximum probability of
\begin{enumerate}
    \item Occurring
    \item Ending up in state $s_k$ at position $i$
    \item Producing output $x_1 x_2 \dots x_i$
\end{enumerate}
$\delta_i(k)$ is probability of $mpp_i(k)$ doing all this.

\subsection*{The Viterbi Algorithm}
Refer to the equations above.\\
Base case $i = 1$ (only one choice for each $k$)
\begin{align*}
    \delta_1(k) &= P(\pi_1 = s_k, x_i | \lambda)\\
    &= P(\pi_1 = s_k | \lambda) P(x_1 | \pi_1 = s_k, \lambda)\\
    &= \tau_k b_k (x_1)
\end{align*}
Inductive case:
\[\delta_{i + 1}(j) = \max_{\pi_1 \pi_2 \dots \pi_i} P(\pi_1 \pi_2 \dots \pi_i, \pi_{i + 1} = s_j, x_1 x_2 \dots x_i x_{i + 1} | \lambda)\]
\begin{center} 
	\includegraphics*[width=\textwidth]{W9_24.png} 
\end{center}
\begin{align*}
    &= \max_{\pi_1 \pi_2 \dots \pi_i} P(\pi_1 \pi_2 \dots \pi_i = s_k, x_1 x_2 \dots x_i | \lambda) P(\pi_{i + 1} = s_j | \pi_i = s_k, \lambda) P(x_{i + 1} | \pi_{i + 1} = s_j, \lambda)\\
    &= \max_k \delta_i(k) a_{kj} b_j (x_{i + 1})
\end{align*}
\begin{itemize}
	\item $\delta_i(k)$ is the previous mpp probability
	\item $a_{kj}$ is the transition probability
	\item $b_j(x_{i + 1})$ is the emission probability for the current observation
\end{itemize}
Now, we define $k^* = \arg \max_{k} \delta_i(k) a_{kj} b_j (x_{i + 1}) mpp_{i + 1}(j) = mpp_{i}(k^*) k^*$
\begin{center} 
	\includegraphics*[width=0.6\textwidth]{W9_25.png} 
\end{center}
Record the maximizing choice for each state $j$ at each position.  To get the Viterbi path, trace back from maximizing choice at the last position.
\begin{itemize}
	\item The time complexity of the Viterbi algorithm is O($nK^2$) since time for one position and state is O($K$) and need to do it for all $nK$ pairs of positions and states.
	\item Note: faster run-times are possible if some transition probabilities are zero.
\end{itemize}

\subsection*{Viterbi Path Example}
\begin{center} 
	\includegraphics*[width=\textwidth]{W9_27.png} \\
    \rule{\textwidth}{2pt}
    \includegraphics*[width=0.9\textwidth]{W9_28.png} \\
\end{center}
To extract the path, we \textbf{record maximizing choice of the last choice, then trace back.}
\begin{center} 
	\includegraphics*[width=0.9\textwidth]{W9_29.png} 
\end{center}

\subsection*{Posterior Decoding vs. Viterbi Path}
\begin{itemize}
	\item Path from posterior decoding can be different from Viterbi path.
	\item Posterior decoding path can have zero probability through model, but still be informative about each position in isolation
	\item Viterbi path could represent very small amount of probability based on all paths
\end{itemize}

\section*{Problem 3: Learning a HMM}
\begin{itemize}
	\item Problem: We are given an observation sequence $x$ and want to learn the HMM parameters (emission, transition, and initial parameters).  For simplicity, we are assuming a single sequence.
	\item There are two settings:
	\begin{itemize}
        \item Supervised Learning: We are also given the true path $\pi$ that generated observation sequence $x$
        \item Unsupervised Learning: We only are given observation sequence $x$
    \end{itemize}
\end{itemize}

\subsection*{HMM Supervised Learning}
\begin{itemize}
	\item Given path $\pi$ and observation sequence $x$ we want to set HMM $\lambda$ parameters to maximize their likelihood.  i.e.,
	\[\arg\max_\lambda P(\pi, x | \lambda)\]
    \item For emission parameters
    \[b_k(\sigma_m) = P(x_i = \sigma_m | \pi_i = s_k)\]
    maximum likelihood estimate is empirical frequency, i.e.,
    \[\frac{\text{number of times in state $k$ and observe symbol $\sigma_m$}}{\text{number of times in state $k$}}\]
    \item For transition parameters
    \[a_{kj} = P(\pi_{i + 1} = s_j | \pi_i = s_k)\]
    maximum likelihood estimate is empirical frequency, i.e.,
    \[\frac{\text{number of times transition from state $k$ to state $j$}}{\text{number of times transition from state $k$}}\]
    \item For initial parameters
    \[\tau_j = P(\pi_i = s_j)\]
    maximum likelihood estimate is empirical frequency, i.e.,
    \[\frac{\text{number of times start a sequence in state $j$}}{\text{number of sequences}}\]
    \item Could optionally use pseudocounts here
\end{itemize}
\begin{center} 
	\includegraphics*[width=\textwidth]{W9_30.png} 
\end{center}
\subsection*{HMM Unsupervised Learning}
\begin{itemize}
	\item Given observation sequence $x$ we want to set HMM $\lambda$ parameters to maximize their likelihood, i.e.
	\[\arg\max_{\lambda} P(x|\lambda)\]
    \item We are trying to estimate the parameters of the HMM just from the observation sequence.  We will assume given the number of state.
    \begin{itemize}
        \item For this example, we will assume there are two states, $s_1$ and $s_2$.
        \item This is analogous to estimating gaussian mixture model cluster parameters for unlabeled data points, or estimating parameters of a PWM from a set of DNA sequences.
    \end{itemize}
\end{itemize}

\subsection*{Unsupervised Learning with EM}
\begin{itemize}
	\item (E-Step) If we had an estimate on the parameters HMM and weren't concerned with time-complexity we could directly compute probability of every path
	\item (M-step) If we had the probability of every path and weren't concerned with time-complexity we could directly compute the most likely parameters of the HMM
	\item This is slow, because we have to do this for every possible sequence of states, which is exponential time.
\end{itemize}

\subsubsection*{Example: Brute Force EM}
Based on the sequence $x$ = AGC, we can fill out the table:
\begin{center} 
	\includegraphics*[width=\textwidth]{W9_31.png} 
\end{center}
Now, we need to re-estimate the HMM parameters.
\begin{itemize}
	\item For emission parameters maximum likelihood estimate of
	\[b_k(\sigma_m) = P(x_i = \sigma_m | \pi_i = s_k)\]
    now becomes:
    \[\frac{\text{expected number of times in state $k$ and observe symbol $\sigma_m$}}{\text{expected number of times in state $k$}}\]
\end{itemize}
For the top left box of the emission parameters:
\begin{align*}
    &\frac{\text{expected number of times in state $B$ and observe symbol $A$}}{\text{expected number of times in state $B$}}\\
    &= \frac{0.737 + 0.109 + 0.024 + 0.129}{3 \times 0.737 + 2 \times 0.109 + 2 \times 0.024 + 1 \times 0.129} = 0.384
\end{align*}
\begin{itemize}
	\item Symbol $A$ appears at $\pi_1$ in the sequence, hence the numerator is the sum of all state $B$ in position 1.
\end{itemize}
Next, for the second from the top box, we get
\begin{align*}
    &\frac{\text{expected number of times in state $B$ and observe symbol $C$}}{\text{expected number of times in state $B$}}\\
    &= \frac{0.737 + 0.024}{3 \times 0.737 + 2 \times 0.109 + 2 \times 0.024 + 1 \times 0.129} = 0.29
\end{align*}
\begin{itemize}
	\item Symbol $C$ appears in the sequence at $\pi_3$ in the sequence, hence the numerator is the sum of all state $B$ in position 3.
\end{itemize}
If we repeat this process for all the cells in the emission table, we get the following:
\begin{center} 
	\includegraphics*[width=0.3\textwidth]{W9_32.png} 
\end{center}
Next, we want to calculate the transitions:
\begin{align*}
    &\frac{\text{expected number of times transition from state $B$ to state $B$}}{\text{expected number of times transition from state $B$}}\\
    &= \frac{2 \times 0.737 + 1 \times 0.109}{2 \times 0.737 + 2 \times 0.109 + 1 \times 0.024 + 1 \times 0.129} = 0.86
\end{align*}
Since $B$ can only transition to $B$ or $P$, and we know the probability for $B$, we can find the probability of transitioning to $P$ since they add to 1  ($P \rightarrow 0.14$).
Repeating the process for $P$,
\begin{align*}
    &\frac{\text{expected number of times transition from state $P$ to state $P$}}{\text{expected number of times transition from state $P$}}\\
    &= \frac{1 \times 0.129}{1 \times 0.024 + 1 \times 0.129} = 0.84
\end{align*}
The probability from $P \rightarrow B$ is 0.16.\\\\
Finally, we want to calculate the starting states.
\begin{itemize}
	\item The probability we start with $B$ is $0.737 + 0.109 + 0.024 + 0.129 = 1$.  (We have a small rounding error in this example.)
    \item The probability we start with $P$ is $0 + 0 + 0 + 0 = 0$.
\end{itemize}
Our final result is:
\begin{center} 
	\includegraphics*[width=\textwidth]{W9_33.png}
\end{center}
Now, we can go back to the E-step.

\subsection*{Efficient E-Step}
How do we make this process efficient?\\\\
For emission parameters $b_k (\sigma_m) = P(x_i = \sigma_m | \pi_i = s_k)$, we need the quantities in the numerator and denominator.
\[\frac{\text{expected number of times in state $k$ and observe symbol $\sigma_m$}}{\text{expected number of times in state $k$}}\]
Recall from the forward-backward algorithm we saw how to efficiently compute the probability we are in state $k$ at position $i$ given the observation sequence, i.e.,
\[P(\pi_i = s_k | x, \lambda)\]
Expected number of times in state $k$:
\[\sum_{i = 1}^N P(\pi_i = s_k | x_1 x_2 \dots x_n, \lambda)\]
We can find the expected number of times in state $k$ and observe $\sigma_m$:
\[\sum_{i s.t. x_i = \sigma_m} P(\pi_i = s_k | x_1 x_2 \dots x_n, \lambda)\]
Now, we need the transition parameters:
\begin{itemize}
	\item The expected number of times transition from state $k$:
	\[\sum_{i = 1}^{n - 1} P(\pi_i = s_k | x_1 x_2 \dots x_n, \lambda)\]
    \item The expected number of times transition from state $k$ to state $j$:
    \[\sum_{i = 1}^{n - 1} P(\pi_i = s_k, \pi_{i + 1} = s_j | x_1 x_2 \dots x_n, \lambda)\]
    \begin{itemize}
        \item Note: we cannot write this as a product of two state estimates since they are not independent.
    \end{itemize}
\end{itemize}
\begin{center} 
	\includegraphics*[width=\textwidth]{W9_34.png} 
\end{center}

\subsection*{Efficient M-step}
\begin{itemize}
	\item For transition parameters maximum likelihood estimate of
	\[a_{kj} = P(\pi_{i + 1} = s_j | \pi_i = s_k)\]
    we have the quantities in the numerator and denominator
    \[\frac{\text{expected number of times transition from state $k$ to state $j$}}{\text{expected number of times transition from state $k$}}\]
    \item For emission parameters
    \[b_k(\sigma_m) = P(x_i = \sigma_m | \pi_i = s_k)\]
    we have the quantities in the numerator and denominator
    \[\frac{\text{expected number of times in state $k$ and observe symbol $\sigma_m$}}{\text{expected number of times in state $k$}}\]
\end{itemize}

\end{document}