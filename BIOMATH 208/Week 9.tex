% document formatting
\documentclass[10pt]{article}
\usepackage[utf8]{inputenc}
\usepackage[left=1in,right=1in,top=1in,bottom=1in]{geometry}
\usepackage[T1]{fontenc}
\usepackage{xcolor}

% math symbols, etc.
\usepackage{amsmath, amsfonts, amssymb, amsthm}
\usepackage{dsfont}

% lists
\usepackage{enumerate}

% images
\usepackage{graphicx} % for images

% code blocks
\usepackage{minted, listings} 

% verbatim greek
\usepackage{alphabeta}

\graphicspath{{./assets/images}}

\newcommand{\solution}{\textbf{Solution:}} 
\newcommand{\example}{\textbf{Example: }}
\newcommand{\R}{\mathbb{R}}
\newcommand{\dd}{\text{d}}

\title{BIOMATH 208 Week 8}

\author{Aidan Jan}
\date{\today}

\begin{document}
\maketitle
\section{Review}
[FILL]

\subsubsection*{Image Registration}
\begin{itemize}
	\item Minimize the sum of square error:
	\[\text{SSE} (T \cdot I, J)\]
    \item $T$ is a translator or affine transformation
    \item For $T$ translator:
    \[\dd f(T) = \int(I(x - T) - J(x)) \cdot \dd I(x - T) \dd x\]
    \item For affine:
    \[\dd f(T) = \int(I(T^{-1}x) - J(x)) \cdot \dd[I(T^{-1}x)]^T \cdot [T^{-1} x]^T \dd x\]
\end{itemize}

\subsubsection*{Metric Manifolds (Riemannian Manifolds)}
[FILL manifold drawing]
\begin{align*}
    L(\gamma) &= \int_0^1 g_{\gamma(t)} (\dot{\gamma} (t), \dot{\gamma}(t)) \dd t\\
    A(\gamma) &= \int_0^1 \text{no square root} \dd t
\end{align*}
\begin{itemize}
	\item Any minimizer of $A$ is also a minimizer of $L$
	\item Any minimizer of $A$ is constant speed
	\item Optimizing over $A$ makes things easier (no square root) and gives a unique parameterization: constant speed geodesics
\end{itemize}

\subsection*{The Constant Speed Geodesic Equation}
In a given coordinate chart, with components of a metric tensor field $g$ written as $g_{ij}$, and its inverse written $g^{ij}$, constant speed geodesics are determined by
\[\ddot{q}^i + \frac{1}{2} g^{ij} (-\partial_j g_{kl} + \partial_k g_{jl} + \partial_l g_{kj}) \dot{q}^k \dot{q}^l = 0\]

\subsubsection*{Proof}
\begin{itemize}
	\item We will consider a path $q(t)$ and a perturbation $q \mapsto q + \epsilon \delta q$.
	\item This denotes a corresponding perturbation in the components of velocity $\dot{q} \mapsto \dot{q} + \epsilon \delta \dot{q}$ (with $\delta \dot{q} = \frac{\dd}{\dd t} \delta q$).
	\item We seek to find stationary solutions via
	\[\frac{\dd}{\dd \epsilon} A(q + \epsilon \delta q) \bigg|_{\epsilon = 0}\]
    \begin{itemize}
        \item e.g., find the directional derivative in the direction of $\delta q$.
    \end{itemize}
\end{itemize}
First, we plug into our equation
\begin{align*}
    &= \frac{\dd}{\dd \epsilon} A(q + \epsilon \delta q) \bigg|_{\epsilon = 0}\\
    \intertext{We'll apply chain rule and get derivatives of $g$}
    &= \frac{\dd}{\dd \epsilon} \int_0^1 g_{ij} (q(t) + \epsilon \delta q(t)) (\dot{q}^i(t) + \epsilon \delta \dot{q}^i(t)) (q^{ij}(t) + \epsilon \delta \dot{q}^{j}(t))
    \intertext{Notice that the last two terms are quadratic in velocity.}
    \intertext{Not trivial: We will apply integration by parts to write things as gradient dot direction for direction = $\delta q$.  e.g., $\delta q(0) = \delta q(1) = 0$ (fixed endpoints).  The boundary terms are zero in integration by parts.}
    &= \int \partial_k g_{ij} (q(t)) \delta g^k(t) \dot{q}^i(t) \dot{q}^j (t) + g_{ij}(q(t)) \delta \dot{q}^i(t) \dot{q}^j(t) + g_{ij} (g(t)) \dot{q}^i(t) \delta \dot{q}^j (t) \dd t
    \intertext{The first term is something acting linearly on the direction.  The next two terms are not, since $\delta \dot{q}^i(t)$ is the time derivative of the direction.}
    &= \int_0^1 \partial_k g_{ij} \dot{q}i (t) \dot{q}^j(t) \delta q^k(t) - \frac{\dd}{\dd t} (g_{ij}(q)\dot{q}^j) \delta q^i - \frac{\dd}{\dd t} (g_{ij}(q) \dot{q}^i) \delta q^j \dd t
    \intertext{Note that $\delta q$ has 3 different indices, but they are just dummy variables}
    &= \int_0^1 \partial_k g_{ij} \dot{q}^i \dot{q}^j \delta q^k - \frac{\dd}{\dd t} (g_{kj}(q) \dot{q}^j) \delta q^k - \frac{\dd}{\dd t} (g_{ik}(q) \dot{q}^i) \delta q^k \dd t\\
    &= \int partial_k g_{ij} (q) \dot{q}^i \dot{q}^j \delta q^k - \partial_i g_{kj}(q) \dot{q}^i \dot{q}^j \delta q^k - g_{kj} (q) \ddot{q} + \delta q^k \\
    &\hspace{0.5cm}- \partial_j g_{ik}(q) \dot{q}^j \dot{q}^i \delta q^k - g_{ik}(q) \dot{q}^i \delta \dot{q}^k \dd t
    \intertext{Now we simplify the equation.  We want to simplify to the form}
    \int_0^1 \text{something} (t)^k \cdot q^k (t) \dd t = 0
    \intertext{Since every term has a $q^k$, we can factor that out.}
    0 &= \partial_k g_{ij} (q) \dot{q}^i \dot{q}^j - \partial_i g_{kj} (q) \dot{q}^i \ddot{q}^j - g_{kj} (q) \ddot{q}^j - \partial_j g_{ik}(g) \dot{q}^j \dot{q}^k - g_{ik}(q) \ddot{q}^i
    \intertext{Now, we can (1) combine the pairs of items that look the same, and (2) act with the matrix inverse of $g$ (sharp map.)}
    0 &= g_{ik}(q) \ddot{g}^i + \frac{1}{2}(-\partial_k g_{ij}(q) + \partial_i g_{kj} (q) + \partial_j g_{ik}(q)) \dot{q}^i \dot{q}^j\\
    0 &= \ddot{q}^l + \frac{1}{2} (g^{-1})^{kl} (-\partial_k g_{ij}(q) + \partial_i g_{kj}(q) + \partial_j g_{ik}(q)) \dot{q}^i \dot{q}^j
\end{align*}
[FILL]

\subsection*{Christoffel Symbols}
We simplify by introducing the notation $\ddot{q}^k + \Gamma_{ij}^k \dot{q}^i \dot{q}^j = 0$.
\textbf{Definition: Christoffel Symbol of the first kind}
\[\Delta_{kij} = \frac{1}{2} (-\partial_k g_{ij} + \partial_i g_{kj} + \partial_j g_{ik})\]
\textbf{Definition: Christoffel symbol of the second kind}
\begin{align*}
    \Gamma_{ij}^l = g^{lk} \Gamma_{kij}\\
    &= \frac{1}{2} g^{lk} (-\partial_k g_{ij} + \partial_i g_{kj} + \partial_j g_{ik})
\end{align*}
Note they are symmetric in the last two indices.

\subsection*{Geodesics in Euclidean Space}
If $g$ is constant everywhere in some chart, then the resulting geodesics are linear equations in this chart
\[q^i(t) = a^i t + b^i\]
\subsubsection*{Proof:}
If $g$ is constant then $\partial_k g_{ij} = 0$ for all $i, j, k$.  The equation becomes
\[\ddot{q}^i = 0\]
Integrating once gives
\[\dot{q}^i = a^i\]
for some $a$.  Integrating again gives the solution for arbitrary $b$.

\subsection*{Geodesics in polar coordinates}
Earlier we showed that the Euclidean dot product in polar coordinates is
\[g_{ij} (r, \theta) = \begin{pmatrix} 1 & 0 \\ 0 & r^2 \end{pmatrix}_{ij}\]
Geodesics in this coordinate system are given by
\begin{align*}
    \ddot{r} - r \dot{\theta}^2 &= 0\\
    \ddot{\theta} + \frac{2}{r} \dot{r} \dot{\theta} &= 0
\end{align*}

\subsubsection*{Proof}
[FILL]

Now, start calculating partial derivatives.  Note they are only nonzero when we take the derivative of the $\theta - \theta$ component ($i = 1, j = 1$), with respect to $r$.
\begin{align*}
    [FILL]
\end{align*}

Now Christoffel symbols of the first kind.
\begin{align*}
    \Gamma_{kij} &= -\frac{1}{2}(-\partial_k g_{ij} + \partial_i g_{kj} + \partial_j g_{ik})\\
    \Gamma_{000} &= 0\\
    \Gamma_{010} &= 0\\
    \Gamma_{100} &= 0\\
    \Gamma_{110} &= \Gamma_{101} = r
    \Gamma_{001} &= [FILL]
    \Gamma_{011} &= 
    \Gamma_{101} &=
    \Gamma_{111} &=
\end{align*}

Now, Christoffel symbols of the second kind
\begin{align*}
    \Gamma_{ij}^l &= g^{lk} \Gamma_{kij}\\ 
    \Gamma_{00}^0 = g^{00} \Gamma_{000} + g^{01} \Gamma_{100} &= 0 \\ 
    \Gamma_{01}^0 = g^{00} \Gamma_{001} + g^{01} \Gamma_{101} &= 0 \\
    \Gamma_{10}^0 = g^{00} \Gamma_{001} + g^{01} \Gamma_{101} &= 0 \\
    \Gamma_{10}^0 = g^{00} \Gamma_{010} + g^{01} \Gamma_{110} &= -r \\
    \Gamma_{00}^1 = g^{10} \Gamma_{000} + g^{11} \Gamma_{100} &= 0 \\
    \Gamma_{01}^1 = g^{10} \Gamma_{001} + g^{11} \Gamma_{101} &= y_r \\
    \Gamma_{10}^1 = g^{10} \Gamma_{001} + g^{11} \Gamma_{101} &= y_r \\
    \Gamma_{10}^1 = g^{10} \Gamma_{010} + g^{11} \Gamma_{110} &= 0
\end{align*}

An example of geodesic curves are shown below:
\begin{center} 
	\includegraphics*[width=\textwidth]{W9_1.png} 
\end{center}

\subsection*{Christoffel Symbols are not Tensors}
Why?  Because a tensor that is 0 in one chart induced basis will be 0 in all others.
\begin{itemize}
	\item Change of basis involved multiplying by Jacobians, these are always invertible matrices (smoothly compatible atlas).
\end{itemize}

\subsection*{Size Data}
Size data is very common in medical imaging, where it is often called volumetry or morphometry.  It gives a means to quantify data in structural images.

\textbf{Geodesics on size}
Consider $\mathcal{M}$ the space of sizes (or the scale group), with the "natural" coordinate chart (i.e., a number in $\R^+$).  Let $g(q) = \frac{1}{q^2}$, and so $g^{-1}(q) = q^2$.  Note this metric is left invariant.  To see this, just push forward vectors with $q^{-1}$.
\[g(q)(u, v) = \frac{1}{q^2}uv = g(I) (\frac{1}{q}u, \frac{1}{q}v)\]
\begin{itemize}
	\item e.g., we push forward, back to identity, with $q^{-1}$.
\end{itemize}
For $a, b \in \R$, geodesics are given by
\[q(t) = \exp(at + b)\]

\subsubsection*{Proof}
First calculate $g^{-1}$
\[\left(\frac{1}{q^2}^{-1}\right) = q^{2} \hspace{1cm} \rightarrow \text{ sharp map}\]
Now, calculate the derivative of $g$.
\[\partial_0 g_{00} = \frac{\dd}{\dd q} \frac{1}{q^2} = -2q^{-3}\]
Now calculate the Christoffel symbol of the first kind.
\[\Gamma_{000} = \frac{1}{2} (-2 q^{-3})\]
Note, in 1D, two of the terms cancel out because of a minus sign.\\\\
Now,calculate the Christoffel symbol of the second kind.
[FILL]

Write the geodesic equation:
\[\ddot{q} - \frac{1}{q} \dot{q}^2 = 0\]
Notice that the coefficients are the Christoffel symbols.\\\\
Show exponentials are a solution:
\begin{align*}
    q(t) &= \exp(at + b)\\
    \dot{q}(t) &= a \exp(at + b)\\
    \ddot{q}(t) &= a^2 \exp(at + b)\\
\end{align*}
Plugging in,
\[a^2 \exp(at + b) - \frac{1}{\exp(at + b)} (a \exp(at + b))^2 = 0\]

\subsection*{Distances on Size Data}
The Riemannian distance between two points $0 < a < b$ is given by the log of their ration
\[d(a, b) = \log(b/a) = \log(b) - \log(a)\]
Note the distance between $sa$ and $sb$ is the same as that between $a$ and $b$ (left invariance).

\subsubsection*{Proof}
Let $q(t)$ be any increasing function with $q(0) = a$ and $q(1) = b$.  Then,
\begin{align*}
    d(a, b) = \int_0^1 \sqrt{\frac{\dot{q}^2(t)}{q^2(t)}} \dd t\\
    [FILL]
\end{align*}

\subsection*{Normal coordinates on sizes}
We chose a base of point 1, and use a chart where
\[x(p) = d(1, p) \text{sign}(p - 1) = \log(p)\]
This chart has the property that the Euclidean distance from $1$ to $p$ in the chart, is the Riemnanian distance on the manifold.\\\\
This property defines "normal coordinates" on a manifold.

\subsection*{Probability data}
Probability data shows up in image segmentation/classification tasks, where an image or a pixel is assigned to a given category with some probability (e.g., foreground vs. background)
\begin{itemize}
	\item Consider $\mathcal{M}$ the space of probabilities, with the "natural" coordinate chart (i.e., a number in (0, 1)).  Let's choose the metric tensor $g(q) = \frac{1}{q^2 (1-q)^2}$ which says that the parameter changes near the endpoints are larger than those near the middle.  Note, $g$ is invariant under $q \mapsto 1 - q$.\\\\
	\item For $a, b \in \mathcal{R}$, geodesics are given by the logistic function (sigmoid)
	\[q(t) = \frac{1}{1 + \exp(at + b)}\]
\end{itemize}




\end{document}