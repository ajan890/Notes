% document formatting
\documentclass[10pt]{article}
\usepackage[utf8]{inputenc}
\usepackage[left=1in,right=1in,top=1in,bottom=1in]{geometry}
\usepackage[T1]{fontenc}
\usepackage{xcolor}

% math symbols, etc.
\usepackage{amsmath, amsfonts, amssymb, amsthm}

% lists
\usepackage{enumerate}

% images
\usepackage{graphicx} % for images

% code blocks
\usepackage{minted, listings} 

% verbatim greek
\usepackage{alphabeta}

\graphicspath{{./assets/images}}

\newcommand{\solution}{\textbf{Solution:}} 
\newcommand{\example}{\textbf{Example: }}
\newcommand{\R}{\mathbb{R}}

\title{BIOMATH 208 Week 5}

\author{Aidan Jan}
\date{\today}

\begin{document}
\maketitle
\subsection*{Review}
A manifold:
\begin{enumerate}
    \item $\mathcal{M}$ a set (often a subset of a bigger space, like a donut)
    \item A topology (open sets, sets that don't include boundary continuous functions)
    \item An atlas which is a set of charts.
    \begin{itemize}
        \item Each chart: $u \subseteq \mathcal{M}$ (coordinate neighborhood)
        \item $x \::\: u \rightarrow \R^d$ (coordinate map)
        \item $x$ must be continuous and invertible and have continuous inverse
        \item "locally" find a subset $u$, can be small $\rightarrow$ "looks like $\R^d$"
    \end{itemize}
    \item Every point on the manifold needs to be in at least one coordinate neighborhood
\end{enumerate}
\begin{center}
    [FILL annotations, drawing with map to x and y]
\end{center}
Compatibility:
\begin{itemize}
    \item Smoothness compatibility: If all chart transition maps are continuously differentiable, the atlas is smoothness compatible, and we have a smooth manifold.
    \item Want smooth manifolds so we can do optimization
    \begin{itemize}
        \item E.g., if $\mathcal{M} = (-1, 1)$ and our charts are $x(p) = p$ and $y(p) = p^3$, this does not work because $y(p)$ is not differentiable at the origin.
    \end{itemize}
\end{itemize}

\section*{Groups}
\subsection*{Motivation}
A common data type in imaging which is not vector valued, are sets called groups.
\begin{itemize}
    \item Typically groups are used to describe transformations, such as those that can be used to align multiple modalities of imaging data.
    \item When the family of transformations we consider also forms a smooth manifold, this is called a Lie (Pronounced: Lee) group.
\end{itemize}
\subsection*{Definition}
A set $G$, together with a binary operations $\circ \::\: G \times G$ is called a group if it satisfies the following properties.  Here, let $f, g, h \in G$:
\begin{itemize}
    \item Associativity: $(f \circ g) \circ h = f \circ (g \circ h) = f \circ g \circ h$
    \item Neutral Element: "Identity Element", $i \in G$, such that $f \circ i = i \circ f = f$
    \item Inverse Element: There exists a "$f^{-1}$", such that $f^{-1} \circ f = f \circ f^{-1} = i$
\end{itemize}
This is very similar to vector spaces with $+$, but there is no C (commutativity).  The $\circ$ operation used here is composition.\\\\
An example of a group is matrix multiplication, because C is missing.
\begin{itemize}
    \item When multiplying matrices, order matters.  Therefore, it is not commutative.
\end{itemize}
\subsection*{Other Properties}
\subsubsection*{Uniqueness of Identity}
There is only one identity.  Suppose that $a$ and $b$ are both identity elements, but are distinct.  Then,
\begin{align*}
    a \circ b &= b \hspace{1.5cm} \text{because a is identity}\\
    a \circ a &= a \hspace{1.5cm} \text{because b is identity}\\
    \intertext{Therefore, by the transistivity of the equals sign,}
    a &= b
\end{align*}
Therefore, $a = b$, which contradicts our assumption that there are two distinct identities.
\subsubsection*{Uniqueness of Inverse}
There is only one inverse.  Suppose that $a$ has two distinct inverses, $b$ and $c$.  Then,
\begin{align*}
    c \circ a \circ b &= (c \circ a) \circ b\\
    &= c \circ (a \circ b)\\
    \intertext{Since c is an inverse of a, we get}
    &= i \circ b
    \intertext{However, since b is an inverse of a, we get}
    &= i \circeq
\end{align*}
Therefore, $b = c$, which contradicts our assumption that there are two distinct inverses.
\subsubsection*{Example: Rotations in 2D}
\[\begin{pmatrix}
    \cos \theta & -\sin \theta \\ \sin \theta & \cos \theta
\end{pmatrix}\]
In this case, $\theta$ and $\theta + 2\pi$ give the same rotations!  However, in terms of groups, they are considered to be the same, since a cosine or sine of adding 2$\pi$ can be simplified.

\subsection*{Cayley Tables}
Rotations about arbitrary angles are infinite groups.\\\\
Definition:
\begin{itemize}
    \item For finite groups, we can list the result of binary operations in a table.  The first input to $\circ$ will be the row, the second input to $\circ$ will be the column, and the result will be in the corresponding cell.
    \begin{center}
        \includegraphics*[scale=0.8]{W5_1.png}    
    \end{center}
    \item Complete representation!  Everything you might want to know about the group.
\end{itemize}

\subsubsection*{Example: One element group}
The simplest group has only one element.  $G = \{a\}$ (the set)
\begin{center}
    \begin{tabular}{c|c}
        $\circ$ & $a$ \\ \hline
        $a$ & $a \circ a = a$
    \end{tabular}
\end{center}
\begin{itemize}
    \item $a$ is identity.
    \item $a = a^{-1}$
\end{itemize}

\subsubsection*{Example: Two element group}
We can build a two element group, for example, modeling reflections
\begin{center}
    \begin{tabular}{c|cc}
        $\circ$ & $a$ & $b$ \\ \hline
        $a$ & $a$ & $a \circ b = b$ \\
        $b$ & $b \circ a = b$ & $b \circ b = a$
    \end{tabular}
\end{center}
We could think of $G = \{1, -1\}$ and $\circ = \cdot$.
\begin{itemize}
    \item For the bottom right corner, $b$ must have an inverse, and it cannot be $a$, therefore it must be $b$.
    \item The same table can represent more than one set and more than one operator.
\end{itemize}

\subsubsection*{Example: Three element group}
We can build a 3 element group, for example, rotations by 120 degrees.
\begin{center}
    \begin{tabular}{c|ccc}
        $\circ$ & 0 & 120 & 240 \\ \hline
        0 & 0 & 120 & 240 \\
        120 & 120 & 240 & 0 \\
        240 & 240 & 0 & 120
        \end{tabular}
\end{center}
We could think of these elements as numbers, and $\circ$ as addition mod 360.\\\\
Or\dots we could think of 0 as $\begin{pmatrix}1 & 0 \\ 0 & 1\end{pmatrix}$, 120 as $\begin{pmatrix}-0.5 & -0.870 \\ 0.870 & -0.5\end{pmatrix}$, etc., and $\circ$ as matrix multiplication.

\subsubsection*{Example: Integers with Plus}
\begin{center}
    \includegraphics[scale=0.9]{W5_2.png}
\end{center}
Every sum of integer is also an integer.  In this case, integers are an infinite group since it is closed if $\circ = +$.  Additionally,
\begin{itemize}
    \item The neutral element of the integer set is 0
    \item The inverse any integer $a$ is $-a$.
\end{itemize}

\subsubsection*{Example: Permutations}
Permutations refer to how we can rearrange the elements.
\begin{center}
    \includegraphics[scale=0.8]{W5_3.png}
\end{center}
These can also be represented as elementary matrices acting by matrix multiplication.  The objects that get transformed are related to the objects doing the transforming.

\subsection*{Group Actions}
Groups often represent transformations, and they therefore act on the objects they transform.
\subsubsection*{Left Group Action}
Let $\mathcal{I}$ be some set of objects we act on, then $\cdot \::\: G \times \mathcal{I} \rightarrow \mathcal{I}$ is called a left group action if it respects group properties.  With $I \in \mathcal{I}$, $f, g \in G$, $i$ = identity $\in G$, we require:
\begin{enumerate}
    \item Identity: $i \cdot I = I$
    \item Compatibility: $ g \cdot (f \cdot I) = (g \circ f) \cdot I$
    \begin{itemize}
        \item LHS of equation: We act on the image twice
        \item RHS of equation: We compose the actions and act on the image once.
    \end{itemize}
\end{enumerate}

\subsubsection*{Right Group Action}
Let $\mathcal{I}$ be some set of objects we act on, then $\cdot \::\: \mathcal{I} \times G \rightarrow \mathcal{I}$ is called a right group action if it respects group properties.  With $I \in \mathcal{I}$, $f, g \in G$, $i$ = identity $\in G$, we require:
\begin{enumerate}
    \item Identity: $I \cdot i = I$
    \item Compatibility: $ (I \cdot f) \cdot g = I \cdot (f \circ g)$
\end{enumerate}
Left and right group actions are the same if the operation is commutative.
\begin{itemize}
    \item This is like multiplication of matrices with vectors (e.g., on the right side), while left group action is like a covector multiplied by a matrix (e.g., on the left side).
\end{itemize}

\subsection*{Permutation and Reflection of Axes}
\begin{itemize}
    \item Discrete images are arrays indexed with three numbers: \texttt{I[i, j, k]}.
    \item Typically, we use a symbol like "RAS" to mean:
    \begin{itemize}
        \item The first axis points from left to right.
        \item The second axis points from posterior to anterior.
        \item The third axis points from inferior to superior.
    \end{itemize}
    \item The permutation group can act on an image (left action) to reorient it: RAS, RSA, ARS, ASR, SRA, SAR.
    \item Permutations and reflections can generate 48 combinations: (R/L, A/P, S/I).
\end{itemize}

\subsection*{Lie Group}
A Lie (pronounced like "lee") group is a group which is also a smooth manifold, which is compatible with its smooth structure.\\\\
Compatible means composition and inverse are differentiable functions of the coordinates.

\subsubsection*{Example: Addition of Reals}
The real numbers with addition is a Lie group.  We know the real numbers form a manifold, so first we can check this is a group:
\begin{itemize}
    \item Associtivity: $(a + b) + c = a + (b + c)$   (and closed under the group operation, e.g., $G \times G \rightarrow G$)
    \item Neutral element: $0$
    \item Inverse element: $a^{-1} = -a$ 
\end{itemize}
Then, we check that its group structure is compatible:  We will pick the natural chart to make this easy, e.g., $x(p) = p$, just use the real number
\begin{itemize}
    \item $\circ$: $\circ (x, y) = x + y$, $\partial_0 \circ(x, y) = 1$, $\partial_1 \circ(x, y) = 1$ 
    \begin{itemize}
        \item These need to be differentiable functions in this chart and in any chart in our smoothly compatible atlas. (which they are)
    \end{itemize}
    \item $^{-1}$: $^{-1}(x) = -x$
\end{itemize}

\subsubsection*{Example: Multiplication of Positive Reals}
The positive real numbers with multiplication is a Lie group.  We know the positive real numbers form a manufold, so first we can check this is a group:
\begin{itemize}
    \item Associativity: $(a \cdot b) \cdot c = a \cdot (b \cdot c)$  (This does meet the closure ($G \times G \rightarrow G$) requirement, since multiplying two positive numbers will yield a positive number.)
    \item Neutral element: $1$
    \item Inverse element: $a^{-1} \cdot \frac{1}{a} \in \R^+$
\end{itemize}
Then we check that its group structure is compatible:
\begin{itemize}
    \item $\circ$:$\circ(x, y) = xy$, $\partial_0 \circ(x, y) = y$, $\partial_1 \circ(x, y) = x$
    \item $^{-1}$: $^{-1}(x) = \frac{1}{x}$, $\partial_0^{-1} (x) = -\frac{1}{x^2}$, which is well defined as long as $x \neq 0$.
\end{itemize}

\subsubsection*{Aside: Is addition of positive reals a lie group?}
$\R^+$ with $+$ (instead of multiplication) is NOT a Lie group, because there is no identity, neither is there an inverse.

\subsection*{Group Homomorphisms}
Let $f$ be a function mapping elements of a group $G$, to elements of a group $H$.  It is called a group homomorphism if it is compatible with the laws of compositions.  If $a, b \in G$, we require:
\[f(a \circ_G b) = f(a) \circ_H f(b)\]
Left side: composition in the group $G$.  Right side: composition in the group $H$.
When we introduced linear maps, we said that it is a function compatible with $+$ and $\cdot$.

\subsubsection*{Example: The Exponential Map}
The exponential function maps $(\R, +) \rightarrow (\R^{+}, \cdot)$.  For $a, b \in \R$, we have
\[\exp(a + b) = \exp(a) \cdot \exp(b)\]
It will be very useful to work with maps like these from a vector space (where it is easy to do computations) to a group (which models our data).

\subsubsection*{Example: Square Matrices}
Square matrices are not a Lie group.\\\\
This is because square matrices have no inverse!  E.g., the 0 matrix doesn't have an inverse.  (All the matrices form a group with $+$).

\subsubsection*{Example: General Linear Groups}
The invertible $n \times n$ matrices do form a Lie group with matrix multiplication.\\\\
To prove:
\begin{enumerate}
    \item This is a smooth manifold
    \item Show that it is a group
    \item Show that matrix multiplication and inverse are differentiable in some chart.
\end{enumerate}
\end{document}