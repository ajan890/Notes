% document formatting
\documentclass[10pt]{article}
\usepackage[utf8]{inputenc}
\usepackage[left=1in,right=1in,top=1in,bottom=1in]{geometry}
\usepackage[T1]{fontenc}
\usepackage{xcolor}

% math symbols, etc.
\usepackage{amsmath, amsfonts, amssymb, amsthm}

% lists
\usepackage{enumerate}

% images
\usepackage{graphicx} % for images

% code blocks
\usepackage{minted, listings} 

% verbatim greek
\usepackage{alphabeta}

\graphicspath{{./assets/images}}

\newcommand{\solution}{\textbf{Solution:}} 
\newcommand{\example}{\textbf{Example: }}
\newcommand{\R}{\mathbb{R}}

\title{BIOMATH 208 Week 3}

\author{Aidan Jan}
\date{\today}

\begin{document}
\maketitle
\section*{Curves and Surfaces in the Brain}
\begin{itemize}
    \item The cortex and subcortical structures of the brain can be well represented by surfaces.
    \item The gyri and sulci of the cortex can be well represented by curves.
\end{itemize}
\begin{center}
    [FILL 5]
\end{center}
\section*{Discrete Curves}
Curves are modeled as piecewise linear, parameterized by a list of vertices, e.g.:
\begin{center}
    [FILL 6]
    [FILL 7]
\end{center}
The first graph is actually in 3D!  Notice how the $z$-coordinate of all the coordinates is zero.

\subsection*{Goal}
\begin{itemize}
    \item Build a vector space structure for discrete curves and surfaces.
    \item Design an inner product and associated norm that is small when two curves (resp. surfaces) and their tangents (normals) are close.
    \item With a norm and inner product, we will be able to use many standard machine learning algorithms.
    \item We follow the approach of "Large Deformation Diffeomorphic Metric Curve Mapping" (2008) or "Surface Matching via Currents" (2005).
\end{itemize}

\subsection*{Curves as Integral Operators}
\textbf{Definition: (Action of a curve on a smooth vector field)}\\
Let our curve be parameterized by a function $\gamma \::\: [0, 1] \rightarrow \mathbb{R}^3$.  Let $v \::\: \R^3 \rightarrow \R^3$ be a smooth vector valued function.  The curve acts:
\[\int_0^1 v(\gamma(t)) \cdot \gamma'(t) \text{d} t\]
where $\cdot$ is the "standard" dot product in $\R^3$.  The term $\gamma'(t)$ is the derivative of $\gamma$ at the point $t$, which can be thought of as a vector tangent to the curve whose magnitude is its speed.

\subsection*{Example: Action close to 0 because far away}
[FILL 9]

\subsection*{Example: Action close to 0 because orthogonal}
[FILL 10]

\subsection*{Example: A large action}
[FILL 11]

\subsection*{Curves are covectors dual to smooth functions}
This action is linear, and therefore curves are covectors.
\begin{itemize}
    \item $+$: [FILL 12]
    \item $\cdot$: [FILL 12]
\end{itemize}

\subsection*{Parameterization Invariance}
If $\gamma \::\: [0, 1] \rightarrow [0, 1]$ is an increasing differentiable function with $\varphi(0) = 0$ and $\varphi(1) = 1$, then a different parameterization could be $\kappa(t) = \gamma(\varphi(t))$.  Parameterization invariance means that
\[\int_0^1 v(\gamma(t)) \cdot \gamma'(t) \text{d}t = \int_0^1 v(\kappa(t)) \cdot \kappa'(t) \text{d}t\]

\subsubsection*{Proof.}
[FILL 13]

\subsection*{Changing Direction}
If $\varphi(t) = 1 - t$ (i.e., we change direction), then a different parameterization could be $\kappa(t) = \gamma(\varphi(t))$.  Then we have, $\kappa = -\gamma$:
\[\int_0^1 v(\gamma(t)) \cdot \gamma'(t) \text{d}t = - \int_0^1 v(\kappa(t)) \cdot \kappa'(t) \text{d}t\]

\subsubsection*{Proof.}
[FILL 14]

\subsection*{Distance between curves}
\begin{itemize}
    \item Two curves are close if their action on all smooth vector fields are similar.  They are (weakly) equal if their action on all smooth vector fields are equal.
    \item How do we deal with the "all" part of this statement?  Consider them all and take the worst one.
\end{itemize}
\textbf{Definition: Operator norm for curves}\\
We will use the operator norm
\[\Vert \gamma \Vert v^* = \sup_{v \in V, \Vert v \Vert_{v = 1}} \gamma(v)\]

\subsection*{The Maximizer}
The maximizer of the previous expression is given by a unit vector in the direction of $\gamma^\sharp$
\[v_{\text{max}} = \frac{\gamma^\sharp}{\vert \gamma^\sharp \vert v}\]

\subsubsection*{Proof.}
[FILL 16]

\subsection*{The Explicit Norm}
Plugging in the maximizer gives
\[\Vert \gamma \Vert v^* = \Vert \gamma^\sharp \Vert v\]
This means we can define a norm on $V^*$, as long as we can define on $V$!
\subsubsection*{Proof.}
[FILL 17]

\subsection*{Why smooth vector fields?  Example: Importance of Smoothness}
[FILL 18]

\subsection*{Enforcing Smoothness}
We will use two approaches to make sure the above cannot occur:
\begin{enumerate}
    \item Parameterize our vector fields as a superposition of smooth functions
    \item Build an inner product so that non-smooth (or rough) functions have an infinite norm
\end{enumerate}

\subsection*{Parameterization}
We restrict ourselves to vector fields that are a superposition of Gaussians, $K(x) = \exp(-\frac{1}{2\sigma^2} |x|^2)$ where $\sigma^2$ controls smoothness.
\[V = \left\{ f \::\: f(x) = \sum_{i = 1}^N p_i K(x - x_i), \hspace{0.5cm} \forall x_i, p_i \in \mathbb{R}^3, \hspace{0.5cm} \forall N \in \mathbb{N}\right\}\]
Any vector field can be parameterized by a list of centers ($x_i$) and positions ($p_i$).
\end{document}