% document formatting
\documentclass[10pt]{article}
\usepackage[utf8]{inputenc}
\usepackage[left=1in,right=1in,top=1in,bottom=1in]{geometry}
\usepackage[T1]{fontenc}
\usepackage{xcolor}

% math symbols, etc.
\usepackage{amsmath, amsfonts, amssymb, amsthm}

% lists
\usepackage{enumerate}

% images
\usepackage{graphicx} % for images

% code blocks
\usepackage{minted, listings} 

% verbatim greek
\usepackage{alphabeta}

\graphicspath{{./assets/images}}

\newcommand{\solution}{\textbf{Solution:}} 
\newcommand{\example}{\textbf{Example: }}
\newcommand{\R}{\mathbb{R}}

\title{BIOMATH 208 Week 4}

\author{Aidan Jan}
\date{\today}

\begin{document}
\maketitle
\section*{Review}
We chose to use the reproducing kernel inner product for our space of smooth functions, and parameterize them as $v(x) = \sum_{i = 1}^N p_i k(x - c_i)$.  ($p_i$ is the direction, not normalized, and $c_i$ is the center of the vector.)  
[FILL]

\subsection*{The Flat Map}
While there are mayn objects included in the dual space, we will focus on the ones that result from the flat map.\\\\
The flat map is given by 
\[\flat(aK(\cdot - x)) = a \delta_x\]
\begin{itemize}
    \item $K$ is a gaussian blob
    \item $a$
\end{itemize}
\subsubsection*{Definition (Linear evaluation functional)}
$\delta_x$ acts linearly on a function, and returns its value at a point.
\[\delta_x(v) = v(x)\]
We define the action of $a\delta_x$ as
\[a\delta_x(v) = a \cdot v(x)\]
\begin{itemize}
    \item The $\cdot$ is a dot product in $\R^3$.
\end{itemize}
\subsubsection*{The flat map for smooth vector fields}
We can expand our definition using linearity
\[\flat\left(\sum_i a_i K(\cdot - x_i)\right) = \sum_i \flat(a_i K(\cdot - x_i)) = \sum_i a_i \delta_{x_i}\]
\subsubsection*{Proof}
This flat map is the one defined by our inner product
\begin{align*}
    [FILL]
\end{align*}

\subsection*{The Sharp Map}
By definition, the sharp map is the inverse of the flat map:
\[\sharp(a \delta_x) = aK(\cdot - x)\]
It is extended to all ("nice") linear evaluation functionals by linearity.

\subsection*{Discrete Line Integrals}
Approximate our curve $\gamma$ with a sequence of points $x_1, \dots, x_N$.  The center of the $i$th edge is $c_i \frac{x_i + x_{i + 1}}{2}$ for $i \in \{1, \dots, N - 1\}$.  The tangent to the $i$th edge is $\tau_i = x_{i + 1} - x_i$.  Then:
\[\gamma(v) = \int v(\gamma(t)) \cdot \gamma'(t) \text{d}t \simeq [FILL]\]
[FILL]

\subsection*{Integrals as evaluation functionals}
We can rewrite this as:
\[\sum_{i = 1}^{N - 1} v(c_i) \cdot \tau_i = \left(\sum_{i = 1}^{N - 1} \tau_i \delta_{c_i}\right)(v)\]
[FILL]

\subsection*{Sharp map for discrete curves}
If $\gamma$ is a discrete curve, then its sharp map is given by
\[\delta^\sharp(x) = \sum_{i = 1}^N \tau_i K(x - c_i)\]
[FILL]

\subsection*{Inner product for discrete curves}
Let $\mu = \sum_{i = 1}^n^{\mu - 1}$... [FILL]

\subsection*{Distance between discrete curves}
The distance between two curves is the norm of their difference.
\begin{align*}
    &\Vert \mu - \nu \Vert^2_{V^*}\\
    &= g_{V^*}(\mu, \mu) - 2g_{V^*}(\mu, \nu) + g_{V^*}(\nu, \nu)\\
    &= \sum_{i, i' = 1}^{n^\mu - 1} K [FILL]
\end{align*}

\section*{The Corpus Callosum}
\begin{center}
    [FILL IMG]
\end{center}
\begin{itemize}
    \item One of the most well-studied parts for the brain is the Corpus Callosum, because it contains such a large amount of white matter
    \item In the past, a common treatment for epilepsy was to sever the Corpus Callosum, as it prevents positive feedback loops between the two sides.
    \item Its shape changes depending on different diseases, phenotypes, etc.
\end{itemize}

\section*{Review - Curve Fitting and Interpolation}
Consider a curve fitting problem:  You have a lot of data points.  The goal is to find a "nice" $f(x)$ that passes through my data.\\\\ 
What does "nice" mean?  It is a curve with no cusps or discontinuities.\\\\
To do this, we first find the minimizer of $f$.
\[\text{argmin}_{f \in V} \langle f, f\rangle_v \text{ such that } f(x_i) = y_i \hspace{1cm} \forall i \in \{1, \dots, N\}\]
This is a costumed plurization, use Lagrange multipliers $p_i$.
\begin{align*}
    L &= \sum_{i = 1}^N p_i \cdot (y_i - f_i) + \langle f, f \rangle_v\\
    &= \sum_{i = 1}^N p_i f(x_i) + \frac{1}{2}\langle f, f \rangle_v + \sum_{i = 1}^N p_i y_i
\end{align*}
for a fixed $p$, find the best $f$.
\begin{align*}
    &= \sum_{i = 1}^N [FILL]
\end{align*}
Therefore, the optimal $f$ is $f(x) = \sum_{i = 1}^N p_i K(x - x_i)$.
\begin{itemize}
    \item We now need to solve for $p$.
    \item In these types of problems, $p$ is a lagrange multiplier, so we would have to refer to the constraints.
\end{itemize}
$y_{j} = f(x_j) = \sum_{i = 1}^N p_i K(x_j - x_i)$
solve for $p$ by solving linear equations!

\section*{Smooth Manifolds}
Motivation: Many useful data types in medical imaging are not elements of a vector space.  (Not closed under $+$ and $\cdot$.)
\begin{itemize}
    \item Rotation matrices
    \item Diffusion tensors
    \item Probabilities
\end{itemize}
We can still analyze them quantitatively by modeling them as elements of a manifold.\\\\
We will discuss two main types of data
\begin{enumerate}
    \item Pixels that are manifold valued objects
    \item Manifold valued objects that act on imaging data
\end{enumerate}
\subsection*{Intuition for Smooth Manifolds}
A manifold is a set (possibly curved), such that if you zoom in close it looks like a (flat) vector space (i.e., $\mathbb{R}^d$ for some $d$).
\begin{itemize}
    \item A classic example is a sphere like the earth.  When we walk around in a small area it looks flat.
\end{itemize}


\subsection*{Example - Not Manifolds}
\begin{center}
    [FILL]
\end{center}

\subsection*{Definition of a Smooth Manifold}
A smooth manifold is a triple
\begin{enumerate}
    \item A set $\mathcal{M}$
    \item A topology $\mathcal{O}$
    \item A collection of smoothly compatible charts called an atlas $\mathcal{A}$, where every point is in at least one chart.
\end{enumerate}

\subsection*{Topologies}
\begin{itemize}
    \item We will not cover topologies in detail in lecture.  Please see the notes if you are interested.
\end{itemize}
Working definition of topologies:  We can think of a topology as a collection of open sets (including $\mathcal{M}$ and $\emptyset$), that allow us to define continuous functions:
\begin{itemize}
    \item A function $f$ is continuous if the inverse image of any open set is also open set.
\end{itemize}

\subsection*{Charts}
Charts will make precise what "looks like $\mathbb{R}^d$ means.
Definition: A chart is a pair $(U, x)$ in $\mathcal{A}$, where $U$ is an open subset of $\mathcal{M}$
[FILL]


\end{document}