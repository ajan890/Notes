% document formatting
\documentclass[10pt]{article}
\usepackage[utf8]{inputenc}
\usepackage[left=1in,right=1in,top=1in,bottom=1in]{geometry}
\usepackage[T1]{fontenc}
\usepackage{xcolor}

% math symbols, etc.
\usepackage{amsmath, amsfonts, amssymb, amsthm}

% lists
\usepackage{enumerate}

% images
\usepackage{graphicx} % for images

% code blocks
\usepackage{minted, listings} 

% verbatim greek
\usepackage{alphabeta}

\graphicspath{{./assets/images}}

\newcommand{\solution}{\textbf{Solution:}} 
\newcommand{\example}{\textbf{Example: }}
\newcommand{\R}{\mathbb{R}}

\title{BIOMATH 208 Week 4}

\author{Aidan Jan}
\date{\today}

\begin{document}
\maketitle
\section*{Review}
\subsection*{Curves and Surfaces}
On a computer:
\begin{itemize}
    \item list of points (vertices)
    \item list of connectivity elements (line segments, triangles)
    \begin{itemize}
        \item Integers (2 for segment, 3 for traingle), which are indices for the array of vertices
        \item orientation matters (we often want consistent tangent or normal vectors)
    \end{itemize}
\end{itemize}
In a vector space:
\begin{itemize}
    \item consider them as integral operators, compute path or flux integrals when acting on a smooth vector field.
    \item Therefore, these are covectors in the space dual to smooth vector fields
    \item We looked at the operator norm 
    \[|\delta|_{v^*} = \sup_{v \in V, |v|_v = 1} \gamma(v) = \Vert \gamma^\sharp \Vert_v\]    
    \item We can model smooth functions instead of weird curves
\end{itemize}
We chose to use the reproducing kernel inner product for our space of smooth functions, and parameterize them as $v(x) = \sum_{i = 1}^N p_i k(x - c_i)$.  ($p_i$ is the direction, not normalized, and $c_i$ is the center of the vector.), for any $N \in \mathbb{N}$.
\begin{itemize}
    \item $k$ as a gaussian with a fixed width 
\end{itemize}
The reproducing kernel inner product is:
\[\langle ak(\cdot - a), bk(\cdot - y) \rangle_v = a \cdot b k(x - y)\]
\begin{itemize}
    \item The $\cdot$ on the right hand side is a dot product in $\mathbb{R}^3$.
    \item This shows that non-smooth functions will have an infinite norm.
\end{itemize}

\subsection*{The Flat Map}
While there are many objects included in the dual space, we will focus on the ones that result from the flat map.\\\\
The flat map is given by 
\[\flat(aK(\cdot - x)) = a \delta_x\]
\begin{itemize}
    \item $K$ is a gaussian blob
    \item $a$ is a vector
    \item $x$ is a center
\end{itemize}
\subsubsection*{Definition (Linear evaluation functional)}
$\delta_x$ acts linearly on a function, and returns its value at a point.
\[\delta_x(v) = v(x)\]
We define the action of $a\delta_x$ as
\[a\delta_x(v) = a \cdot v(x)\]
\begin{itemize}
    \item The $\cdot$ is a dot product in $\R^3$.
\end{itemize}
\subsubsection*{The flat map for smooth vector fields}
We can expand our definition using linearity
\[\flat\left(\sum_i a_i K(\cdot - x_i)\right) = \sum_i \flat(a_i K(\cdot - x_i)) = \sum_i a_i \delta_{x_i}\]
\subsubsection*{Proof}
This flat map is the one defined by our inner product
\begin{align*}
    \flat(aK(\cdot - x)) &= a \delta_x \\
    \intertext{By the definition of inner product,}
    \langle ak(\cdot - x), bk(\cdot - y)\rangle_v &= a \cdot b k(x - y)\\
    &= b(ak(\cdot - x))(bk(\cdot - y))\\
    &= a \delta_x(bk(\cdot - y))
\end{align*}

\subsection*{The Sharp Map}
By definition, the sharp map is the inverse of the flat map:
\[\sharp(a \delta_x) = aK(\cdot - x)\]
It is extended to all ("nice") linear evaluation functionals by linearity.

\subsection*{Discrete Line Integrals}
Approximate our curve $\gamma$ with a sequence of points $x_1, \dots, x_N$.  The center of the $i$th edge is $c_i \frac{x_i + x_{i + 1}}{2}$ for $i \in \{1, \dots, N - 1\}$.  The tangent to the $i$th edge is $\tau_i = x_{i + 1} - x_i$.  Then:
\[\gamma(v) = \int v(\gamma(t)) \cdot \gamma'(t) \text{d}t \simeq \sum_{i=1}^{N-1} v(c_i)\cdot \tau_i\]
\begin{itemize}
    \item Centers: average of two consecutive points
    \item Tangent: difference between two consecutive points
    \item This is like a riemann sum.
    \item We can evaluate the right side of the equation (the summation) with evaluation functionals.
\end{itemize}

\subsection*{Integrals as evaluation functionals}
We can rewrite this as:
\[\sum_{i = 1}^{N - 1} v(c_i) \cdot \tau_i = \left(\sum_{i = 1}^{N - 1} \tau_i \delta_{c_i}\right)(v)\]
The right side element $\left(\sum_{i = 1}^{N - 1} \tau_i \delta_{c_i}\right) \in V^*$, and is a discrete curve that can be written as a "nice" covector, a weighted sum of evaluation functions. 

\subsection*{Sharp map for discrete curves}
If $\gamma$ is a discrete curve, then its sharp map is given by
\[\delta^\sharp(x) = \sum_{i = 1}^N \tau_i K(x - c_i)\]
\begin{itemize}
    \item We just replaced the $\delta_x$ with a $K$.
    \item In the below image, the vector field (black) is the sharp map applied to our curve.
\end{itemize}
\begin{center}
    \includegraphics*[scale=0.7]{W4_1.png}
\end{center}


\subsection*{Inner product for discrete curves}
Let $\mu = \sum_{i = 1}^{n^{\mu} - 1} \tau_i^\mu \delta_{c_i^\mu}$ and $\nu = \sum_{i = 1}^{n^\nu - 1} \tau_i^\nu \delta_{c_i^\nu}$.  The inner product is:
\[g_{V^*} (\mu, \nu) = \sum_{i = 1}^{n^\mu - 1} \sum_{j = 1}^{n^\nu - 1} K(c_i^\mu - c_j^\nu)(\tau_i^\mu \cdot \tau_j^\nu)\]
This equation works due to bilinearity, since
\[\langle\tau_1 \delta_{c_1}, \tau_2\delta_{c_2}\rangle = \langle \tau_1 k(\cdot - c_1), \tau_2 k(\cdot - c_2) \rangle_v = \tau_1 \cdot \tau_2 k(c_1 - c_2)\] 
The double sum we take in this formula represents the sum of all possible pairs of $i$ and $j$.  Think: nested for loops.

\subsection*{Distance between discrete curves}
The distance between two curves is the norm of their difference.
\begin{align*}
    &\Vert \mu - \nu \Vert^2_{V^*}\\
    &= g_{V^*}(\mu, \mu) - 2g_{V^*}(\mu, \nu) + g_{V^*}(\nu, \nu)\\
    &= \sum_{i, i' = 1}^{n^\mu - 1} K(c_i^\nu - c_{i'}^\nu)(\tau_i^\nu \cdot \tau_{i'}^\nu) \\
    &~~- 2 \sum_{i = 1}^{n^\mu - 1} \sum_{j = 1}^{n^\nu - 1} K(c_i^\mu - c_j^\nu)(\tau_i^\mu \cdot \tau_j^\nu)\\
    &~~+ \sum_{j, j' = 1}^{n^\nu - 1} K(c_j^\nu - c_{j'}^\nu)(\tau_j^\nu \cdot \tau_{j'}^\nu)
\end{align*}
We are essentially summing
\begin{itemize}
    \item All the pairs in the first curve, $g_{V^*}(\mu, \mu)$,
    \item all the pairs between the two curves, $2g_{V^*}(\mu, \nu)$,
    \item and all the pairs in the second curve, $g_{V^*}(\nu, \nu)$
\end{itemize}


\section*{The Corpus Callosum}
\begin{center}
    \includegraphics*[width=\textwidth]{W4_2.png}
\end{center}
\begin{itemize}
    \item One of the most visible and well-studied parts for the brain is the Corpus Callosum, because it contains such a large amount of white matter
    \item In the past, a common treatment for epilepsy was to sever the Corpus Callosum, as it prevents positive feedback loops between the two sides.
    \item Its shape changes depending on different diseases, phenotypes, etc.
\end{itemize}

\section*{Review - Curve Fitting and Interpolation}
Consider a curve fitting problem:  You have a lot of data points.  The goal is to find a "nice" $f(x)$ that passes through my data.\\\\ 
What does "nice" mean?  It is a curve with no cusps or discontinuities.\\\\
To do this, we first find the minimizer of $f$.
\[\text{argmin}_{f \in V} \langle f, f\rangle_v \text{ such that } f(x_i) = y_i \hspace{1cm} \forall i \in \{1, \dots, N\}\]
This is a costumed plurization, use Lagrange multipliers $p_i$.
\begin{align*}
    L &= \sum_{i = 1}^N p_i \cdot (y_i - f_i) + \langle f, f \rangle_v\\
    &= \sum_{i = 1}^N p_i f(x_i) + \frac{1}{2}\langle f, f \rangle_v + \sum_{i = 1}^N p_i y_i
\end{align*}
for a fixed $p$, find the best $f$.  (Notice the last term in the above equation is not dependent on $f$.)
\begin{align*}
    &= \sum_{i = 1}^N P_i \delta_{x_i} (f) + \langle f, f \rangle_v + \sum_{i = 1}^N p_i y_i\\
    &= \langle \sum_{i = 1}^N p_i \cdot K(\cdot - x_i), f \rangle_v + \langle f, f \rangle_v + \cdots\\
    &= \langle g, f \rangle_v + \langle f, f \rangle_v + \cdots
\end{align*}
We can solve this by completing the square.  The result would look something like
\[\langle f - g, f - g \rangle_v + \cdots\]
The constants of this equation are independent of $f$.  This equation is minimized when $f = g$.\\\\
Therefore, the optimal $f$ is $f(x) = \sum_{i = 1}^N p_i K(x - x_i)$.
\begin{itemize}
    \item We now need to solve for $p$.
    \item In these types of problems, $p$ is a lagrange multiplier, so we would have to refer to the constraints.
\end{itemize}
$y_{j} = f(x_j) = \sum_{i = 1}^N p_i K(x_j - x_i)$
\begin{itemize}
    \item $p_i$ is a N by 1 array
    \item $k(x_j - x_i)$ is an N by N array.
    \item Therefore, the right side can be thought of as matrix multiplication.
\end{itemize}
solve for $p$ by solving linear equations!

\section*{Smooth Manifolds}
Motivation: Many useful data types in medical imaging are not elements of a vector space.  (Not closed under $+$ and $\cdot$.)
\begin{itemize}
    \item Rotation matrices
    \item Diffusion tensors
    \item Probabilities
\end{itemize}
We can still analyze them quantitatively by modeling them as elements of a manifold.\\\\
We will discuss two main types of data
\begin{enumerate}
    \item Pixels that are manifold valued objects
    \item Manifold valued objects that act on imaging data
\end{enumerate}
\subsection*{Intuition for Smooth Manifolds}
A manifold is a set (possibly curved), such that if you zoom in close it looks like a (flat) vector space (i.e., $\mathbb{R}^d$ for some $d$).
\begin{itemize}
    \item A classic example is a sphere like the earth.  When we walk around in a small area it looks flat.
\end{itemize}

\subsection*{Example - Not Manifolds}
\begin{enumerate}
    \item A line segment:
    \begin{itemize}
        \item Imagine you are standing at the end of a line segment.  You see a cliff!  The segment ends, but $\mathbb{R}$ doesn't.
    \end{itemize}
    \item An "x":
    \begin{itemize}
        \item Consider the intersection between two lines.  If you zoom in on the intersection, it always looks the same!  It does not smooth out.
    \end{itemize}
\end{enumerate}

\subsection*{Definition of a Smooth Manifold}
A smooth manifold is a triple
\begin{enumerate}
    \item A set $\mathcal{M}$
    \item A topology $\mathcal{O}$
    \item A collection of smoothly compatible charts called an atlas $\mathcal{A}$, where every point is in at least one chart.
\end{enumerate}

\subsection*{Topologies}
\begin{itemize}
    \item We will not cover topologies in detail in lecture.  Please see the notes if you are interested.
\end{itemize}
Working definition of topologies:  We can think of a topology as a collection of open sets (including $\mathcal{M}$ and $\emptyset$), that allow us to define continuous functions:
\begin{itemize}
    \item A function $f$ is continuous if the inverse image of any open set is also open set.
\end{itemize}

\subsection*{Charts}
Charts will make precise what "looks like $\mathbb{R}^d$ means.
Definition: A chart is a pair $(U, x)$ in $\mathcal{A}$, where $U$ is an open subset of $\mathcal{M}$ and $x \::\: \mathcal{M} \rightarrow \mathbb{R}^d$ is a continuous and invertible map, with continuous inverse (a homeomorphism), called the coordinate map, for some $d \in \mathbb{N}$.
Definitiion: $d$ is called the dimension of the manifold.

\subsection*{Components}
Definition: We call $x^i$ the $i$-th component map.  It can be thought of as first applying $x$, then identifying the $i$-th component in a basis $x^i = \epsilon^i \circ x$. \textcolor{red}{Note components have an index up!}
\begin{itemize}
    \item We will always choose a standard basis for $\epsilon^i$.
    \item $x$ will input a point on the manifold and output a list of $d$ numbers
    \begin{itemize}
        \item Each item on the list is a component.
    \end{itemize}
    \item Note that a set here refers to a set of functions (which vary with $p$, the point picked), rather than a basis which is a set of vectors.
\end{itemize}

\subsection*{Example: The Open Interval}
Consider the interval $(0, 1) \in \mathbb{R}$.  We can choose $x$ as the "natural" chart.  If $p \in \mathcal{M}$m then $x(p) = p$, and the domain is all of $\mathcal{M}$
\begin{itemize}
    \item We are essentially choosing $x(p)$ as a point on a number line from 0 to 1.  We can represent $x(p)$ by the number $p$.
\end{itemize}
Or we could use the logit (log of odds) chart, $y(p) = \log \left(\frac{p}{1 - p}\right)$.  The domain is again all of $\mathcal{M}$, and the image is all of $\mathbb{R}$. 
\begin{itemize}
    \item In these lectures, log refers to natural log.
\end{itemize}
Note the inverse:
\begin{itemize}
    \item If we represent data this way, we don't have to worry about constraints.
    \[y'(q) = \frac{1}{1 + e^{-q}}\]
    \item This is a (sigmoid) logistic function, derived from the $y(p)$ equation above, setting $y(p) = q$, and solving for $p$.
    \item Any function can be used, this is just an example, as long as the function has a domain from 0 to 1, and is continuous.
\end{itemize}

\subsection*{Example: The Plane}
Consider the ordered tuples in $\mathbb{R}^2$.  We could use the "natural" chart with $x(p) = 0$.  Here, $x^0(p) = p^0$, and $x^1(p) = p^1$.  The domain is all of $\mathcal{M}$.\\\\
Or, we could consider rotated and translated charts.
\begin{align*}
    y^0 &= \cos(\theta) p^0 + \sin(\theta) p^1 + h\\
    y^1 &= -\sin(\theta) p^0 + \cos(\theta) p^1 + k
\end{align*}
for $\theta \in [0, 2\pi]$ and $h, k \in \R$.
\begin{itemize}
    \item $(h, k)$ is a shift, $\theta$ is a rotation.
    \item Given one chart, we could immediately construct an infinite amount of new charts!
    \item Our atlas and charts will never be unique.
\end{itemize}
Or, we could consider the "polar" chart, with
\[z(p) = \left(\sqrt{(p^0)^2 + (p^1)^2}, \text{sign}(p^1) \arccos\left(\frac{p^0}{\sqrt{(p^0)^2 + (p^1)^2}}\right)\right)\]
\begin{itemize}
    \item This coordinate, $z(p)$ is in the format, (radius, angle).
\end{itemize}
\begin{center}
    \includegraphics*[scale=0.9]{W4_3.png}
\end{center}
Here, the domain is $U = \mathbb{R}^2$, minus the "nonpositve $x$-axis".  Since it doesn't cover $\mathcal{M}$, this can't be the only chart in our atlas.
\begin{itemize}
    \item Notice that the theta graph is noncontinuous.  Also at the origin, it is not invertible.
\end{itemize}

\subsection*{Example: The Circle}
Consider a circle with radius 1, centered at the origin in the Cartesian plane:
\[S = \{p \in \R^2\::\: (p^0)^2 + (p^1)^2 = 1\}\]
We will cover the circle with two "polar" charts.\\\\
First, $U = S \backslash (-1, 0)$, $x(p) = \text{sign}(p^1) \arccos(p^0)$.  This is the signed angle counterclockwise from positive $x$-axis.\\
Second, $V = S \backslash (0, -1)$, $y(p) = \text{sign}(p^0) \arccos(o^1)$.  Signed angle clockwise from positive $y$-axis.
\begin{itemize}
    \item The backslash (\textbackslash) represents subtraction of sets.
    \begin{itemize}
        \item e.g., $U$ covers every point on the circle except (-1, 0).
    \end{itemize}
    \item With these two definitions, $U \cup V = \mathcal{M}$.  This makes a nice atlas for the circle using continuous and invertible functions.
\end{itemize}
For another insightful pair of charts, consider the circle sitting on the real line, with a lightbulb on top.  We assign a coordinate where our point casts a shadow.  For points in $U = S \backslash (0, 1)$, we have
\[z(p) = \frac{2p^0}{1 - p^1}\]
\begin{center}
    \includegraphics*[scale=0.9]{W4_4.png}
\end{center}
\begin{itemize}
    \item Here, the bottom half of the circle would cause a shadow between $-2 < x < 2$, and the points on the top half of the circle going to infinity on both sides
    \item The only point not responsible for a shadow is (0, 2)
\end{itemize}
This gives insight as to the meaning of $\pm \infty$!  It's just the one point missing from the top of the circle.
\begin{itemize}
    \item Importantly, this mapping maps all of $\mathbb{R}$ to a circle!  (With one point gone).
\end{itemize}
\subsubsection*{Question: Can the circle be covered with one chart?}
Answer: NO!\\\\
This is pretty easy to prove.  Suppose that there is a function $x$ that maps a circle to $\mathbb{R}$, which is continuous and invertible on the whole circle.
\begin{itemize}
    \item Consider if you remove one point from the circle.  
    \item The circle is still a connected set, with one point removed.
    \item However, $\mathbb{R}$ must have a discontinuity if the point was removed.
    \item Since continuity is always preserved by all continuous functions, the function must not be continuous.  This is a contradiction, and therefore, the mapping cannot exist.
\end{itemize}

\subsection*{Example: The Hemisphere}
Consider the surface of a human skull as a hemisphere,
\[\mathcal{M} = \{p \in \mathbb{R}^3 \::\: (p^0)^2 + (p^1)^2 + (p^2)^2 = 1, p^2 > 0\}\]
\subsubsection*{Example: Parallel Projections}
Imagine a camera high above the patient's head, recording their skull with a video camera.  Points on the head will map to the screen using their "xy coordinate".  This chart can be written as $(U, x)$, where $U = \mathcal{M}$, and $x(p) = (p^0, p^1)$.
\begin{itemize}
    \item The head will look like a circle.  
    \item We don't consider the boundary of the head as part of our set, because it will not be a function with respect to the $z$-axis.
    \begin{itemize}
        \item Essentially, the sides of the head are perpendicular to our camera's view, which violates the vertical line test.
    \end{itemize}
\end{itemize}
In this case, the inverse of $x(p)$ can be written as:
\[x^{-1}(q) = (q^0, q^1, \sqrt{1 - (q^0)^2 - (q^1)^2})\]
The $z$-coordinate represents the height of the pixel.

\subsubsection*{Example: Xray Projection}
Imagine a radiotracer in the center of their head.  A detector (the plane $p^2 = 1$) lies above their head and forms an image of their skull.  This corresponds to the chart (V, y), where $V = \mathcal{M}$ and $y = (p^0 / p^2, p^1 / p^2)$
\begin{center}
    \includegraphics*[scale=0.8]{W4_5.png}
\end{center}
Note this maps he hemisphere to all of $\mathbb{R}^2$, in a similar way the lightbulb on a circle worked.

\subsubsection*{Example: Latitude and Longitude}
Since this is a hemisphere (like part of the earth), it might be natural to work with latitude and longitude.  This corresponds to the chart (W, z), where $W = \mathcal{M} \backslash$ "nonpositve $x$ axis", and 
\[z = \left(\frac{\pi}{2} - \arccos(p^2), \text{sign}(p^1) \arccos(\frac{p^0}{\sqrt{(p^0)^2 + (p^1)^2}})\right)\]
This cannot be our only chart.

\subsubsection*{Summary}
We build three very different charts to describe the same (hemisphere) object.  First we thought of the hemisphere as a graph of a function of two variables.  Then, a map onto all of $\R^2$.  Then, a map onto a finite subset of $\R^2$.  (angles)

\subsection*{Example: The Sphere}
The sphere also cannot be covered with one chart.  One classic approach is to cover it with 6 parallel projection hemisphere charts.
\begin{itemize}
    \item $+x, -x$
    \item $+y, -y$
    \item $+z, -z$
\end{itemize}
Another is latitude and longitude (as above), but we'll have to choose two neighborhoods.\\\\
Another choice is to place a light on the north pole, and cast the shadow of a point onto the plane:
\[x(p) = (2p^0 / (1 - p^2), 2p^1 / (1 - p^2))\]
with $U$ the sphere minus the north pole.\\\\
A fun choice is to use the arcsine function sine $p^i \in (-1, 1)$, $(\mathcal{M}, y)$ where $y = (\arcsin(p^0), \arcsin(p^1))$
\begin{itemize}
    \item You end up with the same problem with the lightbulb on the circle.
\end{itemize}

\subsection*{Atlases}
An atlas is a collection of charts (coordinate neighborhood and homeomorphism to $\R^2$), such that every point on the manifold is in at least one chart.
\subsection*{Chart Transition Maps}
This is essentially a change of basis.  For any two chart (U, x), (V, y) with $U \cap V \neq \emptyset$, the mappings $x \circ y^{-1}$ and $y \circ x^{-1}$ are called chart transition maps.  Note these are maps from $\R^d \rightarrow \R^d$.  There is no $\mathcal{M}$ involved, and these can be studied with regular calculus.

\subsection*{Compatible Atlases}
For a property $\mathcal{P}$, two charts $x$ and $y$ are called $\mathcal{P}$-compatible if $x \circ y^{-1}$ and $y \circ x^{-1}$ have the property $\mathcal{P}$.  If every chart transition map has property $\mathcal{P}$, this is called a $\mathcal{P}$-compatible atlas.

\subsubsection*{Smooth manifolds}
A smooth manifold is one where every chart transition map is a smooth map.  That is, its derivative is a smooth function from $\R^d \rightarrow \R^d$.
\begin{itemize}
    \item We want continuous derivatives so we can do gradient based optimization.
\end{itemize}

\subsection*{Example: The Interval}
\subsubsection*{Example: Smoothly Compatible Charts}
Using two charts: $(, x)$ and $(V, y)$, with $x(p) = p$ and $y(p) = \log(p / (1 - p))$, we have:
\begin{itemize}
    \item $x \circ y^{-1}(u) = \frac{1}{1 + e^{-u}}$ 
    \item $y \circ x^{-1}(u) = \log \frac{u}{1 - u}$
\end{itemize}
These are both smooth maps from $\R \rightarrow \R$.

\subsubsection*{Example: Incompatible Atlas}
Consider a second chart $(W, z)$, with $z(p) = p^3$ in addition to the first chart, $x(p) = p$.  Then,
\begin{itemize}
    \item $x \circ z^{-1}(u) = u^{1/3}$
    \item $z \circ x^{-1}(u) = u^3$
\end{itemize}
These are not both differentiable everywhere.

\subsection*{Example: The Plane}
\subsubsection*{Example: Smoothly Compatible Charts}
Consider the identity chart $(U, x)$ for $U = \mathcal{M}$ and $x(p) = 0$, and the polar chart $(V, y)$ for $V = \mathcal{M} \backslash$ "nonpositive $x$ axis" and $y(p) = \left(\sqrt{(p^0)^2 + (p^1)^2}, \text{sign}(p^1) \arccos(p^0 / \sqrt{(p^0)^2 + (p^1)^2})\right)$.  Then,
\begin{itemize}
    \item $y \circ x^{-1} = y(p)$ (as defined above)
    \item $x \circ y^{-1} = (r \cos(\theta), r \sin(\theta))$, or the inverse of $y(p)$.
\end{itemize}
These are differentiable everywhere except for the slit we cut out.

\subsection*{Example: The Hemisphere}
\subsubsection*{Compatible Charts on the Hemisphere}
Consider $(U, x)$ parallel projection, and $(V, y)$ arcsine angles.\\\\
First, find the inverse of parallel projection
\[x^{-1}(u) = (u^0, u^1, \sqrt{1 - (u^0)^2 - (u^1)^2})\]
Then the inverse of arcsine angles
\[y^{-1}(u) = (\sin u^0, \sin u^1, \text{height})\]

\subsubsection*{Example: Chart Transition Maps 1:}
Consider the map $y \circ x^{-1}$:
\[(x^1, x^0) \rightarrow (\arcsin(x^1), \arcsin(x^2))\]
The derivatives are a 2x2 (diagonal) matrix:
\[\begin{pmatrix}\frac{1}{\sqrt{1 - (x^1)^2}} & 0 \\ 0 & \frac{1}{\sqrt{1 - (x^2)^2}}\end{pmatrix}\]
Similarly, this does not include the boundary.

\subsubsection*{Example: Chart Transition Maps 2:}
Consider the map $x \circ y^{-1}$:
\begin{itemize}
    \item $x$ is the parallel projection, and $y$ is the sines.
\end{itemize}
\[(y^1, y^2) \mapsto (\sin(y^1), \sin(y^2))\]
The derivatives are again a 2x2 (diagonal) matrix.
\[\begin{pmatrix}
-\cos(y^1) & 0 \\ 0 & -\cos(y^0)
\end{pmatrix}\]
This function exists everywhere.  This matrix is continuous and invertible, there it is a smoothly compatible atlas.
\begin{itemize}
    \item A hemisphere with this choice of atlas is a smooth manifold.
\end{itemize}

\subsection*{Extrinsic Coordinates}
Often we will consider manifolds that are subsets of a larger space, such as surfaces in 3D, or matrix groups in $\R^{3\times 3}$.\\\\
It is natural to associate each $p \in \mathcal{M}$ as a point, an "extrinsic coordinate" (e.g., a point in 3D), as well as a chart (e.g., a coordinate in 2D).\\\\
A danger is that the result of computations may not lie in the manifold.

\subsection*{Averages}
\subsection*{Example: Points on a hemisphere}
Find the average of 5 points on a hemisphere.
\begin{center}
    \includegraphics*[width=\textwidth]{W4_6.png}
\end{center}
The easiest way is the way we were taught in middle school - add them up and divide by $n$.  This is the \textbf{average in extrinsic coordinates}.
\begin{itemize}
    \item Summing all the coordinates and dividing by 5 gives \texttt{[-0.111, -0.155, 0.528]}.
    \item Note the distance of this point from the origin is 0.562, so it is not on the hemisphere.
    \item Therefore, this calculation is inappropriate!
\end{itemize}
\subsubsection*{Example: Parallel Projection}
Project all the points onto the $x, y$-plane.  In this case, we just drop the $z$-coordinate.
\begin{itemize}
    \item Finding the average this way (dropping $z$, summing points and dividing by 5), gives \texttt{[-0.111, -0.155, 0.982]}.  The first two cooridnates are the same as before, except the $z$ coordinate is picked so it is on the hemisphere.
    \item This is a better answer.
\end{itemize}
\subsubsection*{Example: Angular Coordinates}
For this one, we use the angle from the north pole and the angle from the $x$-axis.  This gives the following (five, 2D) coordinates:
\begin{verbatim}
    [[ 0.479,  0.839,  0.335,  0.331,  0.9  ]
     [-0.956,  1.908, -2.177,  2.853, -0.281]]
\end{verbatim}
and the following average: \texttt{[0.577, 0.269]}, which, converting back to cartesian coordinates, corresponds to the 3D point \texttt{[0.808, 0.223, 0.545]}.
\begin{itemize}
    \item This is another different answer. 
    \item The first answer (straight average) is wrong.  The other two may be useful, depending on the case.
\end{itemize}






\end{document}
