% document formatting
\documentclass[10pt]{article}
\usepackage[utf8]{inputenc}
\usepackage[left=1in,right=1in,top=1in,bottom=1in]{geometry}
\usepackage[T1]{fontenc}
\usepackage{xcolor}

% math symbols, etc.
\usepackage{amsmath, amsfonts, amssymb, amsthm}

% lists
\usepackage{enumerate}

% images
\usepackage{graphicx} % for images

% code blocks
\usepackage{minted, listings}

% links
\usepackage{hyperref}

\graphicspath{{./assets/images}}

\newcommand{\solution}{\textbf{Solution:}} 

\title{COM SCI C121 Week 3}

\author{Aidan Jan}
\date{\today}

\begin{document}
\maketitle

\section*{Pseudoalignment}
RNA-seq quantification is a computational problem.
\begin{itemize}
    \item \textbf{Goal:} given a known set of isoform targets (genes) and RNA-seq fragments, recover the distribution of RNA molecules.
\end{itemize}
\begin{center}
    \includegraphics*[scale=0.5]{W4_1.png}
\end{center}
\textbf{Alignment} is the (in)exact matching of a subsequence to a reference.
\begin{itemize}
    \item In simplest terms:
    \begin{itemize}
        \item chromosome 1, position 342,215
        \item transcript A, position 32; transcript B, position 3, \dots
    \end{itemize}
    \item Also possible:
    \begin{verbatim}
            TACGGGCCCGCTA-C
            TA---G-CC-CTATC
    \end{verbatim}
\end{itemize}

\subsection*{A Fundamental Problem: Alignment and Counting}
\begin{itemize}
    \item Classical approaches for exact matching are too slow.
    \begin{itemize}
        \item Boyer-Moore $O(\vert R \vert + \vert T \vert)$\
    \end{itemize}
    \item Contemporary methods use heuristics
    \begin{itemize}
        \item Seed and extend
    \end{itemize}
    \item Our approach: use the \textit{redundancy} and structure of the target sequences
\end{itemize}

\subsection*{A Fundamental Problem: Counting and Quantification}
\begin{itemize}
    \item \textbf{Quantification:} given many alignments to a reference transcriptome, what is the likely \textit{relative} abundance of each isoform?
    \begin{itemize}
        \item Complication: most reads will give many, many transcripts
    \end{itemize}
\end{itemize}
\begin{center}
    \includegraphics*[scale=0.75]{W4_2.png}
\end{center}
Equivalence classes are sufficient for quantification.

\subsection*{Quick Aside: k-mer}
A \textbf{k-mer} is a sequence of length $k$ that is a substring of a longer sequence\\\\
Consider 'ACGGT':
\begin{itemize}
    \item k-mers of length 3: ACG, CGG, GGT
    \item k-mers of length 4: ACGG, CGGT
\end{itemize}

\subsection*{The Linear Allocation Problem Likelihood}
\begin{center}
    \includegraphics*[scale=0.75]{W4_3.png}
\end{center}
All observed equivalence classes easily fit into the memory on a laptop.
\begin{itemize}
    \item ~100M kmers
    \item <1M equivalence classes
\end{itemize}

\end{document}