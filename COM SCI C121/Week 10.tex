% document formatting
\documentclass[10pt]{article}
\usepackage[utf8]{inputenc}
\usepackage[left=1in,right=1in,top=1in,bottom=1in]{geometry}
\usepackage[T1]{fontenc}
\usepackage{xcolor}

% math symbols, etc.
\usepackage{amsmath, amsfonts, amssymb, amsthm}

% lists
\usepackage{enumerate}
\usepackage{enumitem}

% images
\usepackage{graphicx} % for images
\usepackage{tikz} % draw stuff

% code blocks
\usepackage{minted, listings}

% links
\usepackage{hyperref}

% \mathbb{1} symbol
\usepackage{dsfont}

\usepackage{tabularx}
\newcolumntype{Y}{>{\centering\arraybackslash}X}

\graphicspath{{./assets/images}}

\newcommand{\solution}{\textbf{Solution:}} 

\title{COM SCI C121 Week 10}

\author{Aidan Jan}
\date{\today}

\begin{document}
\maketitle

\section*{How do we develop a model to analyze data from these experiments?}
\begin{itemize}
    \item "What targets (perturbations) affect the gene of interest (measurement)?"
    \[y_{ng} = \beta_0 + \beta_g x_g\]
    \item where:
    \begin{itemize}
        \item $n_{ng}$ = reporter of protein expression
        \item $\beta_0$ = mean expression under "normal" conditions
        \item $\beta_g$ = effect of perturbation
        \item $x_g$ = indicator if perturbation occurs in this observation
        \item $n$ = replicate
        \item $g$ = target (gene) being perturbed
    \end{itemize}

\subsection*{We have unique data - let's think carefully}
Some goals of our method:
\begin{enumerate}
    \item Identify gene perturbations that change the reporter (gene of interest) distribution
    \begin{itemize}
        \item multiple guides pere gene should show the same trend
    \end{itemize}
    \item Model the sampling distribution in its "native" state
    \item Be well behaved in small sample sizes
    \item Infer experimental specific parameters
    \begin{itemize}
        \item Bin size, guide specific variance, etc.
    \end{itemize}
\end{enumerate}
\end{itemize}

\subsection*{Multiple guides target the \textit{same} gene and shus should be correlated}
Goal 1: Identify gene perturbations that change the reporter (gene of interest) distribution.  Multiple guides should show similar effects.
\begin{itemize}
    \item Suppose a gene we want to knock out is 20 base pairs long.
    \item When we knock it out, maybe the gene we knocked out doesn't matter much.
    \item However, maybe a gene that \textit{actually} matters contains the same 20 base pairs (highly likely considering the entire genome is billions of base pairs long)
    \item This causes the data to show that knocking out that gene has a huge effect, even if it does nothing.
\end{itemize}
Approach: Guides have their own effect, but share a "parent" effect
\begin{align*}
    \mu_T \sim& F(.) \\
    \beta_g \vert \mu_T \sim& H(\mu_T)
    y_{ng} &= \beta_0 + \beta_g x_g
\end{align*}
The equations give:
\begin{align*}
    \mu_T \vert \sigma^2 &\sim N(0, \sigma^2)\\
    \beta_g \vert \mu_T &\sim N(\mu_T, ?)\\
    y_{ng} \vert \beta_g &\sim N(\beta_g x_g, 1)
\end{align*}
From these equations, we solve goal 2.  We don't directly observe the reporter, we observe a noisy, quantized version.\\
Goal 2: Model the sampling distribution in its "native" state.
\begin{center}
    \includegraphics*[scale=0.8]{W10_1.png}
\end{center}    
What if the expected counts per bin wasn't even?  Then, the distribution is probably a Multinomial$(C, p)$.

\subsection*{Dirichlet Multinomial}
Our sampling distribution is an over dispersal Multinomial, aka \textit{Dirichlet Multinomial}.
\[\text{DirMult}(c, \phi p)\]
\[B_{ng} \vert \phi, p \sim \text{DirMult}(c, \phi p)\]
We need to connect $y_{ng}$ to our sampling distribution, so
\[B_{ng} \vert \phi, p \sim \text{DirMult}(c, \phi p(y_{ng}))\]
\begin{itemize}
    \item $B_{ng}$ is the data (observed bin counts)
\end{itemize}
In the previous equations:
\begin{itemize}
    \item $\mu_T \vert \sigma^2 \sim N(0, \sigma^2)$ = Gene level effect
    \item $\beta_g \vert \mu_T \sim N(\mu_T, ?)$ = Guide level effect.  (There's a ? because we don't know the exact guide level effect)
    \item $y_{ng} \vert \beta_g \sim N(\beta_g x_g, 1)$ = Unobserved reporter
    \item $B_{ng} \vert \phi, p \sim \text{DirMult}(c, \phi p(\beta_g))$ = Observed bin counts
    \item This is the perspective from the generator.
\end{itemize}
Goal 3: Behave well in small sample sizes (n = 3!?)
\begin{itemize}
    \item One way to deal with this is to \textit{shrink} parameters close to a \textit{shared} value
    \item Additionally, we can enforce some sparsity
\end{itemize}

\subsection*{Spike-and-Slab Prior}
The spike-and-slab prior enforces sparsity at the gene-level.\\
Does this \textit{gene} have an effect?  $\psi_T \vert \pi \sim \text{Bernoulli}(\pi)$\\
Yes, it does.  $\mu_T \vert \psi_T = 1 \sim N(0, \sigma^2)$.  But at the same time, no, it doesn't.  $\mu_T \vert \psi_T = 0 \equiv 0$.\\\\
What is the $?$, effect of guides?
\[\beta_g \vert \mu_T \sim N(\mu_g, ?)\]
\[B_{ng} \vert \phi, p \sim \text{DirMult}(c, \phi p(\beta_g))\]

\subsection*{Intuitively, we want the guides to be somewhat "similar"}
\begin{itemize}
    \item Guides should be similar
    \item If we had many guides per gene, we could have a different variance for each gene's guides, $\beta_g \vert \mu_T \sim N(\mu_g, \tau^2_T)$
    \begin{itemize}
        \item The number of guides is usually 3-5.
    \end{itemize}
    \item Enter shrinkage: $\beta_g \vert \mu_T \sim N(\mu_g, \tau^2)$
\end{itemize}

\begin{center}
    \includegraphics*[scale=0.8]{W10_2.png}
\end{center}




\end{document}


