% document formatting
\documentclass[10pt]{article}
\usepackage[utf8]{inputenc}
\usepackage[left=1in,right=1in,top=1in,bottom=1in]{geometry}
\usepackage[T1]{fontenc}
\usepackage{xcolor}

% math symbols, etc.
\usepackage{amsmath, amsfonts, amssymb, amsthm}

% lists
\usepackage{enumerate}

% images
\usepackage{graphicx} % for images

% code blocks
\usepackage{minted, listings} 

\graphicspath{{./assets/images}}

\newcommand{\solution}{\textbf{Solution:}} 

\title{COM SCI C121 Week 1}

\author{Aidan Jan}
\date{\today}

\begin{document}
\maketitle
\section*{Biology Review}
\subsection*{Central Dogma of Biology}
\begin{itemize}
    \item DNA is transcribed to RNA, which is then translated to proteins.
    \item During transcription, splicing may occur, so one section of DNA can produce multiple different strands of RNA, which is then translated into different proteins.
    \begin{itemize}
        \item Occurs often in more complex humans (e.g., not bacteria)
    \end{itemize}
    \item The definition of a gene is complicated, since one gene may have multiple exons, which may be spliced into different RNAs.  (How do you quantify the isoforms?)
    \item Difference in splicing, translation, and regulation are part of what defines cell types.
    \begin{itemize}
        \item This means that molecular smapling needs to be done for all different contexts
        \item Computationally, we \textit{need} fast, accurate, and space-efficient algorithms.
    \end{itemize}
\end{itemize}
\subsection*{21st Century Biology Revolution}
\begin{itemize}
    \item High throughput DNA sequencing has revolutionized modern biology
    \item Can sequence billions of DNA fragments for relatively cheap ($\sim$\$1000)
    \item May biological questions can be reduced to sequencing experiments
    \begin{itemize}
        \item e.g., RNA-Seq, ChIP-Seq, Methyl-Seq, RIP-Seq, CNV-Seq
    \end{itemize}
    \item Currently, hundreds (thousands?) of experiments (since $\sim$2008)
    \item If you can reduce your experiment to a sequencing experiment, you can essentially do \textbf{thousands} of experiments at once.
\end{itemize}
\section*{What is DNA?}
There are many types of biomolecules.  (e.g., carbohydrates, lipids, proteins, and nucleic acids).
\begin{itemize}
    \item DNA is a type of nucleic acid.
    \item DNA stores all the genetic information that a particular organism needs to live.
    \item DNA is stored in nearly every human cell.  DNA inside chromosomes, inside nuclei, in cells.
\end{itemize}
\subsection*{DNA, genes, RNA, and proteins}
\begin{itemize}
    \item DNA contains coding and non-coding regions.
    \begin{itemize}
        \item Coding regions are referred to as \textit{exons}.
        \item Non-coding regions are referred to as \textit{introns}.
        \item There are non-coding regions outside of these two groups, but are not discussed in this class.
    \end{itemize}
    \item Introns exist to allow the same DNA section to code for multiple different proteins
    \begin{itemize}
        \item Introns of some proteins may be exons of a different protein.
    \end{itemize}
\end{itemize}
\subsubsection*{DNA Strands}
DNA has two strands - the forward and reverse strands.  Which one is forward strand is arbitrary - someone just picked it.
\begin{itemize}
    \item The forward strand goes from 5' to 3' (these are names for the ends); the numbers represent the direction transcription occurs - transcription always occurs from 5' to 3'.
    \item 5' and 3' are named based on how the carbons are bonded.
\end{itemize}
\subsection*{Random Useful Facts about DNA}
\begin{itemize}
    \item A human "genome" stores about 3.1Gb (just one side of a double helix)
    \item Humans are 99.9\% genetically identical
    \item A great overestimate of a person's variability is 3M genetic variants
    \item If we take the union of all single nuleotide variants, it's only $\sim$8M (> 5\% allele frequency)
\end{itemize}
\section*{Sequencing DNA}
\begin{itemize}
    \item Sanger Sequencing: the first practical method invent by Fred Sanger in 1977.  Initially used to sequence short genomes (e.g., viruses, with 10k base pairs)
    \item 2nd Generation DNA Sequencing: around 2007, companies began sequencing commercially, but no technology can sequence much more than 10000 nucleotides at a reasonable cost, throughput, and accuracy
    \begin{itemize}
        \item As a result, there's a race to create sequencers that can read "short" fragments (100s of nucleotides) efficiently with the best cost and accuracy.
        \item The DNA would be "read" in sections, and then pieced together in the correct order.
    \end{itemize}
\end{itemize}

\subsection*{How DNA is Sequenced}
\begin{enumerate}
    \item Isolate the DNA part that needs to be sequenced
    \item Use DNA polymerase to amplify the DNA (Polymerase Chain Reaction)
    \item Cut the DNA into snippets (sound waves tend to break the DNA into random, small snippets)
    \item Deposit the snippets on the slides (kind of like electrophoresis)
    \item Submerge the snippets with a pool of nucleotides with terminators and polymerase.
    \item As DNA polymerase "builds" the strands of DNA, use a microscopic camera to capture flashes.
    \item Remove the terminators, and repeat.
\end{enumerate}
\begin{itemize}
    \item There could be billions of templates on a single slide!  
    \item This can be parallelized since a single microscope photo captures all the templates simultaneously.
    \item The terminators act as "speed bumps" and keep reactions in sync.
\end{itemize}

\subsection*{Bridge Amplification}
There is a slight issue with the way DNA is sequence, and that is, it's really hard to isolate a single DNA molecule.  In the Amplification stemp, bridge amplification may be used.
This is done by attaching primers to a slide, and attaching a DNA strand in a "bridge" across two primers.  It can then be copied.
\begin{center}
    \includegraphics*[scale=1]{W1_1.png}
\end{center}
By doing this, each snippet is actually a cluster, making reads easier.

\subsection*{Loss}
Between each read, it is not guaranteed that each strand as gained exactly one nucleotide (or is read accurately).  As a result, we get \textit{loss}, a number that describes the quality of the data.
\[Q = -10 \cdot \log_{10} p\]
\begin{itemize}
    \item $Q$ represents base quality
    \item $p$ represents the probability that the base call is incorrect.
\end{itemize}
If $Q = 10$, it means there is a 1/10 chance the call is incorrect.  $Q=100$ means 1 in 100 chance the call is incorrect.  The higher the quality, the better.\\
$p$ refers to the ratio $\text{incorrect} / \text{total count}$.  For example:
\begin{center}
    \includegraphics*[scale=0.5]{W1_2.png}
\end{center}
In the above image, the call would be orange, since it is the most plentiful.  Then,
\[p = \frac{\text{number not orange}}{\text{number in cluster}} = \frac{3}{9} = 0.\bar 3\]
Plugging in and solving for $Q$ yields $Q = 4.77$.\\\\
Furthermore, once a given pixel is marked as incorrect, it is essentially incorrect 'forever', since once behind or ahead, it is unlikely to be synchronized with the rest.  As a result, quality tends to decrease the longer into the strand.
\begin{itemize}
    \item \textbf{Quality decreases as a function of length!}
\end{itemize}
\end{document}