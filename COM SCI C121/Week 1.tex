% document formatting
\documentclass[10pt]{article}
\usepackage[utf8]{inputenc}
\usepackage[left=1in,right=1in,top=1in,bottom=1in]{geometry}
\usepackage[T1]{fontenc}
\usepackage{xcolor}

% math symbols, etc.
\usepackage{amsmath, amsfonts, amssymb, amsthm}

% lists
\usepackage{enumerate}

% images
\usepackage{graphicx} % for images

% code blocks
\usepackage{minted, listings} 

\graphicspath{{./assets/images}}

\newcommand{\solution}{\textbf{Solution:}} 

\title{COM SCI C121 Week 1}

\author{Aidan Jan}
\date{\today}

\begin{document}
\maketitle
\section*{Biology Review}
\subsection*{Central Dogma of Biology}
\begin{itemize}
    \item DNA is transcribed to RNA, which is then translated to proteins.
    \item During transcription, splicing may occur, so one section of DNA can produce multiple different strands of RNA, which is then translated into different proteins.
    \begin{itemize}
        \item Occurs often in more complex humans (e.g., not bacteria)
    \end{itemize}
    \item The definition of a gene is complicated, since one gene may have multiple exons, which may be spliced into different RNAs.  (How do you quantify the isoforms?)
    \item Difference in splicing, translation, and regulation are part of what defines cell types.
    \begin{itemize}
        \item This means that molecular smapling needs to be done for all different contexts
        \item Computationally, we \textit{need} fast, accurate, and space-efficient algorithms.
    \end{itemize}
\end{itemize}
\subsection*{21st Century Biology Revolution}
\begin{itemize}
    \item High throughput DNA sequencing has revolutionized modern biology
    \item Can sequence billions of DNA fragments for relatively cheap ($\sim$\$1000)
    \item May biological questions can be reduced to sequencing experiments
    \begin{itemize}
        \item e.g., RNA-Seq, ChIP-Seq, Methyl-Seq, RIP-Seq, CNV-Seq
    \end{itemize}
    \item Currently, hundreds (thousands?) of experiments (since $\sim$2008)
    \item If you can reduce your experiment to a sequencing experiment, you can essentially do \textbf{thousands} of experiments at once.
\end{itemize}
\section*{What is DNA?}
There are many types of biomolecules.  (e.g., carbohydrates, lipids, proteins, and nucleic acids).
\begin{itemize}
    \item DNA is a type of nucleic acid.
    \item DNA stores all the genetic information that a particular organism needs to live.
    \item DNA is stored in nearly every human cell.  DNA inside chromosomes, inside nuclei, in cells.
\end{itemize}
\subsection*{DNA, genes, RNA, and proteins}
\begin{itemize}
    \item DNA contains coding and non-coding regions.
    \begin{itemize}
        \item Coding regions are referred to as \textit{exons}.
        \item Non-coding regions are referred to as \textit{introns}.
        \item There are non-coding regions outside of these two groups, but are not discussed in this class.
    \end{itemize}
    \item Introns exist to allow the same DNA section to code for multiple different proteins
    \begin{itemize}
        \item Introns of some proteins may be exons of a different protein.
    \end{itemize}
\end{itemize}
\subsubsection*{DNA Strands}
DNA has two strands - the forward and reverse strands.  Which one is forward strand is arbitrary - someone just picked it.
\begin{itemize}
    \item The forward strand goes from 5' to 3' (these are names for the ends); the numbers represent the direction transcription occurs - transcription always occurs from 5' to 3'.
    \item 5' and 3' are named based on how the carbons are bonded.
\end{itemize}
\subsection*{Random Useful Facts about DNA}
\begin{itemize}
    \item A human "genome" stores about 3.1Gb (just one side of a double helix)
    \item Humans are 99.9\% genetically identical
    \item A great overestimate of a person's variability is 3M genetic variants
    \item If we take the union of all single nuleotide variants, it's only $\sim$8M (> 5\% allele frequency)
\end{itemize}






\end{document}