% document formatting
\documentclass[10pt]{article}
\usepackage[utf8]{inputenc}
\usepackage[left=1in,right=1in,top=1in,bottom=1in]{geometry}
\usepackage[T1]{fontenc}
\usepackage{xcolor}

% math symbols, etc.
\usepackage{amsmath, amsfonts, amssymb, amsthm}

% lists
\usepackage{enumerate}

% images
\usepackage{graphicx} % for images
\usepackage{tikz} % draw stuff

% code blocks
\usepackage{minted, listings}

% links
\usepackage{hyperref}

% \mathbb{1} symbol
\usepackage{dsfont}

\usepackage{tabularx}
\newcolumntype{Y}{>{\centering\arraybackslash}X}

\graphicspath{{./assets/images}}

\newcommand{\solution}{\textbf{Solution:}} 

\title{COM SCI C121 Week 8}

\author{Aidan Jan}
\date{\today}

\begin{document}
\maketitle

\section*{Matrices}
\begin{itemize}
    \item A matrix can be used to store data
    \item Consider a matrix of dimension $m \times n$
    \item The rows can be a high-dimensional sample of dimension $n$
    \item That means that each column is \textit{one} dimension across \textit{all} samples, $m$
\end{itemize}
The goal is to summarize the data in the matrix to see relationship across samples.
\begin{itemize}
    \item The matrix itself is described by $m \times n$ data points, which on its own is unwieldy
    \item Naturally, we would want a way to summarize $n$ data points into an "intuitive" representation that we can make sense of
    \item One way to think about this is creating a "faithful" representation of $m \times n$ data points into, say, $n \times 2$ data points while preserving structure in the data.
    \begin{itemize}
        \item Taking this to the extreme, we will look at linear algebra for inspiration
    \end{itemize}
    \item In an ideal case, samples that are more "similar" will "cluster" together.  For example, if I put tumor samples and normal samples, one would expect the tumor samples to be more similar to each other and the normal samples to be more similar to each other.
\end{itemize}

\subsection*{What is a Matrix in Linear Algebra?}
In short, a matrix is code for a linear function (to transform a set of vectors to another).
For example, 
\[\begin{bmatrix} 1 & 1 \\ 0 & 1 \end{bmatrix} \rightarrow \begin{cases} s = x + y & \\ t = 0x + y\end{cases}\]
where $x$ and $y$ are the $x$ and $y$ of the original vector, and $s$ and $t$ are the components of the new vector.

\section*{Singular Value Decomposition (SVD)}
What is singular value decomposition about?
\begin{itemize}
    \item Linear transformations, and their corresponding matrices (which are rectnagular tables filled with numbers), are seemingly complicated and arbitrary.
    \item The singular value decomposition (SVD) says that every matrix is essentially diagonal, i.e., "nice", provided the "right" bases are used for the domain and range spaces.
    \item By finding the "right" bases, the SVD provides fundamental insights into linear transformations and their accompanying matrix representations.
\end{itemize}
\subsection*{SVD Algorithm}
\begin{itemize}
    \item \textbf{Input:} an $m \times n$ matrix
    \item \textbf{Output:} a set of numbers called \textit{singular values} and a two collection of vectors: a set of \textit{right singular vectors} and another set of \textit{left singular vectors}.
\end{itemize}
\[\begin{bmatrix} & & \\ & M & \\ & & \\ & & \end{bmatrix} = \begin{bmatrix} & & & \\ & U & & \\ & & & \\ & & & \end{bmatrix} \times \begin{bmatrix} & & \\ & \Sigma & \\ & & \\ & & \end{bmatrix} \times \begin{bmatrix} & & \\ & V^* & \\ & & \end{bmatrix}\]
\dots where:
\begin{itemize}
    \item $M$ is a $m \times n$ matrix
    \item $U$ is a $m \times m$ matrix
    \item $\Sigma$ is a $m \times n$ matrix
    \item $V^*$ is a $n \times n$ matrix
\end{itemize}   
additionally
\begin{itemize}
    \item $U$ and $V^*$ have the property that their transposes equal their inverses.
\end{itemize}

\subsection*{The meaning of \textit{singular values}}
\begin{itemize}
    \item As a linear map, an $m \times n$ matrix $M$ can be thought of as mapping a vector $x$ from $R^n$ to $R^m$.
    \item A unit sphere in $R^n$ is mapped to an ellipsoid in $R^m$
    \item The non-zero \textit{singular values} of $M$ are the lengths of the \textit{semi-axes} of the ellipsoid.
\end{itemize}
\begin{center}
    \includegraphics*[scale=0.5]{W8_1.png}
\end{center}

\subsection*{Measuring directions of distortion}
\begin{itemize}
    \item The maximal singular value can therefore be understood to be the size of the vector that points in the direction in which the linear transformation corresponding to $M$ has the largesst effect.
    \item Formally, the maximal singular value, which is usually denoted as $\sigma_1$ can be understood to be 
    \[\sigma_1 = \max_{x:\Vert x \Vert = 1} \Vert Mx\Vert\]
    \item Similarly, the smallest non-zero singular value is the size of the smallest semi-axis of the ellipsoid that is the image of the unit sphere under $M$.
\end{itemize}

\subsection*{Calculating SVD}
\begin{itemize}
    \item Let $M = U\Sigma V^T$ and set $M_k = \sum_{i = 1}^k \sigma_i u_i v_i^T$.  The matrix $M_k$ is a good low-rank matrix approximation of $M$.  Specifically,
    \[\min_{rank(X) = k} \Vert A - B \Vert_2 = \Vert A - A_k \Vert_2 = \sigma_{k + 1}\]
    \item This theorem makes precise the intuition that the top singular values, which measure the sizes of the largest of the semi-axes of the ellipsoid resulting from mapping of a sphere by the linear transformation corresponding to a matrix, capture "most" of the transformation.
\end{itemize}

\subsection*{A Centered Matrix can be Transformed to Summarize the Covariance of the Data}
\begin{itemize}
    \item Take a matrix $A$ with $m \times n$ data points.  Create $M$ by subtracting the mean of each column from that corersponding column
    \begin{itemize}
        \item $M_{ij} = A_{ij} - \frac{1}{n} \sum_{k = 1}^n a_{kj}$
        \item M is now a matrix that is \textit{centered}
    \end{itemize}
    \item $M^T M$ is a \textit{covariance matrix} of $A$ with the property:
    \begin{itemize}
        \item $(M^T M)_{ij} = \text{Cov}(A_i, A_j)$
        \item In words, $M_ij$ represents how much samples $i$ and $j$ covary with each other
        \begin{itemize}
            \item If $(M^T M)_{ij} > 0$, samples $i$ and $j$ are positively related
            \item If $(M^T M)_{ij} < 0$, samples $i$ and $j$ are negatively related
            \item If $(M^T M)_{ij} \approx 0$, samples $i$ and $j$ are unrelated.
        \end{itemize}
    \end{itemize}
\end{itemize}

\subsection*{PCA with SVD}
\begin{itemize}
    \item The SVD of a (centered) $M$, given by $M = U\Sigma V^T$, yields a decomposition of $M^T M$ ad $M^T M = V \Sigma^2 V^{-1}$, i.e., eigendecomposition of the covariance matrix $M^T M$ can be performed by SVD of $M$.Set $V_k$ to be the first $k$ columns of $V$, i.e., $V_k = [v_1, v_2, \dots v_k]$.  Then the projection of the points in $M$ by $V_k$, i.e., PCA($k$) = MV$_k$ has numerous useful properties.
\end{itemize}

\section*{An Example of a PCA Projection}
\begin{itemize}
    \item Each dot represents a sample
    \item Remember, $M$ is $m \times n$, $V$ is $n \times n$.
    \item If I take the first two columns of $V$:
    \begin{itemize}
        \item $PCA(2) = MV_2$ results in a $m \times 2$ matrix.
    \end{itemize}
\end{itemize}
\begin{center}
    \includegraphics*[scale=0.65]{W8_2.png}
\end{center}
PCA Steps:
\begin{itemize}
    \item Start with a data matrix $A$.
    \item Center $A$ to get $M$.
    \item $M$ has a singular value decomposition that is derived from viewing $M$ as a linear transformation.  $M = U\Sigma V^T$.
    \item The matrix $V$ consists of the eigenvectors which diagonalize the covariance matrix $M^T M$.
    \item Compute $V$ from $M$ using the SVD.
    \item Let $V_k$ be the truncation of $V$ to its first $k$ columns.  We know from linear algebra that this is a meaningful restriction because $M_k = U_k \Sigma V_k^T$ is a good low-rank approximation to $M$.
    \item Project the data matrix $M$ with $V_k$ to obtain a new set of points: $MV_k$. 
    \item The projection has the property that it will maximize the variance of the projected points.
\end{itemize}

\subsection*{An Application of PCA: the Human Genotype Matrix}
\begin{itemize}
    \item Differences between any pair of human genomes are largely in the same sites, and consist of single nucleotide polymorphisms (SNPs).
    \item Most human SNPs are biallelic.
\end{itemize}
\begin{center}
    \includegraphics*[scale=0.8]{W8_3.png}
\end{center}


\subsection*{"Genes Mirror Geography Within Europe"}
\begin{center}
    \includegraphics*[scale=0.5]{W8_4.png}
\end{center}

\subsection*{Some Properties of the PCA}
\begin{itemize}
    \item Each subsequent dimension explains less variance than the previous
    \begin{itemize}
        \item i.e., there are \textit{diminishing returns} by including additional PCs
    \end{itemize}
    \item The singular values, $\sigma_i$ are related to how much variance each dimension explains
    \begin{itemize}
        \item Proportion of variance explained: $\sigma_p^2 l \sum_{j = 1}^n \sigma_j^2$
    \end{itemize}
\end{itemize}
\pagebreak
\section*{Choosing $k$ in K-means Clustering}
As a review, K-means minimizes the following loss function:
\[L(\mu, \alpha) = \sum_{k = 1}^K \sum_{i = 1}^n \Vert x_i - \mu_k \Vert_2^2 \mathds{1}\{\alpha_i = k\}\]
where
\begin{itemize}
    \item $x_i$ is the data
    \item $\mu_k$ is the cluster center (mean of the data)
    \item $\mathds{1}$ is a function that returns 1 if the condition is true.  In this case, $1$ if the data belongs to cluster $k$, $0$ otherwise.
\end{itemize}
What happens if $k$ increases and approaches $n$?

\subsection*{More thoughts}
\begin{itemize}
    \item Conceptually, you want a clustering that satisfies the \textit{good clustering principle:}
    \begin{itemize}
        \item "Every pair of points from the same cluster should be close to each other than any pair of points from different clusters."
    \end{itemize}
    \item Does setting $k = n$ satisfy this principle?
\end{itemize}

\subsection*{In Reality}
Technically yes.  But then the data would not be useful.
\begin{itemize}
    \item We want to satisfy the good clustering principle with the \textit{simplest} model that sufficiently captures the complexity of the data
\end{itemize}

\subsection*{The Good Clustering Principle Mathematically}
\begin{itemize}
    \item Let $d_{ii'} = \Vert x_i - x_{i'} \Vert_2^2 = \sum_{j = 1}^p (x_{ij} - x_{i'j})^2$, the pairwise distance between two points.
    \item Let $D_r = \sum_{i, i' \in C_r} d_{ii'}$ be the pairwise distance between all points in a cluster $C_r$.
    \item Then
    \[W_k = \sum_{r = 1}^K \frac{1}{2n_r} D_r\]
    \begin{itemize}
        \item where $n_r$ is the number of points in cluster $r$.
        \item What is k-means with respect to W?  K-means minimizes W.  (W is the same objective function as the loss function above.)
    \end{itemize}
\end{itemize}
\begin{center}
    \includegraphics*[scale=0.5]{W8_5.png}
\end{center}
Graphically, we are looking for the value of $k$ at the "elbow".  Unfortunately, a lot of the times, the graph isn't that clean.

\section*{Gap Statistic}
A more principled approach: the Gap statistic is the difference in the expected W under a \textit{reasonable null distribution} and the observed $W$.
\[\text{Gap}_n(k) = E_n^* \{\log(W_k)\} - \log(W_k)\]
To see how to find an appropriate reference distribution, consider for a moment the population version corresponding to the gap statistic in the case of K-means clustering:
\[g(k) = \log\left\{ \frac{\text{MSE}_{X^*}(k)}{\text{MSE}_{X^*}(1)}\right\} - \log\left\{ \frac{\text{MSE}_{X}(k)}{\text{MSE}_{X}(1)}\right\}\]

\end{document}
