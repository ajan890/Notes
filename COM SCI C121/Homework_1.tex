% document formatting
\documentclass[10pt]{article}
\usepackage[utf8]{inputenc}
\usepackage[left=1in,right=1in,top=1in,bottom=1in]{geometry}
\usepackage[T1]{fontenc}
\usepackage{xcolor}

% math symbols, etc.
\usepackage{amsmath, amsfonts, amssymb, amsthm}

% lists
\usepackage{enumerate}

% images
\usepackage{graphicx} % for images

% code blocks
\usepackage{minted, listings} 

\graphicspath{{./assets/images}}

\newcommand{\solution}{\textbf{Solution:}} 

\title{COM SCI C121 Homework 1}

\author{<name, id>}
\date{\today}

\begin{document}
\maketitle

\section*{Problem 1}
\subsection*{Bags and nucleotides}
Assume I have a bag with infinite nunmber of balls, I mean, nuceotides, with bases $\{A, C, G, T\}$.  When I pull each base (nucleotide) out, it is observed correctly, and each base is equally probable.  Further, assume independence between draws, i.e., $P(A, G) = P(A)P(G)$.
\begin{enumerate}[~~(a)]
    \item What is the probability of observing the sequence $AGG$?
    \item What is the probability of observing the sequence $GAA$?
    \item What is the probability of observing $A$, given you saw $GAA$ already, i.e., $P(A \:\vert\: GAA)$?
\end{enumerate}
\solution\\

\section*{Problem 2}
\subsection*{Finite nucleotides}
Now, assume I have a total of 8 bases in my bag, each of them with equal probability.  When I draw a base, I \textit{do not} replace it.  Note, the order matters and assume you go from left to right.  No need to worry about reverse complements here.  Pretend it doesn't exist.
\begin{enumerate}[~~(a)]
    \item What is the probability of observing the sequence $AGG$?
    \item What is the probability of observing the sequence $GAT$?
    \item What is the probability of observing $A$, given you saw $GAA$ already, i.e., $P(A \:\vert\: GAA)$?
    \item What is the probability of observing $G$, given you saw $GAA$ already, i.e., $P(G \:\vert\: GAA)$?
\end{enumerate}
\solution\\

\section*{Problem 3}
\subsection*{Thought experiments about sequencing by synthesis}
Imagine I have an Illumina-style sequencer and my "true" sequence is $s = AAGTA$, but my first observed data is $d_1 = AAGTG$.  That is, $d_1$ has an error in the final base call.
\begin{enumerate}[~~(a)]
    \item Write up two sentences about how the error in $d_1$ could have arisen.  We are not looking for some in-depth biochemistry explanation, simply explaining the logic of one possible case.
    \item We then query the sequencer for another data point, and this time it gives us $d_2 = AACTA$.  Given how sequencing-by-synthesis works, is $d_1$ or $d_2$ more likely?  Again, not looking for an in-depth biochemistry lesson, just explaining the logic of how errors might arise.  Strive for less than two sentences.
\end{enumerate}
\solution\\


\end{document}