% document formatting
\documentclass[10pt]{article}
\usepackage[utf8]{inputenc}
\usepackage[left=1in,right=1in,top=1in,bottom=1in]{geometry}
\usepackage[T1]{fontenc}
\usepackage{xcolor}

% math symbols, etc.
\usepackage{amsmath, amsfonts, amssymb, amsthm}

% lists
\usepackage{enumerate}

% images
\usepackage{graphicx} % for images

% code blocks
\usepackage{minted, listings} 

% verbatim greek
\usepackage{alphabeta}

\graphicspath{{./assets/images}}

\newcommand{\solution}{\textbf{Solution:}} 

\title{COM SCI M151B Week 3}

\author{Aidan Jan}
\date{\today}

\begin{document}
\maketitle

\section*{Pipelining}
\begin{itemize}
    \item Last week, we discussed that a single-cycle processor is simple and relatively to implement, but not the most efficient.  Why isn't it the most efficient?
    \item \textbf{Pipelining} refers to parallelizing resources.
    \item Consider the following analogy: suppose you have four people, A, B, C, D, who have to do laundry.
    \item If one set of laundry takes 2 hours to complete, and the four people go one after another, the entire task takes 8 hours.
    \begin{center}
        \includegraphics*[scale=0.65]{W3_1.png}
    \end{center}
    \item However, if the second person starts as soon as the first is done with the washer, etc., then the process takes much less time.
    \begin{center}
        \includegraphics*[scale=0.65]{W3_2.png}
    \end{center}
    \item Notice that each individual load of laundry did not get faster; but the entire task does.
    \item This is like executing one instruction per cycle!
    \begin{center}
        \includegraphics*[scale=0.5]{W3_3.png}
    \end{center}
\end{itemize}

\subsection*{Single Cycle vs. Pipeline}
\begin{center}
    \includegraphics*{W3_4.png}
\end{center}
\begin{itemize}
    \item Remember that the processor's clock speed is the time it takes for the longest procedure to execute.
    \item Without pipelining, the clock speed would have to be 800 ps, since that instruction takes the longest.
    \item With pipelining, the clock speed can be 200 ps, since the specific part takes the longest and the rest can be pipelined.
    \item CPI remains at 1
\end{itemize}

\subsection*{Pipelined Datapath + Controller}
\begin{center}
    \includegraphics*[scale=0.5]{W3_5.png}
\end{center}
\begin{itemize}
    \item We divide the original datapath and controller into 5 sections.  (The fifth section is writing to registers.)
    \item This lets us decrease the clock cycle since each section can be pipelined.
\end{itemize}
\subsubsection*{Half-Cycle}
Since reading from registers and writing to registers happen at different times, we get an issue:
\begin{center}
    \includegraphics*[scale=0.8]{W3_6.png}
\end{center}
Do we read the old value or new value of the register?
\begin{itemize}
    \item The answer is the new value, because programmers assume the instructions execute sequentially.
    \item The purpose is to abstract concepts like instruction timing.
    \item As a result, we divide into half-cycles.  Every read (from registers or memory) happen in the second half-cycle, and all the writes (registers and memory) happen in the first half-cycle.
\end{itemize}

\subsection*{Hazards}
Consider this scenario:
\begin{verbatim}
    add x5, x1, x2
    sub x20, x5, x4
\end{verbatim}
\begin{center}
    \includegraphics*[scale=0.8]{W3_7.png}
\end{center}
\begin{itemize}
    \item There are situations in pipelining when the next instruction cannot execute in the following clock cycle.  \textbf{These events are called hazards.}
    \item There are \textbf{three} different types of hazards.
    \begin{enumerate}
        \item Data hazard
        \item Control hazard
        \item Structural hazard
    \end{enumerate}
\end{itemize}

\subsubsection*{Data Hazard}
\begin{itemize}
    \item Read after write (RAW)
    \item Writing to a register (rd) and using it (rs1 or rs2) \underline{before} the writing is finished (i.e., \textit{rd} reaches to the \textit{WB stage}).
    \begin{center}
        \texttt{add \textbf{x5}, x1, x2}\\
        \texttt{sub x20, \textbf{x5}, x4}
    \end{center}
    \item For how many cycles we might have RAW hazard?
    \begin{center}
        \includegraphics*[scale=0.5]{W3_8.png}
    \end{center}
\end{itemize}
What about other relations/dependencies?
\begin{itemize}
    \item \textbf{write after write (WAW)} - this is called \textit{output dependency}.  For our pipelined processor, \textit{this won't be an issue} since instructions are executed \textit{\textbf{in order}}.
    \item \textbf{write after read (WAR)} - this is called \textit{false dependency}.  For our pipelined processor \textit{this also won't be an issue for the same reason} (i.e., in-order execution).
\end{itemize}
\textbf{How to solve RAW hazard?}
\begin{itemize}
    \item Stop the processor until the first instruction finishes!
\end{itemize}
\begin{center}
    \includegraphics*[scale=0.5]{W3_9.png}
\end{center}


\begin{itemize}
    \item Stopping (a part of) the pipeline is called a "stall".
    \item To achieve stall, we can \textbf{prevent} pipeline registers from updating (only those stages that we want to freeze/stall)
    \item Since some pipeline registers are not updated, we need to send \textit{something} to the next stage.
    \begin{itemize}
        \item Sending \textit{arbitrary} values could overwrite data and cause problems.
        \item We can send \textbf{NOP} which acts as a bubble (i.e., does nothing).
    \end{itemize}
\end{itemize}
\begin{center}
    \includegraphics*[scale=0.4]{W3_10.png}
\end{center}

\subsubsection*{Forwarding (Bypassing)}
\begin{itemize}
    \item Stalling can hurt performance if RAW often happens (which it may).
    \item We can prevent this by forwarding.
    \item Use the result when it is computed, don't wait for it to be stored in a register
    \item Requires extra connections in the datapath
\end{itemize}
\begin{center}
    \includegraphics*[scale=0.35]{W3_11.png}
\end{center}
\begin{itemize}
    \item Register to register (i.e., updating ALU operands)
    \item From ALU Result to ALU srcs
    \item From Mem to ALU srcs
    \item Can we always avoid stalling with this method?
\end{itemize}
\textbf{Load-Use Data Hazard}\\
\begin{itemize}
    \item Unlike in \textit{reg-to-reg} forwarding, in this case data is \textbf{not yet ready} in the Mem stage to be forwarded to EX stage.
    \item Half cycle!
\end{itemize}
\begin{center}
    \includegraphics*[scale=0.6]{W3_12.png}\\
    \includegraphics*[scale=0.6]{W3_13.png}
\end{center}
\begin{itemize}
    \item We can't avoid stalling completely.  However, with forwarding, we stall for one less cycle.
\end{itemize}

\subsection*{How to detect when to forward?}
\begin{itemize}
    \item Data hazards when:
    \item If the following happen, forward from EX/MEM pipeline reg
    \begin{itemize}
        \item EX/MEM. Rd = ID/EX. Rs1
        \item EX/MEM. Rd = ID/EX. Rs2
        \item EX/MEM.RegWrite = 1
    \end{itemize}
    \item EX = execution
    \item If the following happen, forward from MEM/WB pipeline reg
    \begin{itemize}
        \item MEM/WB. Rd = ID/EX. Rs1
        \item MEM/WB. Rd = ID/EX. Rs2
        \item MEM/WB.RegWrite = 1
    \end{itemize}
\end{itemize}

\subsubsection*{Datapath with RAW Forwarding}
\begin{center}
    \includegraphics*[scale=0.6]{W3_14.png}
\end{center}

\subsection*{How to detect forward in Load-Use}
\begin{itemize}
    \item If the following happen, stall and insert bubble.
    \begin{itemize}
        \item ID/EX. Rd = IF/ID.Rs1
        \item ID/EX. Rd = IF/ID.Rs2
        \item UD/EX.MemRead = 1
    \end{itemize}
    \item (after inserting the bubble, the forwarding condition would be similar to that of RAW)
    \item Why ID/EX instead of EX/MEM?
\end{itemize}
\begin{center}
    \includegraphics*[scale=0.55]{W3_15.png}
\end{center}

\subsection*{What about the other hazards?}
\begin{itemize}
    \item \textbf{Structural:} one unit is busy
    \begin{itemize}
        \item Shared resources (e.g., ALU)
        \item Won't happen in our design (no sharing)
    \end{itemize}
    \item \textbf{Control:} what happens when there is a branch instruction?
\end{itemize}

\subsection*{Control Hazards}
\begin{itemize}
    \item What instruction should be fetched after a branch?
    \item When the flow of instruction addresses is not known/expected, we have \textbf{control hazard}.
    \item \texttt{BEQ} in our design.
    \item This isn't an issue in single cycle because the next instruction is known by each cycle.
    \item With pipelining, you don't know the instruction until possibly multiple cycles later!
\end{itemize}

\subsection*{Resolving Control Hazards}
Answer: we can stall!
\begin{itemize}
    \item For how many cycles?  How?
    \item Any better solution(s)?
\end{itemize}
The \textbf{3 W's:}
\begin{itemize}
    \item \underline{When} the branch happens?
    \begin{itemize}
        \item When to decode an undecoded instruction is a branch (e.g., \texttt{BEQ})
        \item \item End of ID stage
    \end{itemize}
    \item \underline{Where} to branch to?
    \begin{itemize}
        \item What is the target address of the branch (if taken) - i.e., pc + offset
        \item End of EX or Beginning of Mem stage
    \end{itemize}
    \item \underline{Whether} to branch?
    \begin{itemize}
        \item Predict if the branch is taken or not (for conditional)
        \item End of Mem stage
    \end{itemize}
\end{itemize}

\subsubsection*{If we use stalling\dots}
\begin{center}
    \includegraphics*[scale=0.6]{W3_16.png}
\end{center}
\begin{itemize}
    \item 20\% of instructions in a mix are control-flow instructions!
    \item We stall for 3 cycles (meaning a single control flow instruction takes 4 cycles)
    \item CPU Time = $0.8 \times CPI \times CT + 0.2 \times CPI \times CT \times 4$
    \item This is a huge hit to performance!
\end{itemize}

\subsection*{Can we do better?}
\begin{itemize}
    \item How about \textbf{guessing?}
    \item Instead of waiting (stalling), let's pick one of the outcomes (i.e., \{Taken, Not Taken\}), and fetch the next instruction based on that!
    \item \textbf{Speculation:} An approach whereby the compiler or processor \textit{guesses} the outcome of an instruction to remove it as a dependence in executing other instructions.
    \item For the sake of this example, let's always predict \textbf{\textit{Not Taken!}}
\end{itemize}
\begin{center}
    \includegraphics*[scale=0.6]{W3_17.png}
\end{center}
For any \textit{not taken} branch, we don't have to stall at all!\\\\
Some stats:
\begin{itemize}
    \item About 20\% of instruction mix is control-flow
    \begin{itemize}
        \item 50\% of "forward" (i.e., if-then-else conditions) branches are taken.
        \item 90\% of "backward" (i.e., end of loops) branches are taken.
        \item Overall, about 30\% of all branch instructions are not-taken.
        \item By predicting not-taken, we are fine in about 86\% of the time (i.e., 80\% non-branch + 30\% accuracy for branch).
    \end{itemize}
\end{itemize}
What would happen if the branch is \textbf{Taken}?
\begin{itemize}
    \item This is a correctness issue!
    \item We have to clear those incorrectly fetched instructions.  This is called a \textbf{"flush"}.
\end{itemize}
\begin{center}
    \includegraphics*[scale=0.4]{W3_18.png}
\end{center}
\begin{itemize}
    \item If branch is Taken (i.e., Zero == 1), we need to flush.
    \item We need to flush FE, DE, and EX.
    \item At MEM we will realize that we need to flush.
    \item To \textit{flush}, we have to connect Zero (in MEM) back to the controller.  The controller then issues the Flush control signals.
    \item Referring to the program execution image above, we would need to flush instructions i1-i3.
\end{itemize}

\subsubsection*{Branch Miss Penalty}
\begin{itemize}
    \item If we make a wrong decision, we have to flush three instructions (i.e., same as stalling for three cycles).
    \item 70\% of branches are Taken (i.e., our prediction accuracy - among branches - is only 30\%).
    \item Another issue is the added complexity to the datapath!
\end{itemize}
Can we do better than this?

\subsection*{Resolving the Branch Sooner}
\begin{itemize}
    \item It turns out we can do better!
    \item Even when we make a wrong decision, at least flushing can happen sooner.
    \item This can be done as early as \textit{Decode} stage!  The "Where" and "Whether" out of the 3 W's needs to be changed.
\end{itemize}
\textbf{"Where"}
\begin{itemize}
    \item Offset is available in DE stage.
    \begin{itemize}
        \item Move nextPC calculation to DE
        \item Use ALU for PC + offset in DE stage
    \end{itemize}
\end{itemize}
\textbf{"Whether"}
\begin{itemize}
    \item rs1 and rs2 are also available in DE stage (for Zero).
    \begin{itemize}
        \item Move the comparison between rs1 and rs2 after reading from the register file.
    \end{itemize}
\end{itemize}

\subsubsection*{Resolving Branch in DE Stage}
\begin{center}
    \includegraphics*[scale=0.4]{W3_19.png}
\end{center}
\begin{itemize}
    \item New control signals are not shown.
    \begin{itemize}
        \item The equal comparator next to the registers would output to the Control, if it detects branching.
        \item This is how the instruction can be queued correctly.
    \end{itemize}
    \item Only one instruction needs to be flushed!
\end{itemize}
This results in another problem: RAW Data Hazard?
\begin{itemize}
    \item What if rs1 or rs2 are being written in the flushed instruction?  You also have to revert the registers!
\end{itemize}
\begin{center}
    \includegraphics*[scale=0.6]{W3_20.png}
\end{center}
\begin{itemize}
    \item Restore values from the buffer register.
    \item Forwarding condition?
    \begin{itemize}
        \item Similar, only everything shift one cycle to the left.
        \begin{itemize}
            \item check IF/ID.rs1 with EX/MEM or MEM/WB.rd AND RegWrite signal
        \end{itemize}
    \end{itemize}
    \item Hazard and Stalling condition?
    \begin{itemize}
        \item For load-use hazard, we need to \textbf{stall} if needed.
        \item Similar, again only shifted one cycle to the left.
    \end{itemize}
\end{itemize}

\subsection*{Role of the Compiler}
\begin{itemize}
    \item Reordering the instruction
    \begin{itemize}
        \item \textbf{Delay Slot:} moving an independent instruction between the branch and the next instruction.
    \end{itemize}
    \item Increasing the chance of not-taken
    \item Loop unrolling
    \item \dots
\end{itemize}
\end{document}