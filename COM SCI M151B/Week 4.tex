% document formatting
\documentclass[10pt]{article}
\usepackage[utf8]{inputenc}
\usepackage[left=1in,right=1in,top=1in,bottom=1in]{geometry}
\usepackage[T1]{fontenc}
\usepackage{xcolor}

% math symbols, etc.
\usepackage{amsmath, amsfonts, amssymb, amsthm}

% lists
\usepackage{enumerate}

% images
\usepackage{graphicx} % for images

% code blocks
\usepackage{minted, listings} 

% verbatim greek
\usepackage{alphabeta}

\graphicspath{{./assets/images}}

\newcommand{\solution}{\textbf{Solution:}} 

\title{COM SCI M151B Week 4}

\author{Aidan Jan}
\date{\today}

\begin{document}
\maketitle

\section*{Review}
\subsection*{ISA}
\begin{itemize}
    \item First step in the design process
    \item The ISA (instruction set architecture) is essentially the assembly code.  Ex. RISC-V.
    \item ISA converts to machine code using a standard table.
\end{itemize}

\subsection*{The Iron Law of Processor Performance}
\begin{itemize}
    \item For single cycle design:
    \item $\text{CPU Time} = InstructionCount \times CyclePerInstruction \times CycleTime$
    \item Allowing for parallelism (overlaps)
    \begin{center}
        \includegraphics*[scale=0.7]{W4_1.png}
    \end{center}
\end{itemize}

\subsection*{Hazards}
\subsubsection*{Data Hazard - Read after write (RAW)}
\begin{itemize}
    \item Writing to a register (rd) and using it (rs1 or rs2) \textbf{before} the writing is finished (i.e., rd reaches to the WB stage.)
    \item To fix this, we either \textbf{stall} or \textbf{forward}.
\end{itemize}
\textbf{Stalling:}
\begin{center}
    \includegraphics*[scale=0.8]{W3_9.png}
\end{center}
\begin{itemize}
    \item We add bubble instructions (In RISC-V, this is the NOP instruction, which does nothing).
    \item Alternatively, we can add other, unrelated instructions (e.g., instructions that need to be run that don't involve any of the same registers)
\end{itemize}
\textbf{Forwarding:}
\begin{center}
    \includegraphics*[scale=0.8]{W3_11.png}
\end{center}
\begin{itemize}
    \item Instead of writing the resultant value to the register first, pass it to the ALU directly.
    \item Forwarding helps prevent stalls!
    \item See Week 3 Notes for how to detect when to forward.
\end{itemize}

\subsection*{Branches}
\begin{itemize}
    \item If we get hazards in a branch, we have two options:
    \begin{itemize}
        \item Stalling
        \item Speculation (not forwarding!)
    \end{itemize}
    \textbf{Stalling:}
    \begin{itemize}
        \item We must stall for 3 cycles!
            \begin{itemize}
                \item When: End of ID stage
                \item Where: End of EX or Beg. of Mem stage
                \item Whether: End of Mem stage
            \end{itemize}
    \end{itemize}
    \textbf{Speculation/Prediction}
    \begin{itemize}
        \item We predict which branch to take.  To do this, we predict one as always taken and the other as never taken
        \item Not taken is easier to check for, because:
        \begin{itemize}
            \item We don't need branch addresses; not taken means the next address is PC + 4.
            \item Most instructions are not branch so they are not taken too!
        \end{itemize}
        \item If we guess the branch incorrectly, we have to flush!
        \item The penalty for flushing is 3 cycles, unless we can resolve the branch during the DE stage.  This reduces the flush to 1 cycle.
    \end{itemize}
\end{itemize}

\section*{Branch Prediction}
How can we improve branch prediction?
\begin{itemize}
    \item Branch Miss Penalty
    \begin{itemize}
        \item Resolving branches sooner has reduced this penalty
    \end{itemize}
    \item Miss rate?
    \begin{itemize}
    \item Predicting always not-taken has only 30\% accuracy.
    \item What if we predict always taken?
    \begin{itemize}
        \item We need nextPC at Fetch stage!  (we need to be able to run the next instructions down that branch to save time).
    \end{itemize}
    \end{itemize}
\end{itemize}

\subsection*{Guessing Always Taken}
\begin{itemize}
    \item Idea: keep track of previous targets and use that to guess!
    \item If we see a branch instruction before, we know where it jumped.  Remember this if we see the same branch again!
    \item How frequent is this?  Is this always correct?
\end{itemize}

\subsubsection*{Branch Predictor - Where?}
\textbf{Branch Target Buffer (BTB)}
\begin{itemize}
    \item A table that stores \textit{target addresses.}
    \item Entries are indexed by PC.
    \begin{itemize}
        \item The size?  Lower bits of PC (to utilize \textit{locality}).
    \end{itemize}
\end{itemize}
\textbf{Algorithm for BTB}
\begin{itemize}
    \item For a new PC, record the target address in the table (using the PC as the index).
    \item Next time (a recurring PC), look up the table (i.e., by using the same index) and predict (i.e., use the stored value as the next PC).
\end{itemize}
\begin{center}
    \includegraphics*[scale=0.7]{W4_2.png}
\end{center}

\end{document}
