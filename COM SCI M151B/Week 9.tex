% document formatting
\documentclass[10pt]{article}
\usepackage[utf8]{inputenc}
\usepackage[left=1in,right=1in,top=1in,bottom=1in]{geometry}
\usepackage[T1]{fontenc}
\usepackage{xcolor}

% math symbols, etc.
\usepackage{amsmath, amsfonts, amssymb, amsthm}

% lists
\usepackage{enumerate, enumitem}

% images
\usepackage{graphicx} % for images

% code blocks
\usepackage{minted, listings} 

% verbatim greek
\usepackage{alphabeta}

\graphicspath{{./assets/images}}

\newcommand{\solution}{\textbf{Solution:}} 
\newcommand{\example}{\textbf{Example:}}

\title{COM SCI M151B Week 9}

\author{Aidan Jan}
\date{\today}

\begin{document}
\maketitle

\section*{Memory Consistency}
\subsection*{Consistency Problem}
\begin{itemize}
    \item if initially \texttt{M[Z] = M[Y] = 0, x3 = 100, x4 = 200},
    \begin{center}
        \includegraphics*[scale=0.8]{W9_1.png}
    \end{center}
    \item What are the potential values for \texttt{x1} and \texttt{x2}?
    \item RAW?
    \item Correctness?
    \item LSQ?
    \begin{itemize}
        \item If the addresses are not the same, why wait?
        \item Compiler can reorder, too!
    \end{itemize}
    \item SW and LW (to different addresses) within one core can be reordered and that will mess up assumptions in OTHER cores.
    \item The timing between cores might be off! (even without ordering).
\end{itemize}

\subsection*{Memory Consistency vs. Coherency}
\begin{itemize}
    \item Consistency is about ordering of parallel accesses between \textit{different} addresses.
    \item Coherency is about ordering of parallel accesses to the \textit{same address}.
    \item How do we fix this?
\end{itemize}

\subsection*{The Timing Problem}
\begin{itemize}
    \item Timing between two cores is not the hardware or reordering problem, so it's the programmer's job to fix that using synchronization strategies
    \item We will discuss this later!
\end{itemize}
\subsection*{The Reordering Problem}
\begin{itemize}
    \item That is the optimization done by hardware and/or compiler and programmer cannot do anything about this!
\end{itemize}

\subsection*{Memory Model}
\begin{itemize}
    \item We first need to know what are the possibilities.
    \item \textbf{Memory Model:} Interactions between threads and the shared memory.  In other words, \textit{what are the possible values for a load?}
\end{itemize}

\subsection*{Sequential Consistency}
\begin{enumerate}
    \item Processors see their own loads and stores in program order.
    \item Processors see others' loads and stores in program order.
    \item All processors see the same global load/store ordering.
    \item Sequential Consistency is the simplest scenario!
    \item SC makes multicore systems indistinguishable from multi-programmed in-order uniprocessor.
    \item However, it is \textbf{too slow!}
    \begin{itemize}
        \item Too many restrictions!
    \end{itemize}
\end{enumerate}

\subsection*{Memory Operation Ordering}
\begin{itemize}
    \item Four types of memory operation orderings:
    \begin{itemize}
        \item W \textrightarrow~R: write must complete before subsequent read.
        \item R \textrightarrow~R: read must complete before subsequent read.
        \item R \textrightarrow~W: read must complete before subsequent write.
        \item W \textrightarrow~W: write must complete before subsequent write.
        \item (This is for \textit{different} addresses, so now RAW hazards!)
    \end{itemize}
    \item \textbf{What can we do?}
    \begin{itemize}
        \item We really need to be able to re-order memory addresses.
        \begin{itemize}
            \item Most lines are not even shared!
            \item Store does not need to even wait.
            \item We have the ability to do so!
        \end{itemize}
        \item We need to compromise, and use a "weaker" or "relaxed" model.
        \begin{itemize}
            \item Not all loads and stores need to be in-order.
        \end{itemize}
    \end{itemize}
    \item Sequential consistency maintains all four orderings.
    \item Relaxed memory consistency models allow \textit{\textbf{certain}} orderings to be violated.
\end{itemize}

\subsection*{Total Store Order Model}
\begin{itemize}
    \item What do we need?  
    \begin{itemize}
        \item Sequential Consistency
    \end{itemize}
    \item What do we get?
    \begin{itemize}
        \item Store Buffer Model
    \end{itemize}
\end{itemize}
The Total Store Order (TSO) Model is as follows:
\begin{itemize}
    \item Processor P can read B before its write to A is seen by all processors.
    \begin{itemize}
        \item Each processor can move its own reads in front of its own writes.
        \item Remember LSQs?
        \begin{center}
            \includegraphics*[scale=0.6]{W9_2.png}
        \end{center}
    \end{itemize}
    \item NOTE: Read by other processors \textbf{cannot} return new value of A until the write to A is observed by all processors.
\end{itemize}
\subsubsection*{How TSO helps?}
Let \texttt{a = 0, flag = 0}.
\begin{center}
    \includegraphics*[scale=0.4]{W9_3.png}
\end{center}
\begin{itemize}
    \item Can \texttt{load} read 0?
\end{itemize}
\subsubsection*{TSO vs. SC}
\begin{itemize}
    \item if initially, \texttt{M[Z] = M[Y] = 0}
    \begin{center}
        \includegraphics*[scale=0.3]{W9_4.png}
    \end{center}
    \item What are the potential values for \texttt{x1} and \texttt{x2}?
\end{itemize}
\subsubsection*{TSO Summary}
\begin{itemize}
    \item Improving the per-core performance by allowing the reordering within the core!
    \item Still not observable by other cores (timing problem)
\end{itemize}

\subsection*{Weak Consistency}
\begin{itemize}
    \item Different assumptions
    \begin{itemize}
        \item What and when a memory access is visible to the other cores.
    \end{itemize}
    \item Performance vs. Complexity
    \item Different architectures use different models.
    \item More about it soon!
\end{itemize}

\subsection*{RISC-V Memory Model}
\begin{itemize}
    \item RISC-V OoO (BOOM) follows the RVWMO memory consistency model (RISC-V weak memory ordering).
    \item Loads are optimistically fired to memory on arrival to the LSU (for performance)
    \item Simultaneously, the load instruction compares its address with all of the store addresses that it depends on
    \begin{itemize}
        \item If there is a match, the memory request is killed.
        \item If the corresponding store data is present, then the store data is forwarded to the load and the load marks itself as having succeeded.
        \item If the store data is not present, then the load goes to sleep.
        \begin{itemize}
            \item Loads that have been put to sleep are retried at a later time.
        \end{itemize}
    \end{itemize}
    \item BOOM currently exhibits the following behavior:
    \begin{itemize}
        \item Write \textrightarrow~Read constraint is relaxed (newer loads may execute before older stores).
        \item Read \textrightarrow~Read constraint is maintained (loads to the same address appear in order).
        \item A thread can read its own writes early.
    \end{itemize}
\end{itemize}
\subsection*{How to write a correct program?}
\begin{itemize}
    \item The memory model is given.
\end{itemize}
\subsubsection*{The Simple Solution}
\begin{itemize}
    \item Let it be!
    \begin{itemize}
        \item Race-free programming
    \end{itemize}
\end{itemize}
\subsubsection*{The Not-Very-Simple Solution}
\begin{itemize}
    \item Have the hardware to track potential violations!
    \begin{itemize}
        \item Each core reorders but if another core needs that data, it should recover.
        \item Very costly!
    \end{itemize}
\end{itemize}
\subsubsection*{More Realistic Solution}
\begin{itemize}
    \item Use synchronization!
    \begin{itemize}
        \item The programmer (and/or hardware) use some primitives to synchronize different programs.
        \item This solves both timing and ordering problems.
    \end{itemize}
\end{itemize}
\subsection*{Synchronization}
\begin{itemize}
    \item There are two major solutions:
    \begin{enumerate}
        \item Producer-Consumer Model
        \item Mutual Exclusiveness.
    \end{enumerate}
    \begin{center}
        \includegraphics*[scale=0.6]{W9_5.png}
    \end{center}
    \item \textbf{Solution:}
    \begin{itemize}
        \item Use \textit{\textbf{fences!}}
        \begin{itemize}
            \item FENCE is an instruction that enforces all memory operations \textit{before} it to complete before fence is executed
            \item All memory operations \textit{after} FENCE must wait for FENCE to be finished.
            \item This happens \textit{within} each core!
        \end{itemize}
    \end{itemize}
\end{itemize}
\subsubsection*{FENCE in RISC-V}
\begin{itemize}
    \item FENCE (rwio)
    \begin{itemize}
        \item fence r, r
    \end{itemize}
\end{itemize}
\begin{center}
    \includegraphics*[scale=0.6]{W9_6.png}
\end{center}
What about between cores?
\begin{itemize}
    \item Use semaphores!
\end{itemize}
\subsubsection*{Semaphores}
\begin{itemize}
    \item Create some barriers!
    \item This will synchronize all the processors.
    \item Typically implemented in software.
\end{itemize}
\begin{center}
    \includegraphics*[scale=0.4]{W9_7.png}
\end{center}
\textbf{How to implement barrier/semaphore in hardware?}
\begin{itemize}
    \item Hardware-or with one-hot encoding \textrightarrow~ set a flag to one when a core reaches the barrier.  Wait until all flags are one!
\end{itemize}
\subsection*{Mutual Exclusiveness}
\begin{itemize}
    \item Making sure only one process has access to the shared region at any given time.
    \item To do this, use Mutex/Locks!
\end{itemize}
\begin{center}
    \includegraphics*[scale=0.4]{W9_8.png}
\end{center}
\subsection*{Release Consistency}
\begin{itemize}
    \item It is guaranteed that all reads and writes are executed and visible to all cores when the lock is released.
    \item Other than this guarantee, any ordering outside the acquire/release blocks is possible.
\end{itemize}
\subsubsection*{How to implement \texttt{acquire/release}?}
In code:
\begin{minted}{c++}
void acquire(lock) {
    while (lock != 0) {}  //spin lock!
    // lock is acquired at the point this executes.
    lock = 1; // set to 1 to make sure no one else can get it
}

void release(lock) {
    lock = 0;
}
\end{minted}
In assembly:
\begin{verbatim}
acquire:
    lw ra, 0(x1)
    bnez ra, acquire     # <- loops to the beginning
    addi ra, x0, 1
    sw ra 0(x1)

release:
    st r0 -> 0(&lock)
\end{verbatim}

\subsubsection*{Problem?}
\begin{tabular}{ll}
\textcolor{red}{(Core 1)} & \textcolor{blue}{(Core 2)}\\ 
\texttt{acquire:} & \texttt{acquire:}\\
\texttt{~~~~lw ra, 0(x1)} & \texttt{~~~~lw ra, 0(x1)}\\
\texttt{~~~~bnez ra, acquire} & \texttt{~~~~bnez ra, acquire}\\
\texttt{~~~~addi ra, x0, 1} & \texttt{~~~~addi ra, x0, 1}\\
\texttt{~~~~sw ra 0(x1)} & \texttt{~~~~sw ra 0(x1)}
\end{tabular}\\\\
The locks aren't always exclusive!  Since the \texttt{addi ra, x0, 1} is executed after the check for the lock, two cores can both claim the lock at the same time!\\\\
\textbf{The solution: } atomic instructions!
\subsection*{Atomic Instructions}
\begin{itemize}
    \item A hardware primitive that guarantees instructions are executed in-order and without interruption.
\end{itemize}
\subsection*{How to implement Mutex?}
\begin{itemize}
    \item Hardware-enforce instruction (called test and set).
    \begin{verbatim}
TS(int x) {
    oldval = SWAP(x, 1); // atomic swap
    return oldavl;
}
    \end{verbatim}
    \item Other variants: compare-and-swap, fetch-and-add
\end{itemize}
\subsubsection*{How SWAP is implemented?}
\begin{itemize}
    \item \textbf{Load-reserved, store-conditional} (in RISC-V)
    \begin{verbatim}
Label:
    load-link x2, 0(x1)
    addi x2, x0, 1
    store-conditional x2, 0(x0)
    branch-not-zero label           # check for failure
    \end{verbatim}
\end{itemize}

\subsection*{LR and SC}
\begin{itemize}
    \item On load-link, processor remembers address.
    \item Then checks for writes by other processors.
    \begin{itemize}
        \item If write is detected, store-conditional will fail.
    \end{itemize}
    \item For MESI:
    \begin{itemize}
        \item Load-Reserved ensures line in cache in Exclusive/Modified state.
        \item Store-Conditional succeeds if line still in Exclusive/Modified state.
    \end{itemize}
\end{itemize}

\subsection*{Recap}
\begin{itemize}
    \item Locks:
    \begin{itemize}
        \item Test-and-set
        \item Atomic
        \item LR and SC
    \end{itemize}
    \item This can be combined with a very weak consistency model (called release consistency).
\end{itemize}

\subsection*{Optimizations?}
\begin{itemize}
    \item Too many \texttt{SW}s when waiting/spinning.
    \begin{itemize}
        \item Use test-test-and-set
    \end{itemize}
    \item Unfairness?
    \begin{itemize} 
        \item Use queues
    \end{itemize}
    \item Multiple/Parallel Locks
\end{itemize}
\textbf{Bottom Line:}
\begin{itemize}
    \item There is a range of possibilities!
    \item The burden is (mostly) on the developer!
\end{itemize}

\subsubsection*{Test-test-and-set}
\begin{verbatim}
TTS(int x) {
    oldval = x;
    if (oldval == 0)
        oldval = SWAP(x, 1);
    return oldval;
}
\end{verbatim}
\begin{itemize}
    \item To check for lock:
    \item \texttt{while (TTS(lock) == 1);}
\end{itemize}

\end{document}