% document formatting
\documentclass[10pt]{article}
\usepackage[utf8]{inputenc}
\usepackage[left=1in,right=1in,top=1in,bottom=1in]{geometry}
\usepackage[T1]{fontenc}
\usepackage{xcolor}

% math symbols, etc.
\usepackage{amsmath, amsfonts, amssymb, amsthm}

% lists
\usepackage{enumerate}
\usepackage{tabularx}
\usepackage{multicol}
\usepackage[table,xcdraw]{xcolor}

% images
\usepackage{graphicx} % for images

% code blocks
\usepackage{minted, listings} 

% verbatim greek
\usepackage{alphabeta}

\graphicspath{{../images/Week 8}}

\newcommand{\solution}{\textbf{Solution:}} 
\newcommand{\example}{\textbf{Example: }}
\newcommand{\water}{\text{H$_2$O}}
\newcommand{\hydroxide}{\text{OH$^-$}}
\newcommand{\hydronium}{\text{H$_3$O$^+$}}
\newcommand{\proton}{\text{H$^+$}}
\newcommand{\pc}{$^+$}
\newcommand{\nc}{$^-$}
\newcommand{\ka}{\text{$K_\text{a}$}}

% import subfiles
\usepackage{subfiles}

\begin{document}
\subsection*{Glycolysis}
\begin{itemize}
	\item The process of converting one molecule of glucose into two molecules of pyruvate
	\item Occurs in the cytoplasm of cells
	\item Net reaction:
	\begin{center} 
        Glucose + 2 NAD\pc + 2 Pi $\rightarrow$ 2 Pyruvate + 2 NADH + 2 ATP + 2 \water
    \end{center}
\end{itemize}
Glycolysis
\begin{itemize}
	\item is the first step in cellular respiration
	\item generates ATP and NADH for energy
	\item provides intermediates for other metabolic pathways
	\item does not require oxygen
\end{itemize}

\subsection*{The Chemical Logic of the Glycolytic Pathway}
\begin{center} 
	\includegraphics*[width=\textwidth]{L3_1.png}
\end{center}

\pagebreak
\subsection*{Stage I: Preparatory Phase / Energy Investment}
\begin{center}
    Glucose + 2 ATP $\rightarrow$ 2 G3P + 2 ADP
\end{center}
\begin{itemize}
	\item G3P = Glyceraldehyde 3-Phosphate
\end{itemize}
\begin{center} 
	\includegraphics*[width=\textwidth]{L3_2.png}
\end{center}

\pagebreak
\subsection*{Stage II: Payoff Phase}
\begin{center}
    2 G3P + 2 NAD\pc + 4 ADP + 2 Pi $\rightarrow$ 2 Pyruvate + 2 NADH + 2 \proton + 4 ATP + 2 \water\\
    \includegraphics*[width=\textwidth]{L3_3.png}
\end{center}
\begin{itemize}
	\item \textbf{Substrate-level phosphorylation} is a type of \textbf{ATP (or GTP) synthesis} that occurs when a phosphate group is directly transferred from a \textbf{high-energy phosphorylated compound} to \textbf{ADP (or GDP)} to form ATP (or GTP).  This process does \textbf{not} require oxygen or the electron transport chain.
\end{itemize}

\subsection*{High-energy phosphate compounds}
\begin{center}
    \includegraphics*[width=0.7\textwidth]{L3_4.png}
\end{center}
\begin{itemize}
	\item These are high-energy phosphate compounds, and therefore capable of \underline{substrate-level phosphorylation}
	\begin{itemize}
        \item Substrate-level phosphorylation is an oxygen-independent mechanism for ATP synthesis that relies on direct phosphate transfer from high-energy compounds, making it essential for both aerobic and anaerobic energy metabolism
    \end{itemize}
    \item Compounds with $\Delta$ G$^{\circ '}$ more negative than ATP ($\sim$ -30.5 kJ/mol) can transfer phosphate groups to form ATP through \textbf{substrate-level phosphorylation}
    \item Compounds with $\Delta$ G$^{\circ '}$ less negative than ATP cannot phosphorylate ADP to ATP and typically require ATP for phosphorylation.
\end{itemize}
\textbf{Metabolites like glucose are often activated with a high-energy group before their catabolism.} Glycolysis is a nearly universal 10-step metabolic pathway for producing ATP by the oxidation of glucose. In this process, two molecules of ATP are invested to activate glucose, but the products of the pathway include four ATP, as well as NADH (\textbf{a form of reducing power}) and the triose pyruvate, which can be metabolized further in other pathways

\subsection*{ATP and NADH Formation Coupled to Glycolysis}
\begin{itemize}
	\item the overall equation for glycolysis is:
	\begin{center} 
        glucose + 2 NAD\pc + 2 ADP + 2 Pi $\rightarrow$ 2 pyruvate + 2 NADH + 2 \proton + 2 ATP + 2 \water
    \end{center}
    \item the reduction of NAD\pc proceeds by the enzymatic transfer of a hydride ion (\textbf{:}H\nc) from the aldehyde group of glyceraldehyde 3-phosphate (G3P) to the nicotinamide ring of NAD\pc, yielding NADH
\end{itemize}

\subsection*{Noteworthy Chemical Transformations of Glycolysis}
Three noteworthy chemical transformations:
\begin{enumerate}
    \item degradation of the carbon skeleton of glucose to yield \textbf{pyruvate}
    \item \textbf{phosphorylation of ADP to ATP} by compounds with high phosphoryl group transfer potential, formed during glycolysis
    \item transfer of a hydride ion to NAD\pc, forming \textbf{NADH}
\end{enumerate}

\subsection*{Resolving the Equation of Glycolysis into Two Processes}
\begin{itemize}
	\item The conversion of glucose to pyruvate is exergonic:
	\begin{center}
        glucose + 2 NAD\pc $\rightarrow$ 2 pyruvate + 2 NADH + 2 \proton\\
        $\Delta$G$^{'\circ}_1$ = -146 kJ/mol
    \end{center}
    \item The formation of ATP from ADP and Pi is endergonic:
	\begin{center}
        2 ADP + 2 Pi $\rightarrow$ 2 ATP + 2 \water
        $\Delta$G$^{'\circ}_2$ = 2(30.5 kJ/mol) = 61.0 kJ/mol
    \end{center}
\end{itemize}

\subsection*{The Standard Free-Energy Change of Glycolysis, $\Delta$G$^{'\circ}_{\text{Sum}}$}
\begin{itemize}
	\item The sum of the two processes gives the overall standard free-energy change of glycolysis, $\Delta$G$^{'\circ}_{\text{Sum}}$:
\end{itemize}
\begin{align*}
    \Delta \text{G}^{'\circ}_{\text{Sum}} &= \Delta \text{G}^{'\circ}_1 + \Delta \text{G}^{'\circ}_2\\
    &= -146 \text{kJ/mol} + 61.0 \text{kJ/mol}\\
    &= -85 \text{kJ/mol}
\end{align*}
\begin{itemize}
    \item \textbf{under standard and cellular conditions, glycolysis is essentially irreversible}
\end{itemize}

\subsection*{Energy Remaining in Pyruvate}
\begin{itemize}
	\item energy stored in pyruvate can be extracted by:
	\begin{itemize}
        \item \textbf{aerobic processes:}
        \begin{itemize}
            \item oxidative reactions in the citric acid cycle (TCA cycle)
            \item oxidative phosphorylation
        \end{itemize}
        \item \textbf{anaerobic processes:}
        \begin{itemize}
            \item reduction to lactate
            \item reduction to ethanol
        \end{itemize}
    \end{itemize}
    \item pyruvate can provide the carbon skeleton for alanine synthesis or fatty acid synthesis
\end{itemize}

\end{document}