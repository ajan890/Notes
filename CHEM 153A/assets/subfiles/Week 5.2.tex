% document formatting
\documentclass[10pt]{article}
\usepackage[utf8]{inputenc}
\usepackage[left=1in,right=1in,top=1in,bottom=1in]{geometry}
\usepackage[T1]{fontenc}
\usepackage{xcolor}

% math symbols, etc.
\usepackage{amsmath, amsfonts, amssymb, amsthm}

% lists
\usepackage{enumerate}
\usepackage{tabularx}
\usepackage{multicol}
\usepackage[table,xcdraw]{xcolor}

% images
\usepackage{graphicx} % for images

% code blocks
\usepackage{minted, listings} 

% verbatim greek
\usepackage{alphabeta}

\graphicspath{{../images/Week 5}}

\newcommand{\solution}{\textbf{Solution:}} 
\newcommand{\example}{\textbf{Example: }}
\newcommand{\water}{\text{H$_2$O}}
\newcommand{\hydroxide}{\text{OH$^-$}}
\newcommand{\hydronium}{\text{H$_3$O$^+$}}
\newcommand{\proton}{\text{H$^+$}}
\newcommand{\pc}{$^+$}
\newcommand{\nc}{$^-$}
\newcommand{\ka}{\text{$K_\text{a}$}}

\begin{document}
\subsection*{Lipoproteins}
\begin{center}
    \includegraphics*[width=\textwidth]{L3_1.png}
\end{center}

\subsection*{Receptor-Mediated Endocytosis}
apoB-100 of LDL is recognized by receptors in the plasma membrane.
\begin{center}
    \includegraphics*[scale=0.6]{L3_2.png}
\end{center}

\subsection*{Some Common Types of Storage and Membrane Lipids}
\begin{center}
    \includegraphics*[width=\textwidth]{L3_3.png}
\end{center}

\subsection*{Cell Membrane: Fluid Mosaic Model}
\begin{center}
    \includegraphics*[width=\textwidth]{L3_4.png}\\
    \includegraphics*[width=0.8\textwidth]{L3_5.png}
\end{center}

\subsection*{Membrane Permeability}
\begin{itemize}
    \item Low permeability for ions.  
    \item Small non-polar molecules (O$_2$, CO$_2$, H$_2$O) pass more easily.
\end{itemize}
\begin{center}
    \includegraphics*[width=\textwidth]{L3_6.png}
\end{center}

\subsection*{Types of Phospholipids}
\begin{center}
    \includegraphics*[scale=0.7]{L3_7.png}
\end{center}
\subsubsection*{Glycerophospholipids}
\begin{itemize}
    \item Backbone: Glycerol
    \item Structure:  Two fatty acids attached to the glycerol backbone, a phosphate group (PO$_4$) linked to an alcohol (e.g., choline or ethanolamine)
    \item Function: In cell membranes, it has a role in structural integrity and signaling
\end{itemize}
\subsubsection*{Sphingolipids:}
\begin{itemize}
    \item Backbone: Sphingosine
    \item Structure: One fatty acid attached to the sphingosine backbone, a phosphate group (PO$_4$) linked to choline or another alcohol group
    \item Function: In cell membranes, particularly in neural tissues, they contribute to signaling and cellular recognition
\end{itemize}

\subsection*{Glycerophospholipids}
\begin{center}
    \includegraphics*[width=\textwidth]{L3_8.png}
\end{center}
\textbf{Structure:}  Glycerophospholipids are based on a glycerol backbone, with two fatty acids and a phosphate group attached.  The phosphate group is further linked to a polar "head group" (e.g., ethanolamine, choline, serine, glycerol).\\\\
\textbf{Roles:}
\begin{enumerate}
    \item \textbf{Precursor function:} Phosphatidic acid serves as the precursor to other glycerophospholipids
    \item \textbf{Membrane components:} Glycerophospholipids like phosphatidylethanolamine, phosphatidylcholine, and phosphatidylserine are crucial for membrane structure and function
    \item \textbf{Signaling pathways:} Specialized glycerophospholipids (e.g., phosphatidylinositol 4, 5-bisphosphate) participate in intracellular signaling, regulating processes like calcium release and enzyme activation
\end{enumerate}
\begin{itemize}
    \item \textbf{Cardiolipin:} Vital for mitochondrial membrane integrity and function
\end{itemize}

\subsection*{Sphingolipids}
\begin{center}
    \includegraphics*[width=\textwidth]{L3_9.png}
\end{center}
\textbf{Structure:}  Sphingolipids are built on a sphingosine backbone (rather than glycerol), with a single fatty acid and a polar head group.  The head groups can vary, producing different sphingolipid types
\textbf{Roles:}
\begin{enumerate}
    \item \textbf{Precursor function:} Ceramide acts as a precursor for more complex sphingolipids
    \item \textbf{Membrane components:} Sphingomyelin is a major component of myelin sheaths in nerve cells
    \item \textbf{Glycolipids:} These include neutral glycosphingolipids (e.g., glucosylceramide) and more complex gangliosides.  These glycolipids play roles in cell recognition, signaling, and interactions.
\end{enumerate}

\subsection*{Phospholipid Asymmetry}
\begin{itemize}
    \item Cell membranes maintain asymmetry in phospholipid content
    \item Why?  Likely helps maintain different electric environments and contribute to structure - also relevant for biological regulation
    \item If \underline{phosphatidiylserine} (PS) is seen on the outer membrane, it's targeted for phagocytosis
    \begin{itemize}
        \item In platelets, this triggers aggregation
    \end{itemize}
\end{itemize}
\begin{center}
    \includegraphics*[width=0.8\textwidth]{L3_10.png}
\end{center}

\subsection*{Lipids in Organelles}
\begin{center}
    \includegraphics*[width=0.8\textwidth]{L3_11.png}
\end{center}
Abbreviations for the pie charts:
\begin{itemize}
    \item Phosphatidylcholine (PC)
    \item Phosphatidylethanolamine (PE)
    \item Phosphatidylserine (PS)
    \item Phosphatidylglycerol (PG)
    \item Cardiolipin (CL)
    \item Sphingomyelin (SM)
    \item Other - Includes Phosphatidic Acid (PA)
    \item Diacylglycerol (DAG) and Lysolipids
\end{itemize}
\begin{itemize}
    \item Lipids show distinct distributions and functions acroos different organelles, with specific abundances in organelles/compartments such as the \textbf{endoplasmic reticulum (ER), mitochondria, lysosomes, nucleus, Golgi apparatus, plasma membrane (PM), endosomes,} and \textbf{lipid droplets.}
\end{itemize}
\pagebreak
\subsection*{Phospholipid Diffusion}
\begin{center}
    \includegraphics*[scale=0.8]{L3_12.png}
\end{center}
Changes in lipid organization can affect various cellular functions, such as membrane trafficking or signal transduction.  These membrane-related effects can cause disease in living organisms due to genetic alterations, environmental factors (e.g., high dietary intake of saturated fats), or both

\subsection*{Flippase, Floppase, and Scramblase}
\begin{itemize}
    \item \textbf{Flippase and Floppase} maintain asymmetry by putting phospholipids where they're "supposed" to be
    \item \textbf{Scramblase} indiscriminately flips phospholipids both directions - \underline{what would happen if scramblase} \underline{dominated?}
    \begin{itemize}
        \item Phospholipid content would \textbf{even out} (50:50 equilibrium)
        \item This would include PS
    \end{itemize}
    \item \textbf{Scramblase} is activated by levels of Ca$^{2+}$ signaling that trigger in case of damage - beginnings of apoptosis (conversely flippase and floppase are inhibited)
\end{itemize}
\begin{center}
    \includegraphics*[width=\textwidth]{L3_13.png}\\
    \includegraphics*[width=\textwidth]{L3_14.png}
\end{center}

\subsection*{Saturated and Unsaturated Phospholipid Structure}
\begin{center}
    \includegraphics*[width=\textwidth]{L4_1.png}
\end{center}
\begin{itemize}
    \item Saturated phospholipids generate more rigid structures (Left). 
    \item Unsaturated phospholipids generate more fluid structures (Right).
\end{itemize}

\subsection*{Phospholipids - Cell Regulation}
Cells can regulate their own lipid composition to maintain membrane fluidity (called \textbf{homeoviscous adaptation})
\begin{center}
    \includegraphics*[width=\textwidth]{L4_2.png}
\end{center}

\subsection*{Moderation of membrane fluidity by Cholesterol}
\begin{itemize}
    \item As we've already discussed, cholesterol serves as a biosynthetic precursor to sterol hormones
    \item Also serves as an important membrane structural component
    \item Polar head group (-OH) creates hydrogen bonding with sphingolipids, creating \textbf{lipid rafts}
\end{itemize}
\begin{center}
    \includegraphics*[width=\textwidth]{L4_3.png}
\end{center}
\begin{itemize}
    \item At \underline{low temperatures}, phospholipids pack tightly, but cholesterol's rigid four ring structure reduces extent of packing
    \item At \underline{high temperatures}, phospholipids pack loosely, but the intermolecular interactions with cholesterol keeps phospholipids somewhat closer together
    \item All in all, this means that cholesterol moderates membrane fluidity.
\end{itemize}
\begin{center}
    \includegraphics*[width=\textwidth]{L4_4.png}
\end{center}
\begin{itemize}
    \item The mammals have evolved sophisticated and complex mechanisms to maintain plasma cholesterol levels, as well as cell membrane cholesterol levels, within a narrow physiological range.
\end{itemize}
\begin{center}
    \includegraphics*[width=\textwidth]{L4_5.png}
\end{center}

\subsection*{Cholesterol Homeostasis}
\begin{itemize}
    \item Animal cells maintain cholesterol homeostasis by transporting cholesterol from one membrane to another.  Cholesterol derived from low-density lipoprotein (LDL) is taken into cells through endocytosis mediated by LDL receptors (LDLRs)
    \item The LDL-derived cholesterol is released in lysosomes and then transported to the plasma membrane (PM), where it plays a structural role, and to the endoplasmic reticulum (ER) membrane.
\end{itemize}

\subsection*{Membrane Proteins Redux}
\begin{itemize}
    \item We've dealt with how proteins use their primary/secondary structures to associate with membranes, but not with any lipid-based solutions.
    \item Enter \textbf{lipid-anchored proteins}, proteins located on the surface of the cell membrane that are covalently attached to lipids embedded within the cell membrane
\end{itemize}
\begin{center}
    \includegraphics*[width=0.8\textwidth]{L4_6.png}
    \includegraphics*[width=\textwidth]{L4_7.png}
\end{center}


\end{document}
