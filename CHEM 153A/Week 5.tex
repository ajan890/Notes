% document formatting
\documentclass[10pt]{article}
\usepackage[utf8]{inputenc}
\usepackage[left=1in,right=1in,top=1in,bottom=1in]{geometry}
\usepackage[T1]{fontenc}
\usepackage{xcolor}

% math symbols, etc.
\usepackage{amsmath, amsfonts, amssymb, amsthm}

% lists
\usepackage{enumerate}
\usepackage{tabularx}
\usepackage{multicol}
\usepackage[table,xcdraw]{xcolor}

% images
\usepackage{graphicx} % for images

% code blocks
\usepackage{minted, listings} 

% verbatim greek
\usepackage{alphabeta}

\graphicspath{{./assets/images/Week 5}}

\newcommand{\solution}{\textbf{Solution:}} 
\newcommand{\example}{\textbf{Example: }}
\newcommand{\water}{\text{H$_2$O}}
\newcommand{\hydroxide}{\text{OH$^-$}}
\newcommand{\hydronium}{\text{H$_3$O$^+$}}
\newcommand{\proton}{\text{H$^+$}}
\newcommand{\pc}{$^+$}
\newcommand{\nc}{$^-$}
\newcommand{\ka}{\text{$K_\text{a}$}}

\title{CHEM 153A Week 5}

\author{Aidan Jan}
\date{\today}

\begin{document}
\maketitle

\section*{Carbohydrates}
\begin{itemize}
    \item \textbf{Carbohydrates} = aldehydes or ketones with at least two hydroxyl groups, or substances that yield such compounds on hydrolysis
    \item many carbohydrates have the empirical formula (CH$_2$O)$_n$
    \item \textbf{monosaccharides} = simple sugars, consisting of a single polyhydroxy aldehyde or ketone unit
    \begin{itemize}
        \item Example: D-glucose
    \end{itemize}
    \item \textbf{disaccharides} = oligosaccharides with two monosaccharide units
    \begin{itemize}
        \item Example: sucrose (D-glose and D-fructose)
    \end{itemize}
    \item \textbf{Oligosaccharides} = short chains of monosaccharide units, or residues, joined by glycosidic bonds
    \item \textbf{Polysaccharides} = sugar polymers with 10+ monosaccharide units
    \begin{itemize}
        \item Examples: cellulose (linear), glycogen (branched)
    \end{itemize}
\end{itemize}

\subsection*{Functions of Carbohydrates}
\begin{itemize}
    \item Energy sources (e.g., large polymers for energy storage)
    \item Protein targeting
    \item Cell identification/recognition (e.g., determination of blood type)
    \item Protection/structure (e.g., call walls and insect shells)
    \item Components of other biomolecules (e.g., antibiotics, enzyme cofactors, nucleic acids, etc.)
    \item Other (e.g., lubrication in joints)
\end{itemize}

\subsection*{From previous courses:}
\begin{itemize}
    \item What a carbohydrate is
    \item Different kinds of sugars (mono-, di-, poly-)
    \item Basic structures of a few monosaccharides
    \item Basic nomenclature
    \begin{itemize}
        \item Aldoses vs. Ketoses (functional group)
        \item C$_3$ = triose
        \item C$_4$ = tetraose
        \item C$_5$ = pentose
        \item C$_6$ = hexose
    \end{itemize}
\end{itemize}

\subsection*{Aldoses and Ketoses}
\begin{itemize}
    \item \textbf{aldose} = carbonyl group is at an end of the carbon chain (in an aldehyde group)
    \item \textbf{ketose} = carbonyl group is at any other position (in a ketone group)
\end{itemize}
\begin{center}
    \includegraphics*[width=\textwidth]{L1_1.png}
\end{center}

\subsection*{Basic Trioses - D and L}
\begin{itemize}
    \item The below structures are all \textbf{trioses} (three carbon sugars) - basically the simplest sugars possible
    \item For $n$ stereocenters, there are $2$ possible stereoisomers (making $2^n$ stereoisomers total)
    \item Glyceraldehyde has $1$ stereocenter, hence $2$ stereoisomers, notated as D and L
    \begin{itemize}
        \item If the OH on the last chiral carbon points to the right, the sugar is D
        \item If the OH on the last chiral carbon points to the left, the sugar is L
    \end{itemize}
\end{itemize}
\begin{center}
    \includegraphics*[width=\textwidth]{L1_2.png}
\end{center}

\subsubsection*{Enantiomers of Glyceraldehyde}
\begin{center}
    \includegraphics*[width=\textwidth]{L1_3.png}
\end{center}

\subsection*{D Isomers and L Isomers}
\begin{itemize}
    \item \textbf{reference carbon} = chiral center \textit{most distant} from the carbonyl carbon
    \item two groups of stereoisomers:
    \begin{itemize}
        \item D isomers = configuration at reference carbon is the same as D-glyceraldehyde
        \begin{itemize}
            \item on the right (dextro) in a projection formula
            \item most hexoses of living organisms
        \end{itemize}
        \item L isomers = configuration at reference carbon is the same as L-glyceraldehyde
        \begin{itemize}
            \item on the left (levo) in a projection formula
        \end{itemize}
    \end{itemize}
\end{itemize}

\subsection*{Fischer Projection Review}
\begin{itemize}
    \item \textbf{Fischer projections are the preferred representation of linear carbohydrates}
    \item \underline{Vertical lines} represent dashes angled away from you
    \item \underline{Horizontal lines} represent wedges angled towards you
    \item Fischer projections of longer sugars are set up in such a way as to easily transition to cyclic molecules
\end{itemize}
\begin{center}
    \includegraphics*[width=\textwidth]{L1_4.png}
\end{center}

\subsection*{Ketose Family}
The carbons of a sugar are numbered beginning at the end of the chain nearest the carbonyl group
\begin{itemize}
    \item C$_3$ = triose
    \item C$_4$ = tetraose
    \item C$_5$ = pentose
    \item C$_6$ = hexose
\end{itemize}
\begin{center}
    \includegraphics*[scale=0.6]{L1_5.png}
\end{center}
3 stereocenters, so $2^3$ stereoisomers, but we're excluding L sugars so $8/2=4$.

\subsection*{Aldose Family}
\begin{center}
    \includegraphics*[scale=0.6]{L1_6.png}
\end{center}

\subsection*{The Common Monosaccharides Have Cyclic Structures}
\begin{itemize}
    \item in aqueous solution, aldotetroses and all monosaccharides with 5+ backbone carbon atoms occur as cyclic structures
    \begin{itemize}
        \item covalent bond between the carbonyl group and the oxygen of a hydroxyl group $\rightarrow$ cyclic hemiacetal
    \end{itemize}
    \item Linear sugars cyclize via intramolecular hemiacetal formation
    \item Favored ring sizes: 5- and 6-membered rings (furanoses and pyranoses)
\end{itemize}

\subsection*{Formation of the Two Cyclic Forms of D-Glucose}
\begin{itemize}
    \item Reaction between the aldehyde group at C-1 and the hydroxyl group at C-5 forms a \textbf{hemiacetal linkage}
    \item \textbf{mutarotation} = the interconversion of $\alpha$ and $\beta$ anomers
\end{itemize}
\begin{center}
    \includegraphics*[scale=0.6]{L1_7.png}
\end{center}
Mutarotation is the interconversion of $\alpha$ and $\beta$ anomers

\subsection*{Cyclization of Monomeric Sugars}
\begin{itemize}
    \item Previous linear molecule had four stereocenters
    \begin{itemize}
        \item \textbf{Cyclization generates a new stereocenter at C$_1$}
    \end{itemize}
    \item Two possiblities can form, $\alpha$ and $\beta$ anomers
    \begin{itemize}
        \item Notation denotes whether -OH is on the same side of ring as C$_6$
        \begin{itemize}
            \item $\alpha$ is opposite side
            \item $\beta$ is same side
        \end{itemize}
    \end{itemize}
    \item For D-sugars, C$_6$ is above the ring so always the same behavior
    \begin{itemize}
        \item $\alpha$, -OH is down (trans)
        \item $\beta$, -OH is up (cis)
    \end{itemize}
    \item Cyclization is a \textbf{reversible} process
    \begin{itemize}
        \item This means anomers can interchange
        \begin{itemize}
            \item This is called \textbf{mutarotation}
        \end{itemize}
        \item \textbf{C$_1$ is now the \underline{anomeric carbon}}
    \end{itemize}
\end{itemize}

\subsection*{Mutarotation}
\begin{itemize}
    \item In aqueous solution, cyclic sugars like glucose exist in equilibrium with their open-chain form.
    \item  When a sugar molecule opens up, \textbf{the hemiacetal bond at the anomeric carbon is broken}, forming the linear aldehyde (or ketone) form.
    \item This open-chain form can then reclose to form either the $\alpha$- or $\beta$-anomer.
    \item The direction from which the hydroxyl group attacks the carbonyl carbon determines whether the -OH at the anomeric carbon ends up in the axial ($\alpha$) or equatorial ($\beta$) position. Under acidic or basic conditions, this equilibrium is established more rapidly, but it occurs spontaneously even in pure water. Thus, mutarotation is the chemical process by which the configuration at the anomeric center interconverts, switching the $\alpha$- and $\beta$-anomeric forms.    
\end{itemize}

\subsection*{Cyclization of monomeric sugars produces either of two stereoisomeric configurations: $\alpha$ and $\beta$}
\begin{center}
    \includegraphics*[width=\textwidth]{L1_8.png}
\end{center}
The anomeric carbon serves as the \textbf{connection} to other monosaccharides\\\\
The \textbf{anomeric carbon} is highly reactivev because it retains some of the original reactivity of the carbonyl group from the linear form.  This makes it ideal for forming \textbf{glycosidic bonds,} which are the covalent linkages between sugars or between a sugar and another molecule (such as a protein or lipid)

\subsection*{Cyclization of Linear Monosaccharides Mechanism}
\begin{center}
    \includegraphics*[width=\textwidth]{L1_9.png}
\end{center}

\subsection*{Pyranoses and Furanoses}
\begin{center}
    \includegraphics*[scale=0.6]{L1_10.png}
\end{center}
\begin{itemize}
    \item Six-membered ring compounds are called \textbf{pyranoses} because they resemble the six-membered ring compound pyran.  The systematic names for the two ring forms of D-glucose are therefore $\alpha$-D-glucopyranose and $\beta$-D-glucopyranose
    \item Ketohexoses (such as fructose) also occur as cyclic compounds with $\alpha$ and $\beta$ anomeric forms.  In these compounds, the hydroxyl group at C-5 (or C-6) reacts with the heto group at C-2 to form a \textbf{furanose} (or pyranose) ring containing a hemiketal linkage.
    \item D-Fructose readily forms the furanose ring the more common anomer of this sugar in combined forms or in derivatives is $\beta$-D-fructofuranose
\end{itemize}

\subsection*{Furanoses can exist in equilibrium with pyranoses}
\begin{center}
    [FILL 19]
\end{center}

\subsection*{Hemoglobin Glycation}
(so named to distinguish it from glycosylation, the enzymatic transfer of glucose to a protein)

\begin{itemize}
    \item A person's glycated hemoglobin fraction (HbA1c) reflects the average concentration of glucose in the blood (AG) over the past 2-3 months and is the gold standard measure for establishing risk for diabetes-related complications in patients with type 1 or type 2 diabetes
\end{itemize}
\begin{center}
    [FILL 20]
    [FILL 21, edited, combined]
\end{center}

\subsection*{Glycosidic Bond Formation}
\begin{itemize}
    \item \textbf{Glycosidic bonds} are covalent bonds that form between the hemiacetal group of a carbohydrate and a hydroxyl group on another compound (for us, other carbohydrates)
    \item This is how disaccharides, oligosaccharides, and polysaccharides form
    \begin{itemize}
        \item \underline{Glycosidic bonds are notated with:} The carbon they come from, the carbon they're going to, and the $\alpha/\beta$ arrangement (determined by the anomeric carbon)
    \end{itemize}
\end{itemize}
\begin{center}
    [FILL 22]
\end{center}

\subsection*{Nomenclature for Glycosidic Bonds}
\begin{itemize}
    \item Anomeric Configuration:
    \begin{itemize}
        \item $\alpha$: In a Haworth projection of a D-pyranose, the -OH on the anomeric carbon is positioned on the side opposite the CH$_2$OH group (typically shown "down" or in an axial orientation)
        \item $\beta$: The -OH on the anomeric carbon is on the same side as the CH$_2$OH group (typically shown "up" or in an equatorial orientation)
    \end{itemize}
    \item Linkage Positions:
    \begin{itemize}
        \item The notation (e.g., 1$\rightarrow$4) indicates that the bond forms between the anomeric carbon (C1) of one sugar and a specific carbon (e.g., C4) on the adjacent sugar
        \item Other common linkages include:
        \begin{itemize}
            \item 1$ \rightarrow$ 6: For example, in glycogen, the main chain is formed by $\alpha$(1$\rightarrow$4) linkages with branch points at $\alpha$(1$\rightarrow$6) bonds
            \item 1$\rightarrow$ 2: In sucrose, the glycosidic bond links the anomeric carbon of glucose to the anomeric carbon of fructose (commonly described as $\alpha$-D-glucopyranosyl-(1$\rightarrow$2)-$\beta$-D-fructofuranoside)
        \end{itemize}
    \end{itemize}
    \item Examples:
    \begin{itemize}
        \item Amylose (starch): Composed of $\alpha$(1$\rightarrow$4) glycosidic bonds
        \item Cellulose: Consists of $\beta$(1$\rightarrow$4) glycosidic bonds
        \item Glycogen: Features $\alpha$(1$\rightarrow$4) bonds in its main chain with $\alpha$(1$\rightarrow$6) bonds at branch points
        \item Sucrose: Has an $\alpha$(1$\rightarrow$2) glycosidic bond linking the two monosaccharides
    \end{itemize}
\end{itemize}
\begin{center}
    [FILL 24]
\end{center}
Note that the glycosidie bond can be alpha or beta, and that the second carbohydrate can be linked at any of the carbon atoms that contain an -OH.  The glycosidic bond is named as alpha or beta, followed by numbers that correspond to the locations of the carbons involved in the glycosidic bond.

\section*{Polysaccharides}
\textbf{Monomeric subunits, monosaccharides, serve as the building blocks of large carbohydrate polymers.}  The specific sugar, the way the units are linked, and whether the polymer is branched determine its properties and thus its function
\begin{itemize}
    \item Most carbohydrates in nature occur as polysaccharides ($M_r > 20000$)
    \item also called \textbf{glycans}
\end{itemize}
\begin{center}
    [FILL 24]
\end{center}

\subsection*{Homopoolysaccharides and Heteropolysaccharides}
\begin{itemize}
    \item \textbf{Homopolysaccharides} = contain only a single monomeric sugar species
    \begin{itemize}
        \item Serve as storage forms and structural elements
    \end{itemize}
    \item \textbf{Hetoropolysaccharides} = contain 2+ kinds of monomers
    \begin{itemize}
        \item provide extracellular support
    \end{itemize}
\end{itemize}
\begin{center}
    [FILL 26]
\end{center}

\subsection*{Polsaccharides Generally Do Not Have Defined Lengths or Molecular Weights}
\begin{itemize}
    \item This distinction between proteins and polysaccharides is a consequence of the mechanisms of assembly
    \item There is no template for polysaccharide synthesis
    \item The program for polysaccharide synthesis is intrinsic to the enzymes that catalyze the polymerization of monomer units
\end{itemize}
The sequences of complex polysaccharides are determined by the intrinsic properties of the biosynthetic enzymes that add each monomeric unit to the growing polymer.

\subsection*{Some Homopolysaccharides are storage forms of fuel}
\begin{itemize}
    \item storage polysaccharides = starch in plant cells and glycogen in animal cells
    \item \textbf{starch and glycogen molecules are heavily hydrated because they have many exposed hydroxyl groups available to hydrogen bond}
\end{itemize}

\subsection*{Starch (Amylose)}
\begin{itemize}
    \item \textbf{Amylose (starch)} is a \underline{polysaccharide} if $\alpha$-D-glucose that's used for \textbf{energy storage in plants}
    \item Standard 1,4 linkage, no branches
    \item Human enzymes can break down $\alpha$ linkages and utilize for energy
\end{itemize}
\begin{center}
    [FILL 28]
\end{center}
Starch (amylose) isn't a straight chain!
\begin{itemize}
    \item On average, 8 glucose residues per turn.
\end{itemize}
\begin{center}
    [FILL 29]
\end{center}

\subsection*{Glycogen}
\begin{itemize}
    \item \textbf{Glycogen} is a \underline{polysaccharide} of $\alpha$-D-glucose that's used for \textbf{energy storage in animals}
    \item Main linkage is 1,4 but after 8-10 residues there's a 1,6 linkage as well
    \item Multiple branches reduce time it takes for breakdown and utilization
    \item Human enzymes can more easily break down $\alpha$ linkages
\end{itemize}
\begin{center}
    [FILL 30]
    [FILL 31]
\end{center}
Glycogen isn't a straight chain either!
\begin{center}
    [FILL 32]
\end{center}

\subsection*{Some Homopolysaccharides Serve Structural Roles}
\textbf{Cellulose} = tough, fibrous, water-insoluble substance
\begin{itemize}
    \item linear, unbranched homopolysaccharide, consisting of 10000 to 15000 D-glucose units
    \item glucose residues have the $\beta$ configuration
    \item linked by ($\beta 1 \rightarrow 4$) glycosidic bonds
    \item animals do not have the enzyme to hydrolyze ($\beta 1 \rightarrow 4$) glycosidic bonds
\end{itemize}
\begin{center}
    [FILL 33-L]
\end{center}
\textbf{Chitin} = linear homopolysaccharide composed of $n$-acetylglucosamine residues in ($\beta 1 \rightarrow 4$) linkage
\begin{itemize}
    \item Acetylated amino group makes chitin more hydrophobic and water-resistant than cellulose
\end{itemize}
\begin{center}
    [FILL 33-R]
\end{center}

\subsection*{Cellulose}
\begin{itemize}
    \item \textbf{Cellulose} is a \underline{polysaccharide} of $\beta$-D-glucose that's used \textbf{for structure in plants}
    \item Human enzymes can't break down $\beta$ linkages!  No way to use for energy
    \item \textbf{Cellulose has an alternating structure!}  (every other glucose is inverted)
    \item Cellulose chains can form strong intrachain and interchain hydrogen bonds: High tensile strength
\end{itemize}
\begin{center}
    [FILL 34]
    [FILL 35]
\end{center}

\subsection*{Chitin}
\begin{itemize}
    \item \textbf{Chitin} is a \underline{polysaccharide} of N-acetyl-D-glucosamine connected through a $\beta$($1 \rightarrow 4$) linkage
    \item Structure of chitin differs from cellulose because of acetamide groups
    \item Also has alternating structure and strong interchain/intrachain hydrogen bonding
\end{itemize}
\begin{center}
    [FILL 36]
\end{center}
It is a \textbf{structural polysaccharide} found in the exoskeletons of \textbf{insects, crustaceans} (like crabs and shrimp), and the \textbf{cell walls of fungi.}  Chitin is the second most abundant polysaccharide in nature after cellulose.
\begin{center}
    [FILL 37]
\end{center}
\end{document}