% document formatting
\documentclass[10pt]{article}
\usepackage[utf8]{inputenc}
\usepackage[left=1in,right=1in,top=1in,bottom=1in]{geometry}
\usepackage[T1]{fontenc}
\usepackage{xcolor}

% math symbols, etc.
\usepackage{amsmath, amsfonts, amssymb, amsthm}

% lists
\usepackage{enumerate}
\usepackage{tabularx}
\usepackage{multicol}
\usepackage[table,xcdraw]{xcolor}

% images
\usepackage{graphicx} % for images

% code blocks
\usepackage{minted, listings} 

% verbatim greek
\usepackage{alphabeta}

\graphicspath{{./assets/images/Week 9}}

\newcommand{\solution}{\textbf{Solution:}} 
\newcommand{\example}{\textbf{Example: }}
\newcommand{\water}{\text{H$_2$O}}
\newcommand{\hydroxide}{\text{OH$^-$}}
\newcommand{\hydronium}{\text{H$_3$O$^+$}}
\newcommand{\proton}{\text{H$^+$}}
\newcommand{\pc}{$^+$}
\newcommand{\nc}{$^-$}
\newcommand{\ka}{\text{$K_\text{a}$}}

% import subfiles
\usepackage{subfiles}

\title{CHEM 153A Week 9}

\author{Aidan Jan}
\date{\today}

\begin{document}
\maketitle
\section*{Glycolysis (Continued)}
\subsection*{Importance of Phosphorylated Intermediates}
\begin{itemize}
	\item All nine intermediates are phosphorylated
	\item Functions of the phosphoryl groups:
	\begin{itemize}
        \item Prevent glycolytic intermediates from leaving the cell
        \item Serve as essential components in the enzymatic conservation of metabolic energy
        \item Lower the activation energy and increase the specificity of the enzymatic reactions
    \end{itemize}
\end{itemize}

\section*{The Preparatory Phase of Glycolysis Requires ATP}
\begin{itemize}
	\item In the preparatory phase of glycolysis:
	\begin{itemize}
        \item Two molecules of ATP are invested to activate \textbf{glucose} to fructose \textbf{1,6-bisphosphate}
        \item The bond between C-3 and C-4 of fructose 1,6-bisphosphate is then broken to yield two molecules of triose phosphate
    \end{itemize}
\end{itemize}

\subsection*{(Step 1) Phosphorylation of Glucose}
\begin{itemize}
	\item Hexokinase activates glucose by phosphorylating at C-6 to yield \textbf{glucose 6-phosphate}
	\begin{itemize}
        \item ATP serves as the phosphoryl donor
        \item hexokinase requires Mg$^{2+}$ for its activity
        \item irreversible under intracellular conditions
    \end{itemize}
\end{itemize}
\begin{center} 
	\includegraphics*[width=0.8\textwidth]{L1_1.png}
\end{center}
Hexokinase commits glucose to the hexose \underline{phosphate pool} by converting glucose to glucose-6-phosphate (G6P)

\subsection*{Hexokinase mechanism basics}
\begin{itemize}
	\item Hexokinase relies on magnesium for stabilizing triphosphate
	\item Shielding the negative phosphate charges allows for nucleophilic attack by hydroxyl
	\item Example in \underline{metal-ion catalysis}
\end{itemize}
\begin{center} 
	\includegraphics*[width=\textwidth]{L1_2.png}
\end{center}
\textbf{Hexokinase Reaction:}
\begin{center}
    Glucose + ATP $\rightarrow$ Glucose-6-Phosphate (G6P) + ADP
\end{center}
\textbf{Commitment to Metabolic Pool:}
\begin{itemize}
	\item The hexose phosphate pool
	\item Significance:
	\begin{itemize}
        \item Traps glucose inside the cell (G6P cannot cross the cell membrane)
        \item Commits glucose to further metabolism within the cell
    \end{itemize}
\end{itemize}

\subsection*{The pool of hexoses}
\begin{center} 
	\includegraphics*[width=\textwidth]{L1_3.png}
\end{center}

\subsection*{(Step 2) Conversion of Glucose 6-Phosphate to Fructose 6-Phosphate}
\begin{itemize}
	\item Phosphohexose isomerase (phosphoglucose isomerase) catalyzes the reversible isomerization of glucose 6-phosphate to fructose 6-phosphate
	\begin{itemize}
        \item mechanism involves an enediol intermediate
        \item reaction readily proceeds in either direction
    \end{itemize}
    \begin{center} 
        \includegraphics*[width=0.9\textwidth]{L1_4.png}
    \end{center}
    \item The rearrangement of G6P to F6P is critical for th efficient progression of glycolysis.  \textbf{It ensures compatibility with downstream enzymes, facilitates the symmetrical cleavage of the sugar,} and \textbf{prepares the molecule for the energy-investment step catalyzed by PFK-1.}  Without this rearrangement, glycolysis could not proceed in a coordinated or efficient manner.
\end{itemize}

\subsection*{Phosphohexose isomerase mechanism}
\begin{center} 
	\includegraphics*[width=0.9\textwidth]{L1_5.png}
\end{center}

\subsection*{(Step 3) Phosphorylation of Fructose 6-Phosphate to Fructose 1,6-Bisphosphate}
\begin{itemize}
	\item Phosphofructokinase-1 (PFK-1) is a key regulatory enzyme in glycolysis
	\item Catalyzes the transfer of a phosphoryl group from ATP to fructose 6-phosphate to yield fructose 1,6-bisphosphate
	\begin{itemize}
        \item Essentially irreversible under cellular conditions
        \item The first "committed" step in the glycolytic pathway
    \end{itemize}
\end{itemize}
\begin{center} 
	\includegraphics*[width=0.9\textwidth]{L1_6.png}
\end{center}

\subsection*{Allosteric Regulation of PFK-1}
\begin{itemize}
	\item Activity increases when:
	\begin{itemize}
        \item ATP supply is depleted
        \item ADP and AMP accumulate
    \end{itemize}
    \item Fructose 2,6-bisphosphate is a potent allosteric activator
    \item PFK-1 acts as a metabolic "gatekeeper", integrating signals from the cell's energy status and hormonal environment.  This regulation allows glycolysis to be precisely tuned to the cell's energy demands, maintaining metabolic balance and energy homeostasis
    \begin{center} 
        \includegraphics*[scale=0.6]{L1_7.png}
    \end{center}
    \item Fructose 6-Phosphate (F6P), an intermediate of glycolysis, is phosphoylated by phosphofructokinase-2 (PFK-2) to form Fructose 2,6-bisphosphate (F2, 6BP).  F2,6BP is not an intermediate in glycolysis or gluconeogenesis but acts as a potent allosteric regulator of PFK-1, stimulating glycolysis and inhibiting gluconeogenesis.
\end{itemize}

\subsection*{(Step 4) Cleavage of Fructose 1,6-Bisphosphate}
\begin{itemize}
	\item Fructose 1,6-Bisphosphate aldolase (aldolase) catalyzes a reverse aldol condensation and cleaves fructose 1,6-bisphosphate to yield \textbf{glyceraldehyde 3-phosphate} and \textbf{dihydroxyacetone phosphate}
	\item Reversible because reactant concentrations are low in the cell.
\end{itemize}

\begin{center} 
	\includegraphics*[width=\textwidth]{L1_8.png}
\end{center}

\subsection*{The Class I Aldolase Reaction}
\begin{itemize}
	\item Class I = found in animals and plants
	\item Class II = found in fungi and bacteria
	\begin{itemize}
        \item Do not form the Schiff base intermediate
    \end{itemize}
\end{itemize}
\begin{center} 
	\includegraphics*[width=\textwidth]{L1_9.png}
\end{center}

\subsection*{(Step 5) Interconversion of the Triose Phosphates}
\begin{itemize}
	\item \textbf{Triose phosphate isomerase} converts dihydroxyacetone phosphate to glyceralehyde 3-phosphate
	\begin{itemize}
        \item reversible
        \item \underline{final step of the perparatory phase of glycolysis}
    \end{itemize}
\end{itemize}
\begin{center} 
	\includegraphics*[width=0.9\textwidth]{L1_10.png}
\end{center}

\subsection*{Fate of the Glucose Carbons in the Formation of Glyceraldehyde 3-Phosphate}
\begin{itemize}
	\item After Step 5 of glycolysis, the carbon atoms derived from C-1, C-2, and C-3 of the starting glucose are chemically indistinguishable from C-6, C-5, and C-4, respectively
\end{itemize}
\begin{center} 
	\includegraphics*[width=\textwidth]{L1_11.png}
\end{center}

\section*{The Payoff Phase of Glycolysos Yields ATP and NADH}
In the payoff phase of glycolysis:
\begin{itemize}
	\item Each of the two molecules of glyceraldehyde 3-phosphate undergoes \textbf{oxidation at C-1}
	\item Some energy from the oxidation reaction is conserved in the form of one \textbf{NADH and two ATP per triose phosphate oxidized}
\end{itemize}

\subsection*{(Step 6) Oxidation of Glyceraldehyde 3-Phosphate to 1,3-Bisphosphoglycerate}
\begin{itemize}
	\item \textbf{Glyceraldehyde 3-Phosphate Dehydrogenase} catalyzes the oxidation of glyceraldehyde 3-phosphate to \textbf{1,3-bisphosphoglycerate}
	\item This is an energy-conserving reaction
\end{itemize}
\begin{center} 
	\includegraphics*[width=\textwidth]{L1_12.png}
\end{center}
This reduction step stores energy with the formation of the acyl phosphate and in the form of high-energy electrons within NADH

\subsection*{The First Step of the Payoff Phase is an Energy-Conserving Reaction}
\begin{itemize}
	\item Formation of the \textbf{acyl phosphate} group at C-1 of 1,3-bisphosphoglycerate conserves the free energy of oxidation
	\item acyl phosphates have a very high standard free energy of hydrolysis ($\Delta G'^\circ = -49.3$ kJ/mol)
\end{itemize}

\subsection*{The Glyceraldehyde 3-Phosphate Dehydrogenase Reaction}
\begin{center} 
	\includegraphics*[width=\textwidth]{L1_13.png}
\end{center}
\begin{itemize}
	\item First, the thiolate ion attacks the carbonyl group of the substrate to form a thiohemiacetal, which is then oxidized to a thioester by transfer of a hydride ion (a hydrogen with two electrons, H\nc) to an enzyme-bound NAD\pc, with concurrent release of a proton (\proton).  Thus, in effect, two hydrogen atoms are removed from the substrate.
	\item Once NADH is formed, its affinity for the enzyme decreases, so that a free NAD\pc displaces this NADH.  The thioester is an energy-rich intermediate, and by phosphorolysis the high-energy 1,3-bisphosphoglycerate is generated with the release of the free enzyme.  Thus, the substrate aldehyde group is oxidized to a carboxylic acid group, with conservation of most of the energy of oxidation in formation of the anhydride bond between carboxylic and phosphoric acids.
\end{itemize}

\subsection*{Why This Process Works}
\begin{itemize}
	\item The \textbf{thioester intermediate} serves as a critical energy-rich intermediate that conserves the energy released during the oxidation of G3P.  This conserved energy is then used to drive the unfavorable phosphorylation step
	\item NAD\pc not only acts as an electron acceptor, forming NADH, but also activates the cysteine residue for catalysis
	\item The release of NADH ensures that the enzyme is ready to catalyze subsequent reactions efficiently
\end{itemize}

\subsection*{(Step 7) Phosphoryl Transfer from 1,3-Bisphosphoglycerate to ADP}
\begin{itemize}
	\item Phosphoglycerate Kinase transfers the high-energy phosphoryl group from the carboxyl group of 1,3-bisphosphoglycerate to ADP, forming ATP and \textbf{3-phosphoglycerate}
	\item substrate-level phosphorylation
\end{itemize}
\begin{center} 
	\includegraphics*[width=\textwidth]{L1_14.png}\\
    \includegraphics*[width=0.8\textwidth]{L1_15.png}
\end{center}

\subsection*{Steps 6 and 7 of Glycolysis Consistute an Energy-Coupling Process}
\begin{itemize}
	\item The sum of the two reactions is:
	\begin{center}
        Glyceraldehyde 3-Phosphate + ADP + P$_{\text{i}}$ + NAD\pc $\rightleftarrows$ phosphoglycerate + ATP + NADH + \proton\\
        $\Delta G'^\circ$ = -12.2 kJ/mol
    \end{center}
    \item \textbf{substrate-level phosphorylation} = the formation of ATP by phosphoryl group transfer from a substrate different from \textbf{respiration-linked phosphorylation}
    \item G3P dehydrogenase is coupled to phsophoglycerate kinase
    \begin{itemize}
        \item G3P dehydrogenase is forming a high energy phosphate while phosphoglycerate kinase is removing the phosphoryl group and adding it to ADP ($\Delta G < 0$ overall)
    \end{itemize}
\end{itemize}
\begin{center} 
	\includegraphics*[width=\textwidth]{L1_16.png}
\end{center}

\subsection*{(Step 8) Conversion of 3-Phosphoglycerate to 2-Phosphoglycerate}
\begin{itemize}
	\item \textbf{phosphoglycerate mutase} catalyzes a reversible shift of the phosphoryl group between C-2 and C-3 of glycerate
	\begin{itemize}
        \item requires Mg$^{2+}$
    \end{itemize}
\end{itemize}
\begin{center} 
	\includegraphics*[width=0.9\textwidth]{L1_17.png}
\end{center}

\subsection*{The Phosphoglycerate Mutase Reaction}
\begin{center} 
	\includegraphics*[width=0.9\textwidth]{L1_18.png}
\end{center}

\subsection*{(Step 9) Dehydration of 2-Phosphoglycerate to Phosphoenolpyruvate}
\begin{itemize}
	\item \textbf{enolase} promotes reversible removal of a molecule of water from 2-phosphoglycerate to yield \textbf{phosphoenolpyruvate (PEP)}
	\begin{itemize}
        \item energy-conserving reaction
        \item mechanism involves a Mg$^{2+}$-stabilized enolic intermediate
    \end{itemize}
\end{itemize}
\begin{center} 
    \includegraphics*[width=\textwidth]{L1_19.png}
\end{center}

\subsection*{(Step 10) Transfer of the Phosphoryl Group from Phosphoenolpyruvate to ADP}
\begin{itemize}
	\item \textbf{pyruvate kinase} catalyzes the transfer of the phosphoryl group from phosphoenolpyruvate to ADP, yielding \textbf{pyruvate}
	\item Requires K\pc and either Mg$^{2+}$ or Mn$^{2+}$
	\item \textbf{substrate-level phosphorylation} - the formation of ATP by phosphoryl group transfer from a substrate different from \textbf{respiration-linked phosphorylation}
\end{itemize}
\begin{center} 
	\includegraphics*[width=\textwidth]{L1_20.png}\\
    \includegraphics*[width=\textwidth]{L1_21.png}
\end{center}

\subsection*{Pyruvate in its Enol Form Spontaneously Tautomerizes to its Keto Form}
\begin{itemize}
	\item \textbf{pyruvate kinase} catalyzes the transfer of the phosphoryl group from phosphoenolpyruvate to ADP, yielding \textbf{pyruvate}
	\begin{itemize}
        \item requrires K\pc and either Mg$^{2+}$ or Mn$^{3+}$
    \end{itemize}
    \begin{center} 
        \includegraphics*[width=0.7\textwidth]{L1_22.png}
    \end{center}
\end{itemize}

\subsection*{The Overall Balance Sheet Shows a Net Gain of Two ATP and Two ADH per Glucose}
\begin{itemize}
	\item Subtracting the two ATP spent in the preparatory phase, the net equation for the overall process is:
	\begin{center} 
        glucose + 2 NAD\pc + 2 ADP + 2 P$_i$ $\rightarrow$ 2 pyruvate + 2 NADH + 2 \proton + 2 ATP + 2 \water
    \end{center}
\end{itemize}

\subsection*{Glycolysis Overview}
\begin{center} 
	\includegraphics*[width=\textwidth]{L1_23.png}
\end{center}

\subsection*{Energy Remaining in Pyruvate}
\begin{itemize}
	\item Energy stored in pyruvate can be extracted by:
	\begin{itemize}
        \item \textbf{aerobic processes:}
        \begin{itemize}
            \item oxidative reactions in the citric acid cycle (TCA cycle)
            \item oxidative phosphorylation
        \end{itemize}
        \item \textbf{anaerobic processes:}
        \begin{itemize}
            \item reduction to lactate
            \item reduction to ethanol
        \end{itemize}
    \end{itemize}
    \item pyruvate can provide the carbon skeleton for alanine synthesis of fatty acid synthesis
\end{itemize}
\end{document}