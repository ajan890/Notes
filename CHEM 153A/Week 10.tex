% document formatting
\documentclass[10pt]{article}
\usepackage[utf8]{inputenc}
\usepackage[left=1in,right=1in,top=1in,bottom=1in]{geometry}
\usepackage[T1]{fontenc}
\usepackage{xcolor}

% math symbols, etc.
\usepackage{amsmath, amsfonts, amssymb, amsthm}

% lists
\usepackage{enumerate}
\usepackage{tabularx}
\usepackage{multicol}
\usepackage[table,xcdraw]{xcolor}

% images
\usepackage{graphicx} % for images

% code blocks
\usepackage{minted, listings} 

% verbatim greek
\usepackage{alphabeta}

\graphicspath{{./assets/images/Week 10}}

\newcommand{\solution}{\textbf{Solution:}} 
\newcommand{\example}{\textbf{Example: }}
\newcommand{\water}{\text{H$_2$O}}
\newcommand{\hydroxide}{\text{OH$^-$}}
\newcommand{\hydronium}{\text{H$_3$O$^+$}}
\newcommand{\proton}{\text{H$^+$}}
\newcommand{\pc}{$^+$}
\newcommand{\nc}{$^-$}
\newcommand{\ka}{\text{$K_\text{a}$}}

% import subfiles
\usepackage{subfiles}

\title{CHEM 153A Week 10}

\author{Aidan Jan}
\date{\today}

\begin{document}
\maketitle
\section*{Citric Acid Cycle Regulation}
\begin{itemize}
	\item regulation balances the supply of key intermediates with the demands of energy production and biosynthetic processes
	\item regulation occurs at several points:
	\begin{itemize}
        \item PDH complex
        \item citrate synthase
        \item isocitrate dehydrogenase complex
        \item $\alpha$-ketoglutarate dehydrogenase complex
    \end{itemize}
\end{itemize}

\subsection*{Production of Acetyl-CoA by the PDH Complex is Regulated by Allosteric and Covalent Mechanisms}
\begin{itemize}
	\item PDH complex activity is turned off when:
	\begin{itemize}
        \item ample fatty acids and acetyl-CoA are available as fuel
        \item [ATP]/[ADP] and [NADH]/[NAD\pc] ratios are high
    \end{itemize}
	\item PDH complex activity is turned on when:
	\begin{itemize}
        \item energy demands are high
        \item the cell requires greater flux of acetyl-CoA into the citric acid cycle
    \end{itemize}
\end{itemize}

\subsection*{Regulation of Metabolite Flow Through the Citric Acid Cycle}
\textbf{The central role of the citric acid cycle in metabolism requires that it be regulated in coordination with many other pathways.}  Regulation occurs by both allosteric and covalent mechanisms that overlap and interact to achieve homeostasis.
\begin{center} 
	\includegraphics*[width=0.6\textwidth]{L1_1.png}
\end{center}

\subsection*{Covalent Modification of the PDH Complex}
\begin{itemize}
	\item \textbf{PDH Kinase} - inhibits the PDH complex by phosphorylation
	\begin{itemize}
        \item Allosterically activated by products of the complex
        \item Inhibited by substrates of the complex
    \end{itemize}
    \item \textbf{PDH phosphatase} = reverses the inhibition by PDH kinase
\end{itemize}
\begin{center} 
	\includegraphics*[width=0.5\textwidth]{L1_2}
\end{center}

\subsection*{The Citric Acid Cycle is also Regulated at Three Exergonic Steps}
\begin{itemize}
	\item regulation occurs at strongly exergonic steps catalyzed by:
	\begin{itemize}
        \item citrate synthase
        \item isocitrate dehydrogenase complex
        \item $\alpha$-ketoglutarate dehydrogenase complex
    \end{itemize}
    \item fluxes are affected by the concentrations of substrates and products:
    \begin{itemize}
        \item end products ATP and NADH are inhibitory
        \item NAD\pc and ADP are stimulatory
        \item long-chain fatty acids are inhibitory
    \end{itemize}
\end{itemize}

\subsection*{Role of the Citric Acid Cycle in Anabolism}
\begin{itemize}
	\item \textbf{Cataplerosis} describes the series of enzymatic reactions that draw down pools of metabolic intermediates
	\item \textbf{Anaplerosis} describes the series of enzymatic reactions or pathways that replenish pools of metabolic intermediates in the TCA cycle
	\item As intermediates of the citric acid cycle are removed to serve as biosynthesic precursors, they are replenished by \textbf{anaplerotic reactions}
\end{itemize}
\begin{center} 
    \includegraphics*[width=0.8\textwidth]{../Week 9/L4_23.png} 
\end{center}
Intermediates of the citric acid cycle are drawn off as precursors in many biosynthetic pathways.  Shown in red are four anaplerotic rea

\subsection*{Pyruvate Carboxylase}
\begin{itemize}
	\item Catalyzes the first step of \textbf{gluconeogenesis}
	\item Also replenishes oxaloacetate allowing TCA to continue
	\item Allosterically activated by acetyl-CoA
	\begin{itemize}
        \item Fate determination for pyruvate
    \end{itemize}
	\item Uses interesting cofactor called \textbf{biotin} that allows for carbon-carbon bond formation
\end{itemize}
\begin{center} 
	\includegraphics*[width=\textwidth]{L1_3.png}
\end{center}

\pagebreak
\subsection*{Structures of the B Vitamins along with their Role in Cells and the Disease Caused by their Deficiency}
Vitamins are organic compounds required in small amounts for human health, distinct from essential amino acids, fatty acids, and elements.
\begin{center} 
	\includegraphics*[width=\textwidth]{L1_4.png}
\end{center}
All B vitamins indeed act as precursors for \textbf{coenzymes} or are directly involved in enzymatic reactions.  Evolutionarily, animals lost the ability to biosynthesize vitamins, relying on dietary intake instead.

\subsection*{Functions of B-vitamin Coenzymes in Metabolism}
\begin{center} 
	\includegraphics*[width=\textwidth]{L1_5.png}
\end{center}

\subsection*{Cofactors and Their Role in Enzyme Function}
\begin{center} 
	\includegraphics*[width=\textwidth]{L1_6.png}
\end{center}
Cofactors are essential non-protein components that assist enzymes in catalyzing reactions.  They are classified into \textbf{metal cofactors} (e.g., magnesium, zinc, and iron) and \textbf{organic cofactors} (e.g., coenzymes like NAD\pc, FAD, and prosthetic groups).  Metal cofactors can activate enzymes directly, while coenzymes often act as electron carriers.  Together, cofactors and the protein portion of an enzyme (apoenzyme) form an active holoenzyme capable of binding substrates and catalyzing reactions effectively

\pagebreak
\section*{The Mitochondrial Respiratory Chain}
\subsection*{Electrons Are Funneled to Universal Electron Acceptors}
\begin{itemize}
	\item \textbf{respiratory chain} = series of electron carriers
    \item dehydrogenases collect electrons from catabolic pathways and funnel them into universal electron acceptors:
    \begin{itemize}
        \item nicotinamide nucleotides (NAD\pc or NADP\pc)
        \item flavin nucleotides (FMN or FAD)
    \end{itemize}
\end{itemize}

\subsection*{Electrons Pass through a Series of Membrane-Bound Carriers}
\begin{itemize}
	\item Three types of electron transfers occur in oxidative phosphorylation:
	\begin{itemize}
        \item direct transfer of electrons
        \item transfer as a hydrogen atom (\proton + $e^-$)
        \item transfer as a hydride ion (\textbf{:}H$^-$)
    \end{itemize}
    \item \textbf{reducing equivalent} = a single electron equivalent transferred in an oxidation-reduction reaction
\end{itemize}

\subsection*{Electron-Carrying Molecules in the Respiratory Chain}
\begin{itemize}
	\item Five types of electron-carrying molecules:
	\begin{itemize}
        \item NAD
        \item flavoproteins
        \item \textbf{ubiquinone (coenzyme Q or Q)}
        \item \textbf{cytochromes}
        \item \textbf{Iron-sulfur Proteins}
    \end{itemize}
\end{itemize}

\subsection*{Ubiquinone}
\begin{itemize}
	\item Ubiquinone (coenzyme \textbf{Q}) = a lipid-soluble benzoquinone with a long isoprenoid side chain
	\begin{itemize}
        \item Can accept one or two electrons
        \item Freely diffusible within the inner mitochondrial membrane
        \item Plays a central role in coupling electron flow to proton movement
    \end{itemize}
\end{itemize}
\begin{center} 
	\includegraphics*[width=0.4\textwidth]{L1_7.png}
\end{center}

\subsection*{Cytochromes}
\begin{itemize}
	\item \textbf{cytochromes} = proteins with characteristic strong absorption of visible light due to their iron-containing heme prosthetic groups
	\begin{itemize}
        \item one-electron carriers
        \item 3 classes in mitochondria: $a$, $b$, and $c$
        \begin{itemize}
            \item hemes of $a$ and $b$ are not covalently bound to associated proteins
            \item $c$ is covalently attached through Cys residues
        \end{itemize}
    \end{itemize}
\end{itemize}

\subsection*{Prosthetic Groups of Cytochromes}
\begin{center} 
	\includegraphics*[width=0.9\textwidth]{L1_8.png}
\end{center}

\subsection*{Spatial Context 1}
\begin{center} 
	\includegraphics*[width=0.9\textwidth]{L1_9.png}
\end{center}

\subsection*{Spatial Context 2}
\begin{center} 
	\includegraphics*[width=0.6\textwidth]{../Week 9/L3_7.png}
\end{center}
Cristae structure does \textbf{two} things:
\begin{itemize}
	\item Higher surface area allows for more ETC subunits (more ATP production)
	\item Allows for higher localized proton density, creating stronger gradient
\end{itemize}

\subsection*{Reduction Potential}
\begin{itemize}
	\item \textbf{Standard reduction potential} ($E^\circ$) is a measure of the tendency for a chemical species to be reduced
	\begin{itemize}
        \item The more positive the potential, the more favorable the reduction
        \item Completely proportional to $\Delta$G
        \[\Delta G^\circ_{cell} = -nFE^\circ_{cell}\]
        Connects Gibbs free energy change ($\Delta G^\circ_{cell}$) with the reduction potential ($E^\circ_{cell}$) where:
        \begin{itemize}
            \item $\Delta G^\circ_{cell}$ is the standard Gibbs free energy change
            \item $n$ is the number of electrons transferred
            \item $F$ is Faraday's constant (96485 C/mol)
            \item $E^\circ_{cell}$ is the standard cell potential
        \end{itemize}
        A \textbf{positive $E^\circ_{cell}$ results in a negative $\Delta G^\circ_{cell}$} indicating a spontaneous reaction, while a \textbf{negative $E^\circ_{cell}$ leads to a positive $\Delta G^\circ_{cell}$}, meaning the reaction is non-spontaneous
    \end{itemize}
    \item This can help us predict which direction redox reactions will flow naturally
    \begin{itemize}
        \item The less favorable reduction will flip to become an oxidation
    \end{itemize}
    \begin{align*}
        X^+ + e^- &\longrightarrow X \hspace{2cm} &\text{less favorable (lower Eo)}\\
        Y^+ + e^- &\longrightarrow Y \hspace{2cm} &\text{more favorable (higher Eo)}\\
        \\
        Y^+ + X &\longrightarrow Y + X^+ \hspace{2cm} &\text{net reaction}\\
        & &\text{$Y^+$ is reduced, and $X$ is oxidized.}
    \end{align*}
\end{itemize}

\section*{The Electron Transport Chain}
\subsection*{ETC Redox Overview}
Remember: these values indicate the tendency of each molecule to gain electrons (be reduced).
\begin{center} 
	\includegraphics*[width=\textwidth]{L2_1.png}
\end{center}
\begin{itemize}
	\item The flow of electrons in the ETC is "downhill" energetically, moving from molecules with lower Eo values (e.g., NADH at -0.315) to those with higher Eo values (e.g., O$_2$ at 0.815V)
	\item For example, Q at $+0.045$V to Cyt b at $+0.077$V
\end{itemize}

\subsection*{Electron Transport Chain: Complexes I to IV}
\begin{center} 
    \includegraphics*[width=\textwidth]{L2_2.png}
\end{center}

\subsection*{Protein Components of the Mitochondrial Respiratory Chain}
\begin{center} 
	\includegraphics*[width=\textwidth]{L2_3.png}
\end{center}
* generalization from organisms where it has already been studied.  Remember that there are always exceptions.

\subsection*{Complex I: NADH Oxidoreductase}
\begin{itemize}
	\item Also known as NADH oxidoreductase or NADH dehydrogenase
	\item Large, large L-shaped enzyme with >40 polypeptide chains
	\item Accepts 2 electrons from NADH and passes them to FMN (Flavin Mononucleotide)
	\item Then passes electrons through 8+ Fe/S clusters to Ubiquinone \textbf{one at a time}
	\item This complex uses this electrical work to pump 4 \proton ions out of the matrix and into the intermembrane space (likely an induced conformational change)
\end{itemize}
\begin{center} 
	\includegraphics*[width=0.8\textwidth]{L2_4.png}
\end{center}

\subsubsection*{(Review) Ubiquinone}
\begin{itemize}
	\item Ubiquinone, also known as coenzyme Q, is a lipid with a quinone ring structure at the top
	\item Can be reduced with two electrons, then travels to Complex III: freely diffusible within the inner mitochondrial membrane
	\item plays a central role in coupling electron flow to proton movement
\end{itemize}
\begin{center} 
	\includegraphics*[width=0.4\textwidth]{L1_7.png} 
\end{center}

\subsubsection*{Complex I Catalyzes Two Simultaneous and Obligately Coupled Processes}
Complex I catalyzes:
\begin{itemize}
	\item the exergonic transfer of a hydride ion (hydrogen atom with two electrons) from NADH and a proton from the matrix to ubiquinone
	\begin{center}
        NADH + \proton + Q $\rightarrow$ NAD\pc + QH$_2$
    \end{center}
    \item the endergonic transfer of 4 protons from the matrix to the intermembrane space
\end{itemize}

\subsubsection*{Electron Flow in Complex I}
\begin{itemize}
	\item \textbf{N1a} has a unique role compared to other Fe-S clusters: it can accept electrons from FMN but \textbf{does not always participate in the main electron transfer chain to ubiquinone.}  It may act as a reserve or moderate the process.
	\item \textbf{N3} is part of the main pathway and facilitates the transfer of electrons to downstream clusters like \textbf{N2} (which transfers electrons to ubiquinone)
\end{itemize}
\begin{center} 
	\includegraphics*[width=0.5\textwidth]{L2_5.png}
\end{center}

\subsubsection*{Complex I Overview}
\begin{center} 
	\includegraphics*[width=\textwidth]{L2_6.png}
\end{center}

\subsubsection*{Proton Wires}
\textbf{Protons are transported by proton "wires"} - a series of amino acids that undergo protonation and deprotonation
\begin{center} 
    \includegraphics*[width=\textwidth]{L2_7.png}
\end{center}
Proton transfer pathways, outlined by blue arrows.  Membrane arm contains the central axis of charged residues, essential for the proton transfer and the coupling

\subsection*{Pumping Protons and Free Energy}
\begin{center} 
	\includegraphics*[width=\textwidth]{L2_8.png}
\end{center}
\begin{itemize}
	\item Free energy is released on each step, with every reduction.  Oxygen is the final electron acceptor, with the lowest free energy and the highest electronegativity.
	\item To find how much free energy is released between two steps, subtract their values to find $\Delta E^\circ$, then plug into the equation $\Delta G^\circ = -nFE^\circ$.
	\begin{itemize}
        \item For example, to find the amount of free energy released through Complex I, subtract NADH oxidation ($E^\circ = -0.32$ V) from Ubiquinone reduction ($E^\circ = +0.045$ V).
        \item This gives $\Delta E^\circ = +0.36$ V.  Now, plug into the equation.  $n = 2$ since two electrons are transferred, and $F = 96.5$ kJ/mol (Faraday's constant)
        \item The result is $-70$ kJ/mol 
    \end{itemize}
\end{itemize}

\subsection*{The Coupling of Proton Pumping with Electron Flow}
\begin{center} 
	\includegraphics*[width=\textwidth]{L2_9.png}
\end{center}

\subsection*{Complex II: Succinate Dehydrogenase}
\begin{itemize}
	\item As we've discussed - succinate dehydrogenase oxidizes succinate to fumarate as part of the TCA
	\begin{itemize}
        \item The two electrons are passed to FAD, forming FADH$_2$
    \end{itemize}
    \item The electrons are then passed \textbf{one at a time} through 3 Fe/S clusters to ubiquinone (Q)
    \begin{itemize}
        \item Also passes ubiquinol (reduced to Q) to Complex III
        \item Effectively works in parallel with Complex I, \textbf{ETC can start from either Complex}
        \item \textbf{Does not transport protons} - This is why FADH$_2$ produces less ATP than NADH (1.5 vs. 2.5)
    \end{itemize}
\end{itemize}
\begin{center} 
	\includegraphics*[width=\textwidth]{L2_10.png}
\end{center}

\subsubsection*{Why Doesn't Complex II Pump Protons?}
\begin{itemize}
	\item Reduction of Fumarate and reduction of Ubiquinone only has a $\Delta E^\circ$ of $+0.014$ V!  There is not enough energy differential to pump protons.
\end{itemize}

\subsection*{Complex III: Cytochrome $c$ reductase}
\begin{itemize}
	\item Dimer of 11 subunits (22 in total)
	\begin{itemize}
        \item Relevance comes from the cavity in the center of the dimer
        \item \textbf{Cavity has two binding sites for Coenzyme Q molecules}
    \end{itemize}
	\item Uses two electrons from ubiquinol (CoQ) to reduce two molecules of cytochrome $c$ (does so sequentially)
	\item \textbf{Rieske center}, specialized Fe/S center with two His and Cys residues coordinating
	\item \textbf{Issue:} We no longer have access to Flavin cofactors, making it hard for the protein to hold onto two electrons at the same time
	\begin{itemize}
        \item \textbf{Solution:} Releasing one electron at a time to cytochrome $c$ while pumping the second electron through a secondary pathway called the Q cycle.
    \end{itemize}
    \begin{center} 
        \includegraphics*[width=0.9\textwidth]{L2_11.png}\\
        \includegraphics*[width=0.45\textwidth]{L2_12.png}
    \end{center}
\end{itemize}

\subsubsection*{Cytochromes}
\begin{itemize}
	\item Proteins with heme groups that are involved in redox reductions (as opposed to O$_2$ binding)
	\item \textbf{Oxidation state change in central iron allows for electron carrying}
	\item Can be embedded in complexes (e.g., cytochrome $b$, cytochrome $c_1$) or freely moving between them (cytochrome $c$)
\end{itemize}
\begin{center} 
	\includegraphics*[width=0.5\textwidth]{L2_13.png} 
\end{center}

\subsubsection*{The Q Cycle}
\begin{itemize}
	\item The two cavities in Complex III can bind both reduced \textbf{ubiquinol (QH$_2$)} and \textbf{oxidized ubiquinone (Q)}
	\item When ubiquinol (QH$_2$) attaches to its cavity, it releases one electron towards cytochrome $c$ (via the Fe-S cluster and cytochrome $c_1$), but also releases one to cytochrome b chain, \textbf{reducing a bound ubiquinone halfway (semiquinone radical: $Q \cdot^-$)}
	\item Same process occurs with a second ubiquinol, generating a second reduced cytochrome $c$, but \textbf{fully reduces one CoQ}
	\begin{itemize}
        \item This ensures that while two QH$_2$ molecules are oxidized, one QH$_2$ is regenerated, maintaining the balance of the ubiquinone pool
    \end{itemize}
    \begin{center} 
        \includegraphics*[width=0.5\textwidth]{L2_14.png} 
    \end{center}
    \item One reason for this cycle (besides lack of FAD/FMN) is that \proton ions can be donated directly to the intermembrane space from oxidized QH$_2$
    \begin{itemize}
        \item Four protons transported for every two electrons reaching cytochrome $c$
    \end{itemize}
\end{itemize}
\begin{center} 
	2 QH$_2$ + Q + 2 Cyto $c$ (Fe$^{3+}$) + 2 H\pc$_{\text{(matrix)}}$ $\rightarrow$ 2 Q + QH$_2$ + 2 Cyto $c$ (Fe$^{2+}$) + 4 \proton$_{\text{(IMS)}}$\\
    QH$_2$ + 2 Cyto $c$ (Fe$^{3+}$) + 2 \proton$_{\text{(matrix)}}$ $\rightarrow$ Q + 2 Cyto $c$ (Fe$^{2+}$) + 4 \proton$_{\text{(IMS)}}$ \hspace{1cm} \textbf{\underline{NET}}\\
    \includegraphics*[width=\textwidth]{L2_15.png} 
\end{center}

\subsubsection*{Complex III Overview}
\begin{center} 
    \includegraphics*[width=\textwidth]{L2_16.png} 
\end{center}

\subsection*{Complex IV: Cytochrome $c$ Oxidase}
\begin{itemize}
	\item Two cytochrome $c$ molecules each transfer one electron to Complex IV through two redox-active centers: two cytochrome groups (cyt a and cyt a3) and two copper atom groups (Cu$_A$ and Cu$_B$)
	\item One electron is held at the Cu$_B$ center
	\item The other is held at cytochrome a3
	\item Once oxygen ($O_2$) binds to cytochrome a3 and Cu$_B$, it accepts the two electrons, forming a peroxide bridge
\end{itemize}
Two additional cytochrome $c$ molecules donate two more electrons to the system.  These electrons, along with two protons (\proton), break the peroxide bridge, reducing the oxygen to two water molecules (2 \water)
\begin{itemize}
	\item The complete process involves the oxidation of \textbf{four cytochrome $c$ molecules}, transferring four electrons to reduce one molecule of $O_2$ to two molecules of \water.  This process requires the uptake of four protons from the matrix, contributing to the proton gradient necessary for ATP synthesis
\end{itemize}
\begin{center} 
	\includegraphics*[width=\textwidth]{L2_17.png} \\
    \includegraphics*[width=0.5\textwidth]{L2_18.png} 
\end{center}

\subsubsection*{Path of Electron Through Complex IV}
\begin{center} 
    \includegraphics*[width=0.5\textwidth]{L2_19.png} 
\end{center}

\subsubsection*{Complex IV Overview}
\begin{center} 
	\includegraphics*[width=\textwidth]{L2_20.png} 
\end{center}

\subsubsection*{Electrochemical Gradient Across the Inner Membrane}
The free energy made available by "downhill" (exergonic) electron flow is coupled to the "uphill" transport of protons across a proton-impermeable membrane.  The free energy of fuel oxidation is thus conserved as a transmembrane electrochemical potential
\begin{center} 
	\includegraphics*[width=0.8\textwidth]{L2_21.png} 
\end{center}

\subsubsection*{Proton-Motive Force}
\begin{itemize}
	\item \textbf{proton-motive force} = the energy stored in an electrochemical proton gradient across the mitochondrial inner membrane
	\begin{itemize}
        \item composed of chemical and electrical potential energy
    \end{itemize}
\end{itemize}
\begin{center} 
	\includegraphics*[width=0.5\textwidth]{L2_22.png} 
\end{center}

\subsubsection*{In the Chemiosmotic Model, Oxidation and Phosphorylation are Obligately Coupled}
\begin{itemize}
	\item \textbf{chemiosmotic model} = describes the coupling of ATP synthesis to an electrochemical proton gradient (the proton-motive force)
\end{itemize}
\begin{center} 
	\includegraphics*[width=\textwidth]{L2_23.png} 
\end{center}

\subsection*{Complex V: ATP Synthase}
\begin{center} 
	\includegraphics*[width=\textwidth]{L2_24.png} 
\end{center}
\begin{itemize}
	\item ATP synthase uses the electrochemical gradient produced by the protons pumped into the IMS to produce ATP
	\item \underline{Two major portions}
	\begin{itemize}
        \item \textbf{F$_0$ stalk region} that translates the electrochemical energy into mechanical motion
        \begin{itemize}
            \item \underline{Subunit a} and \underline{c ring}
        \end{itemize}
        \item \textbf{F$_1$ catalytic region} that translates mechanical motion into ATP synthesis
        \begin{itemize}
            \item \underline{$\alpha$ and $\beta$} subunits
        \end{itemize}
    \end{itemize}
    \item $\gamma$ (gamma) subunit connects them both - aka it translates the mechanical motion!
\end{itemize}

\subsubsection*{The Structure of the F$_0$F$_1$ Complex}
\begin{center} 
	\includegraphics*[width=\textwidth]{L3_1.png}//
    \includegraphics*[width=\textwidth]{L3_2.png}
\end{center}

\subsubsection*{F$_0$ portion: Proton motor / rotor}
\begin{itemize}
	\item The \underline{a subunit} has two hydrophilic channels, one allowing for the translocation of protons from the intermembrane space and one allowing for translocation to the matrix (both connect to the \underline{c ring})
	\begin{itemize}
        \item The electrochemical gradient pushes protons from the IMS into the a subunit which adds the proton to an un-protonated \underline{c subunit}
    \end{itemize}
    \item The \underline{c ring} is (roughly) 10 alpha helix subunits attached in a circular loop
    \begin{itemize}
        \item Each subunit has a key residue, Asp59 that can accept the proton from the \underline{a channel}
        \item After one full rotation, the second channel becomes available, allowing the proton to escape to the matrix
    \end{itemize}
    \item This proton-powered rotation turns the gamma ($\gamma$) subunit!
\end{itemize}
\begin{center} 
	\includegraphics*[width=\textwidth]{L3_3.png}
\end{center}

\subsubsection*{F$_1$ portion: ATP synthesis}
\begin{itemize}
	\item The $\alpha$ and $\beta$ subunits form a hexamer (3 of each) with the $\gamma$ subunit wedged in the middle
	\item As the $\gamma$ subunit turns, it stimulates conformational change for the $\alpha$ and $\beta$ subunits
	\begin{itemize}
        \item O $\rightarrow$ L $\rightarrow$ T (when counter-clockwise)
    \end{itemize}
    \item The \textbf{open} conformation allows for ADP and P$_i$ to bind (and allows previous ATP to leave)
    \item The \textbf{loose} conformation locks ADP and P$_i$ in place
    \item The \textbf{tight} conformation promotes ATP formation
    \item One full turn produces 3 ATP!
    \item Around 10 protons makes 3 ATP.  (3.33 protons/ATP)
\end{itemize}
\begin{center} 
	\includegraphics*[width=0.6\textwidth]{L3_4.png}
\end{center}

\subsubsection*{Extra Proton Used by Phosphate Translocase Symporter}
\begin{center} 
	\includegraphics*[width=0.6\textwidth]{L3_5.png}
\end{center}

\subsubsection*{O$_2$ Consumption and ATP Synthesis are Tightly Linked}
\begin{itemize}
	\item Let's examine our model of oxidative phosphorylation
	\begin{itemize}
        \item Imagine an experiment with free floating mitochondria where we can control metabolites and measure environmental variables
    \end{itemize}
    \item Need to provide ADP/P$_i$ as well as electron source for ATP synthesis and O$_2$ consumption
    \begin{itemize}
        \item Everything stops if either is missing
    \end{itemize}
    \item O$_2$ consumption and ATP synthesis are tightly linked
    \begin{itemize}
        \item What links them?  The electrochemical gradient
    \end{itemize}
\end{itemize}
\begin{center} 
	\includegraphics*[width=\textwidth]{L3_6.png}
\end{center}

\subsection*{Respiratory Control and ATP Synthesis}
\begin{center} 
	\includegraphics*[width=\textwidth]{L3_7.png}
\end{center}
\begin{itemize}
	\item The electron transport chain pumps protons up their gradient
	\begin{itemize}
        \item When this gradient becomes too high, electron transport becomes more difficult and ETC slows
        \item This phenomenon is called \textbf{respiratory control}
    \end{itemize}
    \item ATP synthase also relies on this gradient - when the gradient is too low, ATP synthesis slows
    \item This causes the correlation we saw earlier!
\end{itemize}

\subsection*{Uncoupling O$_2$ Consumption and ATP Synthesis}
\begin{itemize}
	\item Certain molecules can allow protons to cross the inner membrane of mitochondria
	\begin{itemize}
        \item This would immediately destroy our proton motive force
    \end{itemize}
    \item These molecules are called \textbf{uncouplers} because \textbf{\underline{they uncouple ATP synthesis and O$_2$ consumption}}
\end{itemize}
\begin{center} 
	\includegraphics*[width=\textwidth]{L3_8.png}
\end{center}
Some idiots tried to use a drug like this in the late 20th century to lose weight.  They died.

\subsubsection*{Inhibitors of ETC}
\begin{center} 
	\includegraphics*[width=\textwidth]{L3_9.png}
\end{center}
No need to memorize these.

\subsection*{The Impact of Inhibitors/Uncouplers}
\begin{itemize}
	\item Cyanide inhibits Complex IV (stops ETC $\rightarrow$ O$_2$ consumption and ATP synthesis stop)
	\item Oligomycin inhibits ATP synthase  (ETC slows down due to backed-up proton gradient)
	\item DNP (2,4-Dinitrophenol) is an uncoupler, destroys the proton gradient (ETC runs without ATP production)
\end{itemize}
\begin{center} 
	\includegraphics*[width=\textwidth]{L3_10.png}
\end{center}

\section*{Shuttles}
\subsection*{Glycolysis Products}
\begin{center} 
	\includegraphics*[width=\textwidth]{L3_11.png}
\end{center}

\subsection*{TCA Products}
\begin{center} 
	\includegraphics*[width=\textwidth]{L3_12.png}
\end{center}

\subsection*{ETC Overview}
\begin{center} 
	\includegraphics*[width=\textwidth]{L3_13.png}
\end{center}

\subsection*{Protons to ATP and Yields}
\begin{center} 
    \includegraphics*[width=\textwidth]{L3_14.png}
\end{center}

\subsection*{The Malate-Aspartate Shuttle}
\begin{itemize}
	\item NADH equivalents moved in by the \textbf{malate-aspartate shuttle} enter the respiratory chain at Complex I and yield a P/O ratio of 2.5
	\item These shuttles are going to pass the electrons from cytosolic NADH to an electron carrier in the matrix
	\begin{itemize}
        \item Malate-aspartate passes these electrons to an NAD\pc, reforming NADH
        \item The malate-aspartate shuttle is used by our \textbf{heart} and \textbf{liver} cells
    \end{itemize}
\end{itemize}
\begin{center} 
	\includegraphics*[width=0.9\textwidth]{L3_15.png}\\
    2 NADH (cytosol) $\rightarrow$ 2 NADH (matrix)\\
    2 NADH $\times$ 2.5 = 5 ATP\\
    23 (e\nc carriers) + 4 (SLP) + 5 (shuttle) = 32 ATP
\end{center}

\subsection*{The Glycerol 3-Phosphate Shuttle}
\begin{itemize}
	\item NADH equivalents moved in by the \textbf{glycerol 3-phosphate shuttle} enter the respiratory chain at Complex III and yield a P/O ratio of 1.5.
\end{itemize}
\begin{center} 
	\includegraphics*[width=\textwidth]{L3_16.png}
    2 NADH (cytosol) $\rightarrow$ 2 FADH$_2$\\
    2 FADH$_2$ $\times$ 1.5 = 3 ATP\\
    23 (e\nc carriers) + 4 (SLP) + 3 (shuttle) = 30 ATP
\end{center}
\begin{itemize}
	\item The glycerol-3-phosphate shuttle passes these electrons to a protein in the inner mitochondrial membrane
	\begin{itemize}
        \item Cytosolic NADH reduces DHAP to glycerol-3-phosphate via cytosolic glycerol-3-phosphate dehydrogenase (which oxidizes NADH to NAD\pc)
        \item Glycerol-3-phosphate enters the mitochondria and donates its electrons to \textbf{FAD} via \textbf{mitochondrial glycerol-3-phosphate} dehydrogenase
        \item Electrons are passed to \textbf{Complex III} using \textbf{Coenzyme Q}
        \item The G3P shuttle is used by \textbf{brain} and \textbf{muscle} cells
    \end{itemize}
\end{itemize}

\subsection*{Stoichiometry of Coenzyme Reduction and ATP Formation in Aerobic Oxidation of Glucose}
\begin{center} 
	\includegraphics*[width=\textwidth]{L3_17.png}
\end{center}

\subsection*{Summary of Shuttles}
The \textbf{Malate-Aspartate Shuttle} generates more ATP because it fully conserves the energy of cytosolic NADH by tansferring it to NADH in the matrix, allowing the electrons to enter the ETC at Complex I.  In contrast, the \textbf{Glycerol-3-Phosphate Shuttle} sacrifices some energy by transferring electrons to FADH$_2$, which enters at Complex III, resulting in lower ATP production.  This efficiency tradeoff often depends on the cell type and its metabolic needs.







\end{document}