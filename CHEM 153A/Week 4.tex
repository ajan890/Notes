% document formatting
\documentclass[10pt]{article}
\usepackage[utf8]{inputenc}
\usepackage[left=1in,right=1in,top=1in,bottom=1in]{geometry}
\usepackage[T1]{fontenc}
\usepackage{xcolor}

% math symbols, etc.
\usepackage{amsmath, amsfonts, amssymb, amsthm}

% lists
\usepackage{enumerate}
\usepackage{tabularx}
\usepackage{multicol}
\usepackage[table,xcdraw]{xcolor}

% images
\usepackage{graphicx} % for images

% code blocks
\usepackage{minted, listings} 

% verbatim greek
\usepackage{alphabeta}

\graphicspath{{./assets/images/Week 4}}

\newcommand{\solution}{\textbf{Solution:}} 
\newcommand{\example}{\textbf{Example: }}
\newcommand{\water}{\text{H$_2$O}}
\newcommand{\hydroxide}{\text{OH$^-$}}
\newcommand{\hydronium}{\text{H$_3$O$^+$}}
\newcommand{\proton}{\text{H$^+$}}
\newcommand{\pc}{$^+$}
\newcommand{\nc}{$^-$}
\newcommand{\ka}{\text{$K_\text{a}$}}

\title{CHEM 153A Week 4}

\author{Aidan Jan}
\date{\today}

\begin{document}
\maketitle

\section*{Protein Tertiary and Quarternary Structure}
\begin{itemize}
    \item \textbf{tertiary structure} = overall three-dimensional arrangement of all the atoms in a protein
    \begin{itemize}
        \item weak interactions and covalent bonds hold interacting segments in position
    \end{itemize}
    \item \textbf{quaternary structure} = arrangement of 2+ separate polypeptide chains in three-dimensional complexes
    \begin{center}
        \textbf{Shape \textrightarrow~Function}
    \end{center}
\end{itemize}

\subsection*{Tertiary Structure - What holds it together?}
\begin{itemize}
    \item The global interactions of tertiary structure are formed through the interaction of amino acid side chains
    \item \underline{Electrostatic interactions} forming between charged side chains
    \item \underline{London dispersion forces} forming between nonpolar side chains
    \item \underline{Hydrogen bonds} forming between polar/charged side chains
    \item \underline{Covalent bonds} forming through disulfide bridges
\end{itemize}
\begin{center}
    \includegraphics*[width=\textwidth]{L1_1.png}
\end{center}

\section*{Categories of Tertiary Structure - Globular proteins}
\begin{itemize}
    \item \textbf{Globular proteins} polypeptide chains folded into a \underline{spherical or globular shape}
    \begin{itemize}
        \item fold back on each other
        \item more compact than fibrous proteins
        \item \underline{soluble in water}
        \item mixture of different secondary structures
        \item quaternary structure usually held together by noncovalent forces
        \item regulatory and metabolic roles (basically most proteins you can think of)
        \begin{itemize}
            \item enzymes
            \item transport proteins
            \item motor proteins
            \item regulatory proteins
            \item immunoglobulins
        \end{itemize}
    \end{itemize}
\end{itemize}
\begin{center}
    \includegraphics*[width=\textwidth]{L1_2.png}
\end{center}
\subsection*{Myoglobin Provided Early Clues about the Complexity of Globular Protein Structure}
\begin{itemize}
    \item several structural representations of myoglobin's tertiary structure:
\end{itemize}
\begin{center}
    \includegraphics*[width=\textwidth]{L1_3.png}
    \includegraphics*[width=\textwidth]{L1_4.png}
\end{center}
The bottom picture depicts the original model of the myoglobin molecule, constructed in plasticine in 1957.
\begin{itemize}
    \item This was the first ever model of a protein molecule.  (Won a nobel prize)
    \item In modern day, we have so much information about protein structure that we can train AIs to simulate protein folding.
\end{itemize}

\subsection*{Globular Proteins Have a Variety of Tertiary Structures}
Each globular protein has a distinct structure, adapted for its biological function
\begin{center}
    \includegraphics*[scale=0.4]{L1_5.png}
\end{center}

\section*{Categories of Tertiary Structure - Fibrous proteins}
\begin{itemize}
    \item \textbf{Fibrous proteins} are long (often) rope-like proteins \underline{adapted for strength}.
    \begin{itemize}
        \item Extended structure
        \item Insoluble in water
        \item Simple repetitive structure (often the same secondary structure throughout)
        \item Quaternary structure usually held together by disulfide bonds
        \item Famously involved in a lot of extracellular structures (incl. Tendons, bones, hair, skin)
    \end{itemize}
\end{itemize}
\begin{center}
    \includegraphics*[width=\textwidth]{L1_6.png}
\end{center}

\subsection*{Fibrous Proteins are Adapted for a Structural Function}
\begin{itemize}
    \item give strength and/or flexibility to structures
    \item simple repeating element of secondary structure
    \item \water~insoluble due to high concentrations of hydrophobic residues
\end{itemize}
\begin{center}
    \includegraphics*[scale=0.4]{L1_7.png}
\end{center}

\subsection*{Membrane Proteins} 
\textbf{Membrane proteins} are proteins with polypeptide chains embedded into lipid membranes
\begin{itemize}
    \item Multiple types, commonality is that they contain hydrophobic regions so as to embed themselves
    \item Defined patterns of secondary structure
\end{itemize}
\begin{center}
    \includegraphics*[width=\textwidth]{L1_8.png}
\end{center}

\section*{Quaternary Structure}
\begin{itemize}
    \item Folded proteins can associate forming multi-subunit complexes
    \begin{itemize}
        \item Adds even more informational complexity to proteins as functional units
        \item Variety of possibilities, anything from small oligomeric complexes to massive complexes made from many different proteins
    \end{itemize}
    \item Subunits can be identical or different
    \begin{itemize}
        \item Subunits are symmetrically arranged
    \end{itemize}
    \item \textbf{oligomer = multimer} = multi-subunit protein
\end{itemize}
\begin{center}
    \includegraphics*[scale=0.6]{L1_9.png}\\
    \includegraphics*[scale=0.4]{L1_10.png}
\end{center}

\subsection*{Quaternary Structure - What holds it together?}
\begin{itemize}
    \item Quaternary structure is built by interactions between protein subunits
    \item These take on the same characteristics as tertiary structure (this should make sense if you consider it)
    \item \underline{Electrostatic interactions} forming between charged side chains (more prevalent in quaternary)
    \item \underline{London dispersion forces} forming between nonpolar side chains
    \item \underline{Hydrogen bonds} forming between polar/charged side chains
    \item \underline{Covalent bonds} forming through disulfide bridges (less prevalent in quaternary)
\end{itemize}
\begin{center}
    \includegraphics*[scale=0.6]{L1_11.png}
\end{center}

\subsection*{Visualizing Protein Structure}
Quaternary structure describes the interactions between components of a multisubunit assembly
\begin{center}
    \includegraphics*[scale=0.5]{L1_12.png}
\end{center}

\subsection*{The Protein Data Bank}
The \textbf{Protein Data Bank (PDB)}: www.rcsb.org
\begin{itemize}
    \item archive of experimentally determined three-dimensional structures
    \item structures assigned an identifier called the PDB ID
    \item PDB data files describe:
    \begin{itemize}
        \item the spatial coordinates of each atom
        \item information on how the structure was determined
        \item information on its accuracy (how good the model is)
        \item structure visualization software can convert atomic coordinates to an image of the molecule
    \end{itemize}
\end{itemize}

\subsection*{Folding Patterns of Proteins}
\begin{itemize}
    \item \textbf{motif = fold =} recognizable folding pattern involving 2+ elements of secondary structures and the connections(s)
    \begin{itemize}
        \item can be simple, such as in a $\beta-\alpha-\beta$ loop
        \item can be elaborate, such as in a $\beta$-barrel
    \end{itemize}
\end{itemize}
\begin{center}
    \includegraphics*[scale=0.5]{L1_13.png}
\end{center}

\subsection*{Protein Domains}
\begin{itemize}
    \item \textbf{domain} = part of a polypeptide chain that is independently stable or could undergo movements as a single entity
    \begin{itemize}
        \item domains may appear as distinct or be difficult to discern
        \item small proteins usually have only one domain
    \end{itemize}
\end{itemize}
\begin{center}
    \includegraphics*[scale=0.5]{L1_14.png}
\end{center}

\subsection*{Complex Motifs are Built from Simple Motifs}
\textbf{$\alpha / \beta$ barrel} = series of $\beta-\alpha-\beta$ loops arranged such that the $\beta$ strands form a barrel
\begin{center}
    \includegraphics*[scale=0.5]{L1_15.png}
\end{center}

\subsection*{Intrinsically disordered proteins:}
\begin{itemize}
    \item lack definable structure
    \item often lack a hydrophobic core
    \item high densities of charged residues (Lys, Arg, Glu, and Pro)
    \item facilitates a protein to interact with multiple binding partners
\end{itemize}

\begin{center}
    \includegraphics*[scale=0.5]{L1_16.png}
\end{center}

\section*{Protein Families and Superfamilies}
\begin{itemize}
    \item proteins with significant similarity in primary structure and/or tertiary structure and function are in the same \textbf{protein family}
    \begin{itemize}
        \item $\sim$4000 different protein families in the PDB
        \item strong evolutionary relationship within a family
    \end{itemize}
    \item \textbf{superfamilies} = 2+ families that have little sequence similarity, but the same major structural motif and have functional similarities.
\end{itemize}

\subsection*{Globins are a Family of Oxygen-Binding Proteins}
\begin{itemize}
    \item Globins like myoglobin and hemoglobin belong to the same protein family: \textbf{Globin Family}
    \item Globins are a widespread protein family:
    \begin{itemize}
        \item highly conserved tertiary structure: eight $\alpha$-helical segments connected by bends (\textbf{globin fold})
        \item most function in O$_2$ transport or storage
    \end{itemize}
\end{itemize}
\begin{center}
    \includegraphics*[scale=0.5]{L1_17.png}
\end{center}

\subsection*{Types of Globins}
\begin{itemize}
    \item Four types in humans and other mammals:
    \begin{itemize}
        \item \textbf{myoglobin} = monomeric, facilitates O$_2$ diffusion in muscle tissue
        \item \textbf{hemoglobin} = tetrameric, responsible for O$_2$ transport in the bloodstream
        \item \textbf{neuroglobin} = monomeric, expressed largely in neurons to protect the brain from low O$_2$ or restricted blood supply
        \item \textbf{cytoglobin} = monomeric, regulates levels of nitric oxide, a localized signal for muscle relaxation
    \end{itemize}
\end{itemize}

\subsection*{Globins and Prosthetic Groups}
\begin{itemize}
    \item \textbf{Myoglobin} and \textbf{hemoglobin} are examples of \underline{conjugated proteins}
    \begin{itemize}
        \item Simple proteins only have a polypeptide chain
        \item \textbf{Conjugated proteins} have a non-protein component called a \underline{prosthetic group}
        \item Myoglobin and hemoglobin have a heme prosthetic group that provides them with oxygen binding functionality (amino acids can't bind oxygen well)
    \end{itemize}
    \item The globins are examples of \textbf{hemoproteins} (proteins with heme prosthetic groups) which are subsets of metalloproteins (proteins with metal prosthetic groups)
    \begin{itemize}
        \item This is because heme contains \underline{ferrous iron} (Fe$^{2+}$)
    \end{itemize}
\end{itemize}

\begin{center}
    \includegraphics*[scale=0.5]{L1_18.png}
\end{center}

\subsection*{Oxygen-carrying proteins}
\begin{itemize}
    \item \underline{Myoglobin (Mb)}
    \begin{itemize}
        \item O$_2$ acts as a ligand (can bind max 1 O$_2$)
        \item Only one subunit
        \item Acts as oxygen storage and facilitates diffusion in muscular cells
        \item $\approx$ 64 g of Mb present in an average human body
    \end{itemize}
    \item \underline{Hemoglobin (Hb)}
    \begin{itemize}
        \item O$_2$ acts as a ligand (can bind up to 4 O$_2$)
        \item Four subunits, two subunits of $\alpha$-globin, and two subunits of $\beta$-globin
        \item Transports O$_2$ from lungs to peripheral tissues (carried in erythrocytes)
        \item $\approx$ 775 g of Hb present in an average human body
    \end{itemize}
    \item Both rely on the heme prosthetic group
\end{itemize}
\begin{center}
    \includegraphics*[scale=0.5]{L1_19.png}
\end{center}

\pagebreak
\end{document}
