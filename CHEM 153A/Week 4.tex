% document formatting
\documentclass[10pt]{article}
\usepackage[utf8]{inputenc}
\usepackage[left=1in,right=1in,top=1in,bottom=1in]{geometry}
\usepackage[T1]{fontenc}
\usepackage{xcolor}

% math symbols, etc.
\usepackage{amsmath, amsfonts, amssymb, amsthm}

% lists
\usepackage{enumerate}
\usepackage{tabularx}
\usepackage{multicol}
\usepackage[table,xcdraw]{xcolor}

% images
\usepackage{graphicx} % for images

% code blocks
\usepackage{minted, listings} 

% verbatim greek
\usepackage{alphabeta}

\graphicspath{{./assets/images/Week 4}}

\newcommand{\solution}{\textbf{Solution:}} 
\newcommand{\example}{\textbf{Example: }}
\newcommand{\water}{\text{H$_2$O}}
\newcommand{\hydroxide}{\text{OH$^-$}}
\newcommand{\hydronium}{\text{H$_3$O$^+$}}
\newcommand{\proton}{\text{H$^+$}}
\newcommand{\pc}{$^+$}
\newcommand{\nc}{$^-$}
\newcommand{\ka}{\text{$K_\text{a}$}}

\title{CHEM 153A Week 4}

\author{Aidan Jan}
\date{\today}

\begin{document}
\maketitle

\section*{Protein Tertiary and Quarternary Structure}
\begin{itemize}
    \item \textbf{tertiary structure} = overall three-dimensional arrangement of all the atoms in a protein
    \begin{itemize}
        \item weak interactions and covalent bonds hold interacting segments in position
    \end{itemize}
    \item \textbf{quaternary structure} = arrangement of 2+ separate polypeptide chains in three-dimensional complexes
    \begin{center}
        \textbf{Shape \textrightarrow~Function}
    \end{center}
\end{itemize}

\subsection*{Tertiary Structure - What holds it together?}
\begin{itemize}
    \item The global interactions of tertiary structure are formed through the interaction of amino acid side chains
    \item \underline{Electrostatic interactions} forming between charged side chains
    \item \underline{London dispersion forces} forming between nonpolar side chains
    \item \underline{Hydrogen bonds} forming between polar/charged side chains
    \item \underline{Covalent bonds} forming through disulfide bridges
\end{itemize}
\begin{center}
    \includegraphics*[width=\textwidth]{L1_1.png}
\end{center}

\section*{Categories of Tertiary Structure - Globular proteins}
\begin{itemize}
    \item \textbf{Globular proteins} polypeptide chains folded into a \underline{spherical or globular shape}
    \begin{itemize}
        \item fold back on each other
        \item more compact than fibrous proteins
        \item \underline{soluble in water}
        \item mixture of different secondary structures
        \item quaternary structure usually held together by noncovalent forces
        \item regulatory and metabolic roles (basically most proteins you can think of)
        \begin{itemize}
            \item enzymes
            \item transport proteins
            \item motor proteins
            \item regulatory proteins
            \item immunoglobulins
        \end{itemize}
    \end{itemize}
\end{itemize}
\begin{center}
    \includegraphics*[width=\textwidth]{L1_2.png}
\end{center}
\subsection*{Myoglobin Provided Early Clues about the Complexity of Globular Protein Structure}
\begin{itemize}
    \item several structural representations of myoglobin's tertiary structure:
\end{itemize}
\begin{center}
    \includegraphics*[width=\textwidth]{L1_3.png}
    \includegraphics*[width=\textwidth]{L1_4.png}
\end{center}
The bottom picture depicts the original model of the myoglobin molecule, constructed in plasticine in 1957.
\begin{itemize}
    \item This was the first ever model of a protein molecule.  (Won a nobel prize)
    \item In modern day, we have so much information about protein structure that we can train AIs to simulate protein folding.
\end{itemize}

\subsection*{Globular Proteins Have a Variety of Tertiary Structures}
Each globular protein has a distinct structure, adapted for its biological function
\begin{center}
    \includegraphics*[scale=0.4]{L1_5.png}
\end{center}

\section*{Categories of Tertiary Structure - Fibrous proteins}
\begin{itemize}
    \item \textbf{Fibrous proteins} are long (often) rope-like proteins \underline{adapted for strength}.
    \begin{itemize}
        \item Extended structure
        \item Insoluble in water
        \item Simple repetitive structure (often the same secondary structure throughout)
        \item Quaternary structure usually held together by disulfide bonds
        \item Famously involved in a lot of extracellular structures (incl. Tendons, bones, hair, skin)
    \end{itemize}
\end{itemize}
\begin{center}
    \includegraphics*[width=\textwidth]{L1_6.png}
\end{center}

\subsection*{Fibrous Proteins are Adapted for a Structural Function}
\begin{itemize}
    \item give strength and/or flexibility to structures
    \item simple repeating element of secondary structure
    \item \water~insoluble due to high concentrations of hydrophobic residues
\end{itemize}
\begin{center}
    \includegraphics*[scale=0.4]{L1_7.png}
\end{center}

\subsection*{Membrane Proteins} 
\textbf{Membrane proteins} are proteins with polypeptide chains embedded into lipid membranes
\begin{itemize}
    \item Multiple types, commonality is that they contain hydrophobic regions so as to embed themselves
    \item Defined patterns of secondary structure
\end{itemize}
\begin{center}
    \includegraphics*[width=\textwidth]{L1_8.png}
\end{center}

\section*{Quaternary Structure}
\begin{itemize}
    \item Folded proteins can associate forming multi-subunit complexes
    \begin{itemize}
        \item Adds even more informational complexity to proteins as functional units
        \item Variety of possibilities, anything from small oligomeric complexes to massive complexes made from many different proteins
    \end{itemize}
    \item Subunits can be identical or different
    \begin{itemize}
        \item Subunits are symmetrically arranged
    \end{itemize}
    \item \textbf{oligomer = multimer} = multi-subunit protein
\end{itemize}
\begin{center}
    \includegraphics*[scale=0.6]{L1_9.png}\\
    \includegraphics*[scale=0.4]{L1_10.png}
\end{center}

\subsection*{Quaternary Structure - What holds it together?}
\begin{itemize}
    \item Quaternary structure is built by interactions between protein subunits
    \item These take on the same characteristics as tertiary structure (this should make sense if you consider it)
    \item \underline{Electrostatic interactions} forming between charged side chains (more prevalent in quaternary)
    \item \underline{London dispersion forces} forming between nonpolar side chains
    \item \underline{Hydrogen bonds} forming between polar/charged side chains
    \item \underline{Covalent bonds} forming through disulfide bridges (less prevalent in quaternary)
\end{itemize}
\begin{center}
    \includegraphics*[scale=0.6]{L1_11.png}
\end{center}

\subsection*{Visualizing Protein Structure}
Quaternary structure describes the interactions between components of a multisubunit assembly
\begin{center}
    \includegraphics*[scale=0.5]{L1_12.png}
\end{center}

\subsection*{The Protein Data Bank}
The \textbf{Protein Data Bank (PDB)}: www.rcsb.org
\begin{itemize}
    \item archive of experimentally determined three-dimensional structures
    \item structures assigned an identifier called the PDB ID
    \item PDB data files describe:
    \begin{itemize}
        \item the spatial coordinates of each atom
        \item information on how the structure was determined
        \item information on its accuracy (how good the model is)
        \item structure visualization software can convert atomic coordinates to an image of the molecule
    \end{itemize}
\end{itemize}

\subsection*{Folding Patterns of Proteins}
\begin{itemize}
    \item \textbf{motif = fold =} recognizable folding pattern involving 2+ elements of secondary structures and the connections(s)
    \begin{itemize}
        \item can be simple, such as in a $\beta-\alpha-\beta$ loop
        \item can be elaborate, such as in a $\beta$-barrel
    \end{itemize}
\end{itemize}
\begin{center}
    \includegraphics*[scale=0.5]{L1_13.png}
\end{center}

\subsection*{Protein Domains}
\begin{itemize}
    \item \textbf{domain} = part of a polypeptide chain that is independently stable or could undergo movements as a single entity
    \begin{itemize}
        \item domains may appear as distinct or be difficult to discern
        \item small proteins usually have only one domain
    \end{itemize}
\end{itemize}
\begin{center}
    \includegraphics*[scale=0.5]{L1_14.png}
\end{center}

\subsection*{Complex Motifs are Built from Simple Motifs}
\textbf{$\alpha / \beta$ barrel} = series of $\beta-\alpha-\beta$ loops arranged such that the $\beta$ strands form a barrel
\begin{center}
    \includegraphics*[scale=0.5]{L1_15.png}
\end{center}

\subsection*{Intrinsically disordered proteins:}
\begin{itemize}
    \item lack definable structure
    \item often lack a hydrophobic core
    \item high densities of charged residues (Lys, Arg, Glu, and Pro)
    \item facilitates a protein to interact with multiple binding partners
\end{itemize}

\begin{center}
    \includegraphics*[scale=0.5]{L1_16.png}
\end{center}

\section*{Protein Families and Superfamilies}
\begin{itemize}
    \item proteins with significant similarity in primary structure and/or tertiary structure and function are in the same \textbf{protein family}
    \begin{itemize}
        \item $\sim$4000 different protein families in the PDB
        \item strong evolutionary relationship within a family
    \end{itemize}
    \item \textbf{superfamilies} = 2+ families that have little sequence similarity, but the same major structural motif and have functional similarities.
\end{itemize}

\subsection*{Globins are a Family of Oxygen-Binding Proteins}
\begin{itemize}
    \item Globins like myoglobin and hemoglobin belong to the same protein family: \textbf{Globin Family}
    \item Globins are a widespread protein family:
    \begin{itemize}
        \item highly conserved tertiary structure: eight $\alpha$-helical segments connected by bends (\textbf{globin fold})
        \item most function in O$_2$ transport or storage
    \end{itemize}
\end{itemize}
\begin{center}
    \includegraphics*[scale=0.5]{L1_17.png}
\end{center}

\subsection*{Types of Globins}
\begin{itemize}
    \item Four types in humans and other mammals:
    \begin{itemize}
        \item \textbf{myoglobin} = monomeric, facilitates O$_2$ diffusion in muscle tissue
        \item \textbf{hemoglobin} = tetrameric, responsible for O$_2$ transport in the bloodstream
        \item \textbf{neuroglobin} = monomeric, expressed largely in neurons to protect the brain from low O$_2$ or restricted blood supply
        \item \textbf{cytoglobin} = monomeric, regulates levels of nitric oxide, a localized signal for muscle relaxation
    \end{itemize}
\end{itemize}

\subsection*{Globins and Prosthetic Groups}
\begin{itemize}
    \item \textbf{Myoglobin} and \textbf{hemoglobin} are examples of \underline{conjugated proteins}
    \begin{itemize}
        \item Simple proteins only have a polypeptide chain
        \item \textbf{Conjugated proteins} have a non-protein component called a \underline{prosthetic group}
        \item Myoglobin and hemoglobin have a heme prosthetic group that provides them with oxygen binding functionality (amino acids can't bind oxygen well)
    \end{itemize}
    \item The globins are examples of \textbf{hemoproteins} (proteins with heme prosthetic groups) which are subsets of metalloproteins (proteins with metal prosthetic groups)
    \begin{itemize}
        \item This is because heme contains \underline{ferrous iron} (Fe$^{2+}$)
    \end{itemize}
\end{itemize}

\begin{center}
    \includegraphics*[scale=0.5]{L1_18.png}
\end{center}

\subsection*{Oxygen-carrying proteins}
\begin{itemize}
    \item \underline{Myoglobin (Mb)}
    \begin{itemize}
        \item O$_2$ acts as a ligand (can bind max 1 O$_2$)
        \item Only one subunit
        \item Acts as oxygen storage and facilitates diffusion in muscular cells
        \item $\approx$ 64 g of Mb present in an average human body
    \end{itemize}
    \item \underline{Hemoglobin (Hb)}
    \begin{itemize}
        \item O$_2$ acts as a ligand (can bind up to 4 O$_2$)
        \item Four subunits, two subunits of $\alpha$-globin, and two subunits of $\beta$-globin
        \item Transports O$_2$ from lungs to peripheral tissues (carried in erythrocytes)
        \item $\approx$ 775 g of Hb present in an average human body
    \end{itemize}
    \item Both rely on the heme prosthetic group
\end{itemize}
\begin{center}
    \includegraphics*[scale=0.5]{L1_19.png}
\end{center}

\pagebreak
\subsection*{Oxygen Can Bind to a Heme Prosthetic Group}
Oxygen is:
\begin{itemize}
    \item poorly soluble in aqueous solutions
    \item diffusion through tissues is ineffective over large distances
    \item transition metals have a strong tendency to bind (iron, copper)
\end{itemize}
\textbf{Heme} is prosthetic group incorporated during folding
\begin{itemize}
    \item \underline{Protoporphyrin}
\end{itemize}
Heme is responsible for reversible O$_2$ binding.
\begin{center}
    \includegraphics*[scale=0.4]{L2_1.png}
\end{center}

\subsection*{The Heme Prosthetic Group}
\begin{itemize}
    \item \textbf{Heme} is prosthetic group incorporated during folding
    \begin{itemize}
        \item \underline{Protoporphyrin}
    \end{itemize}
    \item Responsible for reversible O$_2$ binding
    \item Fe$^{2+}$ has \underline{6 coordination sites/bonds}
    \begin{itemize}
        \item 4 are occupied by N of pyrrole rings
        \item 2 sites available perpendicular to protoporphyrin ring
        \begin{itemize}
            \item 1 occupied by proximal His.
        \end{itemize}
        \item The 6th coordination site:
        \begin{itemize}
            \item \textbf{Deoxyhemoglobin}: unoccupied
            \item \textbf{Oxyhemoglobin}: Occupied by O$_2$
            \item \textbf{Carboxyhemoglobin}: Occupied by CO
        \end{itemize}
    \end{itemize}
\end{itemize}
\begin{center}
    \includegraphics*[width=\textwidth]{L2_2.png}
\end{center}

\subsection*{Perpendicular Coordination Bonds}
Two perpendicular coordination bonds:
\begin{itemize}
    \item one is occupied by a side-chain nitrogen of a highly conserved \textbf{proximal His} residue
    \item one is the binding site for molecular oxygen (O$_2$)
    \begin{itemize}
        \item Fe$^{2+}$ (ferrous iron) binds O$_2$ reversibly
        \item Fe$^{3+}$ (ferric iron) does not bind O$_2$
    \end{itemize}
\end{itemize}
\begin{center}
    \includegraphics*[scale=0.5]{L2_3.png}
\end{center}

\subsection*{Globin Contributions to O$_2$ Binding}
Why can't we just have heme on its own?
\begin{itemize}
    \item \underline{Proximal histidine} occupies 5th coordination site, holding heme in place (Mb and Hb)
    \item \underline{Other heme stabilizers} - Val and Phe (Mb) or Val and Leu (Hb)
    \item \underline{Distal histidine} stabilizes O$_2$ binding and acts as gate for ligand entry - encouraging specificity (Mb and Hb)
    \item Another contribution is specific to hemoglobin and deals with \underline{cooperativity}
\end{itemize}

\subsection*{Role of the Distal Histidine}
\begin{itemize}
    \item For \textbf{O$_2$ binding:} The distal histidine stabilizes the oxygen molecule when it binds to the iron ion in the heme group via a \textbf{hydrogen bond}.  This prevent the iron from being oxidized to the Fe$^{3+}$ state, which cannot bind oxygen
    \item For \textbf{CO binding:} The distal histidine \textbf{hinders carbon monoxide binding} by forcing the CO molecule into a less favorable binding orientation:
    \begin{itemize}
        \item CO prefers to bind in a straight geometry to the iron ion, but the distal histidine forces it into a bent geometry, reducing the affinity of hemoglobin for CO.  While CO still binds more strongly than oxygen, the distal histidine helps prevent complete dominance by CO, providing a protective effect.
    \end{itemize}
\end{itemize}
\begin{center}
    \includegraphics*[width=\textwidth]{L2_4.png}
\end{center}

\subsection*{Carbon Monoxide acts as a competitive inhibitor}
A \textbf{competitive inhibitor} is a compound that binds to the same site as the intended molecule (substrate or ligand), blocking the site
\begin{itemize}
    \item Carbon monoxide acts as a competitive inhibitor to oxygen with respect to Hb
\end{itemize}

\section*{Myoglobin Has a Single Binding Site for Oxygen}
\begin{itemize}
    \item myoglobin:
    \begin{itemize}
        \item 153 residues + one moleule of heme
        \item bends named after the $\alpha$-helical segments they connect
    \end{itemize}
    \item His$^{93}$ = ninety-third residue from the amino terminal end
    \item His F8 = eighth residue in $\alpha$ helix F    
\end{itemize}

\begin{center}
    \includegraphics*[width=\textwidth]{L2_5.png}
\end{center}

\subsection*{Protein-Ligand Interactions Can Be Describbed Quantitively}
A simple \textbf{equilibrium expression} describes the reversible binding of a protein (P) to a ligand (L):
\begin{center}
    \includegraphics*[width=\textwidth]{L2_6.png}
\end{center}

\section*{Association Constant}
The \textbf{association constant} ($K_a$) provides a measure of the affinity of the ligand L for the protein
\begin{itemize}
    \item higher $K_a$ = higher affinity
    \item equivalent to the ratio of the rates of the forward (association) and the reverse (dissociation) reactions that form the PL complex
    \item $K_a$ and $K_d$ are the forward and reverse rate constants
    \[k_a [P] \cdot [L] = k_d[PL]\]
    \[K_a = \frac{[PL]}{[P][L]} = \frac{k_a}{k_d}\]
\end{itemize}

\subsection*{Dissociation Constant}
The \textbf{dissociation constant} ($K_d$) is the reciprocal of $K_a$.
\begin{itemize}
    \item equilibrium constant for the release of ligand
    \item lower $K_d$ = higher affinity.
\end{itemize}
\[K_d = \frac{[P][L]}{[PL]} = \frac{k_d}{k_a}\]
\begin{itemize}
    \item $K_d$ is often used to describe protein-ligand \underline{interactions in practice}
    \item A \textbf{low $K_d$} means strong binding (high affinity) because the protein-ligand complex is less likely to dissociate
    \item A \textbf{high $K_d$} means weak binding (low affinity)
\end{itemize}

\subsection*{Understanding Protein-Ligand Binding via Equilibrium Constants}
\begin{itemize}
    \item At \textbf{equilibrium}, the ratio of bound ligand (PL) to free ligand (P and L) is constant, and this ratio is determined by $K_a$ or $K_d$.
    \item $K_a$ gives us an idea of how readily the protein binds the ligand, while $K_d$ tells us how easily the complex falls apart.
\end{itemize}

\subsection*{Representative $K_d$ Values}
Avidin-biotin complex is the strongest known non-covalent interaction ($K_d = 10^{-15}$M) between a protein and ligand
\begin{center}
    \includegraphics*[width=\textwidth]{L2_7.png}
\end{center}

\subsection*{Binding Equilibrium}
Let's consider binding equilibrium from the standpoint of the fraction of binding sites on the protein that are occupied by ligand.
\begin{itemize}
    \item \textbf{Fractional Occupancy} (Y or $\theta$): is the fraction of the total protein bound to the ligand at any given ligand concentration (fraction of binding sites occupied)
    \[Y = \frac{\text{binding sites occupied}}{\text{total binding sites}} = \frac{[PL]}{[PL] + [P]}\]
    \item $Y$ ranges from 0 to 1.
    \begin{itemize}
        \item $Y = 1$ means all binding sites are occupied (fully saturated with ligand)
        \item $Y = 0$ means no binding sites are occupied (all protein is in its unbound state)
    \end{itemize}
\end{itemize}
\begin{align*}
    K_a &= \frac{[PL]}{[P][L]}\\
    K_a[L][P] &= [PL]
    \intertext{substititing $K_a[L][P]$ for $[PL]$\dots}
    Y &= \frac{K_a [L][P]}{K_a[L][P] + [P]}\\
    &= \frac{K_a[L]}{K_a[L] + 1}\\
    &= \frac{[L]}{[L] + \frac{1}{K_a}}\\
    &= \frac{[L]}{[L] + K_d}
\end{align*}
\begin{itemize}
    \item This expression describes a hyperbola!
\end{itemize}

\subsection*{Graphical Representations of Ligand Binding}
\[Y = \frac{[L]}{[L] + \frac{1}{K_a}} = \frac{[L]}{[L] + K_d}\]
\begin{itemize}
    \item [L] at which half of the available ligand-binding sites are occupied ($Y = 0.5$) corresponds to $K_d$ or $1 / K_a$.
    \item Therefore, our best method for finding $K_d$ is to find [L] when $Y = 0.5$.
\end{itemize}
\begin{center}
    \includegraphics*[scale=0.5]{L2_8.png}
\end{center}

\subsection*{Binding of O$_2$ to Myoglobin}
\begin{itemize}
    \item substituting the [O$_2$] for [$L$]
    \[Y = \frac{[O_2]}{[O_2] + K_d}\]
    \item $K_d$ equals the [O$_2$] at which half of the available ligand-binding sites are occupied, or [O$_2$]$_{0.5}$.
    \[Y = \frac{[O_2]}{[O_2] + [O_2]_{0.5}}\]
\end{itemize}

\subsection*{Partial Pressure of O$_2$}
\begin{itemize}
    \item partial pressure of O$_2$ (pO$_2$) is easier to measure than [O$_2$].
    \item Defining the partial pressure of oxygen at [O$_2$]$_{0.5}$ as $P_{50}$:
    \[Y = \frac{[O_2]}{[O_2] + P_{50}}\]
    \item Thus, $P_{50}$ refers to the \textbf{partial pressure of oxygen (pO$_2$)} at which the \textbf{fractional saturation (Y)} of a protein (like hemoglobin or myoblobin) is $50\%$.  In other wordsd, $P_{50}$ \textbf{is the partial pressure of oxygen at which half of the binding sites on the protein are occupied by O$_2$ molecules.}
\end{itemize}
\begin{center}
    \includegraphics*[scale=0.6]{L2_9.png}
\end{center}
\begin{itemize}
    \item In physiology, pO$_2$ is directly related to how gases exchange in the lungs and tissues.  Hemoglobin saturation and oxygen transport are functions of pO$_2$ rather than the direct concentration of oxygen.  Therefore, pO$_2$ is more relevant and practical to measure in the context of oxygen binding and release in biological systems, especially for processes like respiration and oxygen transport.
    \item P$_{50}$ \textbf{is a measure of the affinity of the protein for oxygen.}  A \textbf{lower P$_{50}$} means that the protein has a \textbf{higher affinity} for oxygen (it can bind oxygen more easily at a lower pO$_2$), while a \textbf{higher P$_{50}$} indicates a \textbf{lower affinity} (it requires a higher pO$_2$ to achieve 50\% saturation)
\end{itemize}

\pagebreak
\subsection*{Binding of O$_2$ to Myoglobin and Hemoglobin}
\begin{itemize}
    \item \textbf{Mylglobin} has a \textbf{very low P$_{50}$} indicating a \textbf{high affinity for oxygen.}  It binds oxygen tightly, even at low partial pressures, which is important for oxygen storage in muscle tissues
    \item \textbf{Hemoglobin}, on the other hand, \textbf{has a higher P$_{50}$ than myoglobin}, reflecting its lower affinity for oxygen compared to myoglobin.  This \textbf{allows hemoglobin to release oxygen more easily in tissues where the partial pressure of oxygen is lower}, while still binding oxygen efficiently in the lungs, where the partial pressure is higher
\end{itemize}

\subsection*{Behavior of Myoglobin with respect to partial pressure of oxygen in the body}
\begin{center}
    \includegraphics*[scale=0.8]{L2_10.png}
\end{center}
\begin{itemize}
    \item At low pO$_2$, myoglobin has a high affinity for oxygen, enabling it to effectively bind oxygen in tissues even when oxygen levels are low.
    \item Myoglobin is primarily found in muscle tissues and serves as an \textbf{oxygen storage molecule}.  
    \item In resting muscle, O$_2$ is moderate ($\sim$4-5 kPa), and myoglobin maintains a high saturation level, acting as a reservoir.  In active muscle, pO$_2$ rops to very low levels (<1 kPa), prompting myoglobin to release its stored oxygen to support metabolic activity
    \item \textbf{Venous blood} (blue): low partial pressure of oxygen (pO$_2$), which is oxygen-poor and returning to the lungs.
    \item \textbf{Arterial blood} (red): high partial pressure of oxygen (pO$_2$), which is oxygem-rich and delivering oxygen to tissues.
\end{itemize}

\subsection*{Oxygen Binding to Myoglobin}
\begin{center}
    \includegraphics*[width=\textwidth]{L2_11.png}
\end{center}
\begin{itemize}
    \item At 13.3 kPa (lungs/alveoli) reflects oxygen-rich conditions in the lungs
    \begin{itemize}
        \item Myoglobin is almost fully saturated with oxygen at this high partial pressure
        \item Binds oxygen very efficiently, acting as an oxygen reservoir
        \item The binding curve is hyperbolic, approaching 100\% saturation at these oxygen levels
    \end{itemize}
    \item At 5.3 kPa (capillary blood/tissues) Oxygen pressure is lower, myoglobin remains highly saturated acting as an oxygen reservoir, essential for maintaining oxygen supply in tissues
    \item At <1 kPa (mitochondria):
    \begin{itemize}
        \item At very low pO$_2$ levels inside mitochondria, myoglobin begins releasing oxygen (crucial for cellular respiration)
        \item Myoglobin's high affinity ensures oxygen is available exactly where it's needed for energy production, such as in muscle cells during activity
    \end{itemize}
\end{itemize}

\subsection*{Why doesn't myoglobin carry O$_2$ through the body?}
\begin{itemize}
    \item Myoglobin tightly binds to O$_2$ at low pO$_2$ \textbf{and} high pO$_2$
    \item An efficient carrier of O$_2$ has to \textbf{let go of it} when exposed to loewr pO$_2$
\end{itemize}
\begin{center}
    \includegraphics*[width=\textwidth]{L2_12.png}
\end{center}

\subsection*{We need a different kind of curve}
\begin{itemize}
    \item Myoglobin can't let go in O$_2$ in tissues so why don't we lower its affinity?
    \begin{itemize}
        \item Low affinity too inefficient
    \end{itemize}
    \item We need an oxygen-binding protein that can flip its behavior \textbf{sigmoidally} when moving from low affinity to high affinity
    \item Has to have:
    \begin{itemize}
        \item Multiple binding sites
        \item Binding sites must be able to "communicate to each other"
        \begin{itemize}
            \item We're going to call the \textbf{cooperativity}
        \end{itemize}
    \end{itemize}
\end{itemize}
\begin{center}
    \includegraphics*[scale=0.6]{L2_13.png}
\end{center}

\end{document}
