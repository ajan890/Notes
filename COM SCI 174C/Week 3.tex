\documentclass{article}
\usepackage{setspace}
\usepackage[utf8]{inputenc}
\usepackage[bb=boondox]{mathalfa}
\usepackage{graphicx}

%%Let's you change margins
\usepackage[left=1in,right=1in,top=1in,bottom=1in]{geometry}

%%Math symbols, proof environments
\usepackage{amsmath,amsthm,amssymb, graphicx, tikz}

%%Use this package for matrices
\usepackage{array}

%%Commands for common sets
\newcommand{\R}{\mathbb{R}} %Real numbers
\newcommand{\Z}{\mathbb{Z}} %Integers
\newcommand{\dP}{\mathbb{P}}
\newcommand{\example}{\textbf{Example: }}
\newcommand{\E}{\mathbb{E}} %Expectation Value

\graphicspath{{./assets/images/Week 3}}

\title{CS 174C Week 3} 

\author{Aidan Jan}

\date{\today}
\begin{document}
\maketitle
\section*{Motion Curves}
\begin{itemize}
    \item The most basic capability of an animation package is to let the user set animation variables in each frame
    \begin{itemize}
        \item Not so easy - major HCI challenges in designing an effective user interface
        \item We will not consider HCI issues
    \end{itemize}
    \item The next is to support keyframing: Computer automatically interpolates in-between frames
    \item A motion curve is what you get when you plot an animation variable against time
    \begin{itemize}
        \item The computer must come up with motion curves that interpolate your keyframe values
    \end{itemize}
\end{itemize}

\subsection*{Different Forms of Curve Functions}
\begin{itemize}
    \item Explicit: $y = f(x)$
    \begin{itemize}
        \item Cannot get multiple values for single x or infinite slopes
    \end{itemize}
    \item Implicit: $f(x, y) = 0$
    \begin{itemize}
        \item Cannot easily compare tangent vectors at joints
        \item In/Out test, normals from gradient
    \end{itemize}
    \item Parametric: $x = f_x(t), y = f_y(t), z = f_z(t)$
    \begin{itemize}
        \item Most convenient for motion representation
    \end{itemize}
\end{itemize}

\subsection*{Describing Curves by Means of Polynomials}
Reminder:
\begin{itemize}
    \item L$^{\text{th}}$ degree polynomial
    \item $p(t) = a_0 + a_1 t + a_2 t^2 + \cdots + a_L t^L$
    \item $a_0, \dots, a_L$ are the coefficients
    \item $L$ is the degree
    \item $(L + 1)$ is the "order" of the polynomial
\end{itemize}

\subsection*{Polynomial Curves of Degree 1}
Parametric and implicit forms are linear
\begin{align*}
    x(t) &= at + b \\
    y(t) &= ct + d
\end{align*}
\begin{center}
    \includegraphics*[scale=0.8]{W3_1.png}
\end{center}

\subsection*{Polynomial Curves of Degree 2}
\textbf{Parametric}
\begin{itemize}
    \item $x(t) = at^2 + 2bt + c$
    \item $y(t) = dt^2 + 2et + f$
    \item For any choice of constants
    \begin{itemize}
        \item $a, b, c, d, e, f \rightarrow$ parabola
    \end{itemize}
\end{itemize}
\textbf{Rational Parametric}
\begin{itemize}
    \item $P(t) = \frac{P_0(1 - t)^2 + 2wP_1 t(1-t) + P_2 t^2}{(1-t)^2 + 2wt(1 - t) + t^2}$
    \item $w<1$: ellipse
    \item $w=1$: parabola
    \item $w>1$: hyperbola
\end{itemize}
\subsection*{Curves From Geometric Constraints}
\textbf{Geometric Approach}
\begin{itemize}
    \item Constraints $\rightarrow$ Polynomial $\rightarrow$ Curve
    \item $P_0, \dots, P_L \rightarrow $ (Curve Generation) $\rightarrow P(t)$
    \begin{itemize}
        \item $P_i$: control points
        \item $P_0, \dots, P_L$: control polygon
    \end{itemize}
\end{itemize}
\textbf{Interpolation vs. Approximation}
\begin{center}
    \includegraphics*[scale=0.8]{W3_2.png}
\end{center}

\section*{Bezier Curves and the De Casteljau Algorithm}
\subsection*{Tweening}
When there are two points:
\begin{itemize}
    \item $A(t) = (1 - t) P_0 + tP_1$
    \item $P(t) = A(t)$
    \item Essentially, $A$ is a point on the line between $P_0$ and $P_1$
\end{itemize}
When there are three points:
\begin{itemize}
    \item $A(t) = (1 - t) P_0 + t P_1$
    \item $B(t) = (1 - t) P_1 + t P_2$
    \item $A(t)$ is a point between $P_0$ and $P_1$ and $B(t)$ is a point between $P_1$ and $P_2$.
    \item Now, place another point, $P(t)$ on the line between $A(t)$ and $B(t)$.
    \begin{itemize}
        \item $P(t) = (1 - t)A + tB = (1 - t)^2 P_0 + 2t(1-t) P_1 + t^2 P_2$
    \end{itemize}
\end{itemize}
When we move the value of $t$ from 0 to 1, $P(t)$ would move from $P_0$ to $P_2$ along a curved path, defined by the quadratic equation.
\begin{center}
    \includegraphics*[scale=1]{W3_3.png}
\end{center}
If we repeat the same process for $P(t)$ but instead of four points, then
\[P(t) = (1 - t)^3 P_0 + 3(1 - t)^2 t P_1 + 3(1 - t) t^2 P_2 + t^3 P_3\]
\begin{center}
    \includegraphics*[scale=1]{W3_4.png}
\end{center}

\section*{Cubic Berstein Polynomials}
\[P(t) = (1 - t)^3 P_0 + 3(1-t)^2 t P_1 + 3(1 - t) t^2 P_2 + t^3 P_3\]
\begin{align*}
B^3_0 (t) &= (1-t)^3\\
B^3_1 (t) &= 3(1-t)^2 t\\
B^3_2 (t) &= 3(1-t) t^2\\
B^3_3 (t) &= t^3
\end{align*}
Expansion of $[(1-t) + t]^3 = (1-t)^3 + 3(1 - t)^2 t + 3(1-t) t^2 + t^3 \rightarrow \sum_k B^3_k (t) = 1, k = 0, 1, 2, 3$\\
An affine combination of points

\subsection*{Berstein Polynomials of Degree L}
Degree $L$ implies $L + 1$ control points.
\[P(t) = \sum_{k=0}^L B_k^L (t) P_k\]
where:
\begin{itemize}
    \item $B_k^L(t) = {L \choose k} (1-t)^{L - k} t^k$
    \item ${L \choose k} = \frac{L!}{k!(L - k)!}$, for $L > k$
    \item $\sum_{k = 0}^L B^L_k (t) = 1$, for all $t$
\end{itemize}
Expansion of $[(1 - t) + t]^L$\\\\
\textbf{Common Berstein Polynomials}
\begin{center}
    \includegraphics*[scale=1]{W3_5.png}
\end{center}
\begin{itemize}
    \item Always positive
    \item Zero only at $t = 0$ or $t = 1$
    \item Berstein polynomials can also be defined recursively
    \begin{align*}
        B_i^n(t) &= (1-t) B_i^{n-1}(t) + t B_{i - 1}^{n-1} (u)\\
        B_0^0 (t) &= 1
    \end{align*}
\end{itemize}

\subsection*{Properties of Bezier Curves}
\begin{itemize}
    \item End point Interpolation
    \item Affine Invariance: $T(P(t)) = \sum_{k = 0}^L B_k^L(t) T(P)_k$
    \item Invariance under affine transformation of the parameter
    \item Convex Hull property for t in [0, 1]: $P = \sum_{k = 0}^L a_k P_k$ where $\sum_{k = 0}^L a_k = 1$ and $a_k > 0$
    \item Linear precision by collapsing convex hull
    \item Variation Diminishing property: No straight line cuts the curve more times than it cuts the control polygon
\end{itemize}

\subsection*{Derivatives of Bezier Curves}
It can be shown that:
\begin{itemize}
    \item Velocity is also a Bezier curve of lower degree
    \[P'(t) = L \sum_{k = 0}^{L - 1} B_k^{L - 1}(t) \Delta P_k\text{, where } \Delta P_k = P_{k + 1} - P_k\]
    \item Acceleration
    \[P''(t) = L(L - 1) \sum_{k = 0}^{L - 2} B_k^{L - 2}(t) \Delta^2 P_k \text{, where } \Delta^2 P_k = \Delta P_{k + 1} - \Delta P_k\]
\end{itemize}

\subsection*{Cubic Parametric Curves}
\begin{align*}
    x(t) &= a_3 t^3 + a_2 t^2 + a_1 t + a_0\\
    y(t) &= b_3 t^3 + b_2 t^2 + b_1 t + b_0\\
    z(t) &= c_3 t^3 + c+2 t^2 + c_1 t + c_0\\
    t &\in [0, 1]
\end{align*}
As a matrix,
\[x(t) = \begin{bmatrix}t^3 & t^2 & t^1 & 1\end{bmatrix} \begin{bmatrix} a_3 \\ a_2 \\ a_1 \\ a_0 \end{bmatrix}\]
\begin{align*}
    x(t) &= TA\\
    y(t) &= TB\\
    z(t) &= TC
\end{align*}

\subsection*{Derivative of Cubic Parameter Curves}
Simply take the derivative of the matrix in terms of t.
\begin{align*}
    x(t) &= \begin{bmatrix}t^3 & t^2 & t^1 & 1\end{bmatrix} \begin{bmatrix} a_3 \\ a_2 \\ a_1 \\ a_0 \end{bmatrix}\\
    x'(t) &= \begin{bmatrix}3t^2 & 2t & 1 & 0\end{bmatrix} \begin{bmatrix} a_3 \\ a_2 \\ a_1 \\ a_0 \end{bmatrix}
\end{align*}

\subsection*{How does the magnitude of the tangent affect the curve?}
\begin{itemize}
    \item Same lower tangent direction but different magnitude.
    \item The magnitude defines how fast the curve assumes the tangent direction
    \item (remember: tangent $\rightarrow$ velocity in parametric space)
\end{itemize}

\subsection*{Parametric Cubic Curves From Constraints}
\textbf{Example:} Endpoints and a tangent at midpoint\\
Constraints:
\begin{itemize}
    \item $x(t) = \begin{bmatrix}t^3 & t^2 & t^1 & 1\end{bmatrix}A$
    \item $x'(t) = \begin{bmatrix}3t^2 & 2t & 1 & 0\end{bmatrix}A$
    \item $x(0) = \begin{bmatrix} 0 & 0 & 0 & 1 \end{bmatrix}A$
    \item $x(0.5) = \begin{bmatrix} 0.5^3 & 0.5^2 & 0.5 & 1 \end{bmatrix}A$
    \item $x'(0.5) = \begin{bmatrix} 3(0.5^2) & 2(0.5) & 1 & 0 \end{bmatrix}A$
    \item $x(1) = \begin{bmatrix} 1 & 1 & 1 & 1 \end{bmatrix} A$
    \item $G_x = BA$
\end{itemize}
This can be written as:
\begin{align*}
    x(t) &= \begin{bmatrix}t^3 & t^2 & t^1 & 1\end{bmatrix}A = TA\\
    x'(t) &= \begin{bmatrix}3t^2 & 2t & 1 & 0\end{bmatrix}A = T'A\\
    \begin{bmatrix}x_0 \\ x_{0.5} \\ x'_{0.5} \\ x_1\end{bmatrix} &= \begin{bmatrix}0 & 0 & 0 & 1 \\ 0.5^3 & 0.5^2 & 0.5  & 1 \\ 3(0.5)^2 & 2(0.5) & 1 & 0 \\ 1 & 1 & 1 & 1\end{bmatrix}A\\
    G_x &= BA \Rightarrow A = B^{-1} G_x\\
    x(t) = TA \Rightarrow \boxed{x(t) = TB^{-1} G_x}
\end{align*}
Final form:
\begin{itemize}
    \item Basis matrix
        \begin{align*}
            x(t) &= TB^{-1} G_x \\
            \text{Set }&M = B^{-1}\\
            & x(t) = TMG_x\\
            & y(t) = TMG_y\\
            & z(t) = TMG_z\\
            P(t) &= TMG
        \end{align*}
    \item For the example
        \[M = \begin{bmatrix}-4 & 0 & -4 & 4 \\ 8 & -4 & 6 & -4 \\ -5 & 5 & -2 & 1 \\ 1 & 0 & 0 & 0\end{bmatrix}\]
\end{itemize}

\subsection*{Blending Functions}
Given by TM:
\begin{itemize}
    \item $x(t) = TMG_x \Rightarrow x(t) = \begin{bmatrix}f_1(t) & f_2(t) & f_3(t) & f_4(t)\end{bmatrix} G_x$
\end{itemize}
For the example:
\begin{itemize}
    \item $f_1(t) = -4t^3 + 8t^2 - 5t + 1$
    \item $f_2(t) = -4t^2 + 4t$
    \item $f_3(t) = -4t^3 + 6t^2 - 2t$
    \item $f_4(t) = 4t^3 - 4t^2 + t$
\end{itemize}
Each blending function weights the contribution of one of the constraints.
\begin{center}
    \includegraphics*[scale=1]{W3_6.png}
\end{center}
\section*{Hermite Curves}
\textbf{Constraints:} Two endpoints and two tangents
\begin{center}
    \includegraphics*[scale=0.8]{W3_7.png}
\end{center}
\begin{flalign*}
    G_h &= \begin{bmatrix} P_1 & P_4 & R_1 & R_4\end{bmatrix}&&\\
    x(t) &= T A_h = T M_h G_h&&\\
    &&\\
    x(0) &= P_1 = \begin{bmatrix} 0 & 0 & 0 & 1 \end{bmatrix}A_h&&\\
    x(1) &= P_4 = \begin{bmatrix} 1 & 1 & 1 & 1 \end{bmatrix}A_h&&\\
    x'(0) &= R_1 = \begin{bmatrix} 0 & 0 & 1 & 0 \end{bmatrix}A_h&&\\
    x'(1) &= R_4 = \begin{bmatrix} 3 & 2 & 1 & 0 \end{bmatrix}A_h&&\\
    G_h &= B_h A_h&&\\
    A_h &= B_h^{-1} G_h&&\\
    x(t) &= T A_h
\end{flalign*}
\subsection*{Blending Functions}
\begin{center}
    \includegraphics*[scale=0.7]{W3_8.png}
\end{center}
\begin{align*}
M_h &= B_h ^{-1} = \begin{bmatrix} 2 & -2 & 1 & 1 \\ -3 & 3 & -2 & -1 \\ 0 & 0 & 1 & 0 \\ 1 & 0 & 0 & 0\end{bmatrix}\\
x(t) &= T M_h G_h \Rightarrow x(t) = \begin{bmatrix} f_1 (t) & f_2 (t) & f_3 (t) & f_4 (t) \end{bmatrix} G_h\\
& \\
f_1(t) &= 2t^3 - 3t^2 + 1\\
f_2(t) &= -2t^3 + 3t^2\\
f_3(t) &= t^3 - 2t^2 + t\\
f_4(t) &= t^3 - t^2
\end{align*}
\section*{Bezier Curves}
Bezier curves are a special case of Hermite curves.
\begin{align*}
    P_{1, h} &= P_1\\
    P_{4, h} &= P_4\\
    R_{1, h} &= 3(P_2 - P_1)\\
    R_{4, h} &= 3(P_4 - P_3)\\
\end{align*}
As a matrix:
\[\begin{bmatrix}P_{1, h} \\ P_{4, h} \\ R_{1, h} \\ R_{4, h}\end{bmatrix} = \begin{bmatrix} 1 & 0 & 0 & 0 \\ 0 & 0 & 0 & 1 \\ -3 & 3 & 0 & 0 \\ 0 & 0 & -3 & 3 \end{bmatrix}\begin{bmatrix}P_1 \\ P_2 \\ P_3 \\ P_4 \end{bmatrix}\]
\[G_h = M_{bh}G_b\]
\[P(t) = TM_h G_h \Rightarrow P(t) = TM_h M_{bh} G_b \Rightarrow P(t) = TM_b G_b\]
We can verify that $TM_b$ are the Berstein Polynomials.\\
Recall that:
\begin{itemize}
    \item $f_1(t) = (1 - t)^3$
    \item $f_2(t) = 3t(1-t)^2$
    \item $f_3(t) = 3t^2(1-t)$
    \item $f_4(t) = t^3$
\end{itemize}
\section*{Splines}
\begin{itemize}
    \item Splines are the standard way to generate a smooth curve which interpolates given values
    \item A spline function is just a piecewise polynomial function
    \begin{itemize}
        \item Split up the real line into intervals
        \item Over each interval, pick a different polynomial
    \end{itemize}
    \item If the polynomials are small degree (Typically at most cubics) the curve is easy and fast to compute
\end{itemize}
\subsection*{Connecting piecewise curves to splines:}
\begin{center}
    \includegraphics*[scale=0.9]{W3_9.png}
\end{center}
\subsection*{Continuity}
\begin{itemize}
    \item Parametric $C^k$ Continuity
    \begin{itemize}
        \item Derivatives $P^{(i)}$ for $i = 0, \dots, k$ exist and are continuous for $t$ in $[a, b]$
        \item Terminology:
            \begin{itemize}
                \item $P$ is $k$-smooth
                \item $P$ has $k^{th}$-order continuity
            \end{itemize}
    \end{itemize}
    \item Geometric $G^k$ Continuity
    \begin{itemize}
        \item $P^{(i)}(t-) = c_i P^{(i)} (t+)$
            \begin{itemize}
                \item for i = 0, $\dots$, k and
                \item for some constants $c_i$
                \item for $t$ in $[a, b]$
            \end{itemize}
    \end{itemize}
\end{itemize}

\subsection*{Knots and Control Points}
\begin{itemize}
    \item The ends of the intervals, where one polynomial ends and another one starts, are called "knots"
    \item A control point is a knot together with associated shape control variables
    \item The spline either interpolates (goes through) or approximates (goes near) the control points    
\end{itemize}

\subsection*{Spline Curve}
Different definitions exist.\\
\textbf{Ours:}
\begin{itemize}
    \item A spline curve is an affine blend of points weighted by piecewise polynomial functions.  It must be continuous at the knots, but may have discontinuous derivatives.
\end{itemize}

\subsubsection*{Hermite Splines}
\begin{itemize}
    \item Hermite splines have even richer control points:
    \item In addition to a function value, a slope (derivative) is specified.
    \begin{itemize}
        \item So the Hermite spline interpolates the control values and must match the control slopes at the knots
    \end{itemize}
    \item Particularly useful for animation - more control over slow in/out, etc.
\end{itemize}

\subsection*{Smoothness}
\begin{itemize}
    \item Each polynomial in a spline is infinitely differentiable (very smooth)
    \item But at the junction between two polynomials, the spline isn't necessarily even continuous!
    \item We need to enforce constraints on the polynomials to get the degree of smoothness we want
    \begin{itemize}
        \item Values match: $C^0$ continuous
        \item Slopes (first derivatives) match: $C^1$ continuous
        \item Second derivatives match: $C^2$ continuous
        \item Etc.
    \end{itemize}
\end{itemize}

\subsubsection*{Example: Piecewise Linear Spline}
\begin{itemize}
    \item Restrict all polynomials to be linear
    \begin{itemize}
        \item $f(t) = a_i (t - t_i) + b_i$ in $[t_i, t_{i + 1}]$
    \end{itemize}
    \item Enforce continuity: make each line segment interpolate the control point on either specified
    \begin{itemize}
        \item $a_i (t - t_i) + b_i = y_i$ and $a_i (t_{i + 1} - t_i) + b_i = y_{i + 1}$
    \end{itemize}
    \item Solve to get
    \begin{itemize}
        \item $a_i = \frac{y_{i + 1} - y_i}{t_{i + 1} - t_i}$; $b_i = y_i$
    \end{itemize}
    \item End result:
    \begin{itemize}
        \item straight line segments connecting the control points
    \end{itemize}
    \item $C^0$ but not $C^1$
\end{itemize}
\begin{center}
    \includegraphics*[scale=1]{W3_10.png}
\end{center}
To do better than this, we need higher degree polynomials.\\
For motion curves, cubic splines are almost always used.\\
We now have three main choices:
\begin{itemize}
    \item Hermite splines: Interpolating, up to $C^1$
    \item Catmull-Rom: Interpolating, $C^1$
    \item B-Splines: Approximating, $C^2$
\end{itemize}
\subsection*{Cubic Hermite Splines}
\begin{itemize}
    \item Our generic cubic in an interval $[t_i, t_{i + 1}]$ is:
    \begin{itemize}
        \item $q_i(t) = a_i(t - t_i)^3 + b_i (t - t_i)62 + c_i (t - t_i) + d_i$ with $t - t_i$ in $[0, 1]$
    \end{itemize}
    \item Make it interpolate endpoints:
    \begin{itemize}
        \item $q_i(t_i) = y_i$ and $q_i(t_{i + 1}) = y_{i + 1}$
    \end{itemize}
    \item And make it match given slopes:
    \begin{itemize}
        \item $q_i'(t_i) = s_i$ and $q_i'(t_{i + 1}) = s_{i + 1}$
    \end{itemize}
    \item Work it out to get
    \begin{align*}
        a_i &= \frac{-2(y_{i + 1} - y_i)}{(t_{i + 1} - t_i)^3} + \frac{s_i + s_{i + 1}}{(t_{i + 1} - t_i)^2} \hspace{1cm} c_i = s_i\\
        b_i &= \frac{3(y_{i + 1} - y_i)}{(t_{i + 1} - t_i)^2} - \frac{2s_i + s_{i + 1}}{t_{i + 1} - t_i} \hspace{1cm} d_i = y_i
    \end{align*}
\end{itemize}
\subsection*{Hermite Basis}
Rearrange the solution to get:
\begin{align*}
    y(t) = &y_i\left(\frac{2(t - t_i)^3}{(t_{i + 1} - t_i)^3} - \frac{3(t - t_i)^2}{(t_{i + 1} - t_i)^2} + 1 \right) + y_{i + 1}\left(\frac{-2(t - t_i)^3}{(t_{i + 1} - t_i)^3} - \frac{3(t - t_i)^2}{(t_{i + 1} - t_i)^2}\right) \\
    &+ s_i \left(\frac{(t - t_i)^3}{(t_{i + 1} - t_i)^2} - \frac{2(t - t_i)^2}{t_{i + 1} - t_i} + (t - t_i)\right) + s_{i + 1} \left(\frac{(t - t_i)^3}{(t_{i + 1} - t_i)^2} - \frac{(t - t_i)^2}{t_{i + 1} - t_i}\right)
\end{align*}
for $i \in [t_0, t_n]$\\
That is, we're taking a linear combination of four basis functions
\begin{itemize}
    \item Note the functions and their slopes are either 0 or 1 at the start and end of the interval
\end{itemize}

\subsection*{Breaking Hermite Splines}
\begin{itemize}
    \item Usually just specify one slope at each knot
    \item But a useful capability: use a different slope on each side of a knot
    \begin{itemize}
        \item We break $C^1$ smoothness, but gain control
        \item Can create motions that abruptly change, like collisions
    \end{itemize}
\end{itemize}

\subsection*{Catmull-Rom Splines}
\begin{itemize}
    \item Given a set of points in space, suppose we want a spline that:
    \begin{itemize}
        \item Interpolates the points (rules out Bezier)
        \item With $C^1$ continuity (Hermite: Lots of tweaking)
    \end{itemize}
    \item This is a common situation in animation
    \item We start with the given set of points $P_0, \dots, P_n$
    \item Define tangents $r_i = s(P_{i + 1} - P_{i - 1})$
    \begin{center}
        \includegraphics*[scale=0.5]{W3_11.png}
    \end{center}
    \item This is really just a $C^1$ Hermite spline with an automati choice of slopes
    \begin{itemize}
        \item Use a 2nd order finite difference formula to estimate slope from values
        \[s_i = \left(\frac{t_i - t_{i - 1}}{t_{i + 1} - t_{i - 1}}\right) \div \frac{y_{i + 1} - y_i}{t_{i + 1} - t_i} - \left(\frac{t_{i + 1} - t_i}{t_{i + 1} - t_{i - 1}}\right) \div \frac{y_i - y_{i - 1}}{t_i - t_{i - 1}}\]
        \item For equally spaced knots, it simplifies to:
        \[s_i = \frac{y_{i + 1} - y_{i - 1}}{t_{i + 1} - t_{i - 1}}\]
    \end{itemize}
\end{itemize}

\subsection*{Catmull-Rom Boundaries}
\begin{itemize}
    \item Need to use slightly different formulas for the Boundaries
    \item For example, 2nd order accurate finite difference at the start of the interval:
    \[s_0 = \left(\frac{t_2 - t_0}{t_2 - t_1}\right)\div \frac{y_1 - y_0}{t_1 - t_0} - \left(\frac{t_1 - t_0}{t_2 - t_1}\right) \div \frac{y_2 - y_0}{t_2 - t_0}\]
    \begin{itemize}
        \item Symmetric formula for end of interval
    \end{itemize}
    \item Which simplifies for equally spaced knots:
    \[s_0 = 2\frac{y_1 - y_0}{\Delta t} - \frac{y_2 - y_0}{2\Delta t}\]
\end{itemize}

\subsection*{Adding Tension Control}
\[P'(t_k) = \frac{1}{2}(1 - v_k)(P_{k + 1} - P_{k - 1})\]
\begin{center}
    \includegraphics*[scale=0.7]{W3_12.png}
\end{center}

\subsection*{Adding Bias Control}
Kochanek-Bartels Splines
\begin{align*}
    P'(t_k) &= \frac{1}{2}(P_k - P_{k - 1}) + \frac{1}{2}(P_{k + 1} - P_k)\\
    P'(t_k) &= \frac{1}{2}(1 - b_k)(P_k - P_{k - 1}) + \frac{1}{2}(1 + b_k)(P_{k + 1} - P_k)
\end{align*}
\begin{center}
    \includegraphics*[scale=0.7]{W3_13.png}
\end{center}

\subsection*{Adding Continuity Control}
Kochanek-Bartels Splines
\begin{align*}
    R'_{k - 1}(1) &= \frac{1}{2}(1 - c_k)(P_k - P_{k - 1}) + \frac{1}{2}(1 + c_k)(P_{k + 1} - P_k)\\
    R'_k(0) &= \frac{1}{2}(1 + c_k)(P_k - P_{k - 1}) + \frac{1}{2}(1 - c_k)(P_{k + 1} - P_k)
\end{align*}
\begin{center}
    \includegraphics*[scale=0.7]{W3_14.png}
\end{center}

\section*{Splines Summary}
\begin{itemize}
    \item General form of the splines we have segment
    \[P(t) = \sum_{k = 0}^L P_k f_k(t),\:t \in \Re\]
    \item Often f is a translated version of a single function
    \begin{align*}
        P(t) =& \sum_{k = 0}^L P_k f(t - t_k)\\
        \text{knots} :& \:[t_0, \dots, t_L]
    \end{align*}
    \item For Hermite splines, the knots coincided with the control points
\end{itemize}

\section*{B-Splines}
\begin{itemize}
    \item Like Catmull-Rom splines, start with sequence of control points $P_0, \dots, P_n$
    \item Drop the interpolating condition and instead design a spline curve that is $C^{m - 2}$ smooth, where $m$ is the order
    \begin{itemize}
        \item Curve segments meet at knots
        \item For Hermite and others, the knots were always at control points
    \end{itemize}
    \item Now basis functions overlap more than one know interval
\end{itemize}

\subsection*{General Form of B-Splines}
\begin{itemize}
    \item Let's make the order $m$ explicit
    \[P(t) = \sum_{k = 0}^L P_k N_{k, m}(t)\]
    \item Fundamental formula
    \begin{itemize}
        \item generalization of successive linear Interpolation
        \[N_{k, m}(t) = \left(\frac{t - t_k}{t_{k + m - 1} - t_k}\right) N_{k, m - 1}(t) + \left(\frac{t_{k + m} - t}{t_{k + m} - t_{k + 1}}\right)N_{k + 1, m - 1}(t)\]
        \[t \in [t_k, t_{k + m}]\]
    \end{itemize}
    \item For equispaced knots $0, \dots, L$
    \[P(t) = \sum_{k = 0}^L P_k N_{k, m}(t) = \sum_{k = 0}^L P_k N_{k, m} (t - k)\]
\end{itemize}

\subsection*{Constant B-Splines (Order m = 1)}
\begin{itemize}
    \item With knots $t_0 = 0$, $t_1 = 1$
    \item Constant (a.k.a. Haar) function,
    \begin{align*}
        N_1(t) &= \begin{cases}1 & \text{if } 0 \leq t < 1 \\ 0 & \text{otherwise}\end{cases}\\
        N_{k, 1}(t) &= N_{0, 1}(t - k)
    \end{align*}
    \item For non-equispaced knots
    \[N_{k, 1}(t) = \begin{cases}1 & \text{if } t_k \leq t < t_{k + 1} \\ 0 & \text{otherwise}\end{cases}\]
    \item Each function has support over $[t_k, t_{k + 1})$
\end{itemize}

\subsection*{Linear B-Splines (Order m = 2)}
\begin{itemize}
    \item With knots $t_0 = 0, t_1 = 1, \dots$
\end{itemize}
\begin{align*}
    N_{0, 2}(t) &= \frac{t}{1} N_{0, 1} + \frac{2 - t}{1}N_{1, 1}\\
    &= \begin{cases}t \times 1 + (2 - t) \times 0 & \text{if } t \in [0, 1] \\ (2 - t) \times 0 + (2 - t) \times 1 & \text{if } t \in [1, 2] \\ 0 & \text{otherwise}\end{cases}\\
    &= \begin{cases} t & \text{if } t \in [0, 1) \\ 2 - t & \text{if } t \in [1, 2] \\ 0 & \text{otherwise}\end{cases}
    & \\
    N_{k, 2}(t) &= N_{0, 2}(t - t_k)
\end{align*}    
\begin{center}
    \includegraphics*[scale=0.8]{W3_15.png}
\end{center}


\end{document}