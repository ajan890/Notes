\documentclass{article}
\usepackage{setspace}
\usepackage[utf8]{inputenc}
\usepackage[bb=boondox]{mathalfa}
\usepackage{graphicx}
\usepackage{hyperref}

%%Let's you change margins
\usepackage[left=1in,right=1in,top=1in,bottom=1in]{geometry}

%%Math symbols, proof environments
\usepackage{amsmath,amsthm,amssymb, graphicx, tikz}

%%Use this package for matrices
\usepackage{array}

%%Commands for common sets
\newcommand{\R}{\mathbb{R}} %Real numbers
\newcommand{\Z}{\mathbb{Z}} %Integers
\newcommand{\dP}{\mathbb{P}}
\newcommand{\example}{\textbf{Example: }}
\newcommand{\E}{\mathbb{E}} %Expectation Value

\newcommand{\der}{\text{d}}

\graphicspath{{./assets/images/Week 5}}

\title{CS 174C Week 5} 

\author{Aidan Jan}

\date{\today}
\begin{document}
\maketitle
\section*{Physically Based Animation}
\subsection*{What is a Particle}
The most basic particle has a position $x$ in space.\\
There may be other attributes, such as:
\begin{itemize}
    \item Velocity $v$
    \item Acceleration $a$
    \item Mass $m$
    \item Orientation 
    \item Radius
    \item Temperature
    \item averageColor
    \item Type
\end{itemize}
The sky is the limit - e.g., AI agent models of behavior

\subsection*{Basic Particle Animation}
\begin{itemize}
    \item Specify a velocity field $v(x, t)$ for any point in space $x$ at any time $t$
    \item Break time $t$ into discrete steps $\Delta t$
    \begin{itemize}
        \item E.g., one step per frame - $\Delta t$ = 1/30th second
        \item Or several steps per frame
    \end{itemize}
    \item Change the particle's current position $x(t)$ by integrating over the time step
    \begin{itemize}
        \item Forward Euler Integration:
        \begin{itemize}
            \item Since velocity $v(t) = \frac{\der x}{\der t} = \lim_{\Delta t \rightarrow 0} \frac{x(t + \Delta t) - x(t)}{\Delta t}$,
            \[x(t + \Delta t) \approx x(t) + \Delta t \cdot v(x(t), t)\]
        \end{itemize}
    \end{itemize}
\end{itemize}

\subsection*{Velocity Fields}
\begin{itemize}
    \item Velocity field can be a combination of pre-designed velocity elements
    \begin{itemize}
        \item E.g., explosions, vortices, \dots
    \end{itemize}
    \item Or from "noise"
    \begin{itemize}
        \item Smooth random number field
    \end{itemize}
    \item Or from a simulation
    \begin{itemize}
        \item Interpolate velocity from a computed grid
        \item E.g., smoke simulation
    \end{itemize}
\end{itemize}
\subsection*{Seeding Particles}
\begin{itemize}
    \item Need to add (or seed) particles to the scene.
    \item Where?
    \begin{itemize}
        \item At a point source
        \item Randomly on a surface or within a shaped volume
        \item Where there currently are insufficient particles
    \end{itemize}
    \item When?
    \begin{itemize}
        \item At the start
        \item Several per frame
        \item Whenever there are insufficient particles somewhere
    \end{itemize}
    \item Need to figure out other attributes, not just position
    \begin{itemize}
        \item E.g., velocity pointing outwards in an explosion
    \end{itemize}
\end{itemize}

\subsection*{Basic Particle Rendering}
\begin{itemize}
    \item Draw a dot for each particle
    \item But what do you do with several particles per pixel?
    \begin{itemize}
        \item Add: models each point emitting (but not absorbing light)
        \begin{itemize}
            \item Good for sparks, fire, \dots
        \end{itemize}
        \item More generally, compute depth order, do alpha-compositing
        \begin{itemize}
            \item and worry about shadows, etc.
        \end{itemize}
    \end{itemize}
    \item Anti-aliasing
    \begin{itemize}
        \item Blur edges of particle; make sure they are blurred to cover at least a pixel
    \end{itemize}
    \item Particle with radius: Kernel function
\end{itemize}

\subsection*{Motion Blur}
Easy for simple particles
\begin{itemize}
    \item Instead of a dot, draw a line from old position to new position
    \begin{itemize}
        \item During the "open shutter" time
    \end{itemize}
    \item May involve decrease in alpha
    \item More accurately, draw a spline curve
    \item May also need to take particle radius into account
\end{itemize}

\section*{Physics-Based Motion of a Particle}
Second-order dynamic motion of mass particles
\begin{itemize}
    \item Newton's law $f = ma$
    \item Need to specify force(s) f
    \begin{itemize}
        \item gravity, collisions, ...
    \end{itemize}
    \item Divide net force by particle mass $m$ to obtain acceleration $a$
    \item Update velocity $v$ using acceleration
    \item Update position $x$ using velocity
\end{itemize}

\subsection*{Integrating the Equations of Motion}
\begin{itemize}
    \item Newtons law:
    \[a(t) = \frac{1}{m} f(x(t), v(t), t)\]
    \item We solve this by numerical simulation
    \begin{itemize}
        \item Integrate second-order ODE forward in time
        \begin{itemize}
            \item Compute $v(t)$, $x(t)$, at $t = t_0, t_0 + \Delta t, t_0 + 2\Delta t, \dots, t_0 + n\Delta t, \dots$
            \item Symplectic Euler integration step at each time t:
            \begin{align*}
            v(t + \Delta t) &= v(t) + \Delta t a(t)\\
            x(t + \Delta t) &= x(t) + \Delta t v(t + \Delta t)
            \end{align*}
        \end{itemize}
    \end{itemize}
\end{itemize}

\subsection*{Mass-Spring-Damper Systems}
\begin{itemize}
    \item Particles interconnected by springs and dampers
    \begin{itemize}
        \item Simplest mass-spring-damper system:
        \begin{center}
            \includegraphics*[scale=0.8]{W5_1.png}
        \end{center}
        \begin{itemize}
            \item Parallel spring and dampler: Voigt visco-elastic (v-e) element
        \end{itemize}
        \item Equations of motion with $f_{ij}^e = f_{ij}^s + f_{ij}^d$
        \begin{align*}
            m_i a_i &= f_i + f_{ij}^e\\
            m_j a_j &= f_j + f_{ji}^e
        \end{align*}
    \end{itemize}
\end{itemize}

\subsection*{Viscoelastic Forces}
\begin{center}
    \includegraphics*[scale=0.8]{W5_2.png}
\end{center}
\begin{align*}
    d_{ij} &= x_j - x_i\\
    d_{ij_m} &= \vert\vert d_{ij} \vert \vert\\
    \hat d_{ij} &= d_{ij} / d_{ij_m}\\
    v_{ij} &= v_j - v_i
\end{align*}


\begin{itemize}
    \item Spring force (elastic)
    \begin{itemize}
        \item With spring constant $k_{ij}^s$ and natural length $l_{ij}$
        \[f_{ij}^s = k_{ij}^s(d_{ij_m} - l_{ij}) \hat d_{ij}; \hspace{1.5cm} f_{ji}^s = -f_{ij}^s\]
    \end{itemize}
    \item Damper force (viscous)
    \begin{itemize}
        \item With damping constant $k_{ij}^d$
        \[f_{ij}^d = -k_{ij}^d(v_{ij} \cdot \hat d_{ij}) \hat d_{ij}; \hspace{1.5cm} f_{ji}^d = -f_{ij}^d\]
    \end{itemize}
    \item Visco-elastic element force: $f_{ij}^e = f_{ij}^s + f_{ij}^d$
\end{itemize}

\pagebreak

\subsubsection*{Bigger Systems}
Each line represents a viscoelastic element
\begin{center}
    \includegraphics*[scale=0.8]{W5_3.png}
\end{center}

\subsection*{Putting Everything Together}
\begin{itemize}
    \item Particle system parameters
    \begin{itemize}
        \item An array $P[i]$ of particle data structures, each structure containing
        \begin{itemize}
            \item particle mass $m$
            \item particle position $x$
            \item particle velocity $v$
            \item net force $f$ acting on particle
        \end{itemize}
        \item An array $S[k]$ of v-e element data structures, each structure containing
        \begin{itemize}
            \item pointer to particle $i$
            \item pointer to particle $j$
            \item spring constant $k^s$ and natural length $l_{ij}$
            \item damper constant $k^d$
        \end{itemize}
    \end{itemize}
    \item Emvironmental parameters
    \begin{itemize}
        \item Perhaps gravity $g$, other external forces $f_i$ and/or force fields, etc.
    \end{itemize}
    \item Initial conditions: $P[i].x = x_i(0); P[i].v = v_i(0);$
\end{itemize}
\textbf{Simulation loop} (starting from $t = 0$ and initial particle positions and velocities)
\begin{enumerate}
    \item For all particles $i$, initialize force variables $P[i].f = 0$
    \item For all v-e elements $k$, using particles $P[i]$ and $P[j]$ connected by $S[k]$, compute visco-elastic force $f_{ij}^e$; accumulate $f_{ij}^e$ on $P[i]$ and $-f_{ij}^e$ on $P[j]$
    \item For all particles $i$, accumulate all external forces acting on particle $P[i]$
    \item For all particles $i$, update
    \begin{align*}
        P[i].v &:= P[i].v + \Delta t \frac{P[i].f}{P[i].m}\\
        P[i].x &:= P[i].x + \Delta t P[i].v
    \end{align*}
    \item Update $t \leftarrow t + \Delta t$ and go to (1).
\end{enumerate}

\section*{Particle-Based Fluid Flow Simulation}
\subsection*{Lenard-Jones Force Profile}
\begin{center}
    \includegraphics*[scale=0.8]{W5_4.png}
\end{center}

\subsection*{Discrete Fluid Model}
The total force on a particle $i$ due to all other particles:
\[g_i(t) = \sum_{j \neq i} g_{ij}(t)\]
where
\[g_{ij}(t) = m_i m_j \left(\frac{\alpha}{(d_{ij} + \varepsilon)^a} - \frac{\beta}{d_{ij}^b}\right) d_{ij}\]
\begin{itemize}
    \item Factors $\alpha$ and $\beta$ scale the strength of the attraction and repulsion ofrces
    \item Exponents often set to $a = 2$ and $b = 4$
    \item $\varepsilon$ is minimum required separation of particles
\end{itemize}

\subsection*{Heat (Diffusion) Equation}
Diffusion of heat in materials
\[\frac{\partial}{\partial t}(\mu \sigma \theta) - \nabla \cdot (C \nabla \theta) = q\]
\begin{itemize}
    \item $q$: rate of heat generation/loss per unit volume
    \item $\mu$: mass density, kg/m$^3$
    \item $\sigma$: specific heat, Joule/(kg $\cdot$ Kelvin), material property
    \item $\theta$: temperature, Kelvin
    \item $C$: thermal conductivity matrix, material property
    \item $\nabla = \left[\frac{\partial}{\partial u}, \frac{\partial}{\partial v}, \frac{\partial}{\partial w}\right]$
\end{itemize}

\subsection*{Thermoelasticity}
\begin{center}
    \includegraphics*[scale=1]{W5_5.png}    
\end{center}
\begin{itemize}
    \item Spring constant $k_{ij}$ is inversely related to $(\theta_i + \theta_j) / 2$
    \item In the molten state, we can use Lenard-Jones inter-particle forces.
\end{itemize}
\begin{center}
    \includegraphics*[scale=0.8]{W5_6.png}
\end{center}

\subsection*{Time Integration Methods}
\begin{itemize}
    \item Issues
    \begin{itemize}
        \item Stability
        \item Accuracy 
        \item Efficiency
    \end{itemize}
    \item Two main categories
    \begin{itemize}
        \item Explicit Methods
        \item Implicit Methods
    \end{itemize}
\end{itemize}

\subsubsection*{Explicit (Forward) Euler Method}
\begin{itemize}
    \item Simple and fast for integrating First-order ODE
    \[\frac{\der x(t)}{\der t} = v(x(t), t)\]
    \[x(t + \Delta t) = x(t) + \Delta t \cdot v(x(t), t)\]
    \item However,
    \begin{itemize}
        \item Only modest accuracy
        \item Relatively unstable
        \begin{itemize}
            \item If $\Delta t$ is too large, accumulating error makes successive approximations spiral away
        \end{itemize}
    \end{itemize}
\end{itemize}

\subsubsection*{Midpoint Method}
\begin{itemize}
    \item Better accuracy and stability
    \begin{itemize}
        \item Use more accurate central approximations
        \[v\left(t +\frac{\Delta t}{2}\right)\approx \frac{x(t + \Delta t) - x(t)}{\Delta t}\]
        \item Obtain update step
        \[x(t + \Delta t) = x(t) + \Delta t \cdot v \left(x\left(t + \frac{\Delta t}{2}\right), t + \frac{\Delta t}{2}\right)\]
        \item where from 1$^{\text{st}}$-order Taylor expansion
        \[x\left(t + \frac{\Delta t}{2}\right) \approx x(t) + \frac{\Delta t}{2} v(t) = x(t) + \frac{\Delta t}{2} v(x(t), t)\]
    \end{itemize}
\end{itemize}

\subsubsection*{Symplectic Euler Method}
\begin{itemize}
    \item For 2$^{\text{nd}}$-order dynamical systems
    \begin{itemize}
        \item Forward Euler:
        \begin{align*}
            v(t + \Delta t) &= v(t) + \Delta t \frac{1}{m} f(x(t), v(t), t)\\
            x(t + \Delta t) &= x(t) + \Delta t \cdot v(t)
        \end{align*}
        \item Better, Symplectic Euler
        \begin{align*}
            v(t + \Delta t) &= v(t) + \Delta t \frac{1}{m} f(x(t), v(t), t)\\
            x(t + \Delta t) &= x(t) + \Delta t \cdot v(t + \Delta t)
        \end{align*}
    \end{itemize}
\end{itemize}

\subsubsection*{Verlet Integration Method}
\begin{itemize}
    \item Commonly used for 2$^{\text{nd}}$-order dynamical systems
    \begin{itemize}
        \item Forward 1$^{\text{st}}$ difference
        \[v(t) \approx \frac{ x(t + \Delta t) - x(t)}{\Delta t}\]
        \item Backward 1$^{\text{st}}$ difference
        \[v(t - \Delta t) \approx \frac{ x(t) - x(t - \Delta t)}{\Delta t}\]
        \item Centered 2$^{\text{nd}}$ difference
        \[a(t) \approx \frac{v(t) - v(t - \Delta t)}{\Delta t} = \frac{1}{\Delta t^2}(x(t + \Delta t) - 2x(t) + x(t - \Delta t))\]
    \end{itemize}
    \item Rearranging terms:
    \[x(t + \Delta t) = 2x(t) - x(t - \Delta t) + \frac{\Delta t^2}{m} f(x(t), v(t), t)\]
\end{itemize}

\subsection*{Velocity Verlet Method}
\begin{itemize}
    \item (see \url{https://en.wikipedia.org/wiki/Verlet_integration})
    \item Rather than two position vectors, need to store position and velocity vectors
    \begin{itemize}
        \item Assuming $a(t) = \frac{1}{m} f(x(t))$, iterate as follows:
        \begin{align*}
            x(t + \Delta t) &= x(t) + \Delta t v(t) + \frac{\Delta t^2}{2} a(t)\\
            v(t + \Delta t) &= v(t) + \Delta t \frac{a(t) + a(t + \Delta t)}{2}
        \end{align*}
    \end{itemize}
\end{itemize}

\subsection*{Implicit Methods}
\begin{itemize}
    \item Main points
    \begin{itemize}
        \item Implicit methods are unconditionally stable
        \item They require the solution of systems of linear algebraic equations: $Ax = b$
        \begin{itemize}
            \item So, large models can be problematic
        \end{itemize}
        \item Taking time steps $\Delta t$ that are too large \textit{reduces accuracy}
        \begin{itemize}
            \item Seen as excessive damping
        \end{itemize}
    \end{itemize}
\end{itemize}

\end{document}
