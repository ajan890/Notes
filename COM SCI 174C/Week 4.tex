\documentclass{article}
\usepackage{setspace}
\usepackage[utf8]{inputenc}
\usepackage[bb=boondox]{mathalfa}
\usepackage{graphicx}

%%Let's you change margins
\usepackage[left=1in,right=1in,top=1in,bottom=1in]{geometry}

%%Math symbols, proof environments
\usepackage{amsmath,amsthm,amssymb, graphicx, tikz}

%%Use this package for matrices
\usepackage{array}

%%Commands for common sets
\newcommand{\R}{\mathbb{R}} %Real numbers
\newcommand{\Z}{\mathbb{Z}} %Integers
\newcommand{\dP}{\mathbb{P}}
\newcommand{\example}{\textbf{Example: }}
\newcommand{\E}{\mathbb{E}} %Expectation Value

\graphicspath{{./assets/images/Week 4}}

\title{CS 174C Week 4} 

\author{Aidan Jan}

\date{\today}
\begin{document}
\maketitle
\section*{Motion Curves}
\textbf{Examples:}
\begin{itemize}
    \item The position of an object: $x(t), y(t), z(t)$
    \begin{itemize}
        \item 3 separate splines
    \end{itemize}
    \item The angle of a simple joint (e.g., elbow)
    \begin{itemize}
        \item 1 spline
    \end{itemize}
    \item The angles of a complex joint (e.g., hip)
    \begin{itemize}
        \item 2 or more splines
    \end{itemize}
    \item The size of an object
    \begin{itemize}
        \item 1 spline, or maybe a spline for each of 3 different axes
    \end{itemize}
    \item The color of an object
    \begin{itemize}
        \item 3 splines for R, G, B
    \end{itemize}
    \item Etc.
\end{itemize}

\subsection*{Using Motion Curves}
\begin{itemize}
    \item Simplest usage:
    \begin{itemize}
        \item Look at every parameter that changes during the animation
        \item Use Hermite interpolation (initialized as a Catmull-Rom spline) in time
        \item Allow animator to adjust values, adjust slopes, break continuity, add knots, move knots. ...
    \end{itemize}
\end{itemize}

\subsection*{Problem}
\begin{itemize}
    \item Re-timing animations is not so simple
    \begin{itemize}
        \item If you adjust a knot position, it changes the shape of the spline, not just the speed
        \item Particularly for Hermite splines, slopes will be off
    \end{itemize}
    \item Partial solution: Separate the shape of the motion curve fro the timing of the motion along the curve
\end{itemize}

\subsection*{Time as a Motion Curve}
\begin{itemize}
    \item Rename parameter of motion curves to "u"
    \item Then make a motion curve for time: $u(t)$
    \item Could have one global timing curve $u(t)$ or separately adjust timing for each variable, or group of variables.
\end{itemize}

\subsection*{Parameterization}
\begin{itemize}
    \item Unsatisfactory still: $u$ doesn't really have a good meaning
    \item For example, to make an object move with constant speed along an arc, $u(t)$ may need to be complicated
    \item Solution: We will introduce a new map based on arc length $s$
    \begin{itemize}
        \item Can easily control the speed of an object\
        \item Timing curve will now be $s(t)$, where $s$ means distance traveled along the curve
    \end{itemize}
\end{itemize}

\subsection*{Arc Length of a Curve}
\begin{itemize}
    \item Arc length is just the length of a curve
    \begin{itemize}
        \item Think of laying a tape measure along the curve $P(u)$
        \item Where $P(u)$ is the 3D position of the curve at parameter value $u$
        \begin{itemize}
            \item Really three curves: $X(u), Y(u), Z(u)$
        \end{itemize}
    \end{itemize}
    \item Parametric Definition:
    \[s(u) = \int_{u_1}^{u_2} \left\vert \frac{\text{d}P}{\text{d}u}\right\vert\text{d}u\]
    \begin{itemize}
        \item Recall how to measure vector norm:
        \[\left\vert \frac{\text{d}P}{\text{d}u}\right\vert = \sqrt{\left(\frac{\text{d}X}{\text{d}u}\right)^2 + \left(\frac{\text{d}Y}{\text{d}u}\right)^2 + \left(\frac{\text{d}Z}{\text{d}u}\right)^2}\]
    \end{itemize}
\end{itemize}

\subsection*{2D Example}
\begin{itemize}
    \item Compute a differential length
    \begin{center}
        \includegraphics*[scale=0.8]{W4_1.png}
    \end{center}
    \item Integrate to compute the total arc length
\end{itemize} 

\subsection*{Parametric Circle}
\begin{itemize}
    \item Parametric equation of a circle:
    \[[x(t), y(t)] = [R \cos(t), R\sin(t)], \hspace{1cm} t \in [0, 2\pi]\]
    \item Differential arc length:
    \[\text{d}s = \sqrt{\left(\frac{\text{d}x}{\text{d}t}\right)^2 + \left(\frac{\text{d}y}{\text{d}t}\right)^2} \text{d}t\]
    \item Substituting and differentiating:
    \[\text{d}s = \sqrt{R^2 \sin^2(t) + R^2 \cos^2(t)} \text{d}t = R \text{d}t\]
    \item Integrating: 
    \[S(t) = \int_0^t R\text{d}t = tR \vert_0^{2\pi}\]
\end{itemize}

\subsection*{Cubic Polynomial Splines}
\begin{align*}
    P(u) &= \begin{bmatrix} x(u) \\ y(u) \\ z(u) \end{bmatrix} = \begin{bmatrix}a_x u^3 + b_x u^2 + c_x u + d_x \\ a_y u^3 + b_y u^2 + c_y u + d_y \\ a_z u^3 + b_z u^2 + c_z u + d_z \end{bmatrix}  
    S(u) &= \int_{u_1}^{u_2} \sqrt{\left(\frac{x(u)}{\text{d}u}\right)^2 + \left(\frac{y(u)}{\text{d}u}\right)^2 + \left(\frac{z(u)}{\text{d}u}\right)^2} \text{d}u\\
    &= \int_{u_1}^{u_2} \sqrt{Au^4 + Bu^3 + Cu^2 + Du + E}\text{d}u
\end{align*}

\subsection*{Numerical Computation}
\begin{itemize}
    \item Analytic approach is hopeless
    \begin{itemize}
        \item Even analytically solving the integral $s(u)$ is hard, solving for $u$ in terms of $s$ is worse.
    \end{itemize}
    \item Numerical approach
    \item Use approximate evaluation of $s(u)$ to get a table of values
    \begin{itemize}
        \item Cut up curve into small line segments and add up their lengths
    \end{itemize}
    \item Then interpolate a smooth curve through the values (Catmull-Rom)
    \begin{itemize}
        \item Use table of $s$ values as knots and associated $u$ values as control points
    \end{itemize}
\end{itemize}

\subsection*{Forward Differencing}
\begin{center}
    \includegraphics*[scale=0.8]{W4_2.png}
\end{center}

Typically $t_0 = 0$, $t_{i + 1} - t_i = \Delta t$\\
\begin{itemize}
    \item Red line is a linear approximation of the blue curve.
    \item To do this, you would find the position at for example, 30 points, and linearly interpolate them.  The connected line segments approximate the curve.
    \begin{itemize}
        \item Once this is done, create a table that maps the parametric entry $u$ to the arc length on the line, $s$.  Then, when animating, interpolate between the arc lengths.
        \item Example:
            [INSERT TABLE]
    \end{itemize}
\end{itemize}

\subsection*{Calculate $s$ With Rounding}
\begin{itemize}
    \item Given parametric entry find arc length
    \[i = (int)\left(\frac{parametric\:\:entry\:\:(u)}{distance\:\:between\:\:entries} + 0.5\right)\]
    \item For $u = 0.73$
    \[i = (int) \left(\frac{0.73}{0.05}+ 0.5\right) = 15\]
    \item So, arc length = s[i] = s[15] = 0.959
    \item Instead of rounding, interpolate:
    \[i = (int)\left(\frac{parametric\:\:entry\:\:(u)}{distance\:\:between\:\:entries}\right) = 14\]
    \[i + 1 = 15\]
    \begin{center}
        That is $u \in [u[i], u[i + 1]]$
    \end{center}
    \begin{align*}
        S(u) &= S[i] + \frac{u - u[i]}{u[i + 1] - u[i]}(S[i + 1] - S[i])\\
        &= 0.944 + \frac{0.73 - 0.70}{0.75 - 0.70} (0.959 - 0.944)
    \end{align*}
\end{itemize}

\subsection*{How Can We Control the Sampling Error?}
\begin{itemize}
    \item Adaptively subdivide
    \item First compare:
    \begin{itemize}
        \item length(start, middle) + length(middle, end) to length (start, end)
        \item Essentially, sample the points more finely when the line has higher curvature.
    \end{itemize}
    \item If the difference is too large, subdivide
    \item Use linked list to store data
    \item Finally, copy to a fixed-size table
\end{itemize}

\subsection*{Numerical Computation}
\begin{itemize}
    \item Numerical methods exist to approximate the integral of a curve given sample points and derivatives
    \item Instead of using the sum of linear segments, use a numerical method to compute the sum of curved segments
\end{itemize}

\subsection*{Gaussian Quadrature}
\begin{itemize}
    \item Commonly used to integrate a function between -1 and +1
    \begin{itemize}
        \item Function to be integrated is evaluated at specific points within [-1, +1] and is multiplied by pre-calculated weights
        \[\int_{-1}^1 f(u) = \sum_i w_i f(u_i)\]
        \item More sample points lead to better accuracy
    \end{itemize}
    \item Compute arc length of cubic curve
    \begin{itemize}
        \item Adaptive Gaussian integration monitors errors to subsample
        \item Build a table that maps parameter values to arc length values
        \item To solve arc length at $u$, find $u_i$, and $u_{i + 1}$ that bound $u$
        \item Use Gaussian quadrature to compute intermediate arc length between those at $u_i$ and $u_{i + 1}$
    \end{itemize}
\end{itemize}

\subsection*{Inverse Map}
\begin{itemize}
    \item However, the question we really want to answer is what value of $u$ gives us a specific arc length $s$ along the curve $P(u)$?
    \begin{itemize}
        \item i.e., we want to invert the arc length function $s(u)$
        \item Let's call this $u(s)$
    \end{itemize}
    \item Then the timing curve is $s(t)$, which feeds into $u(s)$, which feeds into motion curve $P(u)$:
    \begin{itemize}
        \item Position at time $t$ is $P(u(s(t)))$
    \end{itemize}
    \item Question remains: How to calculate $u(s)$?
\end{itemize}

\subsection*{First Approach}
\begin{itemize}
    \item Use the previous table of samples
    \item Problem?
    \item Arc Length $s$ is not a linear function of $u$
    \begin{itemize}
        \item Given $s$, we must search to find the value of $t$ (index) in the table
    \end{itemize}
\end{itemize}

\subsection*{Computs $u$ Given $s$}
\begin{itemize}
    \item Compute $u$ such that $f(u) = s - \text{length}(u_0, u) = 0$.
    \item This is a classic problem of root finding
    \item Methods?
    \begin{itemize}
        \item Bisection
        \item Newton-Raphson
        \item Secant
    \end{itemize}
\end{itemize}

\subsection*{Bisection Method}
\begin{itemize}
    \item Need two initial guesses that bracket the root
    \begin{itemize}
        \item let $x_c = \frac{1}{2}(x_a + x_b)$
        \item if $f_c = f(x_c) = 0$ then $x = x_c$ is an exact solution: return $x_c$
        \item else, if $f_a \cdot f_c < 0$ then the root lies in the interval $(x_a, x_c)$,
        \item else the root lies in the interval $(x_c, x_b)$.
        \item replace the interval $(x_a, x_b)$ with either $(x_a, x_c)$ or $(x_c, x_b)$, whichever brackets the root
        \item repeat until $f(x_c) < $ threshold
        \item return $x_c$
    \end{itemize}
\end{itemize}
\begin{center}
    \includegraphics*[scale=0.8]{W4_3.png}
\end{center}

\subsection*{Newton-Raphson Method}
\begin{itemize}
    \item Based on Taylor expansion:
    \begin{align*}
        f(x) &= f(x_0) + (x - x_0) f'(x_0) + \frac{(x-x_0)^2}{2} f''(x_0) + \cdots + \frac{(x - x_0)^n}{n!} f^{(n)}(x_0)\\
        f(x) \approx f(x_0) + (x - x_0) f'(x_0) + O(\vert x - x_0 \vert ^3)
    \end{align*}
    \item Set $f(x) = 0$ and solve for $x = x_r$
    \begin{align*}
        0 = f(x_r) &\approx f(x_0) + (x_r - x_0) f'(x_0)\\
        \rightarrow 0 &\approx f(x_0) + (x_r - x_0) f'(x_0)\\
        \rightarrow x_r &\approx x_0 - f(x_0) / f'(x_0)
    \end{align*}
    \item Iteratively converge to the root
    \[x_{n + 1} = x_n - f(x_n) / f'(x_n)\]
\end{itemize}
\begin{center}
    \includegraphics*[scale=0.8]{W4_4.png}
\end{center}

\subsection*{Secant (Chord)}
\begin{itemize}
    \item Same as Newton-Raphson except that the derivative is approximated assets
    \[f'(x_n) = \frac{f(x_n) - f(x_{n - 1})}{x_n - x_{n - 1}}\]
    \begin{itemize}
        \item Substituted in Newton-Raphson, this gives
        \[x_{n + 1} = x_n - \frac{(x_n - x_{n - 1})}{f(x_n) - f(x_{n - 1})} f(x_n)\]
    \end{itemize}
\end{itemize}

\subsection*{Controlling the Speed Along a Curve}
\begin{center}
    \includegraphics*[scale=0.8]{W4_5.png}
\end{center}
\begin{itemize}
    \item \textbf{Arc length is not proportional to time.}
    \item Equal Arc lengths
    \begin{itemize}
        \item Constant speed
    \end{itemize}
    \item Smooth motion
    \begin{itemize}
        \item Easy in - Easy out
        \item Acceleration - Deceleration
    \end{itemize}
    \item Relate time to arc length: easy(t)
\end{itemize}
\subsubsection*{Speed Control}
\begin{itemize}
    \item Given arc-length parameterized curve
    \item Let $s(t)$ be the distance along the curve at time $t$
    \begin{itemize}
        \item Normalize total distance to 1.0
        \item Monotonically increasing and continuous
    \end{itemize}
    \item To ease in/out, make $s(t) = easy(t)$ that looks like this:
    \begin{center}
        \includegraphics*[scale=0.7]{W4_6.png}
    \end{center}
\end{itemize}
\subsection*{Approach 1 for Easy-In/Easy-Out}
\begin{itemize}
    \item Let's use the sine function
    \item Scale so $s$ and $t$ in $[0, 1]$
    \[s(t) = ease(t) = \frac{\sin(\pi t - \frac{\pi}{2}) + 1}{2}\]
\end{itemize}
\subsection*{Approach 2: Slow in, Constant Middle, Slow Out}
\begin{itemize}
    \item Let's use 2 sinusoidal segments and a line
    \begin{itemize}
        \item First part: Sine from $-\pi/2$ to $0$
        \item Second part: line (45 deg. slope)
        \item Third part: Sine from $0$ to $\pi/2$
    \end{itemize}
    \item We can actually use the previous easy-in/easy-out function between $[0, 0.5]$ and $[0.5, 1]$
    \begin{center}
        \includegraphics*[scale=0.7]{W4_7.png}
    \end{center}
    \item We need to scale the timing and the values to ensure continuous first derivatives!
\end{itemize}
\subsubsection*{Ensuring Continuous Derivatives}
\begin{itemize}
    \item Constraints
    \begin{align*}
        S(k_1) &= k_1\\
        S(k_2) &= k_2\\
        S'(k_1+) &= S'(k_1-)\\
        S'(k_2+) = S'(k_2-)
    \end{align*}
\end{itemize}
\subsection*{Approach 3: Parabolic Ease-in/Ease-out}
\begin{itemize}
    \item Sine functions are expensive, so\dots
    \begin{itemize}
        \item Constant Acceleration
        \item Constant intermediate velocity
        \item Constant deceleration
        \begin{center}
            \includegraphics*[scale=0.7]{W4_8.png}
        \end{center}
    \end{itemize}
\end{itemize}

\subsection*{Velocity and Distance}
\begin{itemize}
    \item Velocity:
    \begin{center}
        \includegraphics*[scale=0.8]{W4_9.png}
    \end{center}
    \begin{itemize}
        \item $v = v_0 \frac{t}{t_1}$               $0.0 < t < t_1$
        \item $v = v_0$                          $t_1 < t < t_2$
        \item $v = v_0 (1.0 - \frac{t - t_2}{1 - t_2})$                $t_2 < t < 1$
    \end{itemize}
    \item Distance:
    \begin{itemize}
        \item The distance traveled is the area under the velocity curve
        \[1. = \frac{1}{2} v_0 t_1 + v_0 (t_2 - t_1) + \frac{1}{2} v_0 (1 - t_2)\]
        \item Integrate the velocity to find the distance at time $t$
        \begin{align*}
            d(t) &= v_0 \frac{t^2}{2t_1},\hspace{4.8cm} 0 < t < t_1\\
            d(t) &= v_0 \frac{t_1}{2} + v_0(t - t_1), \hspace{3.1cm} t_1 < t < t_2\\
            d(t) &= v_0 \frac{t_1}{2} + \left(v_0 - \frac{v_0 \frac{t - t_2}{1 - t_2}}{2}\right)(t - t_2), \hspace{0.95cm} t_2 < t < 1
        \end{align*}
    \end{itemize}
\end{itemize}

\subsection*{General Velocity Curves}
\begin{itemize}
    \item Velocity curves can have arbitrary shapes; for example, Splines
    \item If the animator works with velocity Curves
    \begin{itemize}
        \item The curve must maintain the average velocity otherwise the distance changes
    \end{itemize}
\end{itemize}

\section*{Path Following}
\begin{itemize}
    \item What's wrong with this?
\end{itemize}
\begin{center}
    \includegraphics*[scale=0.7]{W4_10.png}
\end{center}
\subsection*{Frenet Frame}
\begin{itemize}
    \item Partial solution: Moving right-handed system along path: $(u, v, w)$
    \begin{center}
        \includegraphics*[scale=0.7]{W4_11.png}
    \end{center}
\end{itemize}
\subsection*{How Do WE Compute $u, v, w$?}
\begin{align*}
    w &= P'(t)                             &\text{tangent along the derivative}\hspace{4.45cm}\\
    u &= P'(t) \times P''(t)               &\text{perpendicular to both tangent }P'(t) \text{ and curvature } P''(t)\\
    v &= w \times u                        &\text{perpendicular to both }w \text{ and }u\hspace{4cm}
\end{align*}
\begin{itemize}
    \item Normalize the vectors!
\end{itemize}

\subsection*{Discontinuous Curvature}
\begin{itemize}
    \item Example: Two semi-circles
    \begin{itemize}
        \item Curvature discontinuous at P
        \begin{center}
            \includegraphics*[scale=0.7]{W4_12.png}
        \end{center}
    \end{itemize}
\end{itemize}

\subsection*{Straight Segments}
\begin{center}
    \includegraphics*[scale=0.7]{W4_13.png}
\end{center}
\begin{itemize}
    \item $P''(t) = 0$
    \item Solution: Interpolate frames before and after straight segments
    \begin{itemize}
        \item They only differ by a rotation around $w$
    \end{itemize}
\end{itemize}

\subsection*{Other Limitations}
\begin{itemize}
    \item Camera motion
    \begin{itemize}
        \item Bumpy ride makes the camera look up / down
        \item Often people look ahead and not along the tangent; e.g., driving along a curve
    \end{itemize}
\end{itemize}

\subsection*{Adding a Center of Interest}
\begin{center}
    \includegraphics*[scale=0.5]{W4_14.png}
    \hspace{1cm}
    \includegraphics*[scale=0.5]{W4_15.png}
\end{center}
$w = COI - P(t)$\\
$u = y-axis \times w$\\
$v = w \times u (\text{parallel to $y$-axis})$
\begin{itemize}
    \item (The textbook has the formulas for negative $w$)
\end{itemize}



\end{document}
