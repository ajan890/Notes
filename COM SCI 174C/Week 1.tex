\documentclass{article}
\usepackage{setspace}
\onehalfspacing
\usepackage[utf8]{inputenc}
\usepackage[bb=boondox]{mathalfa}
\usepackage{graphicx}

%%Let's you change margins
\usepackage[left=1in,right=1in,top=1in,bottom=1in]{geometry}

%%Math symbols, proof environments
\usepackage{amsmath,amsthm,amssymb, graphicx, tikz}

%%Use this package for matrices
\usepackage{array}

%%Commands for common sets
\newcommand{\R}{\mathbb{R}} %Real numbers
\newcommand{\Z}{\mathbb{Z}} %Integers
\newcommand{\dP}{\mathbb{P}}
\newcommand{\example}{\textbf{Example: }}
\newcommand{\E}{\mathbb{E}} %Expectation Value

\title{CS 174C Week 1} 

\author{Aidan Jan}

\date{\today}

\begin{document}
\maketitle
\section*{Animation Basics}
\begin{itemize}
    \item Vision
        \begin{itemize}
            \item Foveal vs. Peripheral vision
        \end{itemize}
    \item Persistence of Vision
        \begin{itemize}
            \item We see a bright flash for a while after it's gone
            \item A sequence of images shown fast enough is hard to distinguish from continuous motion
                \begin{itemize}
                    \item What is fast enough?
                    \item However fast the eye is able to see with fine acuity
                \end{itemize}
        \end{itemize}
    \item Frame rate (fps = frames per second)
        \begin{itemize}
            \item Legacy Film: 24
                \begin{itemize}
                    \item Often shown at 48, each frame twice, to reduce flicker
                    \item Sometimes animated "on twos" = 12fps, "on threes" = 8fps, or even slower
                \end{itemize}
            \item Legacy TV: ~30 for NTSC, 25, for PAL (European, Asian)
                \begin{itemize}
                    \item Interlaced - double the speed to reduce flicker
                \end{itemize}
            \item Computers: 60Hz or more, gamers prefer 60fps, 120fps, ...
        \end{itemize}
\end{itemize}

\section*{Motion Blur}
Since light persists in our vision for a while, fast moving objects leave a blurred streak.  Similarly, film/video cameras leave "shutter" open for a while.
\begin{itemize}
    \item Moving objects blurred from position a start of shutter time to position at end.
\end{itemize}
Without motion blur (or tricks to simulate it) we get strobing effect
\begin{itemize}
    \item Temporal aliasing, akin to "jaggies"
    \item Spinning wheels rotate backwards
    \item Movie Camera vs Stop-Motion
    \begin{itemize}
        \item At the start, stop-motion was used for all movies since it was the best you could do.  (Ex. King Kong)
    \end{itemize}
\end{itemize}
Motion blur is required for "realism".  We use motion blur to "fool" the eye, to make the image as real as possible.

\section*{Animation Principles}
\subsection*{Squash and Stretch}
\textbf{Rigid objects look robotic, so let them deform to make the motion more natural and fluid.}\\
Accounts for the physics of deformation
\begin{itemize}
    \item Think of a tennis ball...
    \item Communicates to viewer what the object is made of, how heavy it is, etc.
    \item Usually large deformations conserve volume: if you squash in one dimension, stretch in another to keep volume constant
\end{itemize}
Also accounts for the persistence of vision
\begin{itemize}
    \item Fast moving objects leave an elongated streak on our retinas
\end{itemize}

\subsection*{Timing}
\begin{itemize}
    \item Pay careful attention to how long an action takes - how many frames
    \item How something moves defines its weight and mood to the audience.
    \item Also think dramatically: give the audience time to understand one event before going to the next, but don't bore them
\end{itemize}

\subsection*{Anticipation}
The preparation before a motion.
\begin{itemize}
    \item E.g., crouching before jumping, pitcher winding up to throw a ball
\end{itemize}
Often physically necessary, and indicates how much effort a character is making.\\\\
Also essential for controlling the audience's attention, to make sure they don't miss the action.
\begin{itemize}
    \item Signals something is about to happen, and where it is going to happen
\end{itemize}

\subsection*{Staging}
\begin{itemize}
    \item Make the action clear
    \item Avoid confusing the audience by having two or more things happen at the same time
    \item Select a camera viewpoint, and pose the characters, so that visually you can't mistake what is going on
    \begin{itemize}
        \item Clear enough so you can tell what's happening just from the silhouettes (highest contrast)
    \end{itemize}
\end{itemize}

\subsection*{Follow-through and Secondary Motion}
Again, physics demands follow-through - the inertia that is carried over after an action
\begin{itemize}
    \item E.g., knees bending after a jump
    \item Also helps define weight, rigidity, etc.
\end{itemize}
Secondary motion is movement that's not part of the main action, but is physically necessary to support it.
\begin{itemize}
    \item E.g., arms swinging in a jump
\end{itemize}
Just about everything should always be in motion.
\begin{itemize}
    \item Animator has to give the audience an impression of reality, or things look stilted and rigid.
\end{itemize}

\subsection*{Overlapping Actions and Asymmetry}
\textbf{Overlapping action:} start the next action before the current one finishes
\begin{itemize}
    \item Otherwise looks scripted and robotic instead of natural and fluid
\end{itemize}
\textbf{Asymmetry:} natural motion is rarely exactly the same on both sides of the body, or for 2 or more characters
\begin{itemize}
    \item People very good at spotting "twins", synchronization, etc.
    \item Break up symmetries to avoid scripted or robotic feel
\end{itemize}

\subsection*{Slow In and Out}
Also called "Easing in" and "Easing out"
\begin{itemize}
    \item More physics: objects generally smoothly accelerate and decelerate, depending on mass and forces
\end{itemize}

\subsection*{Arcs}
Natural motions tend not to be in straight lines, instead should be curved arcs
\begin{itemize}
    \item Just doing straight-line interpolation gives weird, robotic movement
\end{itemize}
Also part of physics
\begin{itemize}
    \item gravity causes parabolic trajectories
\end{itemize}

\subsection*{Exaggeration}
Obvious in the old Loony Tunes cartoons - "cartoon physics"
\begin{itemize}
    \item Not so obvious, but necessary ingredient in photo-realistic special effects
    \item If you're too subtle, even if that is accurate, the audience will miss it: confusing and boring
    \item Think of stage make-up, movie lighting, and other "photo surrealistic" techniques
\end{itemize}
Don't worry about being physically accurate: convey the correct psychological impression as effectively as possible.

\subsection*{Appeal}
\begin{itemize}
    \item Make animations that people enjoy watching
    \item Appealing characters aren't necessarily attractive, just well designed and rendered
    \begin{itemize}
        \item All the principles of art still apply to each still frame
        \item E.g., controlling symmetry - avoid "twins", avoid needless complexity
    \end{itemize}
    \item Present scenes that are clear and communicate the story effectively.
\end{itemize}

\subsection*{Straight-Ahead vs. Pose-to-Pose}
\begin{itemize}
    \item "Straight Ahead" means making one frame after the other
    \begin{itemize}
        \item Especially suited for rapid, unpredictable motion
    \end{itemize}
    \item "Pose-to-Pose" means planning it out, making "key frames" of the most important poses, then interpolating in between the key frames later
    \begin{itemize}
        \item The typical approach for most scenes
    \end{itemize}
\end{itemize}

\subsection*{Extremes}
Keyframes are also called "Extremes", since they usually define the extremes positions of a character.
\begin{itemize}
    \item The frames in between (or "inbetweens") introduce relatively little new - watching the keyframes should reveal the action
    \item May add additional keyframes to add some interest and/or better control the interpolated motion
    \item E.g., for a sit-to-stand animation:
    \begin{itemize}
        \item Sitting
        \item Pushing off
        \item Straighten up
    \end{itemize}
\end{itemize}
 
\subsection*{Layering}
\begin{itemize}
    \item Work out the big picture first 
    \begin{itemize}
        \item e.g., where the characters need to be and when
    \end{itemize}
    \item Then layer by layer add more details
    \begin{itemize}
        \item Which way the characters face
        \item Move their limbs and heads
        \item Move their fingers and faces
        \item Add small details like wrinkles in clothing, hair, etc.
    \end{itemize}
\end{itemize}
    

\section*{Computer Animation}
\begin{itemize}
    \item The task boils down to setting various animation parameters in each frame
    \item Can mix the straight-ahead and pose-to-pose methods
    \begin{itemize}
        \item Keyframe some variables, do others straight-ahead.
    \end{itemize}
\end{itemize}
\end{document}