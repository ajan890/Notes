\documentclass{article}
\usepackage{setspace}
\usepackage[utf8]{inputenc}
\usepackage[bb=boondox]{mathalfa}
\usepackage{graphicx}
\usepackage{hyperref}

%%Let's you change margins
\usepackage[left=1in,right=1in,top=1in,bottom=1in]{geometry}

%%Math symbols, proof environments
\usepackage{amsmath,amsthm,amssymb, graphicx, tikz}

%%Use this package for matrices
\usepackage{array}

%%Commands for common sets
\newcommand{\R}{\mathbb{R}} %Real numbers
\newcommand{\Z}{\mathbb{Z}} %Integers
\newcommand{\dP}{\mathbb{P}}
\newcommand{\example}{\textbf{Example: }}
\newcommand{\E}{\mathbb{E}} %Expectation Value

\newcommand{\der}{\text{d}}

\graphicspath{{./assets/images/Week 6}}

\title{CS 174C Week 6} 

\author{Aidan Jan}

\date{\today}
\begin{document}
\maketitle
\section*{Collision Detection and Response}
\subsection*{Aside: Object Geometry}
\textbf{When considering particles colliding with objects, we need to know how to represent objects, and how to answer:}
\begin{itemize}
    \item Is a particle inside/outside an object?
    \item Does the particle trajectory cross into the object?
    \item What is the normal vector at some point on the object's surface?
    \item What is the distance/direction to the surface in space?
\end{itemize}
\textbf{Standard geometric representations:}
\begin{itemize}
    \item Special geometries: plane, sphere, cylinder, prism, \dots
    \item Height fields: $z = h(x, y)$
    \item Triangle meshes - closed or open
    \item Implicit functions: $f(x, y, z) = 0$
\end{itemize}
\subsection*{Ground Collisions}
\subsubsection*{Collisions With a Plane}
\textbf{Represent the plane with a point $P$ on the plane, and the (outward) normal $n$ of the plane}
\begin{itemize}
    \item Often simply $P = [0, 0, 0]^T$, $n = [0, 1, 0]^T$ - the ground plane
    \item Particle at position $x$ is "inside" plane if $(x - P) \cdot n < 0$
    \item Trajectory crosses if $(x - P) \cdot n$ changes sign
    \item Distance to surface:
    \begin{itemize}
        \item if $n$ is unit length, $(x - P) \cdot n$ is the "signed distance"
        \begin{itemize}
            \item $vert ( - P) \cdot n \vert$ is the regular distance
        \end{itemize}
        \item $n$ or $-n$ is direction to the closest point on the surface
        \begin{itemize}
            \item $-n$ if $x$ is "outside"
        \end{itemize}
    \end{itemize}
\end{itemize}

\subsubsection*{Collisions With a Sphere}
\textbf{Represent the sphere with a center point $C$ and a radius $r$}
\begin{itemize}
    \item Particle at position $x$ is inside if $\Vert x - C \Vert - r < 0$
    \item Trajectory cross is complicated
    \begin{itemize}
        \item Need to solve quadratic equation for intersection of straight line trajectory\dots
    \end{itemize}
    \item Outward object normal is $(x - C) / r$
    \item Signed distance is $\vert x - C \vert - r$
    \item Direction to closest point on surface is $\pm(x - C) / \vert x - C \vert$
    \begin{itemize}
        \item Sign depends on whether inside or outside
        \item Beware of divide by zero at $x = C$
        \item Note: matches up with object normal again!
    \end{itemize}
\end{itemize}

\subsubsection*{Collisions With Height Fields}
\textbf{Especially good for terrain - a 2D array of heights}
\begin{itemize}
    \item Maybe stored as an image
    \begin{itemize}
        \item i.e., a displacement map from a plane
    \end{itemize}
\end{itemize}
\textbf{Split up plane into triangles}
\begin{itemize}
    \item Particle inside:
    \begin{itemize}
        \item Figure out which triangle $(x, y)$ belongs to, check $z$ against equation of triangle's plane
    \end{itemize}
    \item Trajectory cross (for a stationary height field):
    \begin{itemize}
        \item Check all triangles along path (use 2D line-drawing algorithm to figure out which cells to check)
    \end{itemize}
    \item Object normal: get from the triangle
    \item Distance, etc.: not so easy, but vertical distance is easy for shallow height fields
\end{itemize}

\subsubsection*{Collisions With Triangle Meshes}
\textbf{For any decent sized mesh, will need to use an acceleration structure}
\begin{itemize}
    \item Could use background (hash-)grid, octree, kd-tree
    \item Can also use bounding volume (BV) hierarchy
    \begin{itemize}
        \item Spheres, axis-aligned bounding boxes, oriented bounding boxes, polytopes, \dots
    \end{itemize}
    \item More exotic structures exist
\end{itemize}
\textbf{Particle inside (for a closed mesh):}
\begin{itemize}
    \item Shoot a ray out to infinity and count the number of crossings
\end{itemize}
\textbf{Trajectory cross (for a stationary mesh):}
\begin{itemize}
    \item For each candidate triangle (from acceleration structure), check a sequence of determinants
\end{itemize}

\section*{Collision Resolution}
\begin{itemize}
    \item We can now detect collisions.  Now what?
\end{itemize}
\subsection*{Repulsion Velocity Fields}
\textbf{How do we create a repulsion velocity field?}
\begin{itemize}
    \item $v(x) = f(\text{distance}(x))n(x)$
    \begin{itemize}
        \item $n(x)$ is the outward unit normal to surface at $x$
        \item $f$ is some function that monotonically decreases to zero
        \begin{itemize}
            \item Exponential $f(d) = e^{-kd}$
            \item Linear drop, truncated to zero: $f(d) = \text{max}(0, m - kd)$
            \item Or more complicated
        \end{itemize}
        \item Outward direction is plus/minus direction to closest point
    \end{itemize}
\end{itemize}
\textbf{Aside: useful for more than just collisions}
\begin{itemize}
    \item e.g., firing particles streaming out of an object
\end{itemize}

\subsection*{Repulsion Force Fields}
\textbf{Can do exactly the same trick for force-based motion}
\begin{itemize}
    \item Add repulsion field to $f(x)$
\end{itemize}
\textbf{Simple, often works, but there are sometimes problems}
\begin{itemize}
    \item What are you trying to model?
    \item Robustness - high velocity impacts can penetrate arbitrarily far
    \begin{itemize}
        \item High velocity impacts may go straight through thin objects!
    \end{itemize}
    \item How much of a rebound do you want?
\end{itemize}

\subsection*{Penalty Methods}
\textbf{Penalize Penetration:}
\begin{center}
    \includegraphics*[scale=0.85]{W6_1.png}
\end{center}
\begin{itemize}
    \item Springs and damers to the rescue!
    \begin{itemize}
        \item Attach a zero-length "virtual" viscoelastic element at the point of entry
    \end{itemize}
\end{itemize}
\textbf{For a Plane:}
\begin{center}
    \includegraphics*[scale=0.8]{W6_2.png}
\end{center}
\begin{itemize}
    \item Normal force:
    \[f_n = k_s \left((P_g - x(t)) \cdot \hat n\right) \hat n - k_d(v(t) \cdot \hat n) \hat n\]
    \[f_n \cdot \hat n > 0\]
\end{itemize}

\subsection*{Spring and Damping Constants}
\textbf{How do you come up with reasonable values for spring constants and damping constants?}
\begin{itemize}
    \item And how do you pick good step sizes for differential equation solver (Symplectic Euler, etc.)
\end{itemize}
\textbf{Look at 1-dimensional, simplified model}
\begin{itemize}
    \item $ma = F = -Kx - Dv$
    \item where\dots
    \begin{itemize}
        \item $m$ is the mass, $a$ is the acceleration, $F$ is the force
        \item $K$ is the spring constant, $D$ is the damping constant
    \end{itemize}
    \item Can solve it analytically.
\end{itemize}

\subsubsection*{Aside: Underdamped}
\[D^2 - 4MK < 0\]
\begin{itemize}
    \item Oscillation with frequency:
    \[\omega \sim \frac{1}{2\pi} \sqrt{\frac{K}{M}}\]
    \item Characteristic time:
    \[t \sim 2\pi \sqrt{\frac{M}{K}}\]
    \item Exponentially decays at rate:
    \[r = - \frac{D}{2M}\]
    \item Characteristic time:
    \[t \sim \frac{2M}{D}\]
\end{itemize}

\subsubsection*{Aside: Overdamped}
\[D^2 - 4MK > 0\hspace{1.5cm}D = 2\sqrt{MK}\]
\begin{itemize}
    \item No continued oscillation
    \item Fastest decay possible at rate:
    \[r = -\frac{D}{2M}\]
    \item Characteristic time:
    \[t \sim \frac{2M}{D}\]
\end{itemize}

\subsubsection*{Aside: Critically Damped}
\[D^2 - 4MK = 0\]
\begin{itemize}
    \item No continued oscillation
    \item Exponentially decays at rates: 
    \[r \sim -\frac{K}{D}, -\frac{D}{M}\]
    \item Characteristic times:
    \[t \sim \frac{D}{K}, \frac{M}{D}\]
\end{itemize}

\subsection*{Numerical Time Steps}
\textbf{Should be proportional to minimum characteristic time}
\begin{itemize}
    \item Implicit time integration methods like Backward Euler actually let you take larger steps with stability, but kill all hope of accuracy for systems with small charateristic time
\end{itemize}
\textbf{For nonlinear multi-dimensional forces, what are $K$ and $D$?}
\begin{itemize}
    \item Estimate them by figuring out what is the fastest $\vert F \vert$ can change if you modify $x$ or $v$ respectively
    \item This is all very approximate, so don't get hung up on getting the "right" answer
    \item Anyhow, will ultimately need a "fudge factor" (from trial-and-error experiments)
\end{itemize}

\pagebreak
\section*{Friction}
\textbf{Coulomb friction:} $f_\text{f} \leq \mu f_\text{n}$
\begin{center}
    \includegraphics*[scale=0.8]{W6_3.png}
\end{center}
\begin{itemize}
    \item Coefficient of friction $\mu$
    \item $f_f$ opposes $f_a$:
    \[f_f \leq -\mu \Vert f_n\Vert \frac{f_a}{\Vert f_a \Vert}\]
    \item Traction: $f_f = -f_a$ until $f_f > f_T$\dots, then motion occurs
\end{itemize}

\subsection*{Friction Models}
\textbf{Static (Coulomb) friction:} $\Vert f_s \Vert = \mu_s \Vert f_n \Vert$
\begin{itemize}
    \item $f_n = -(f \cdot n) n$ is the normal component of the force f
    \item Must exceed static friction for object to start moving
\end{itemize}
\textbf{For object in motion,}
\begin{itemize}
    \item Kinetic (constant) friction model: $f_k = -\mu_k \Vert f_n\Vert \frac{v_t}{\Vert v_t \Vert}$
    \begin{itemize}
        \item Tangential component of velocity: $v_t = v - v_n$
        \item Normal component of velocity: $v_n = (v \cdot n)n$
    \end{itemize}
    \item Viscous (linear) friction model: $f_v = -\mu_k \Vert f_n \Vert v_t$
    \begin{itemize}
        \item i.e., a drag force that acts tangentially with magnitude proportional to the tangential velocity and normal force
    \end{itemize}
\end{itemize}
\pagebreak
\section*{Collisions}
\subsection*{Impulsive Collisions}
\textbf{Model collision as a discrete event - a bounce}
\begin{itemize}
    \item Input: incoming velocity, object normal
    \item Output: outgoing velocity
\end{itemize}
\textbf{Need some idea of how "elastic" the collision}
\begin{itemize}
    \item Fully elastic - reflection
    \item Fully inelastic - sticks (or slides)
\end{itemize}

\subsection*{Newtonian Collisions}
\textbf{For normal vector $n$, compute the components of the velocity $v$}
\begin{itemize}
    \item Normal component $v_N = (v \cdot n)n$
    \item Tangential component $v_T = v - v_N$
    \item Reflect normal component to obtain rebound velocity with coefficient of restitution $r$
    \[v_{new} = v_t - r v_n\]
\end{itemize}

\subsection*{Relative Velocity in Collisions}
\textbf{What if the particle hits a moving object?}
\begin{itemize}
    \item Now process collisions in terms of the relative velocity
    \begin{itemize}
        \item $v_{\text{ref}} = v_{\text{particle}} - v_{\text{object}}$
        \item Resolve normal and tangential components of the relative velocity
        \item Reflect normal part appropriately to get new $v_{\text{rel}}$
        \item Then new $v_{\text{particle}} = v_{\text{object}} + v_{\text{rel}}^{\text{new}}$
    \end{itemize}
\end{itemize}


\end{document}


