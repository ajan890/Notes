\documentclass{article}
\usepackage{setspace}
\usepackage[utf8]{inputenc}
\usepackage[bb=boondox]{mathalfa}
\usepackage{graphicx}
\usepackage{hyperref}
\usepackage{listings}

%%Let's you change margins
\usepackage[left=1in,right=1in,top=1in,bottom=1in]{geometry}

%%Math symbols, proof environments
\usepackage{amsmath,amsthm,amssymb, graphicx, tikz}

%%Use this package for matrices
\usepackage{array}

%%Commands for common sets
\newcommand{\R}{\mathbb{R}} %Real numbers
\newcommand{\Z}{\mathbb{Z}} %Integers
\newcommand{\dP}{\mathbb{P}}
\newcommand{\example}{\textbf{Example: }}
\newcommand{\E}{\mathbb{E}} %Expectation Value

\newcommand{\der}{\text{d}}

\graphicspath{{./assets/images/Week 7}}

\title{CS 174C Week 7} 

\author{Aidan Jan}

\date{\today}
\begin{document}
\maketitle
\section*{Articulated-Body Kinematics}
\begin{itemize}
    \item When animating characters, their movements should look realistic.  Joints must bend in the correct directions, and all the parts should be animated together, not separately.
\end{itemize}

\subsection*{Hierarchies}
\begin{itemize}
    \item Hierarchies are used in order to determine how objects are linked together.
    \item Each individual object is referred to as a node, and one object is designated as a root node.
    \item From here, how objects are linked can be drawn as a graph, similar to a file directory.
\end{itemize}
Example: \href{https://www.youtube.com/watch?v=q54fbMvl3GE}{Building a Robot}
\begin{center}
    \includegraphics*[scale=0.3]{W7_1.png}
\end{center}
\section*{Joints}
\begin{itemize}
    \item A 2-rigid-body system has $2 \cdot 6 = 12$ degrees of freedom (DOF)
    \item Joints are essentially constraints that remove degrees of freedom
    \begin{itemize}
        \item Implicitly (through forces)
        \item Explicitly (through parameterization)
    \end{itemize}
\end{itemize}
\subsection*{Slider Joints}
\begin{itemize}
    \item 1 Degree of freedom
    \begin{itemize}
        \item One translational, defined by the axis
    \end{itemize}
\end{itemize}
\begin{center}
    \includegraphics*[scale=0.8]{W7_2.png}
\end{center}

\subsection*{Hinge Joints}
\begin{itemize}
    \item 1 Degree of freedom
    \begin{itemize}
        \item One rotational, defined by axis and anchor point
    \end{itemize}
\end{itemize}
\begin{center}
    \includegraphics*[scale=0.8]{W7_3.png}
\end{center}

\subsection*{Hinge2 Joints}
\begin{itemize}
    \item 2 Degrees of freedom
    \begin{itemize}
        \item Two rotational, defined by axis 1, axis 2, and anchor
    \end{itemize}
\end{itemize}
\begin{center}
    \includegraphics*[scale=0.8]{W7_4.png}
\end{center}

\subsection*{Universal Joints}
\begin{itemize}
    \item 2 Degrees of freedom
    \begin{itemize}
        \item Two rotational, defined by axis 1, axis 2, and anchor
    \end{itemize}
\end{itemize}
\begin{center}
    \includegraphics*[scale=0.8]{W7_5.png}
\end{center}

\subsection*{Ball and Socket Joints}
\begin{itemize}
    \item 3 Degrees of freedom
    \begin{itemize}
        \item Three rotational, defined by anchor point
        \item Usually represented as a quaternion or exponential map
    \end{itemize}
\end{itemize}
\begin{center}
    \includegraphics*[scale=0.8]{W7_6.png}
\end{center}

\subsection*{Planar Joints}
\begin{itemize}
    \item Point confined to move on a plane
    \begin{itemize}
        \item Can be used to model non-penetration constraints
    \end{itemize}
\end{itemize}
\begin{center}
    \includegraphics*[scale=0.8]{W7_7.png}
\end{center}

\subsection*{Free Joints}
\begin{itemize}
    \item 6 Degrees of freedom
    \begin{itemize}
        \item Three translational
        \item Three rotational
    \end{itemize}
    \item Example: Free joint between root object and ground
\end{itemize}
\begin{center}
    \includegraphics*[scale=0.8]{W7_8.png}
\end{center}

\subsection*{Example - SCARA Robotic Arm}
\begin{center}
    \includegraphics*[scale=0.8]{W7_9.png}
\end{center}

\section*{Parametric Representation of a Human Character}
\begin{itemize}
    \item Local Coordinate systems
    \begin{itemize}
        \item Frames
    \end{itemize}
    \item Child links can move with respect to their parent links
    \begin{itemize}
        \item Links are jointed by transformation matrices
    \end{itemize}
\end{itemize}
\begin{center}
    \includegraphics*[scale=0.6]{W7_10.png}
    \includegraphics*[scale=0.6]{W7_11.png}
\end{center}


\section*{Forward Kinematics (FK) - Determine coordinates of a point $P$}
In 2D: From Link 2 Frame to the "World Frame"
\begin{align*}
    v_w &= T_0 R(\theta_0) v_0\\
    v_w &= T_0 R(\theta_0) T_1 R (\theta_1) v_1\\
    v_w &= T_0 R(\theta_0) T_1 R (\theta_1) T_2 R (\theta_2) v_2
\end{align*}
\begin{center}
    \includegraphics*[scale=0.5]{W7_12.png}
\end{center}

In 3D: 4x4 Rotation matrices:
\begin{itemize}
    \item $R_i = R(\theta_x, \theta_y, \theta_z)$ or $R_i = F(quaternion)$
\end{itemize}
\begin{align*}
    v_w &= T_0 R_0 v_0\\
    v_w &= T_0 R_0 T_1 R_1 v_1\\
    v_w &= T_0 R_0 T_1 R_1 T_2 R_2 v_2
\end{align*}

Let $_{i - 1}M_i = T_i R_i$.  Then, $v_w = _w M_0 \:_0M_1 \:_1M_2 v_2$.

\textbf{In general, for a chain of n links:}
\begin{align*}
    &P_{i - 1} = _{i - 1}M_i P_i &P_i = \:_{i - 1}M^{-1}_i P_{i - 1} &\:\\
    &P_w = \left(\prod_{i = 0}^n \:_{i - 1} M_i\right) P_n &P_n = \left(\prod_{i = 0}^n \:_{i - 1} M_i\right)^{-1} P_w &\:
\end{align*}
where $i - 1 = w$ for $i = 0$.\\
If there is no scaling and shearing:
\[M = \begin{bmatrix}R_{3 \times 3} & T_{3 \times 1} \\ 0_{1 \times 3} & 1\end{bmatrix}\]

\section*{Articulated Model Data Structures}
\begin{itemize}
    \item Node:
    \begin{itemize}
        \item dataPtr: Data (possibly shared by other nodes) that represent the geometry of this part of the figure
        \item Tmatrix: Matrix to transform the node data into position to be articulated (e.g., put the point of rotation at the origin)
        \item arcPtr: Pointer to a single child Arc
    \end{itemize}
    \item Arc
    \begin{itemize}
        \item nodePtr: Pointer to a node holding data to be articulated by the arcPtr
        \item Lmatrix: Matrix that locates the following (child) node relative to the previous (parent) node
        \item Amatrix: Matrix that articulates the node data; this is the matrix that is changed in order to (animate) articulate the linkage
        \item arcPtr: Pointer to a sibling arc (another child of this arc's parent node; this is NULL if there are no more siblings)
    \end{itemize}
\end{itemize}

\section*{Evaluation of an Articulated Model}
\begin{itemize}
    \item By a depth-first traversal from root to leaf nodes of the model hierarchy tree
    \begin{itemize}
        \item Traverse from root node to leaf nodes
        \item Backtrack up the tree until unexplored arc
        \item Traversing arc down: Concatenate transform to that of parent node
        \item Traversing arc up: Restore transform
    \end{itemize}
    \item Implemented as a stack of transformations with push (down) and pop (up) operations
\end{itemize}

Example code:
\begin{center}
    \includegraphics*[scale=0.9]{W7_13.png}
\end{center}

\section*{Inverse Kinematics (IK)}
Given the DOFs, e.g., $q = [T\:\: R\:\: \theta]$, compute the position of any point of interest (e.g., the end effector)
\begin{itemize}
    \item $x = f(q)$
    \item \texttt{traverse(rootArcPtr, I);}, where \texttt{I} is the identity matrix
\end{itemize}

\subsection*{A Simple (Analytic) Example}
Direction IK solution, given $p = (p_x, p_y)$
\begin{itemize}
    \item Solve for $\theta = (\theta_1, \theta_2)$
\end{itemize}
\begin{center}
    \includegraphics*[scale=0.6]{W7_14.png}
\end{center}

\textbf{Problems:}
\begin{itemize}
    \item Multiple Solutions
    \item Unreachable goals
    \item Most structures are too complex to solve analytically
\end{itemize}

\section*{Aside: The Jacobian}
\begin{itemize}
    \item Derivative of a one-variable scalar function
    \[y = f(x) \rightarrow \frac{\der f}{\der x} = \lim_{x \rightarrow 0} \frac{\Delta y}{\Delta x} = \frac{\delta y}{\delta x} \rightarrow \delta y = \frac{\partial f}{\partial x} \delta x\]
    \item Extension to multivariable vector functions
    \[y = F(x) = \begin{pmatrix}y_1 \\ y_2 \\ y_3 \end{pmatrix} = \begin{pmatrix}f_1(x_1, x_2, x_3, x_4) \\ f_2 (x_1, x_2, x_3, x_4) \\ f_3 (x_1, x_2, x_3, x_4)\end{pmatrix}\]
    \[\delta y_i = \frac{\partial f_i}{\partial x_1} \delta x_1 + \frac{\partial f_i}{\partial x_2} \delta x_2 + \frac{\partial f_i}{\partial x_3} \delta x_3 + \frac{\partial f_i}{\partial x_4} \delta x_4, \quad i = 1, 2, 3\]
    \item Jacobian Matrix
    \[J = \frac{\partial F}{\partial x} = \begin{bmatrix} 
        \frac{\partial f_1}{\partial x_1} & \frac{\partial f_1}{\partial x_2} & \frac{\partial f_1}{\partial x_3} & \frac{\partial f_1}{\partial x_4} \\
        \\
        \frac{\partial f_2}{\partial x_1} & \frac{\partial f_2}{\partial x_2} & \frac{\partial f_2}{\partial x_3} & \frac{\partial f_2}{\partial x_4} \\
        \\
        \frac{\partial f_3}{\partial x_1} & \frac{\partial f_3}{\partial x_2} & \frac{\partial f_3}{\partial x_3} & \frac{\partial f_3}{\partial x_4}
    \end{bmatrix}\]
    \item This gives a linear mapping between instantaneous velocities
    \item At any instant in time, the Jacobian provides a linear mapping between the velocities in the neighborhood of x: $y' = J(x) \cdot x'$
    \item The Jacobian is a function of x.
    \item It enables us to linearize $y(x)$ with respect to $x$ around the neighborhood of $x$.
    \[\delta y = J(x) \delta x \Rightarrow \Delta y \approx J\Delta x\]
    \[y(x) \approx y(x_0) + J(x_0)(x - x_0)\]
\end{itemize}

\section*{Back to Inverse Kinematics}
\begin{itemize}
    \item Move end effector to a given goal position $P_g$
    \item Error vector: $E = P - P_g$
    \begin{center}
        \includegraphics*[scale=0.8]{W7_15.png}
    \end{center}
    \item We want the error $E(\theta) = P - P_g$ to be 0.
\end{itemize}

\subsection*{Newton's Method}
\begin{center}
    \includegraphics*[scale=0.8]{W7_16.png}
\end{center}

\subsection*{For the Simple (Analytic) Example}
Two-link planar arm:
\begin{center} 
    \includegraphics*[scale=0.8]{W7_17.png}
\end{center}

\section*{The More General Formulation}
\begin{itemize}
    \item End effector position and orientation
    \begin{itemize}
        \item $x = f(\theta)$ where:
        \begin{itemize}
            \item $x = [p_x, p_y, p_z, \theta_x, \theta_y, \theta_z]^T$
            \item $\theta = [\theta_1, \dots, \theta_n]^T$
        \end{itemize}
    \end{itemize}
    \item End effector velocity
    \begin{itemize}
        \item $x' = [v_x, v_y, v_z, \omega_x, \omega_y, \omega_z]$
    \end{itemize}
    \item Joint velocities
    \begin{itemize}
        \item $\theta' = [\theta_1', \dots, \theta_n']^T$
    \end{itemize}
    \item Velocity relationship
    \begin{itemize}
        \item $x' = J(\theta) \theta'$, where $J = \frac{\partial f}{\partial x}$ is a $6 \times n$ matrix.
    \end{itemize}
\end{itemize}

\subsection*{Inverting the Jacobian}
For inverse kinematics, we ideally need $\theta' = J^{-1}(\theta)x'$.  However, for $m$ functions and $n$ DOFs, $J_{m \times n}$ is not square
\begin{itemize}
    \item $J^{-1}$ is not defined
    \item We use the pseudoinverse $J^+$
\end{itemize}
For full rank matrices:
\begin{itemize}
    \item $m > n$: $J^+_ = (J^T J)^{-1} J^T$  Overconstrained, minimizes $\Vert J\theta' - x'\Vert$ 
    \item $m < n$: $J^+ = J^T(JJ^T)^{-1}$  Underconstrained, minimuzes $\Vert \theta' \Vert$
    \begin{itemize}
        \item For rank deficient matrices, use the SVD or other methods
    \end{itemize}
\end{itemize}

\subsection*{Secondary Tasks}
\begin{itemize}
    \item Obstacle Avoidance
    \item Joint limit constraints
    \item Singularity Avoidance
    \begin{itemize}
        \item Pseudoinverse is unstable around singularities: $\det(J) = 0$
    \end{itemize}
\end{itemize}

\subsection*{Adding Control}
How can we bias the angular velocities without affecting the end effector velocity?
\begin{itemize}
    \item Consider the control expression
    \[\theta' = (J^+ J - I)z\]
    \item Since $x' = J\theta'$
    \[x' = J(J^+ J - I) z = (JJ^+J - J)z = (J - J)z = 0\]
    \item Therefore, $z$ does not change the end effector velocity
\end{itemize}
What is $z$?
\begin{itemize}
    \item Let $\theta_i^c$ be preferred angles at each joint
\end{itemize}
\[H = \frac{1}{2} \sum_{i = 1}^n \alpha_i(\theta_i - \theta_i^c)^2\]
\[z = \nabla_\theta H = [\alpha_1(\theta_1 - \theta_1^c), \dots, \alpha_n(\theta_n - \theta_n^c)]^T\]
\begin{itemize}
    \item Where $\alpha_i$ are the gains
\end{itemize}
How is $z$ used?
\begin{itemize}
    \item System
    \begin{align*}
        \theta' &= J^+ x' + (J^+J - I) \nabla_\theta H\cdots\\
        \theta' &= J^T[(JJ^T)^{-1}(x' + J\nabla_\theta H)] - \nabla_\theta H
    \end{align*}
    \item Set
        \[\beta = (JJ^T)^{-1}(x' + J\nabla_\theta H)\]
    \item Solve for $\beta$
    \item Then, controlled angle velocities are:
    \begin{align*}
        \theta' &= J^T\beta - \nabla_\theta H\\
        x' &= (JJ^T)\beta - J\nabla_\theta H
    \end{align*}
\end{itemize}

\subsection*{Jacobian Transpose Method}
\begin{itemize}
    \item Use the transpose of the Jacobian matrix rather than the pseudoinverse
    \begin{itemize}
        \item Rather than: $\Delta \theta = J^+ \Delta x$
        \item Find $\Delta \theta$ by: $\Delta \theta = J^T \Delta x$
    \end{itemize}
    \item Avoids expensive inversion
    \item Avoids singularity problems
\end{itemize}
But why does it work?\\\\

\textbf{Principal of Virtual Work}
\begin{itemize}
    \item Virtual because amount is infinitesimal
    \item Work = force $\times$ distance
    \item Work = torque $\times$ angle
    \begin{center}
        \includegraphics*[scale=0.9]{W7_18.png}
    \end{center}
\end{itemize}

\subsection*{Jacobian Transpose Method}
\begin{itemize}
    \item The good and bad of $J^T$
    \item Pros:
    \begin{itemize}
        \item Cheaper evaluation step than computing pseudoinverse
        \item No singularities
    \end{itemize}
    \item Cons:
    \begin{itemize}
        \item Slower to converge than $J^+$
        \item Scaling problems
        \begin{itemize}
            \item J' has nice property that the solution has minimal norm at every step
            \item $J^T$ doesn't have this property.  Joints far from the end effector experience larger torques, hence take larger steps
            \item Can introduce a constant diagonal scaling matrix to counteract some scaling problems: $\der\theta/\der t = KJ^TF(\theta)$
        \end{itemize}
    \end{itemize}
\end{itemize}

\subsection*{Cyclic Coordinate Descent}
A simple idea:
\begin{itemize}
    \item Solve 1-DOF IK problems repeatedly up the chain
    \item 1-DOF problems are simple and have analytical solutions
    \begin{center}
        \includegraphics*[scale=0.8]{W7_19.png}
    \end{center}
\end{itemize}
The good and bad of CCD:
\begin{itemize}
    \item Pros:
    \begin{itemize}
        \item Simple to implement
        \item Often effective
        \item Stable around singular configuration
        \item Computationally cheap
        \item Can combine withother more accurate optimization methods (such as Broyden-Fletcher-Shanno (BFS) when close enough)
    \end{itemize}
    \item Cons:
    \begin{itemize}
        \item Can lead to odd solutions if per-step deltas are not limited, making the method slow to converge
        \item Doesn't necessarily lead to smooth motion
    \end{itemize}
\end{itemize}

\subsection*{Comparison of the IK Methods}
\begin{center}
    \includegraphics*[scale=0.9]{W7_20.png}
\end{center}
\end{document}


